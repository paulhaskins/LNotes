\documentclass{report}

%%%%%%%%%%%%%%%%%%%%%%%%%%%%%%%%%
% PACKAGE IMPORTS
%%%%%%%%%%%%%%%%%%%%%%%%%%%%%%%%%


\usepackage[tmargin=2cm,rmargin=1in,lmargin=1in,margin=0.85in,bmargin=2cm,footskip=.2in]{geometry}
\usepackage{amsmath,amsfonts,amsthm,amssymb,mathtools}
\usepackage[varbb]{newpxmath}
\usepackage{xfrac}
\usepackage[makeroom]{cancel}
\usepackage{bookmark}
\usepackage{enumitem}
\usepackage{hyperref,theoremref}
\hypersetup{
	pdftitle={Assignment},
	colorlinks=true, linkcolor=doc!90,
	bookmarksnumbered=true,
	bookmarksopen=true
}
\usepackage[most,many,breakable]{tcolorbox}
\usepackage{xcolor}
\usepackage{varwidth}
\usepackage{varwidth}
\usepackage{tocloft}
\usepackage{etoolbox}
\usepackage{derivative} %many derivativess partials
%\usepackage{authblk}
\usepackage{nameref}
\usepackage{multicol,array}
\usepackage{tikz-cd}
\usepackage[ruled,vlined,linesnumbered]{algorithm2e}
\usepackage{comment} % enables the use of multi-line comments (\ifx \fi) 
\usepackage{import}
\usepackage{xifthen}
\usepackage{pdfpages}
\usepackage{transparent}
\usepackage{verbatim}

\newcommand\mycommfont[1]{\footnotesize\ttfamily\textcolor{blue}{#1}}
\SetCommentSty{mycommfont}
\newcommand{\incfig}[1]{%
    \def\svgwidth{\columnwidth}
    \import{./figures/}{#1.pdf_tex}
}
\usepackage[tagged, highstructure]{accessibility}
\usepackage{tikzsymbols}
\renewcommand\qedsymbol{$\Laughey$}


%\usepackage{import}
%\usepackage{xifthen}
%\usepackage{pdfpages}
%\usepackage{transparent}


%%%%%%%%%%%%%%%%%%%%%%%%%%%%%%
% SELF MADE COLORS
%%%%%%%%%%%%%%%%%%%%%%%%%%%%%%



\definecolor{myg}{RGB}{56, 140, 70}
\definecolor{myb}{RGB}{45, 111, 177}
\definecolor{myr}{RGB}{199, 68, 64}
\definecolor{mytheorembg}{HTML}{F2F2F9}
\definecolor{mytheoremfr}{HTML}{00007B}
\definecolor{mylenmabg}{HTML}{FFFAF8}
\definecolor{mylenmafr}{HTML}{983b0f}
\definecolor{mypropbg}{HTML}{f2fbfc}
\definecolor{mypropfr}{HTML}{191971}
\definecolor{myexamplebg}{HTML}{F2FBF8}
\definecolor{myexamplefr}{HTML}{88D6D1}
\definecolor{myexampleti}{HTML}{2A7F7F}
\definecolor{mydefinitbg}{HTML}{E5E5FF}
\definecolor{mydefinitfr}{HTML}{3F3FA3}
\definecolor{notesgreen}{RGB}{0,162,0}
\definecolor{myp}{RGB}{197, 92, 212}
\definecolor{mygr}{HTML}{2C3338}
\definecolor{myred}{RGB}{127,0,0}
\definecolor{myyellow}{RGB}{169,121,69}
\definecolor{myexercisebg}{HTML}{F2FBF8}
\definecolor{myexercisefg}{HTML}{88D6D1}


%%%%%%%%%%%%%%%%%%%%%%%%%%%%
% TCOLORBOX SETUPS
%%%%%%%%%%%%%%%%%%%%%%%%%%%%

\setlength{\parindent}{1cm}
%================================
% THEOREM BOX
%================================

\tcbuselibrary{theorems,skins,hooks}
\newtcbtheorem[number within=section]{Theorem}{Theorem}
{%
	enhanced,
	breakable,
	colback = mytheorembg,
	frame hidden,
	boxrule = 0sp,
	borderline west = {2pt}{0pt}{mytheoremfr},
	sharp corners,
	detach title,
	before upper = \tcbtitle\par\smallskip,
	coltitle = mytheoremfr,
	fonttitle = \bfseries\sffamily,
	description font = \mdseries,
	separator sign none,
	segmentation style={solid, mytheoremfr},
}
{th}

\tcbuselibrary{theorems,skins,hooks}
\newtcbtheorem[number within=chapter]{theorem}{Theorem}
{%
	enhanced,
	breakable,
	colback = mytheorembg,
	frame hidden,
	boxrule = 0sp,
	borderline west = {2pt}{0pt}{mytheoremfr},
	sharp corners,
	detach title,
	before upper = \tcbtitle\par\smallskip,
	coltitle = mytheoremfr,
	fonttitle = \bfseries\sffamily,
	description font = \mdseries,
	separator sign none,
	segmentation style={solid, mytheoremfr},
}
{th}


\tcbuselibrary{theorems,skins,hooks}
\newtcolorbox{Theoremcon}
{%
	enhanced
	,breakable
	,colback = mytheorembg
	,frame hidden
	,boxrule = 0sp
	,borderline west = {2pt}{0pt}{mytheoremfr}
	,sharp corners
	,description font = \mdseries
	,separator sign none
}

%================================
% Corollery
%================================
\tcbuselibrary{theorems,skins,hooks}
\newtcbtheorem[number within=section]{Corollary}{Corollary}
{%
	enhanced
	,breakable
	,colback = myp!10
	,frame hidden
	,boxrule = 0sp
	,borderline west = {2pt}{0pt}{myp!85!black}
	,sharp corners
	,detach title
	,before upper = \tcbtitle\par\smallskip
	,coltitle = myp!85!black
	,fonttitle = \bfseries\sffamily
	,description font = \mdseries
	,separator sign none
	,segmentation style={solid, myp!85!black}
}
{th}
\tcbuselibrary{theorems,skins,hooks}
\newtcbtheorem[number within=chapter]{corollary}{Corollary}
{%
	enhanced
	,breakable
	,colback = myp!10
	,frame hidden
	,boxrule = 0sp
	,borderline west = {2pt}{0pt}{myp!85!black}
	,sharp corners
	,detach title
	,before upper = \tcbtitle\par\smallskip
	,coltitle = myp!85!black
	,fonttitle = \bfseries\sffamily
	,description font = \mdseries
	,separator sign none
	,segmentation style={solid, myp!85!black}
}
{th}


%================================
% LENMA
%================================

\tcbuselibrary{theorems,skins,hooks}
\newtcbtheorem[number within=section]{Lenma}{Lenma}
{%
	enhanced,
	breakable,
	colback = mylenmabg,
	frame hidden,
	boxrule = 0sp,
	borderline west = {2pt}{0pt}{mylenmafr},
	sharp corners,
	detach title,
	before upper = \tcbtitle\par\smallskip,
	coltitle = mylenmafr,
	fonttitle = \bfseries\sffamily,
	description font = \mdseries,
	separator sign none,
	segmentation style={solid, mylenmafr},
}
{th}

\tcbuselibrary{theorems,skins,hooks}
\newtcbtheorem[number within=chapter]{lenma}{Lenma}
{%
	enhanced,
	breakable,
	colback = mylenmabg,
	frame hidden,
	boxrule = 0sp,
	borderline west = {2pt}{0pt}{mylenmafr},
	sharp corners,
	detach title,
	before upper = \tcbtitle\par\smallskip,
	coltitle = mylenmafr,
	fonttitle = \bfseries\sffamily,
	description font = \mdseries,
	separator sign none,
	segmentation style={solid, mylenmafr},
}
{th}


%================================
% PROPOSITION
%================================

\tcbuselibrary{theorems,skins,hooks}
\newtcbtheorem[number within=section]{Prop}{Proposition}
{%
	enhanced,
	breakable,
	colback = mypropbg,
	frame hidden,
	boxrule = 0sp,
	borderline west = {2pt}{0pt}{mypropfr},
	sharp corners,
	detach title,
	before upper = \tcbtitle\par\smallskip,
	coltitle = mypropfr,
	fonttitle = \bfseries\sffamily,
	description font = \mdseries,
	separator sign none,
	segmentation style={solid, mypropfr},
}
{th}

\tcbuselibrary{theorems,skins,hooks}
\newtcbtheorem[number within=chapter]{prop}{Proposition}
{%
	enhanced,
	breakable,
	colback = mypropbg,
	frame hidden,
	boxrule = 0sp,
	borderline west = {2pt}{0pt}{mypropfr},
	sharp corners,
	detach title,
	before upper = \tcbtitle\par\smallskip,
	coltitle = mypropfr,
	fonttitle = \bfseries\sffamily,
	description font = \mdseries,
	separator sign none,
	segmentation style={solid, mypropfr},
}
{th}


%================================
% CLAIM
%================================

\tcbuselibrary{theorems,skins,hooks}
\newtcbtheorem[number within=section]{claim}{Claim}
{%
	enhanced
	,breakable
	,colback = myg!10
	,frame hidden
	,boxrule = 0sp
	,borderline west = {2pt}{0pt}{myg}
	,sharp corners
	,detach title
	,before upper = \tcbtitle\par\smallskip
	,coltitle = myg!85!black
	,fonttitle = \bfseries\sffamily
	,description font = \mdseries
	,separator sign none
	,segmentation style={solid, myg!85!black}
}
{th}



%================================
% Exercise
%================================

\tcbuselibrary{theorems,skins,hooks}
\newtcbtheorem[number within=section]{Exercise}{Exercise}
{%
	enhanced,
	breakable,
	colback = myexercisebg,
	frame hidden,
	boxrule = 0sp,
	borderline west = {2pt}{0pt}{myexercisefg},
	sharp corners,
	detach title,
	before upper = \tcbtitle\par\smallskip,
	coltitle = myexercisefg,
	fonttitle = \bfseries\sffamily,
	description font = \mdseries,
	separator sign none,
	segmentation style={solid, myexercisefg},
}
{th}

\tcbuselibrary{theorems,skins,hooks}
\newtcbtheorem[number within=chapter]{exercise}{Exercise}
{%
	enhanced,
	breakable,
	colback = myexercisebg,
	frame hidden,
	boxrule = 0sp,
	borderline west = {2pt}{0pt}{myexercisefg},
	sharp corners,
	detach title,
	before upper = \tcbtitle\par\smallskip,
	coltitle = myexercisefg,
	fonttitle = \bfseries\sffamily,
	description font = \mdseries,
	separator sign none,
	segmentation style={solid, myexercisefg},
}
{th}

%================================
% EXAMPLE BOX
%================================

\newtcbtheorem[number within=section]{Example}{Example}
{%
	colback = myexamplebg
	,breakable
	,colframe = myexamplefr
	,coltitle = myexampleti
	,boxrule = 1pt
	,sharp corners
	,detach title
	,before upper=\tcbtitle\par\smallskip
	,fonttitle = \bfseries
	,description font = \mdseries
	,separator sign none
	,description delimiters parenthesis
}
{ex}

\newtcbtheorem[number within=chapter]{example}{Example}
{%
	colback = myexamplebg
	,breakable
	,colframe = myexamplefr
	,coltitle = myexampleti
	,boxrule = 1pt
	,sharp corners
	,detach title
	,before upper=\tcbtitle\par\smallskip
	,fonttitle = \bfseries
	,description font = \mdseries
	,separator sign none
	,description delimiters parenthesis
}
{ex}

%================================
% DEFINITION BOX
%================================

\newtcbtheorem[number within=section]{Definition}{Definition}{enhanced,
	before skip=2mm,after skip=2mm, colback=red!5,colframe=red!80!black,boxrule=0.5mm,
	attach boxed title to top left={xshift=1cm,yshift*=1mm-\tcboxedtitleheight}, varwidth boxed title*=-3cm,
	boxed title style={frame code={
					\path[fill=tcbcolback]
					([yshift=-1mm,xshift=-1mm]frame.north west)
					arc[start angle=0,end angle=180,radius=1mm]
					([yshift=-1mm,xshift=1mm]frame.north east)
					arc[start angle=180,end angle=0,radius=1mm];
					\path[left color=tcbcolback!60!black,right color=tcbcolback!60!black,
						middle color=tcbcolback!80!black]
					([xshift=-2mm]frame.north west) -- ([xshift=2mm]frame.north east)
					[rounded corners=1mm]-- ([xshift=1mm,yshift=-1mm]frame.north east)
					-- (frame.south east) -- (frame.south west)
					-- ([xshift=-1mm,yshift=-1mm]frame.north west)
					[sharp corners]-- cycle;
				},interior engine=empty,
		},
	fonttitle=\bfseries,
	title={#2},#1}{def}
\newtcbtheorem[number within=chapter]{definition}{Definition}{enhanced,
	before skip=2mm,after skip=2mm, colback=red!5,colframe=red!80!black,boxrule=0.5mm,
	attach boxed title to top left={xshift=1cm,yshift*=1mm-\tcboxedtitleheight}, varwidth boxed title*=-3cm,
	boxed title style={frame code={
					\path[fill=tcbcolback]
					([yshift=-1mm,xshift=-1mm]frame.north west)
					arc[start angle=0,end angle=180,radius=1mm]
					([yshift=-1mm,xshift=1mm]frame.north east)
					arc[start angle=180,end angle=0,radius=1mm];
					\path[left color=tcbcolback!60!black,right color=tcbcolback!60!black,
						middle color=tcbcolback!80!black]
					([xshift=-2mm]frame.north west) -- ([xshift=2mm]frame.north east)
					[rounded corners=1mm]-- ([xshift=1mm,yshift=-1mm]frame.north east)
					-- (frame.south east) -- (frame.south west)
					-- ([xshift=-1mm,yshift=-1mm]frame.north west)
					[sharp corners]-- cycle;
				},interior engine=empty,
		},
	fonttitle=\bfseries,
	title={#2},#1}{def}



%================================
% Solution BOX
%================================

\makeatletter
\newtcbtheorem{question}{Question}{enhanced,
	breakable,
	colback=white,
	colframe=myb!80!black,
	attach boxed title to top left={yshift*=-\tcboxedtitleheight},
	fonttitle=\bfseries,
	title={#2},
	boxed title size=title,
	boxed title style={%
			sharp corners,
			rounded corners=northwest,
			colback=tcbcolframe,
			boxrule=0pt,
		},
	underlay boxed title={%
			\path[fill=tcbcolframe] (title.south west)--(title.south east)
			to[out=0, in=180] ([xshift=5mm]title.east)--
			(title.center-|frame.east)
			[rounded corners=\kvtcb@arc] |-
			(frame.north) -| cycle;
		},
	#1
}{def}
\makeatother

%================================
% SOLUTION BOX
%================================

\makeatletter
\newtcolorbox{solution}{enhanced,
	breakable,
	colback=white,
	colframe=myg!80!black,
	attach boxed title to top left={yshift*=-\tcboxedtitleheight},
	title=Solution,
	boxed title size=title,
	boxed title style={%
			sharp corners,
			rounded corners=northwest,
			colback=tcbcolframe,
			boxrule=0pt,
		},
	underlay boxed title={%
			\path[fill=tcbcolframe] (title.south west)--(title.south east)
			to[out=0, in=180] ([xshift=5mm]title.east)--
			(title.center-|frame.east)
			[rounded corners=\kvtcb@arc] |-
			(frame.north) -| cycle;
		},
}
\makeatother

%================================
% Question BOX
%================================

\makeatletter
\newtcbtheorem{qstion}{Question}{enhanced,
	breakable,
	colback=white,
	colframe=mygr,
	attach boxed title to top left={yshift*=-\tcboxedtitleheight},
	fonttitle=\bfseries,
	title={#2},
	boxed title size=title,
	boxed title style={%
			sharp corners,
			rounded corners=northwest,
			colback=tcbcolframe,
			boxrule=0pt,
		},
	underlay boxed title={%
			\path[fill=tcbcolframe] (title.south west)--(title.south east)
			to[out=0, in=180] ([xshift=5mm]title.east)--
			(title.center-|frame.east)
			[rounded corners=\kvtcb@arc] |-
			(frame.north) -| cycle;
		},
	#1
}{def}
\makeatother

\newtcbtheorem[number within=chapter]{wconc}{Wrong Concept}{
	breakable,
	enhanced,
	colback=white,
	colframe=myr,
	arc=0pt,
	outer arc=0pt,
	fonttitle=\bfseries\sffamily\large,
	colbacktitle=myr,
	attach boxed title to top left={},
	boxed title style={
			enhanced,
			skin=enhancedfirst jigsaw,
			arc=3pt,
			bottom=0pt,
			interior style={fill=myr}
		},
	#1
}{def}



%================================
% NOTE BOX
%================================

\usetikzlibrary{arrows,calc,shadows.blur}
\tcbuselibrary{skins}
\newtcolorbox{note}[1][]{%
	enhanced jigsaw,
	colback=gray!20!white,%
	colframe=gray!80!black,
	size=small,
	boxrule=1pt,
	title=\textbf{Note:-},
	halign title=flush center,
	coltitle=black,
	breakable,
	drop shadow=black!50!white,
	attach boxed title to top left={xshift=1cm,yshift=-\tcboxedtitleheight/2,yshifttext=-\tcboxedtitleheight/2},
	minipage boxed title=1.5cm,
	boxed title style={%
			colback=white,
			size=fbox,
			boxrule=1pt,
			boxsep=2pt,
			underlay={%
					\coordinate (dotA) at ($(interior.west) + (-0.5pt,0)$);
					\coordinate (dotB) at ($(interior.east) + (0.5pt,0)$);
					\begin{scope}
						\clip (interior.north west) rectangle ([xshift=3ex]interior.east);
						\filldraw [white, blur shadow={shadow opacity=60, shadow yshift=-.75ex}, rounded corners=2pt] (interior.north west) rectangle (interior.south east);
					\end{scope}
					\begin{scope}[gray!80!black]
						\fill (dotA) circle (2pt);
						\fill (dotB) circle (2pt);
					\end{scope}
				},
		},
	#1,
}

%%%%%%%%%%%%%%%%%%%%%%%%%%%%%%
% SELF MADE COMMANDS
%%%%%%%%%%%%%%%%%%%%%%%%%%%%%%


\newcommand{\thm}[2]{\begin{Theorem}{#1}{}#2\end{Theorem}}
\newcommand{\cor}[2]{\begin{Corollary}{#1}{}#2\end{Corollary}}
\newcommand{\mlenma}[2]{\begin{Lenma}{#1}{}#2\end{Lenma}}
\newcommand{\mprop}[2]{\begin{Prop}{#1}{}#2\end{Prop}}
\newcommand{\clm}[3]{\begin{claim}{#1}{#2}#3\end{claim}}
\newcommand{\wc}[2]{\begin{wconc}{#1}{}\setlength{\parindent}{1cm}#2\end{wconc}}
\newcommand{\thmcon}[1]{\begin{Theoremcon}{#1}\end{Theoremcon}}
\newcommand{\ex}[2]{\begin{Example}{#1}{}#2\end{Example}}
\newcommand{\dfn}[2]{\begin{Definition}[colbacktitle=red!75!black]{#1}{}#2\end{Definition}}
\newcommand{\dfnc}[2]{\begin{definition}[colbacktitle=red!75!black]{#1}{}#2\end{definition}}
\newcommand{\qs}[2]{\begin{question}{#1}{}#2\end{question}}
\newcommand{\pf}[2]{\begin{myproof}[#1]#2\end{myproof}}
\newcommand{\nt}[1]{\begin{note}#1\end{note}}

\newcommand*\circled[1]{\tikz[baseline=(char.base)]{
		\node[shape=circle,draw,inner sep=1pt] (char) {#1};}}
\newcommand\getcurrentref[1]{%
	\ifnumequal{\value{#1}}{0}
	{??}
	{\the\value{#1}}%
}
\newcommand{\getCurrentSectionNumber}{\getcurrentref{section}}
\newenvironment{myproof}[1][\proofname]{%
	\proof[\bfseries #1: ]%
}{\endproof}

\newcommand{\mclm}[2]{\begin{myclaim}[#1]#2\end{myclaim}}
\newenvironment{myclaim}[1][\claimname]{\proof[\bfseries #1: ]}{}

\newcounter{mylabelcounter}

\makeatletter
\newcommand{\setword}[2]{%
	\phantomsection
	#1\def\@currentlabel{\unexpanded{#1}}\label{#2}%
}
\makeatother




\tikzset{
	symbol/.style={
			draw=none,
			every to/.append style={
					edge node={node [sloped, allow upside down, auto=false]{$#1$}}}
		}
}


% deliminators
\DeclarePairedDelimiter{\abs}{\lvert}{\rvert}
\DeclarePairedDelimiter{\norm}{\lVert}{\rVert}

\DeclarePairedDelimiter{\ceil}{\lceil}{\rceil}
\DeclarePairedDelimiter{\floor}{\lfloor}{\rfloor}
\DeclarePairedDelimiter{\round}{\lfloor}{\rceil}

\newsavebox\diffdbox
\newcommand{\slantedromand}{{\mathpalette\makesl{d}}}
\newcommand{\makesl}[2]{%
\begingroup
\sbox{\diffdbox}{$\mathsurround=0pt#1\mathrm{#2}$}%
\pdfsave
\pdfsetmatrix{1 0 0.2 1}%
\rlap{\usebox{\diffdbox}}%
\pdfrestore
\hskip\wd\diffdbox
\endgroup
}
\newcommand{\dd}[1][]{\ensuremath{\mathop{}\!\ifstrempty{#1}{%
\slantedromand\@ifnextchar^{\hspace{0.2ex}}{\hspace{0.1ex}}}%
{\slantedromand\hspace{0.2ex}^{#1}}}}
\ProvideDocumentCommand\dv{o m g}{%
  \ensuremath{%
    \IfValueTF{#3}{%
      \IfNoValueTF{#1}{%
        \frac{\dd #2}{\dd #3}%
      }{%
        \frac{\dd^{#1} #2}{\dd #3^{#1}}%
      }%
    }{%
      \IfNoValueTF{#1}{%
        \frac{\dd}{\dd #2}%
      }{%
        \frac{\dd^{#1}}{\dd #2^{#1}}%
      }%
    }%
  }%
}
\providecommand*{\pdv}[3][]{\frac{\partial^{#1}#2}{\partial#3^{#1}}}
%  - others
\DeclareMathOperator{\Lap}{\mathcal{L}}
\DeclareMathOperator{\Var}{Var} % varience
\DeclareMathOperator{\Cov}{Cov} % covarience
\DeclareMathOperator{\E}{E} % expected

% Since the amsthm package isn't loaded

% I prefer the slanted \leq
\let\oldleq\leq % save them in case they're every wanted
\let\oldgeq\geq
\renewcommand{\leq}{\leqslant}
\renewcommand{\geq}{\geqslant}

% % redefine matrix env to allow for alignment, use r as default
% \renewcommand*\env@matrix[1][r]{\hskip -\arraycolsep
%     \let\@ifnextchar\new@ifnextchar
%     \array{*\c@MaxMatrixCols #1}}


%\usepackage{framed}
%\usepackage{titletoc}
%\usepackage{etoolbox}
%\usepackage{lmodern}


%\patchcmd{\tableofcontents}{\contentsname}{\sffamily\contentsname}{}{}

%\renewenvironment{leftbar}
%{\def\FrameCommand{\hspace{6em}%
%		{\color{myyellow}\vrule width 2pt depth 6pt}\hspace{1em}}%
%	\MakeFramed{\parshape 1 0cm \dimexpr\textwidth-6em\relax\FrameRestore}\vskip2pt%
%}
%{\endMakeFramed}

%\titlecontents{chapter}
%[0em]{\vspace*{2\baselineskip}}
%{\parbox{4.5em}{%
%		\hfill\Huge\sffamily\bfseries\color{myred}\thecontentspage}%
%	\vspace*{-2.3\baselineskip}\leftbar\textsc{\small\chaptername~\thecontentslabel}\\\sffamily}
%{}{\endleftbar}
%\titlecontents{section}
%[8.4em]
%{\sffamily\contentslabel{3em}}{}{}
%{\hspace{0.5em}\nobreak\itshape\color{myred}\contentspage}
%\titlecontents{subsection}
%[8.4em]
%{\sffamily\contentslabel{3em}}{}{}  
%{\hspace{0.5em}\nobreak\itshape\color{myred}\contentspage}



%%%%%%%%%%%%%%%%%%%%%%%%%%%%%%%%%%%%%%%%%%%
% TABLE OF CONTENTS
%%%%%%%%%%%%%%%%%%%%%%%%%%%%%%%%%%%%%%%%%%%

\usepackage{tikz}
\definecolor{doc}{RGB}{0,60,110}
\usepackage{titletoc}
\contentsmargin{0cm}
\titlecontents{chapter}[3.7pc]
{\addvspace{30pt}%
	\begin{tikzpicture}[remember picture, overlay]%
		\draw[fill=doc!60,draw=doc!60] (-7,-.1) rectangle (-0.9,.5);%
		\pgftext[left,x=-3.5cm,y=0.2cm]{\color{white}\Large\sc\bfseries Chapter\ \thecontentslabel};%
	\end{tikzpicture}\color{doc!60}\large\sc\bfseries}%
{}
{}
{\;\titlerule\;\large\sc\bfseries Page \thecontentspage
	\begin{tikzpicture}[remember picture, overlay]
		\draw[fill=doc!60,draw=doc!60] (2pt,0) rectangle (4,0.1pt);
	\end{tikzpicture}}%
\titlecontents{section}[3.7pc]
{\addvspace{2pt}}
{\contentslabel[\thecontentslabel]{2pc}}
{}
{\hfill\small \thecontentspage}
[]
\titlecontents*{subsection}[3.7pc]
{\addvspace{-1pt}\small}
{}
{}
{\ --- \small\thecontentspage}
[ \textbullet\ ][]

\makeatletter
\renewcommand{\tableofcontents}{%
	\chapter*{%
	  \vspace*{-20\p@}%
	  \begin{tikzpicture}[remember picture, overlay]%
		  \pgftext[right,x=15cm,y=0.2cm]{\color{doc!60}\Huge\sc\bfseries \contentsname};%
		  \draw[fill=doc!60,draw=doc!60] (13,-.75) rectangle (20,1);%
		  \clip (13,-.75) rectangle (20,1);
		  \pgftext[right,x=15cm,y=0.2cm]{\color{white}\Huge\sc\bfseries \contentsname};%
	  \end{tikzpicture}}%
	\@starttoc{toc}}
\makeatother


%From M275 "Topology" at SJSU
\newcommand{\id}{\mathrm{id}}
\newcommand{\taking}[1]{\xrightarrow{#1}}
\newcommand{\inv}{^{-1}}

%From M170 "Introduction to Graph Theory" at SJSU
\DeclareMathOperator{\diam}{diam}
\DeclareMathOperator{\ord}{ord}
\newcommand{\defeq}{\overset{\mathrm{def}}{=}}

%From the USAMO .tex files
\newcommand{\ts}{\textsuperscript}
\newcommand{\dg}{^\circ}
\newcommand{\ii}{\item}

% % From Math 55 and Math 145 at Harvard
% \newenvironment{subproof}[1][Proof]{%
% \begin{proof}[#1] \renewcommand{\qedsymbol}{$\blacksquare$}}%
% {\end{proof}}

\newcommand{\liff}{\leftrightarrow}
\newcommand{\lthen}{\rightarrow}
\newcommand{\opname}{\operatorname}
\newcommand{\surjto}{\twoheadrightarrow}
\newcommand{\injto}{\hookrightarrow}
\newcommand{\On}{\mathrm{On}} % ordinals
\DeclareMathOperator{\img}{im} % Image
\DeclareMathOperator{\Img}{Im} % Image
\DeclareMathOperator{\coker}{coker} % Cokernel
\DeclareMathOperator{\Coker}{Coker} % Cokernel
\DeclareMathOperator{\Ker}{Ker} % Kernel
\DeclareMathOperator{\rank}{rank}
\DeclareMathOperator{\Spec}{Spec} % spectrum
\DeclareMathOperator{\Tr}{Tr} % trace
\DeclareMathOperator{\pr}{pr} % projection
\DeclareMathOperator{\ext}{ext} % extension
\DeclareMathOperator{\pred}{pred} % predecessor
\DeclareMathOperator{\dom}{dom} % domain
\DeclareMathOperator{\ran}{ran} % range
\DeclareMathOperator{\Hom}{Hom} % homomorphism
\DeclareMathOperator{\Mor}{Mor} % morphisms
\DeclareMathOperator{\End}{End} % endomorphism

\newcommand{\eps}{\epsilon}
\newcommand{\veps}{\varepsilon}
\newcommand{\ol}{\overline}
\newcommand{\ul}{\underline}
\newcommand{\wt}{\widetilde}
\newcommand{\wh}{\widehat}
\newcommand{\vocab}[1]{\textbf{\color{blue} #1}}
\providecommand{\half}{\frac{1}{2}}
\newcommand{\dang}{\measuredangle} %% Directed angle
\newcommand{\ray}[1]{\overrightarrow{#1}}
\newcommand{\seg}[1]{\overline{#1}}
\newcommand{\arc}[1]{\wideparen{#1}}
\DeclareMathOperator{\cis}{cis}
\DeclareMathOperator*{\lcm}{lcm}
\DeclareMathOperator*{\argmin}{arg min}
\DeclareMathOperator*{\argmax}{arg max}
\newcommand{\cycsum}{\sum_{\mathrm{cyc}}}
\newcommand{\symsum}{\sum_{\mathrm{sym}}}
\newcommand{\cycprod}{\prod_{\mathrm{cyc}}}
\newcommand{\symprod}{\prod_{\mathrm{sym}}}
\newcommand{\Qed}{\begin{flushright}\qed\end{flushright}}
\newcommand{\parinn}{\setlength{\parindent}{1cm}}
\newcommand{\parinf}{\setlength{\parindent}{0cm}}
% \newcommand{\norm}{\|\cdot\|}
\newcommand{\inorm}{\norm_{\infty}}
\newcommand{\opensets}{\{V_{\alpha}\}_{\alpha\in I}}
\newcommand{\oset}{V_{\alpha}}
\newcommand{\opset}[1]{V_{\alpha_{#1}}}
\newcommand{\lub}{\text{lub}}
\newcommand{\del}[2]{\frac{\partial #1}{\partial #2}}
\newcommand{\Del}[3]{\frac{\partial^{#1} #2}{\partial^{#1} #3}}
\newcommand{\deld}[2]{\dfrac{\partial #1}{\partial #2}}
\newcommand{\Deld}[3]{\dfrac{\partial^{#1} #2}{\partial^{#1} #3}}
\newcommand{\lm}{\lambda}
\newcommand{\uin}{\mathbin{\rotatebox[origin=c]{90}{$\in$}}}
\newcommand{\usubset}{\mathbin{\rotatebox[origin=c]{90}{$\subset$}}}
\newcommand{\lt}{\left}
\newcommand{\rt}{\right}
\newcommand{\bs}[1]{\boldsymbol{#1}}
\newcommand{\exs}{\exists}
\newcommand{\st}{\strut}
\newcommand{\dps}[1]{\displaystyle{#1}}

\newcommand{\sol}{\setlength{\parindent}{0cm}\textbf{\textit{Solution:}}\setlength{\parindent}{1cm} }
\newcommand{\solve}[1]{\setlength{\parindent}{0cm}\textbf{\textit{Solution: }}\setlength{\parindent}{1cm}#1 \Qed}

%--------------------------------------------------
% LIE ALGEBRAS
%--------------------------------------------------
\newcommand*{\kb}{\mathfrak{b}}  % Borel subalgebra
\newcommand*{\kg}{\mathfrak{g}}  % Lie algebra
\newcommand*{\kh}{\mathfrak{h}}  % Cartan subalgebra
\newcommand*{\kn}{\mathfrak{n}}  % Nilradical
\newcommand*{\ku}{\mathfrak{u}}  % Unipotent algebra
\newcommand*{\kz}{\mathfrak{z}}  % Center of algebra

%--------------------------------------------------
% HOMOLOGICAL ALGEBRA
%--------------------------------------------------
\DeclareMathOperator{\Ext}{Ext} % Ext functor
\DeclareMathOperator{\Tor}{Tor} % Tor functor

%--------------------------------------------------
% MATRIX & GROUP NOTATION
%--------------------------------------------------
\DeclareMathOperator{\GL}{GL} % General Linear Group
\DeclareMathOperator{\SL}{SL} % Special Linear Group
\newcommand*{\gl}{\operatorname{\mathfrak{gl}}} % General linear Lie algebra
\newcommand*{\sl}{\operatorname{\mathfrak{sl}}} % Special linear Lie algebra

%--------------------------------------------------
% NUMBER SETS
%--------------------------------------------------
\newcommand*{\RR}{\mathbb{R}}
\newcommand*{\NN}{\mathbb{N}}
\newcommand*{\ZZ}{\mathbb{Z}}
\newcommand*{\QQ}{\mathbb{Q}}
\newcommand*{\CC}{\mathbb{C}}
\newcommand*{\PP}{\mathbb{P}}
\newcommand*{\HH}{\mathbb{H}}
\newcommand*{\FF}{\mathbb{F}}
\newcommand*{\EE}{\mathbb{E}} % Expected Value

%--------------------------------------------------
% MATH SCRIPT, FRAKTUR, AND BOLD SYMBOLS
%--------------------------------------------------
\newcommand*{\mcA}{\mathcal{A}}
\newcommand*{\mcB}{\mathcal{B}}
\newcommand*{\mcC}{\mathcal{C}}
\newcommand*{\mcD}{\mathcal{D}}
\newcommand*{\mcE}{\mathcal{E}}
\newcommand*{\mcF}{\mathcal{F}}
\newcommand*{\mcG}{\mathcal{G}}
\newcommand*{\mcH}{\mathcal{H}}

\newcommand*{\mfA}{\mathfrak{A}}  \newcommand*{\mfB}{\mathfrak{B}}
\newcommand*{\mfC}{\mathfrak{C}}  \newcommand*{\mfD}{\mathfrak{D}}
\newcommand*{\mfE}{\mathfrak{E}}  \newcommand*{\mfF}{\mathfrak{F}}
\newcommand*{\mfG}{\mathfrak{G}}  \newcommand*{\mfH}{\mathfrak{H}}

\usepackage{bm} % Ensure bold math works correctly
\newcommand*{\bmA}{\bm{A}}
\newcommand*{\bmB}{\bm{B}}
\newcommand*{\bmC}{\bm{C}}
\newcommand*{\bmD}{\bm{D}}
\newcommand*{\bmE}{\bm{E}}
\newcommand*{\bmF}{\bm{F}}
\newcommand*{\bmG}{\bm{G}}
\newcommand*{\bmH}{\bm{H}}

%--------------------------------------------------
% FUNCTIONAL ANALYSIS & ALGEBRA
%--------------------------------------------------
\DeclareMathOperator{\Aut}{Aut} % Automorphism group
\DeclareMathOperator{\Inn}{Inn} % Inner automorphisms
\DeclareMathOperator{\Syl}{Syl} % Sylow subgroups
\DeclareMathOperator{\Gal}{Gal} % Galois group
\DeclareMathOperator{\sign}{sign} % Sign function


%\usepackage[tagged, highstructure]{accessibility}
\usepackage{tocloft}
\usepackage{arydshln}




\begin{document}
\title{Linear Algebra I}
\author{Lecture Notes Provided by Dr.~Miriam Logan.}
\date{}
\maketitle
\tableofcontents
\newpage
\section{}
XXX INCLUDE START OF W4
\\
\\
\section{Identity Matrix}
\dfn{Identity Matrix:}{
        The $n\times n$ Identity matrix is the $n\times n$ matrix whose $\left( i,j \right) $ entry is 1  $i=j$ and $0$ if $i\neq j$ i.e. it is the $n\times n$ matrix where the diagonal entries are all $1$ and all the other entries are zero. \\
$ I_n =\begin{bmatrix}
    1 & 0 & 0 & \dots  & 0 \\
    0 & 1 & 0 & \dots  & 0 \\
    \vdots & \vdots & \vdots & \ddots & \vdots \\
    0 & 0 & 0 & \dots  & 1\end{bmatrix} \text{This diagonal is known as the main diagonal of an $n\times n$ matrix.}$
}
For each $m\times n$ matrix we $A$, we have that $I_mA =A$ and $A I_n =A$
 \ex{}{
Let 
\begin{align*}
        A = \begin{bmatrix}
        1 & -5 & 2\\
        -3 & 0 & 7\\
        \end{bmatrix}\\
        A I_3 =  \begin{bmatrix}
        1 & -5 & 2\\
        -3 & 0 & 7\
        \end{bmatrix} \begin{bmatrix}
        1 & 0 & 0\\
        0 & 1 & 0\\
        0 & 0 & 1\\
        \end{bmatrix} = \begin{bmatrix}
        1 & -5 & 2\\
        -3 & 0 & 7\\
        \end{bmatrix}\\
        I_2 A = \begin{bmatrix}
        1 & 0\\
        0 & 1\\
        \end{bmatrix}\begin{bmatrix}
        1 & -5 & 2\\
        -3 & 0 & 7\\
        \end{bmatrix}= \begin{bmatrix}
        1 & -5 & 2\\
        -3 & 0 & 7\\
        \end{bmatrix}   
.\end{align*}}
\section{Invertible Matrices}
\dfn{Invertible Matrix :}{
An $nxn$ matrix $A$ is said to be invertible if there exists an $n\times n$ matrix $B$ such that $AB =BA =I_n$ \\
The matrix $B$ is called the \underline{inverse } of $A$ and is denoted by $A^{-1}$
}
\nt{
\begin{itemize}
        \item $A^{-1}$ is unique.
                \pf{Proof:}{(proof by contradiction)\\
 Suppose there existed another $n\times n$ matrix $C $ such that $AC = CA =I_n$ then
\[
C= CI_n = C \left( A A^{-1} \right) = \left( CA \right) \left( A^{-1} \right) =I_nA^{-1}=A^{-1} \text{  i.e. } C =A^{-1}
.\] }
\item If $A$ and $B$ are both invariable $n\times  n$ matrices the $AB$ is invertible and $\left( AB \right) ^{-1}= B ^{-1}A^{-1}$ 
        \pf{Proof:}{
        \[
        \left( AB \right) \left( B^{-1} A^{-1} \right) = A \left( B B^{-1} \right) A^{-1} = A \left( I_n \right) A^{-1}=I_n
        .\] This shows that $B^{-1}A^{-1}$ is an inverse of the matrix $AB$ and since inverses are unique we conclude that $\left( AB \right) ^{-1}= B^{-1}A^{-1}$.
        }
\item If $A$ is an invertible $n\times  n $ matrix then \[
\left( A^{-1} \right) ^{-1}=A

.\] 
\pf{Proof:}{
Since $A A^{-1}= A^{-1}A= I_n$\\
$\implies$ $A$ is the inverse of $A^{-1}$ i.e. $A = \left( A^{-1} \right) ^{-1}$.

}

\end{itemize}
}
\section{Inverse of a $2\times 2$ Matrix}\\
Let $A = \begin{bmatrix}
a & b\\
c & d\\
\end{bmatrix}$. We want to find a $2\times 2$ matrix $A^{-1}= \begin{bmatrix}
x & y\\
z & w\\
\end{bmatrix}$ such that $A A^{-1}= A^{-1}A= I_2$.\\
i.e. $\begin{bmatrix}
a & b\\
c & d\\
\end{bmatrix}\begin{bmatrix}
x & y\\
z & w\\
\end{bmatrix}= \begin{bmatrix}
1 & 0\\
0 & 1\\
\end{bmatrix}$
\begin{multicols}{2}
\begin{align*}
        ax+bz=1\\
        ay+bw=0\\
        cx+dz=0\\
        cy+dw=1\\
.\end{align*}

\break
\begin{align*}
        ax=1-bz \implies x= \frac{1-bz}{a}\\
        cx=-dz \implies x= - \frac{dz}{c}\\
        \implies \frac{1-bz}{a}= - \frac{dz}{c}\\
        \frac{1}{a} - \frac{b}{a}z + \frac{dz}{c}=0\\
        z \left( \frac{d}{c}-\frac{b}{a} \right) = - \frac{1}{a}\\
        z \left( \frac{ad-bc}{ac} \right) =-\frac{1}{a}\\
        \implies z= \frac{-ac}{a \left( ad-bc \right) }\\
        z= \frac{-c}{ad-bc}
.\end{align*}
\end{multicols}
Hence, $x = - \frac{d }{c}\left(  - \frac{c}{ad-bc} \right) = \frac{d}{ad-bc}$\\
Similarly it can be show that\\
$y= - \frac{b}{ad-bc}$ and $w = \frac{a}{ad-bc}$ \\
\[
\implies A^{-1}= \begin{bmatrix}
\frac{d}{ad-bc} & \frac{-b}{ad-bc}\\
 \frac{-c}{ad-bc}& \frac{a}{ad-bc}\\
\end{bmatrix} = \frac{1}{ad-bc} \begin{bmatrix}
d & -b\\
-c & d\\
\end{bmatrix}
.\] 
\nt{This definition depends on what  $ad-bc \neq 0$ i.e. the inverse is not defined if $ad-bc=0$.}
\dfn{Determinant of a $2\times 2$ Matrix :}{
For a $2\times 2$ matrix $A = \begin{bmatrix}
a & b\\
c & d\\
\end{bmatrix}$ the quantity $ad-bc$ is known as the \underline{determinant of $A$.} It is denoted by $det \left( A \right) $ \[
\text{det}\begin{bmatrix}
a & b\\
c & d\\
\end{bmatrix} = ad-bc
.\]  If $ad-bc=0$ then $A $ is not invertible. 
}
\ex{}{

\begin{enumerate}[label=(\alph*)]
        \item  Let $A = \begin{bmatrix}
        1 & 3\\
        2 & 7\\
        \end{bmatrix}$ Find $A^{-1}$ :\\
        \[
        A^{-1}= \frac{1}{7-6}\begin{bmatrix}
        7 & -3\\
        -2 & 1\\
        \end{bmatrix}= \begin{bmatrix}
        7 & -3\\
        -2 & 1\\
        \end{bmatrix}
        .\] 
  \item  Rewrite the following linear system as a matrix equation and hence solve:\\
          \begin{multicols}{2}
          \begin{align*}
                  x_1+3x_2=0\\
                  2x_1+7x_2=-1\\
          .\end{align*}
          
          \break
          \begin{align*}
                \Big\to  \begin{bmatrix}
                  1 & 3\\
                  2 & 7\\
                  \end{bmatrix}\begin{bmatrix}
                  x_1 \\
                  x_2\\
                  \end{bmatrix}= \begin{bmatrix}
                  0 \\
                  -1 \\
                  \end{bmatrix}
          .\end{align*}
          \end{multicols}
          We have the form $A \vec{x} = \vec{b } $ : multiplying both sides on the left by $A^{-1}$ we get
          \begin{align*}
                  A^{-1}\left( A\vec{x}  \right) = A^{-1}\vec{b}\\
                  \left( A^{-1}A \right) \vec{x} = A^{-1}\vec{b}\\
                  \vec{x} = A^{-1}\vec{b}\\
                  \implies \begin{bmatrix}
                  x_1\\
                  x_2\\
                  \end{bmatrix}
                = \begin{bmatrix}
                7 & -3\\
                -2 & 1\\
                \end{bmatrix}\begin{bmatrix}
                0\\
                -1\\
                \end{bmatrix}
                = \begin{bmatrix}
                3\\
                -1\\
                \end{bmatrix}
                \text{ i.e. } x_1=3 \text{ } x_2=-1
          .\end{align*}
          
\end{enumerate}
}
\thm{}{Suppose $A$ is an invertible $n \times n$ matrix. For every $\vec{b} \in \mathbb{R}^{n}$ the equation 
        \[
        A \vec{x} =\vec{b} 
        .\] has a unique solution $\vec{x} =A^{-1}\vec{b} $
}
\pf{Proof:}{
First we will show that $A^{-1}\vec{b} $ is a solution to $A \vec{x} = \vec{b } $, then we will show that it is the unique solution.\\
\[
A \left( A^{-1}\vec{b}  \right) = \left( A A^{-1} \vec{b}  \right) = I_n \vec{b } = \vec{b } 
.\] 

i.e. $A^{-1}\vec{b} $ is a solution to the matrix equation $A \vec{x} =\vec{b} $.\\
\underline{Uniqueness:} Suppose $\vec{u} \in \mathbb{R}^{n}$ is also a solution to $A\vec{x} =\vec{b} $ i.e.  $A\vec{u} = \vec{b} $.\\
Multiplying both sides of this equation by $A^{-1}$ on the left we get     
}

\begin{align*}
        A^{-1} \left( A \vec{u }  \right) = A^{-1}\vec{b}\\
        \text{ie} I_n \vec{u} = A^{-1}\vec{b } \\
        \text{ie} \vec{u}  = A^{-1} \vec{b } \\
.\end{align*}
Hence, $A^{-1} \vec{b}  $ is a unique solution to $A \vec{x} = \vec{b } $.\\
\section{Elementary Matrices}
\dfn{Elementary matrices :}{
An \underline{elementary matrix} 
 is a matrix that is obtained by performing a single elementary row operation on the identity matrix.
}
\ex{}{
\begin{align*}
        \begin{bmatrix}
        1 & 0 & 0\\
        0 & 1 & 0\\
        0 & 0 & 1\\
        \end{bmatrix} \xrightarrow[r_3-2r_1]{} \begin{bmatrix}
        1 & 0 & 0\\
        0 & 1 & 0\\
        -2 & 0 & 1\\
        \end{bmatrix}\\
        \begin{bmatrix}
         1& 0 & 0 & 0\\
        0 & 1 & 0 & 0\\
        0 & 0 & 1 & 0\\
        0 & 0 & 0 & 1\\
        \end{bmatrix} \xrightarrow[r_2 \times 5]{}
        \begin{bmatrix}
        1 & 0 & 0 & 0\\
        0 & 5 & 0 & 0\\
        0 & 0 & 1 & 0\\
        0 & 0 & 0 & 1\\
        \end{bmatrix}\\
        \begin{bmatrix}
        1 & 0\\
        0 & 1\
        \end{bmatrix} \xrightarrow[r_1\leftrightarrow r_2]{} \begin{bmatrix}
        0 & 1\\
        1 & 0\\
        \end{bmatrix}
.\end{align*}
}
\ex{}{
Let $A = \begin{bmatrix}
a & b & c\\
d & e &f \\
 g& h & i\\
\end{bmatrix}$ and Let
\[
E_1 = \begin{bmatrix}
1 & 0 & 0\\
0 & 1 & 0\\
-2 & 0 & 1\\
\end{bmatrix} \text{ } E_2 = \begin{bmatrix}
1 & 0 & 0\\
0 & 5 & 0\\
0 & 0 & 1\\
\end{bmatrix} \text{ } E_3 = \begin{bmatrix}
1 & 0 & 0\\
0 & 0 & 1\\
0 & 1 & 0\\
\end{bmatrix}
.\] 
Find $E_1A$, $E_2A$ and $E_3 A$ and describe the effect that multiplying $A$ on the left by these matrices has on the rows of $A$.\\
\[
E_1A = \begin{bmatrix}
a & b & c\\
d  & e  & f\\
g-2a & h-2b & i-2c\\
\end{bmatrix}           
.\] $r_3$ is replaced with  $r_3 - 2r_1$. Note that  $r_3 -2r_1$ was exactly the row operation performed on the identity matrix to obtain $E_1 $ 
\[
E_2A = \begin{bmatrix}
a & b & c\\
5d  & 5e & 5f\\
g & h & i\\
\end{bmatrix}
.\] 
$r_2$ is replaced with  $5r_2$. Note that $5r_2$ was exactly the row operation performed on the identity matrix to obtain $E_2$.
\[
E_3A = \begin{bmatrix}
a & b & c\\
g & h & i\\
 d & e &f \\
\end{bmatrix}
.\] 
$r_2$ and $r_3 $ of $A $are interchanged $\left( r_2 \lefrrightarrow r_3 \right) $.\\
Note that $r_2 \leftrightarrow r_3$ was exactly the row operation performed on the identity matrix to obtain $E_3$.
}
\textbf{Conclusions:}\\
\begin{enumerate}[label=(\roman*)]
  \item Row operations can be performed on a matrix $A$ by multiplying $ A$ on the left by an elementary matrix.
  \item The row operation performed on $I_n$ to obtain an elementary matrix  $E$ is the same as the row operation that $E$ induces on $A$ by multiplying $A$  on the left by $E$.\\
  \end{enumerate}
  \textbf{Question:}\\
  Are row operations reversible?\\
  Yes!\\
  \begin{itemize}
          \item Scaling: (non-zero scalar) 
                  $r_i \to \alpha r_i $,  $\alpha \neq 0 $ can be reversed by $r_i \to \frac{1}{\alpha }r_i$
          \item Replacement: $r_i \to r_i + \beta r_j$ can be reversed by  $r_i \to r_i - \beta r_j$, $\left( \beta \neq 0 \right) $.
          \item Interchanging: $r_i \leftrightarrow r_j$can be reversed by repeating this again. 
  \end{itemize}
  \textit{What does this tell us about elementary matrices?} \\
  They are invertible i.e.  for each elementary matrix $E$, the exists $E^{-1}$ such that $E E^{-1}= E^{-1}E=I$.
\[
E_1 = \begin{bmatrix}
1 & 0 & 0\\
0 & 1 & 0\\
-2 & 0 & 1\\
\end{bmatrix} \text{ } E_2 = \begin{bmatrix}
1 & 0 & 0\\
0 & 5 & 0\\
0 & 0 & 1\\
\end{bmatrix} \text{ } E_3 = \begin{bmatrix}
1 & 0 & 0\\
0 & 0 & 1\\
0 & 1 & 0\\
\end{bmatrix}
.\] Find the inverses of the elementary matrices above.\\
\[
E_1^{-1}= \begin{bmatrix}
1 & 0 & 0\\
0 & 1 & 0\\
2 & 0 & 1\\
\end{bmatrix} \text{  } E_2^{-1} = \begin{bmatrix}
1 & 0 & 0\\
0 & \frac{1}{5} & 0\\
0 & 0 & 1\\
\end{bmatrix} \text{  } E_3^{-1} = \begin{bmatrix}
1 & 0 & 0\\
0 & 0 & 1\\
0 & 1 & 0\\
\end{bmatrix}   
.\] 
\nt{Each elementary matrix  $E$ is invertible. The inverse of $E$ is an elementary matrix of the same type (replacement, scaling, interchanging) that transforms  $E$ back into the identity}
\nt{The reduced row echelon form of an $n \times n$ matrix is either:
\begin{itemize}
        \item the $n \times n$ identity matrix ( if there are n pivots)
        \item \underline{or}  a matrix whose last row has all zeros.
\end{itemize}}
\thm{}
{
An $n \times n$ matrix $A$ is invertible if and only if $A$ is row equivalent to the $n \times n$ identity matrix $I_n$
}
\pf{Proof:}{\\
$\implies$\\
Suppose $A$ is invertible (we want to show that $A$ is equivalent to  $I_n $ ). Then $A\vec{x} =\vec{b}$ has a unique solution for every $\vec{b} \in \mathbb{R}^{n}$. This implies that $A$ has a pivot position in each column. Since  $A$ is a square matrix it also has a pivot in every row. The  $n$ pivots lie along the main diagonal of $A$ (the pivot positions $a_{ij}$ where $i=j$ ). Thus the reduced row echelon form of A has leading $1$'s along the diagonal and zeros above and below them i.e.  reduced row echelon form of $A$ is $I_n$. \\
$\impliedby$ \\
Conversely, suppose $A $ is row equivalent to  $I_n$. This means that there exists a finite number of elementary matrices $E_1, \ldots ,E_p$ such that \[
E_pE_{p-1}\ldots E_2E_1A =I_n
.\] 

Since each $E_j$ is invertible, the product of  $p$ elementary matrices is invertible.\\
We multiply both sides of the equation above on the left by  $\left( E_pE_{p-1}\ldots E_2E_1 \right) ^{-1}$. to get 
\[
\left( E_pE_{p-1}\ldots E_2E_1 \right) ^{-1} \left( E_pE_{p-1}\ldots E_2E_1 \right) A = \left( E_pE_{p-1}\ldots E_2E_1 \right) ^{-1} I_n
.\] 
\[
\implies A = \left( E_pE_{p-1}\ldots E_2E_1 \right) ^{-1}
.\] 
Since $A$ is equal to the inverse of an invariable matrix we can conclude that $A$ is invertible
}

What's more is that this theorem gives us a formula for calculating $A^{-1}$.
\begin{align*}
        A^{-1}= \left( E_pE_{p-1}\ldots E_2E_1 \right)\\
.\end{align*}
Therefore the same sequence of row operations that turned $A$ into $I_n$ will turn $I_n $ into $A^{-1}$.\\
$\implies$ In order to find $A^{-1}$ we row reduce the $n \times 2n$ matrix $\left[ A, I_n \right] $ to $\left[ I_n, A^{-1} \right] $.
\ex{}{
\begin{enumerate} [label=(\alph*)]
  \item Let $A = \begin{bmatrix}
  0 & 1 & 2\\
  1 & 0 & 3\\
  4 & -3 & 8\\
  \end{bmatrix}$ Find $A^{-1}$ \\
  \begin{align*}
          \left[
          \begin{array}{ccc;{2pt/2pt}ccc}  
            0 & 1 & 2 & 1 & 0 & 0\\
            1 & 0 & 3 & 0 & 1 & 0\\
            4 & -3 & 8 & 0 & 0 & 1\\
          \end{array}
          \right] \xrightarrow[r_1 \leftrightarrow r_2]{}
          \left[
          \begin{array}{ccc;{2pt/2pt}ccc}  
            1 & 0 & 3 & 0 & 1 & 0\\
            0 & 1 & 2 & 1 & 0 & 0\\
            4 & -3 & 8 & 0 & 0 & 1\\
          \end{array}
          \right]\\
          \xrightarrow[r_3-4r_1]{}
          \left[
          \begin{array}{ccc;{2pt/2pt}ccc}  
            1 & 0 & 3 & 0 & 1 & 0\\
            0 &1  &2  &1  &0  &0 \\
            0 & -3 & -4 & 0 & -4 & 1\\
          \end{array}
          \right] \xrightarrow[r_3+3r_2]{}
          \left[
          \begin{array}{ccc;{2pt/2pt}ccc}  
            1 & 0 & 3 & 0 & 1 & 0\\
            0 & 1 & 2 & 1 & 0 & 0\\
            0 & 0 & 2 & 3 & -4 & 1\\
          \end{array}
          \right]       \\
          \xrightarrow[r_2-r_3]{}
          \left[
          \begin{array}{ccc;{2pt/2pt}ccc}  
            1 & 0 & 3 & 0 & 1 & 0\\
            0 & 1 & 0 & -2 & 4 & -1\\
            0 & 0 & 2 & 3 & -4 & 1\\
          \end{array}
          \right] \xrightarrow[r_3 \times  \frac{1}{2}]{}
          \left[
          \begin{array}{ccc;{2pt/2pt}ccc}  
            1 & 0 & 3 & 0 & 1 & 0\\
            0 & 1 & 0 & -2 & 4 & -1\\
            0 & 0 & 1 & \frac{3}{2} & -2 & \frac{1}{2}\\
          \end{array}
          \right]\\
          \xrightarrow[r_1-3r_2]{}
          \left[
          \begin{array}{ccc;{2pt/2pt}ccc}  
            1 & 0 & 0 & -\frac{9}{2} & 7 & -\frac{3}{2}\\
            0 & 1 & 0 & -2 & 4 & -1\\
            0 & 0 & 1 & \frac{3}{2} & -2 & \frac{1}{2}\\
          \end{array}
          \right] \implies A^{-1}= \begin{bmatrix}
          - \frac{9}{2} & 7 & - \frac{3}{2}\\
          -2 & 4 & -1\\
          \frac{3}{2} & -2 & \frac{1}{2}\\
          \end{bmatrix}\\
  .\end{align*}
  \item  Find a sequence of elementary matrices $E_1,E_2,\ldots, E_p$ such that $E_p E_{p-1}\ldots E_2E_1A =I_3$.\\
          \[
          E_1 = \begin{bmatrix}
          0 & 1 & 0\\
          1 & 0 & 0\\
          0 & 0 & 1\\
          \end{bmatrix} \text{ , } E_2= \begin{bmatrix}
          1 & 0 & 0\\
          0 &1  0& \\
          -4 & 0 & 1\\
          \end{bmatrix} \text{, } E_3 = \begin{bmatrix}
          1 & 0 & -3\\
          0 & 1 & 0\\
          0 &0  &1 \\
          \end{bmatrix}
          .\] 
  \end{enumerate}
  
}

\section{The Invertible Matrix Theorem}
Let $A $ be an $n \times n$ matrix. The following statements are equivalent:
\begin{enumerate} [label=(\alph*)]
  \item $A$ is an invertible matrix
  \item $A$ is row equivalent to $I_n$
  \item $A$ has $n$ pivot positions
  \item The system $A \vec{x} = \vec{b} $ has a unique solutions $\forall \vec{b} \in \mathbb{R}^{n}$
  \item The system $A \vec{x} = \vec{0} $ has the unique solution $\vec{x} =\vec{0} $ 
  \item There is an $n \times n$ matrix $A^{-1}$ such that $A^{-1}A =I_n$
  \end{enumerate}
  \section{Inverse matrices and the determinant}
          Suppose $A = \begin{bmatrix}
          a_{11} & a_{12} & a_{13}\\
          a_{21} & a_{22} & a_{23}\\
          a_{31} & a_{32} & a_{33}\\
          \end{bmatrix}$ is an invertible $3 \times  3 $ matrix.\\
          Row reducing $ A$ to reduced echelon form $\left( I_3 \right) $ we discover that the following conditions on $A $ must be satisfied for $A^{-1}$ to exist:
          \[
          a_{11} det \begin{bmatrix}
          a_{22} & a_{23}\\
           a_{32}& a_{33}\\
          \end{bmatrix} - a_{12} det \begin{bmatrix}
          a_{21} & a_{23}\\
          a_{31} & a_{33}\\
          \end{bmatrix} + a_{13} det \begin{bmatrix}
          a_{21} & a_{22}\\
          a_{31} & a_{32}\\
          \end{bmatrix}\neq 0
          .\] 
          The above is one definition of the determinant of a $3 \times  3$ matrix.\\
          \nt{\textbf{Notation:} Suppose $A$ is an $n \times n$ matrix. We use $A^{ij}$ to denote the $\left( n-1 \right) \times \left( n-1 \right) $ matrix obtained from $A$ by deleting the $i $th row and the $j$th column.}
          \nt{The determinant above could thus be written as
\[
a_{11} det A^{11}- a_{12} det A^{12} + a_{13} det A^{13}
.\] 
Informally the determinant of the $3 \times  3$ matrix $A$ is a signed sum of all possible products of $3$ entries from  $A$, chosen in a way that every row and every column is represented in the product only once.
          }
          
          
  















\end{document}
