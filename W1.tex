\documentclass{report}

\input{preamble}
\input{macros}
%--------------------------------------------------
% LIE ALGEBRAS
%--------------------------------------------------
\newcommand*{\kb}{\mathfrak{b}}  % Borel subalgebra
\newcommand*{\kg}{\mathfrak{g}}  % Lie algebra
\newcommand*{\kh}{\mathfrak{h}}  % Cartan subalgebra
\newcommand*{\kn}{\mathfrak{n}}  % Nilradical
\newcommand*{\ku}{\mathfrak{u}}  % Unipotent algebra
\newcommand*{\kz}{\mathfrak{z}}  % Center of algebra

%--------------------------------------------------
% HOMOLOGICAL ALGEBRA
%--------------------------------------------------
\DeclareMathOperator{\Ext}{Ext} % Ext functor
\DeclareMathOperator{\Tor}{Tor} % Tor functor

%--------------------------------------------------
% MATRIX & GROUP NOTATION
%--------------------------------------------------
\DeclareMathOperator{\GL}{GL} % General Linear Group
\DeclareMathOperator{\SL}{SL} % Special Linear Group
\newcommand*{\gl}{\operatorname{\mathfrak{gl}}} % General linear Lie algebra
\newcommand*{\sl}{\operatorname{\mathfrak{sl}}} % Special linear Lie algebra

%--------------------------------------------------
% NUMBER SETS
%--------------------------------------------------
\newcommand*{\RR}{\mathbb{R}}
\newcommand*{\NN}{\mathbb{N}}
\newcommand*{\ZZ}{\mathbb{Z}}
\newcommand*{\QQ}{\mathbb{Q}}
\newcommand*{\CC}{\mathbb{C}}
\newcommand*{\PP}{\mathbb{P}}
\newcommand*{\HH}{\mathbb{H}}
\newcommand*{\FF}{\mathbb{F}}
\newcommand*{\EE}{\mathbb{E}} % Expected Value

%--------------------------------------------------
% MATH SCRIPT, FRAKTUR, AND BOLD SYMBOLS
%--------------------------------------------------
\newcommand*{\mcA}{\mathcal{A}}
\newcommand*{\mcB}{\mathcal{B}}
\newcommand*{\mcC}{\mathcal{C}}
\newcommand*{\mcD}{\mathcal{D}}
\newcommand*{\mcE}{\mathcal{E}}
\newcommand*{\mcF}{\mathcal{F}}
\newcommand*{\mcG}{\mathcal{G}}
\newcommand*{\mcH}{\mathcal{H}}

\newcommand*{\mfA}{\mathfrak{A}}  \newcommand*{\mfB}{\mathfrak{B}}
\newcommand*{\mfC}{\mathfrak{C}}  \newcommand*{\mfD}{\mathfrak{D}}
\newcommand*{\mfE}{\mathfrak{E}}  \newcommand*{\mfF}{\mathfrak{F}}
\newcommand*{\mfG}{\mathfrak{G}}  \newcommand*{\mfH}{\mathfrak{H}}

\usepackage{bm} % Ensure bold math works correctly
\newcommand*{\bmA}{\bm{A}}
\newcommand*{\bmB}{\bm{B}}
\newcommand*{\bmC}{\bm{C}}
\newcommand*{\bmD}{\bm{D}}
\newcommand*{\bmE}{\bm{E}}
\newcommand*{\bmF}{\bm{F}}
\newcommand*{\bmG}{\bm{G}}
\newcommand*{\bmH}{\bm{H}}

%--------------------------------------------------
% FUNCTIONAL ANALYSIS & ALGEBRA
%--------------------------------------------------
\DeclareMathOperator{\Aut}{Aut} % Automorphism group
\DeclareMathOperator{\Inn}{Inn} % Inner automorphisms
\DeclareMathOperator{\Syl}{Syl} % Sylow subgroups
\DeclareMathOperator{\Gal}{Gal} % Galois group
\DeclareMathOperator{\sign}{sign} % Sign function

%\usepackage[tagged, highstructure]{accessibility}
\usepackage{tocloft}

\begin{document}
\title{Linear Algebra I}
\author{Lecture Notes Provided by Dr.~Miriam Logan.}
\date{}
\maketitle
\tableofcontents
\newpage
\\
\section{Vectors in $\mathbb{R}^{N}$}
\subsection{Properties of Vectors}
\subsubsection{Vectors, Points and Magnitudes}
$\mathbb{R}^2 = \{ \left( x,y \right) ~|~x,y \in \mathbb{R}  \} = \{ \langle x , y\rangle~| ~x,y \in \mathbb{R}  \}  $ which is equal to the set of all ordered types of real numbers.\\
\underline{Notation}: We will use $\left( x,y \right) $ to denote a point in $\mathbb{R}^2$ and $\langle x ,y \rangle $ to denote a vector in $\mathbb{R}^2$.\\
These two notations are interchangeable since a point determines a vector (a directed line segment joining the origin to that point).\\
\\
\[
\mathbb{R}^{3} = \{ \vec{v} = \langle  v_1,v_2,v_3 \rangle |_{}^{} ~v_1, v_2, v_3 \in \mathbb{R} \} = \text{ the set of all ordered triples of real numbers}
.\] 
\[
\mathbb{R}^{n}= \{ \vec{v} = \langle v_1,v_2,\ldots,v_n  \rangle ~|~ v_1,v_2,\ldots,v_n \in \mathbb{R}, n \in \mathbb{N} \} = \text{ the set of all ordered n-tuples of real numbers}
.\] Geometrically: A vector is a directed line segment. Algebraically: a vector is defined by its co-ordinates.\\
\\
\textit{In what follows the results (unless stated otherwise) hold for all vectors $\vec{v} $ in $\mathbb{R}^{n}$}\\
On this 'new' set of objects (vectors) we define the operations of addition, scalar multiplication, dot product (also known as scalar product) and cross product for vectors in $\mathbb{R}^{3}$ only.\\
\\
\textbf{ Length/ Magnitude} : $\|\vec{v} \|= \sqrt{v_1^2+v_2^2+\ldots+v_n^2}  \text{ is the distance from $\vec{0} $ to } \left( v_1,v_2,\ldots,v_n \right) $.\\
\\
\subsection{Addition of vectors in $\mathbb{R}^{n}$ :   } Algebraically- add respective co-ordinates.\\
\[
\vec{v} +\vec{w} = \langle v_1+w_1, v_2+w_2 +\ldots, v_n+w_n  \rangle 
.\] 
Geometrically: $\vec{v} +\vec{w}$ is the directed line segment which forms the long diagonal of the parallelogram formed by $\vec{v} $ and $\vec{w} $.
\\
XXX TEXIT
\\
\\
\textbf{Scalar Multiplication: } (multiplying a vector by a scalar)\\
Let $c \in \mathbb{R}$, $\vec{v} \in \mathbb{R}^{n}$.\\
XXX TEXIT\\
\\
Algebraically: $c \vec{v} $ is defined to be 
\[
c \vec{v} = \langle cv_1, cv_2, \ldots, cv_n \rangle 
.\] 
Geometrically: $c\vec{v} $ is a vector who length is $|c|$times the length of  $\vec{v} $ and whos direction is the same as $\vec{v} $ if $c >0$ and the opposite to $\vec{v} $ if $ c<0$.\\
\\
If  $c=0$ then $c \vec{v} =\vec{0} = \langle 0,0,\ldots,0  \rangle $ which is known as the \underline{null}  \underline{vector} .\\
\\
\textbf{Unit Vectors:}  A \underline{unit } \underline{vector}  is a vector of length 1. For a general vector $\vec{v} \in \mathbb{R}^{n} $ $\frac{1}{\|\vec{v} \|}\vec{v} $ is a unit vector pointing in the direction of $\vec{v} $.\\
\\
The \underline{Standard} \underline{basis} \underline{vectors}  are the unit vectors along with the co-ordinate axis e.g. In $\mathbb{R}^2$ $\langle 1,0  \rangle $ and $\langle 0,1  \rangle $.\\
\\
You may have encountered them in $\mathbb{R}^3$ under a different disguise in physics,
\[
\langle 1,0,0  \rangle = \hat{i}\\
.\] 
\[
\langle 0,1,0  \rangle = \hat{j}\\
.\] 
\[
\langle 0,0,1  \rangle = \hat{k}
.\] 
\\
\textbf{Scalar Product/ Dot Product} \\
\underline{Algebraic:}  Let $\vec{v} = \langle v_1,v_2,\ldots,v_n  \rangle $, $\vec{w} = \langle w_1,w_2,\ldots, w_n  \rangle $. Then $\vec{v} \cdot \vec{w} $ is defined to be 
\[
\vec{v}  \cdot  \vec{w} = v_1w_1_ v_2w_2 + \ldots + w_nw_n
.\] Note that the dot product of two vectors results in a scalar.\\
We will talk about the geometric interpretation of the dot product once we have another view on it.\\
\\
\textbf{Properties of the dot Product} \\
\textit{How do all of the operations interact with each other?} \\
\thm{}{
Let $\vec{v} $, $\vec{w}  $, $\vec{q} $ $\in \mathbb{R}^{n}$ and let $c \in \mathbb{R}$ \\
\\
\begin{enumerate}[label=(\roman*)]
  \item $\vec{v} \cdot  \vec{w} =  \vec{w} \cdot \vec{v} $ (the dot product is commutative)
  \item $\vec{v} \cdot \left( \vec{w} +\vec{q}  \right) = \vec{v} \cdot \vec{w} + \vec{v} \cdot \vec{q} $
  \item $\vec{v} \cdot \left( c \vec{w}  \right) = c \left( \vec{v} \cdot \vec{w}  \right) $ 
  \item $\vec{v} \cdot \vec{v} = \|\vec{v} \|^2$
  \item $\vec{v}\cdot \vec{w} = \|\vec{v} \|\|\vec{w} \|\cos\phi   $ where $\phi $ is an angle between $\vec{v} $ and $\vec{w} $, $0 \le \phi  \le \pi$

\end{enumerate}

}



\thm{}{
    \label{thm:sec-\thesection.\arabic{theorem}?}
    Let $\vec{u} $, $\vec{v} $, $\vec{w} \in \mathbb{R}^{3} $ and $c\in \mathbb{R}$ we have

    \begin{enumerate}[label=(\roman*)]
      \item $\vec{v} \times \vec{w} = -  \vec{w} \times \vec{v} $
      \item $\vec{v} \times  \left( \vec{w} +\vec{u}  \right) = \vec{v} \times \vec{w} + \vec{v} \times  \vec{u}                  $
      \item $\vec{v} \times \left( c \vec{w}  \right) = c \left( \vec{v} \times  \vec{w}  \right) $ 
      \item $\vec{v} \times \vec{v} = \vec{0} $, $\|\vec{v} \times \vec{v} \|=0$, $\implies \vec{v} \times  \vec{v} =\vec{0} $.
      \item $\vec{u} \cdot \left( \vec{v} \times \vec{w}  \right) = - \vec{v} \cdot \left( \vec{u} \times \vec{w}  \right) $ 
      \item $\vec{u} \times \left( \vec{v} \times \vec{w}  \right) = \left( \vec{u} **\vec{v}  \right) \vec{w}  $
      \item Jacobi Identity: $\left( \vec{u} \times  \vec{v}  \right)\times \vec{w} + \left( \vec{v} \times \vec{w}  \right) \times  \vec{u}  + \left( \vec{w} \times \vec{u} \right) \times \vec{v}  = \vec{0}   $
    \end{enumerate}
}
    \textbf{Proof} {

Let \[
\vec{v} = \langle v_1,v_2,v_3  \rangle \quad \vec{w} = \langle w_1,w_2,w_3  \rangle \] \[
\vec{v} \times \vec{w} = \langle v_2w_3-w_2v_3, v_3w_1-v_1w_3, v_1w_2-v_2w_1  \rangle  
%= \langle v_2\left( w_3+u_3 \right) -\left( w_2+u_2 \right) v_3, v_3 \left( w_1+u_1 \right) -v_1\left( w_3+u_3 \right)   \rangle 
.\] 
\[
=\langle -1 \left( w_2v_3-w_3v_2, -\left( w_3v_1-w_1v_3 \right) , -\left( w_1v_2-w_2v_1 \right)  \right)   \rangle 
.\] \[
= - \left( \vec{w} \times \vec{v}  \right) 
.\] 
    }

    ii
    \\
    \\
    $\vec{v} \times  \left( \vec{w} +\vec{u}  \right) = \langle v_1,v_2,v_3  \rangle \times  \left(  \langle w_1+u_1, w_2+u_2, w_3+u_3  \rangle  \right) $ 
    \[
    \langle v_2\left( w_3+u_3 \right) - \left( w_2+u_2 \right) v_3, v_3\left( w_1+u_1 \right) - v_1\left( w_3+u_3 \right) , v_1\left( w_2+u_2 \right) - v_2\left( w_1+u_1 \right)   \rangle 
    .\] \[
    = \langle v2w_3-w2v_3+v_2u_3-u2v_3, v3w_1-v_1w_3+v_3u_1-v_1u_3, v_1,w_2-v_2w_1+v_1u_2-v_2u_1  \rangle 
    .\] 
\[
= \langle v_2w_3-w_2v_3, v_3w_1-v_1w_3, v_1w_2-v_2w_1  \rangle + \langle v_2u_3-u_2v_3, v_3u_1-v_1u_3, v_1u_2-v_2u_1  \rangle 
.\] \[
= \vec{v} \times \vec{w} +\vec{ v} \times \vec{u} 
.\] 
\\
\\
iii\\
\\
\[
\vec{v} \times \left( c\vec{w} \right) = \langle v_1,v_2,v_3  \rangle \times  \langle cw_1,cw_2,cw_3  \rangle 
.\] \[
= \langle c \left( v_2w_3 -w_2v_3 \right) , c\left( v_2w_1-v_1w_3 \right) , c\left( v_1w_2-v_2w_1 \right)   \rangle 
.\] \[
=c \left( \vec{v} \times  \vec{w}  \right) 
.\] 
\\
\\
iv
\\
\\
\[
\vec{v} \times \vec{v} = \langle v_2v_3- v_2v_3, v_3v_1- v_1v_3, v_1v_2-v_2v_1  \rangle = \langle 0,0,0  \rangle 
.\] 

Note the property we take for granted in multiplying real numbers: $\left( ab \right)c = a \left( bc  \right)   $ (associativity) is not satisfied for cross products i.e
\[
\left( \vec{u} \times  \vec{v}  \right) \times  \vec{w}  \neq \vec{u} \times  \left( \vec{v}  \times  \vec{w}  \right) 
.\] 
\textbf{Proof} (i ) $\to$ (iv) can check algebraically\ geometrically and think about geometrically.
\\
\\
(v)  \[
\vec{u} \cdot  \left(  \vec{v} \times  \vec{w}  \right)= u_1 \left( v_2w_3-w_2v_3  \right) + u_2 \left( v_3w_1-v_1w_3 \right) + u_3 \left(  v1 w_2 -v_2w_1 \right)  
.\] \[
\vec{v} \cdot  \left( \vec{u} \times  \vec{w}  \right) = v_1 \left(  u_2w_3-w_2v_3 \right) + v_2\left( u_3w_1-u_1w_3 \right) + v_3 \left( u_1w_2-u_2w_1 \right) 
.\] 
Note: These quantities add together to give zero.\\

(vi) and (vii) The Jacobi identity if solved algebraically would be a mess of many terms $\implies$ we won't inflict that upon ourselves \\
\\
Instead we'll introduce one of the key ideas of linear algebra- that of linearity and therefore simplify our prof enormously. \\
\\
\textbf{Multi linearity:}  Let $\vec{v_1} ,\vec{v_2} ,\ldots, \vec{v_m} , \vec{w_1} ,\vec{w_2} ,\ldots, \vec{w_m} \in \mathbb{R} ^{n} $

\textbf{Definition:} A function $f$ of several vectors is said to be \underline{linear if all its arguments}or multilinear if 
\[
f \left( \vec{v_1} +\vec{w_1} , \vec{v_2} ,\ldots, \vec{v_m}  \right) = f \left( \vec{v_1} ,\vec{v_2} ,..,\vec{v_m}  \right) +f \left( \vec{w_1} ,\vec{v_2} ,\ldots,\vec{v_m}  \right) 
.\] 

\\
\chapter{Section 2}
\section{section 2}
\subsection{subsection 2}










\end{document}
