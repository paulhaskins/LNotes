\documentclass{report}

%%%%%%%%%%%%%%%%%%%%%%%%%%%%%%%%%
% PACKAGE IMPORTS
%%%%%%%%%%%%%%%%%%%%%%%%%%%%%%%%%


\usepackage[tmargin=2cm,rmargin=1in,lmargin=1in,margin=0.85in,bmargin=2cm,footskip=.2in]{geometry}
\usepackage{amsmath,amsfonts,amsthm,amssymb,mathtools}
\usepackage[varbb]{newpxmath}
\usepackage{xfrac}
\usepackage[makeroom]{cancel}
\usepackage{bookmark}
\usepackage{enumitem}
\usepackage{hyperref,theoremref}
\hypersetup{
	pdftitle={Assignment},
	colorlinks=true, linkcolor=doc!90,
	bookmarksnumbered=true,
	bookmarksopen=true
}
\usepackage[most,many,breakable]{tcolorbox}
\usepackage{xcolor}
\usepackage{varwidth}
\usepackage{varwidth}
\usepackage{tocloft}
\usepackage{etoolbox}
\usepackage{derivative} %many derivativess partials
%\usepackage{authblk}
\usepackage{nameref}
\usepackage{multicol,array}
\usepackage{tikz-cd}
\usepackage[ruled,vlined,linesnumbered]{algorithm2e}
\usepackage{comment} % enables the use of multi-line comments (\ifx \fi) 
\usepackage{import}
\usepackage{xifthen}
\usepackage{pdfpages}
\usepackage{transparent}
\usepackage{verbatim}

\newcommand\mycommfont[1]{\footnotesize\ttfamily\textcolor{blue}{#1}}
\SetCommentSty{mycommfont}
\newcommand{\incfig}[1]{%
    \def\svgwidth{\columnwidth}
    \import{./figures/}{#1.pdf_tex}
}
\usepackage[tagged, highstructure]{accessibility}
\usepackage{tikzsymbols}
\renewcommand\qedsymbol{$\Laughey$}


%\usepackage{import}
%\usepackage{xifthen}
%\usepackage{pdfpages}
%\usepackage{transparent}


%%%%%%%%%%%%%%%%%%%%%%%%%%%%%%
% SELF MADE COLORS
%%%%%%%%%%%%%%%%%%%%%%%%%%%%%%



\definecolor{myg}{RGB}{56, 140, 70}
\definecolor{myb}{RGB}{45, 111, 177}
\definecolor{myr}{RGB}{199, 68, 64}
\definecolor{mytheorembg}{HTML}{F2F2F9}
\definecolor{mytheoremfr}{HTML}{00007B}
\definecolor{mylenmabg}{HTML}{FFFAF8}
\definecolor{mylenmafr}{HTML}{983b0f}
\definecolor{mypropbg}{HTML}{f2fbfc}
\definecolor{mypropfr}{HTML}{191971}
\definecolor{myexamplebg}{HTML}{F2FBF8}
\definecolor{myexamplefr}{HTML}{88D6D1}
\definecolor{myexampleti}{HTML}{2A7F7F}
\definecolor{mydefinitbg}{HTML}{E5E5FF}
\definecolor{mydefinitfr}{HTML}{3F3FA3}
\definecolor{notesgreen}{RGB}{0,162,0}
\definecolor{myp}{RGB}{197, 92, 212}
\definecolor{mygr}{HTML}{2C3338}
\definecolor{myred}{RGB}{127,0,0}
\definecolor{myyellow}{RGB}{169,121,69}
\definecolor{myexercisebg}{HTML}{F2FBF8}
\definecolor{myexercisefg}{HTML}{88D6D1}


%%%%%%%%%%%%%%%%%%%%%%%%%%%%
% TCOLORBOX SETUPS
%%%%%%%%%%%%%%%%%%%%%%%%%%%%

\setlength{\parindent}{1cm}
%================================
% THEOREM BOX
%================================

\tcbuselibrary{theorems,skins,hooks}
\newtcbtheorem[number within=section]{Theorem}{Theorem}
{%
	enhanced,
	breakable,
	colback = mytheorembg,
	frame hidden,
	boxrule = 0sp,
	borderline west = {2pt}{0pt}{mytheoremfr},
	sharp corners,
	detach title,
	before upper = \tcbtitle\par\smallskip,
	coltitle = mytheoremfr,
	fonttitle = \bfseries\sffamily,
	description font = \mdseries,
	separator sign none,
	segmentation style={solid, mytheoremfr},
}
{th}

\tcbuselibrary{theorems,skins,hooks}
\newtcbtheorem[number within=chapter]{theorem}{Theorem}
{%
	enhanced,
	breakable,
	colback = mytheorembg,
	frame hidden,
	boxrule = 0sp,
	borderline west = {2pt}{0pt}{mytheoremfr},
	sharp corners,
	detach title,
	before upper = \tcbtitle\par\smallskip,
	coltitle = mytheoremfr,
	fonttitle = \bfseries\sffamily,
	description font = \mdseries,
	separator sign none,
	segmentation style={solid, mytheoremfr},
}
{th}


\tcbuselibrary{theorems,skins,hooks}
\newtcolorbox{Theoremcon}
{%
	enhanced
	,breakable
	,colback = mytheorembg
	,frame hidden
	,boxrule = 0sp
	,borderline west = {2pt}{0pt}{mytheoremfr}
	,sharp corners
	,description font = \mdseries
	,separator sign none
}

%================================
% Corollery
%================================
\tcbuselibrary{theorems,skins,hooks}
\newtcbtheorem[number within=section]{Corollary}{Corollary}
{%
	enhanced
	,breakable
	,colback = myp!10
	,frame hidden
	,boxrule = 0sp
	,borderline west = {2pt}{0pt}{myp!85!black}
	,sharp corners
	,detach title
	,before upper = \tcbtitle\par\smallskip
	,coltitle = myp!85!black
	,fonttitle = \bfseries\sffamily
	,description font = \mdseries
	,separator sign none
	,segmentation style={solid, myp!85!black}
}
{th}
\tcbuselibrary{theorems,skins,hooks}
\newtcbtheorem[number within=chapter]{corollary}{Corollary}
{%
	enhanced
	,breakable
	,colback = myp!10
	,frame hidden
	,boxrule = 0sp
	,borderline west = {2pt}{0pt}{myp!85!black}
	,sharp corners
	,detach title
	,before upper = \tcbtitle\par\smallskip
	,coltitle = myp!85!black
	,fonttitle = \bfseries\sffamily
	,description font = \mdseries
	,separator sign none
	,segmentation style={solid, myp!85!black}
}
{th}


%================================
% LENMA
%================================

\tcbuselibrary{theorems,skins,hooks}
\newtcbtheorem[number within=section]{Lenma}{Lenma}
{%
	enhanced,
	breakable,
	colback = mylenmabg,
	frame hidden,
	boxrule = 0sp,
	borderline west = {2pt}{0pt}{mylenmafr},
	sharp corners,
	detach title,
	before upper = \tcbtitle\par\smallskip,
	coltitle = mylenmafr,
	fonttitle = \bfseries\sffamily,
	description font = \mdseries,
	separator sign none,
	segmentation style={solid, mylenmafr},
}
{th}

\tcbuselibrary{theorems,skins,hooks}
\newtcbtheorem[number within=chapter]{lenma}{Lenma}
{%
	enhanced,
	breakable,
	colback = mylenmabg,
	frame hidden,
	boxrule = 0sp,
	borderline west = {2pt}{0pt}{mylenmafr},
	sharp corners,
	detach title,
	before upper = \tcbtitle\par\smallskip,
	coltitle = mylenmafr,
	fonttitle = \bfseries\sffamily,
	description font = \mdseries,
	separator sign none,
	segmentation style={solid, mylenmafr},
}
{th}


%================================
% PROPOSITION
%================================

\tcbuselibrary{theorems,skins,hooks}
\newtcbtheorem[number within=section]{Prop}{Proposition}
{%
	enhanced,
	breakable,
	colback = mypropbg,
	frame hidden,
	boxrule = 0sp,
	borderline west = {2pt}{0pt}{mypropfr},
	sharp corners,
	detach title,
	before upper = \tcbtitle\par\smallskip,
	coltitle = mypropfr,
	fonttitle = \bfseries\sffamily,
	description font = \mdseries,
	separator sign none,
	segmentation style={solid, mypropfr},
}
{th}

\tcbuselibrary{theorems,skins,hooks}
\newtcbtheorem[number within=chapter]{prop}{Proposition}
{%
	enhanced,
	breakable,
	colback = mypropbg,
	frame hidden,
	boxrule = 0sp,
	borderline west = {2pt}{0pt}{mypropfr},
	sharp corners,
	detach title,
	before upper = \tcbtitle\par\smallskip,
	coltitle = mypropfr,
	fonttitle = \bfseries\sffamily,
	description font = \mdseries,
	separator sign none,
	segmentation style={solid, mypropfr},
}
{th}


%================================
% CLAIM
%================================

\tcbuselibrary{theorems,skins,hooks}
\newtcbtheorem[number within=section]{claim}{Claim}
{%
	enhanced
	,breakable
	,colback = myg!10
	,frame hidden
	,boxrule = 0sp
	,borderline west = {2pt}{0pt}{myg}
	,sharp corners
	,detach title
	,before upper = \tcbtitle\par\smallskip
	,coltitle = myg!85!black
	,fonttitle = \bfseries\sffamily
	,description font = \mdseries
	,separator sign none
	,segmentation style={solid, myg!85!black}
}
{th}



%================================
% Exercise
%================================

\tcbuselibrary{theorems,skins,hooks}
\newtcbtheorem[number within=section]{Exercise}{Exercise}
{%
	enhanced,
	breakable,
	colback = myexercisebg,
	frame hidden,
	boxrule = 0sp,
	borderline west = {2pt}{0pt}{myexercisefg},
	sharp corners,
	detach title,
	before upper = \tcbtitle\par\smallskip,
	coltitle = myexercisefg,
	fonttitle = \bfseries\sffamily,
	description font = \mdseries,
	separator sign none,
	segmentation style={solid, myexercisefg},
}
{th}

\tcbuselibrary{theorems,skins,hooks}
\newtcbtheorem[number within=chapter]{exercise}{Exercise}
{%
	enhanced,
	breakable,
	colback = myexercisebg,
	frame hidden,
	boxrule = 0sp,
	borderline west = {2pt}{0pt}{myexercisefg},
	sharp corners,
	detach title,
	before upper = \tcbtitle\par\smallskip,
	coltitle = myexercisefg,
	fonttitle = \bfseries\sffamily,
	description font = \mdseries,
	separator sign none,
	segmentation style={solid, myexercisefg},
}
{th}

%================================
% EXAMPLE BOX
%================================

\newtcbtheorem[number within=section]{Example}{Example}
{%
	colback = myexamplebg
	,breakable
	,colframe = myexamplefr
	,coltitle = myexampleti
	,boxrule = 1pt
	,sharp corners
	,detach title
	,before upper=\tcbtitle\par\smallskip
	,fonttitle = \bfseries
	,description font = \mdseries
	,separator sign none
	,description delimiters parenthesis
}
{ex}

\newtcbtheorem[number within=chapter]{example}{Example}
{%
	colback = myexamplebg
	,breakable
	,colframe = myexamplefr
	,coltitle = myexampleti
	,boxrule = 1pt
	,sharp corners
	,detach title
	,before upper=\tcbtitle\par\smallskip
	,fonttitle = \bfseries
	,description font = \mdseries
	,separator sign none
	,description delimiters parenthesis
}
{ex}

%================================
% DEFINITION BOX
%================================

\newtcbtheorem[number within=section]{Definition}{Definition}{enhanced,
	before skip=2mm,after skip=2mm, colback=red!5,colframe=red!80!black,boxrule=0.5mm,
	attach boxed title to top left={xshift=1cm,yshift*=1mm-\tcboxedtitleheight}, varwidth boxed title*=-3cm,
	boxed title style={frame code={
					\path[fill=tcbcolback]
					([yshift=-1mm,xshift=-1mm]frame.north west)
					arc[start angle=0,end angle=180,radius=1mm]
					([yshift=-1mm,xshift=1mm]frame.north east)
					arc[start angle=180,end angle=0,radius=1mm];
					\path[left color=tcbcolback!60!black,right color=tcbcolback!60!black,
						middle color=tcbcolback!80!black]
					([xshift=-2mm]frame.north west) -- ([xshift=2mm]frame.north east)
					[rounded corners=1mm]-- ([xshift=1mm,yshift=-1mm]frame.north east)
					-- (frame.south east) -- (frame.south west)
					-- ([xshift=-1mm,yshift=-1mm]frame.north west)
					[sharp corners]-- cycle;
				},interior engine=empty,
		},
	fonttitle=\bfseries,
	title={#2},#1}{def}
\newtcbtheorem[number within=chapter]{definition}{Definition}{enhanced,
	before skip=2mm,after skip=2mm, colback=red!5,colframe=red!80!black,boxrule=0.5mm,
	attach boxed title to top left={xshift=1cm,yshift*=1mm-\tcboxedtitleheight}, varwidth boxed title*=-3cm,
	boxed title style={frame code={
					\path[fill=tcbcolback]
					([yshift=-1mm,xshift=-1mm]frame.north west)
					arc[start angle=0,end angle=180,radius=1mm]
					([yshift=-1mm,xshift=1mm]frame.north east)
					arc[start angle=180,end angle=0,radius=1mm];
					\path[left color=tcbcolback!60!black,right color=tcbcolback!60!black,
						middle color=tcbcolback!80!black]
					([xshift=-2mm]frame.north west) -- ([xshift=2mm]frame.north east)
					[rounded corners=1mm]-- ([xshift=1mm,yshift=-1mm]frame.north east)
					-- (frame.south east) -- (frame.south west)
					-- ([xshift=-1mm,yshift=-1mm]frame.north west)
					[sharp corners]-- cycle;
				},interior engine=empty,
		},
	fonttitle=\bfseries,
	title={#2},#1}{def}



%================================
% Solution BOX
%================================

\makeatletter
\newtcbtheorem{question}{Question}{enhanced,
	breakable,
	colback=white,
	colframe=myb!80!black,
	attach boxed title to top left={yshift*=-\tcboxedtitleheight},
	fonttitle=\bfseries,
	title={#2},
	boxed title size=title,
	boxed title style={%
			sharp corners,
			rounded corners=northwest,
			colback=tcbcolframe,
			boxrule=0pt,
		},
	underlay boxed title={%
			\path[fill=tcbcolframe] (title.south west)--(title.south east)
			to[out=0, in=180] ([xshift=5mm]title.east)--
			(title.center-|frame.east)
			[rounded corners=\kvtcb@arc] |-
			(frame.north) -| cycle;
		},
	#1
}{def}
\makeatother

%================================
% SOLUTION BOX
%================================

\makeatletter
\newtcolorbox{solution}{enhanced,
	breakable,
	colback=white,
	colframe=myg!80!black,
	attach boxed title to top left={yshift*=-\tcboxedtitleheight},
	title=Solution,
	boxed title size=title,
	boxed title style={%
			sharp corners,
			rounded corners=northwest,
			colback=tcbcolframe,
			boxrule=0pt,
		},
	underlay boxed title={%
			\path[fill=tcbcolframe] (title.south west)--(title.south east)
			to[out=0, in=180] ([xshift=5mm]title.east)--
			(title.center-|frame.east)
			[rounded corners=\kvtcb@arc] |-
			(frame.north) -| cycle;
		},
}
\makeatother

%================================
% Question BOX
%================================

\makeatletter
\newtcbtheorem{qstion}{Question}{enhanced,
	breakable,
	colback=white,
	colframe=mygr,
	attach boxed title to top left={yshift*=-\tcboxedtitleheight},
	fonttitle=\bfseries,
	title={#2},
	boxed title size=title,
	boxed title style={%
			sharp corners,
			rounded corners=northwest,
			colback=tcbcolframe,
			boxrule=0pt,
		},
	underlay boxed title={%
			\path[fill=tcbcolframe] (title.south west)--(title.south east)
			to[out=0, in=180] ([xshift=5mm]title.east)--
			(title.center-|frame.east)
			[rounded corners=\kvtcb@arc] |-
			(frame.north) -| cycle;
		},
	#1
}{def}
\makeatother

\newtcbtheorem[number within=chapter]{wconc}{Wrong Concept}{
	breakable,
	enhanced,
	colback=white,
	colframe=myr,
	arc=0pt,
	outer arc=0pt,
	fonttitle=\bfseries\sffamily\large,
	colbacktitle=myr,
	attach boxed title to top left={},
	boxed title style={
			enhanced,
			skin=enhancedfirst jigsaw,
			arc=3pt,
			bottom=0pt,
			interior style={fill=myr}
		},
	#1
}{def}



%================================
% NOTE BOX
%================================

\usetikzlibrary{arrows,calc,shadows.blur}
\tcbuselibrary{skins}
\newtcolorbox{note}[1][]{%
	enhanced jigsaw,
	colback=gray!20!white,%
	colframe=gray!80!black,
	size=small,
	boxrule=1pt,
	title=\textbf{Note:-},
	halign title=flush center,
	coltitle=black,
	breakable,
	drop shadow=black!50!white,
	attach boxed title to top left={xshift=1cm,yshift=-\tcboxedtitleheight/2,yshifttext=-\tcboxedtitleheight/2},
	minipage boxed title=1.5cm,
	boxed title style={%
			colback=white,
			size=fbox,
			boxrule=1pt,
			boxsep=2pt,
			underlay={%
					\coordinate (dotA) at ($(interior.west) + (-0.5pt,0)$);
					\coordinate (dotB) at ($(interior.east) + (0.5pt,0)$);
					\begin{scope}
						\clip (interior.north west) rectangle ([xshift=3ex]interior.east);
						\filldraw [white, blur shadow={shadow opacity=60, shadow yshift=-.75ex}, rounded corners=2pt] (interior.north west) rectangle (interior.south east);
					\end{scope}
					\begin{scope}[gray!80!black]
						\fill (dotA) circle (2pt);
						\fill (dotB) circle (2pt);
					\end{scope}
				},
		},
	#1,
}

%%%%%%%%%%%%%%%%%%%%%%%%%%%%%%
% SELF MADE COMMANDS
%%%%%%%%%%%%%%%%%%%%%%%%%%%%%%


\newcommand{\thm}[2]{\begin{Theorem}{#1}{}#2\end{Theorem}}
\newcommand{\cor}[2]{\begin{Corollary}{#1}{}#2\end{Corollary}}
\newcommand{\mlenma}[2]{\begin{Lenma}{#1}{}#2\end{Lenma}}
\newcommand{\mprop}[2]{\begin{Prop}{#1}{}#2\end{Prop}}
\newcommand{\clm}[3]{\begin{claim}{#1}{#2}#3\end{claim}}
\newcommand{\wc}[2]{\begin{wconc}{#1}{}\setlength{\parindent}{1cm}#2\end{wconc}}
\newcommand{\thmcon}[1]{\begin{Theoremcon}{#1}\end{Theoremcon}}
\newcommand{\ex}[2]{\begin{Example}{#1}{}#2\end{Example}}
\newcommand{\dfn}[2]{\begin{Definition}[colbacktitle=red!75!black]{#1}{}#2\end{Definition}}
\newcommand{\dfnc}[2]{\begin{definition}[colbacktitle=red!75!black]{#1}{}#2\end{definition}}
\newcommand{\qs}[2]{\begin{question}{#1}{}#2\end{question}}
\newcommand{\pf}[2]{\begin{myproof}[#1]#2\end{myproof}}
\newcommand{\nt}[1]{\begin{note}#1\end{note}}

\newcommand*\circled[1]{\tikz[baseline=(char.base)]{
		\node[shape=circle,draw,inner sep=1pt] (char) {#1};}}
\newcommand\getcurrentref[1]{%
	\ifnumequal{\value{#1}}{0}
	{??}
	{\the\value{#1}}%
}
\newcommand{\getCurrentSectionNumber}{\getcurrentref{section}}
\newenvironment{myproof}[1][\proofname]{%
	\proof[\bfseries #1: ]%
}{\endproof}

\newcommand{\mclm}[2]{\begin{myclaim}[#1]#2\end{myclaim}}
\newenvironment{myclaim}[1][\claimname]{\proof[\bfseries #1: ]}{}

\newcounter{mylabelcounter}

\makeatletter
\newcommand{\setword}[2]{%
	\phantomsection
	#1\def\@currentlabel{\unexpanded{#1}}\label{#2}%
}
\makeatother




\tikzset{
	symbol/.style={
			draw=none,
			every to/.append style={
					edge node={node [sloped, allow upside down, auto=false]{$#1$}}}
		}
}


% deliminators
\DeclarePairedDelimiter{\abs}{\lvert}{\rvert}
\DeclarePairedDelimiter{\norm}{\lVert}{\rVert}

\DeclarePairedDelimiter{\ceil}{\lceil}{\rceil}
\DeclarePairedDelimiter{\floor}{\lfloor}{\rfloor}
\DeclarePairedDelimiter{\round}{\lfloor}{\rceil}

\newsavebox\diffdbox
\newcommand{\slantedromand}{{\mathpalette\makesl{d}}}
\newcommand{\makesl}[2]{%
\begingroup
\sbox{\diffdbox}{$\mathsurround=0pt#1\mathrm{#2}$}%
\pdfsave
\pdfsetmatrix{1 0 0.2 1}%
\rlap{\usebox{\diffdbox}}%
\pdfrestore
\hskip\wd\diffdbox
\endgroup
}
\newcommand{\dd}[1][]{\ensuremath{\mathop{}\!\ifstrempty{#1}{%
\slantedromand\@ifnextchar^{\hspace{0.2ex}}{\hspace{0.1ex}}}%
{\slantedromand\hspace{0.2ex}^{#1}}}}
\ProvideDocumentCommand\dv{o m g}{%
  \ensuremath{%
    \IfValueTF{#3}{%
      \IfNoValueTF{#1}{%
        \frac{\dd #2}{\dd #3}%
      }{%
        \frac{\dd^{#1} #2}{\dd #3^{#1}}%
      }%
    }{%
      \IfNoValueTF{#1}{%
        \frac{\dd}{\dd #2}%
      }{%
        \frac{\dd^{#1}}{\dd #2^{#1}}%
      }%
    }%
  }%
}
\providecommand*{\pdv}[3][]{\frac{\partial^{#1}#2}{\partial#3^{#1}}}
%  - others
\DeclareMathOperator{\Lap}{\mathcal{L}}
\DeclareMathOperator{\Var}{Var} % varience
\DeclareMathOperator{\Cov}{Cov} % covarience
\DeclareMathOperator{\E}{E} % expected

% Since the amsthm package isn't loaded

% I prefer the slanted \leq
\let\oldleq\leq % save them in case they're every wanted
\let\oldgeq\geq
\renewcommand{\leq}{\leqslant}
\renewcommand{\geq}{\geqslant}

% % redefine matrix env to allow for alignment, use r as default
% \renewcommand*\env@matrix[1][r]{\hskip -\arraycolsep
%     \let\@ifnextchar\new@ifnextchar
%     \array{*\c@MaxMatrixCols #1}}


%\usepackage{framed}
%\usepackage{titletoc}
%\usepackage{etoolbox}
%\usepackage{lmodern}


%\patchcmd{\tableofcontents}{\contentsname}{\sffamily\contentsname}{}{}

%\renewenvironment{leftbar}
%{\def\FrameCommand{\hspace{6em}%
%		{\color{myyellow}\vrule width 2pt depth 6pt}\hspace{1em}}%
%	\MakeFramed{\parshape 1 0cm \dimexpr\textwidth-6em\relax\FrameRestore}\vskip2pt%
%}
%{\endMakeFramed}

%\titlecontents{chapter}
%[0em]{\vspace*{2\baselineskip}}
%{\parbox{4.5em}{%
%		\hfill\Huge\sffamily\bfseries\color{myred}\thecontentspage}%
%	\vspace*{-2.3\baselineskip}\leftbar\textsc{\small\chaptername~\thecontentslabel}\\\sffamily}
%{}{\endleftbar}
%\titlecontents{section}
%[8.4em]
%{\sffamily\contentslabel{3em}}{}{}
%{\hspace{0.5em}\nobreak\itshape\color{myred}\contentspage}
%\titlecontents{subsection}
%[8.4em]
%{\sffamily\contentslabel{3em}}{}{}  
%{\hspace{0.5em}\nobreak\itshape\color{myred}\contentspage}



%%%%%%%%%%%%%%%%%%%%%%%%%%%%%%%%%%%%%%%%%%%
% TABLE OF CONTENTS
%%%%%%%%%%%%%%%%%%%%%%%%%%%%%%%%%%%%%%%%%%%

\usepackage{tikz}
\definecolor{doc}{RGB}{0,60,110}
\usepackage{titletoc}
\contentsmargin{0cm}
\titlecontents{chapter}[3.7pc]
{\addvspace{30pt}%
	\begin{tikzpicture}[remember picture, overlay]%
		\draw[fill=doc!60,draw=doc!60] (-7,-.1) rectangle (-0.9,.5);%
		\pgftext[left,x=-3.5cm,y=0.2cm]{\color{white}\Large\sc\bfseries Chapter\ \thecontentslabel};%
	\end{tikzpicture}\color{doc!60}\large\sc\bfseries}%
{}
{}
{\;\titlerule\;\large\sc\bfseries Page \thecontentspage
	\begin{tikzpicture}[remember picture, overlay]
		\draw[fill=doc!60,draw=doc!60] (2pt,0) rectangle (4,0.1pt);
	\end{tikzpicture}}%
\titlecontents{section}[3.7pc]
{\addvspace{2pt}}
{\contentslabel[\thecontentslabel]{2pc}}
{}
{\hfill\small \thecontentspage}
[]
\titlecontents*{subsection}[3.7pc]
{\addvspace{-1pt}\small}
{}
{}
{\ --- \small\thecontentspage}
[ \textbullet\ ][]

\makeatletter
\renewcommand{\tableofcontents}{%
	\chapter*{%
	  \vspace*{-20\p@}%
	  \begin{tikzpicture}[remember picture, overlay]%
		  \pgftext[right,x=15cm,y=0.2cm]{\color{doc!60}\Huge\sc\bfseries \contentsname};%
		  \draw[fill=doc!60,draw=doc!60] (13,-.75) rectangle (20,1);%
		  \clip (13,-.75) rectangle (20,1);
		  \pgftext[right,x=15cm,y=0.2cm]{\color{white}\Huge\sc\bfseries \contentsname};%
	  \end{tikzpicture}}%
	\@starttoc{toc}}
\makeatother


%From M275 "Topology" at SJSU
\newcommand{\id}{\mathrm{id}}
\newcommand{\taking}[1]{\xrightarrow{#1}}
\newcommand{\inv}{^{-1}}

%From M170 "Introduction to Graph Theory" at SJSU
\DeclareMathOperator{\diam}{diam}
\DeclareMathOperator{\ord}{ord}
\newcommand{\defeq}{\overset{\mathrm{def}}{=}}

%From the USAMO .tex files
\newcommand{\ts}{\textsuperscript}
\newcommand{\dg}{^\circ}
\newcommand{\ii}{\item}

% % From Math 55 and Math 145 at Harvard
% \newenvironment{subproof}[1][Proof]{%
% \begin{proof}[#1] \renewcommand{\qedsymbol}{$\blacksquare$}}%
% {\end{proof}}

\newcommand{\liff}{\leftrightarrow}
\newcommand{\lthen}{\rightarrow}
\newcommand{\opname}{\operatorname}
\newcommand{\surjto}{\twoheadrightarrow}
\newcommand{\injto}{\hookrightarrow}
\newcommand{\On}{\mathrm{On}} % ordinals
\DeclareMathOperator{\img}{im} % Image
\DeclareMathOperator{\Img}{Im} % Image
\DeclareMathOperator{\coker}{coker} % Cokernel
\DeclareMathOperator{\Coker}{Coker} % Cokernel
\DeclareMathOperator{\Ker}{Ker} % Kernel
\DeclareMathOperator{\rank}{rank}
\DeclareMathOperator{\Spec}{Spec} % spectrum
\DeclareMathOperator{\Tr}{Tr} % trace
\DeclareMathOperator{\pr}{pr} % projection
\DeclareMathOperator{\ext}{ext} % extension
\DeclareMathOperator{\pred}{pred} % predecessor
\DeclareMathOperator{\dom}{dom} % domain
\DeclareMathOperator{\ran}{ran} % range
\DeclareMathOperator{\Hom}{Hom} % homomorphism
\DeclareMathOperator{\Mor}{Mor} % morphisms
\DeclareMathOperator{\End}{End} % endomorphism

\newcommand{\eps}{\epsilon}
\newcommand{\veps}{\varepsilon}
\newcommand{\ol}{\overline}
\newcommand{\ul}{\underline}
\newcommand{\wt}{\widetilde}
\newcommand{\wh}{\widehat}
\newcommand{\vocab}[1]{\textbf{\color{blue} #1}}
\providecommand{\half}{\frac{1}{2}}
\newcommand{\dang}{\measuredangle} %% Directed angle
\newcommand{\ray}[1]{\overrightarrow{#1}}
\newcommand{\seg}[1]{\overline{#1}}
\newcommand{\arc}[1]{\wideparen{#1}}
\DeclareMathOperator{\cis}{cis}
\DeclareMathOperator*{\lcm}{lcm}
\DeclareMathOperator*{\argmin}{arg min}
\DeclareMathOperator*{\argmax}{arg max}
\newcommand{\cycsum}{\sum_{\mathrm{cyc}}}
\newcommand{\symsum}{\sum_{\mathrm{sym}}}
\newcommand{\cycprod}{\prod_{\mathrm{cyc}}}
\newcommand{\symprod}{\prod_{\mathrm{sym}}}
\newcommand{\Qed}{\begin{flushright}\qed\end{flushright}}
\newcommand{\parinn}{\setlength{\parindent}{1cm}}
\newcommand{\parinf}{\setlength{\parindent}{0cm}}
% \newcommand{\norm}{\|\cdot\|}
\newcommand{\inorm}{\norm_{\infty}}
\newcommand{\opensets}{\{V_{\alpha}\}_{\alpha\in I}}
\newcommand{\oset}{V_{\alpha}}
\newcommand{\opset}[1]{V_{\alpha_{#1}}}
\newcommand{\lub}{\text{lub}}
\newcommand{\del}[2]{\frac{\partial #1}{\partial #2}}
\newcommand{\Del}[3]{\frac{\partial^{#1} #2}{\partial^{#1} #3}}
\newcommand{\deld}[2]{\dfrac{\partial #1}{\partial #2}}
\newcommand{\Deld}[3]{\dfrac{\partial^{#1} #2}{\partial^{#1} #3}}
\newcommand{\lm}{\lambda}
\newcommand{\uin}{\mathbin{\rotatebox[origin=c]{90}{$\in$}}}
\newcommand{\usubset}{\mathbin{\rotatebox[origin=c]{90}{$\subset$}}}
\newcommand{\lt}{\left}
\newcommand{\rt}{\right}
\newcommand{\bs}[1]{\boldsymbol{#1}}
\newcommand{\exs}{\exists}
\newcommand{\st}{\strut}
\newcommand{\dps}[1]{\displaystyle{#1}}

\newcommand{\sol}{\setlength{\parindent}{0cm}\textbf{\textit{Solution:}}\setlength{\parindent}{1cm} }
\newcommand{\solve}[1]{\setlength{\parindent}{0cm}\textbf{\textit{Solution: }}\setlength{\parindent}{1cm}#1 \Qed}

%--------------------------------------------------
% LIE ALGEBRAS
%--------------------------------------------------
\newcommand*{\kb}{\mathfrak{b}}  % Borel subalgebra
\newcommand*{\kg}{\mathfrak{g}}  % Lie algebra
\newcommand*{\kh}{\mathfrak{h}}  % Cartan subalgebra
\newcommand*{\kn}{\mathfrak{n}}  % Nilradical
\newcommand*{\ku}{\mathfrak{u}}  % Unipotent algebra
\newcommand*{\kz}{\mathfrak{z}}  % Center of algebra

%--------------------------------------------------
% HOMOLOGICAL ALGEBRA
%--------------------------------------------------
\DeclareMathOperator{\Ext}{Ext} % Ext functor
\DeclareMathOperator{\Tor}{Tor} % Tor functor

%--------------------------------------------------
% MATRIX & GROUP NOTATION
%--------------------------------------------------
\DeclareMathOperator{\GL}{GL} % General Linear Group
\DeclareMathOperator{\SL}{SL} % Special Linear Group
\newcommand*{\gl}{\operatorname{\mathfrak{gl}}} % General linear Lie algebra
\newcommand*{\sl}{\operatorname{\mathfrak{sl}}} % Special linear Lie algebra

%--------------------------------------------------
% NUMBER SETS
%--------------------------------------------------
\newcommand*{\RR}{\mathbb{R}}
\newcommand*{\NN}{\mathbb{N}}
\newcommand*{\ZZ}{\mathbb{Z}}
\newcommand*{\QQ}{\mathbb{Q}}
\newcommand*{\CC}{\mathbb{C}}
\newcommand*{\PP}{\mathbb{P}}
\newcommand*{\HH}{\mathbb{H}}
\newcommand*{\FF}{\mathbb{F}}
\newcommand*{\EE}{\mathbb{E}} % Expected Value

%--------------------------------------------------
% MATH SCRIPT, FRAKTUR, AND BOLD SYMBOLS
%--------------------------------------------------
\newcommand*{\mcA}{\mathcal{A}}
\newcommand*{\mcB}{\mathcal{B}}
\newcommand*{\mcC}{\mathcal{C}}
\newcommand*{\mcD}{\mathcal{D}}
\newcommand*{\mcE}{\mathcal{E}}
\newcommand*{\mcF}{\mathcal{F}}
\newcommand*{\mcG}{\mathcal{G}}
\newcommand*{\mcH}{\mathcal{H}}

\newcommand*{\mfA}{\mathfrak{A}}  \newcommand*{\mfB}{\mathfrak{B}}
\newcommand*{\mfC}{\mathfrak{C}}  \newcommand*{\mfD}{\mathfrak{D}}
\newcommand*{\mfE}{\mathfrak{E}}  \newcommand*{\mfF}{\mathfrak{F}}
\newcommand*{\mfG}{\mathfrak{G}}  \newcommand*{\mfH}{\mathfrak{H}}

\usepackage{bm} % Ensure bold math works correctly
\newcommand*{\bmA}{\bm{A}}
\newcommand*{\bmB}{\bm{B}}
\newcommand*{\bmC}{\bm{C}}
\newcommand*{\bmD}{\bm{D}}
\newcommand*{\bmE}{\bm{E}}
\newcommand*{\bmF}{\bm{F}}
\newcommand*{\bmG}{\bm{G}}
\newcommand*{\bmH}{\bm{H}}

%--------------------------------------------------
% FUNCTIONAL ANALYSIS & ALGEBRA
%--------------------------------------------------
\DeclareMathOperator{\Aut}{Aut} % Automorphism group
\DeclareMathOperator{\Inn}{Inn} % Inner automorphisms
\DeclareMathOperator{\Syl}{Syl} % Sylow subgroups
\DeclareMathOperator{\Gal}{Gal} % Galois group
\DeclareMathOperator{\sign}{sign} % Sign function


%\usepackage[tagged, highstructure]{accessibility}
\usepackage{tocloft}
\usepackage{arydshln}
\usetikzlibrary{arrows.meta, decorations.pathreplacing}
\usepackage{tikz-cd}



\begin{document}
\title{Linear Algebra I}
\author{Lecture Notes Provided by Dr.~Miriam Logan.}
\date{}
\maketitle
\tableofcontents
\newpage  

\section{Linear Operators}
         \dfn{Linear Operator :}{
         A \underline{linear operator} $ \phi $, is a transformation mapping from a vector space $ V $ to itself, i.e.  $ \phi : V \to V$
         }
         \mlem{}{
           Let $ \phi : V \to V$ be a linear operator and $ \mathcal{B}$ and $ \mathcal{F} $ be two bases for $ V $. Then 
           \[
             \left[ \phi  \right] _{ \mathcal{F} \mathcal{F}} = P _{ \mathcal{B} \to \mathcal{F}} \left[ \phi \right] _{ \mathcal{B} \mathcal{B}} P _{ \mathcal{F} \to \mathcal{B}}
           .\] 
           \[
              \left[ \phi  \right] _{ \mathcal{F} \mathcal{F}} = \left( P _{ \mathcal{F} \to \mathcal{B}} \right) ^{ -1} \left[ \phi \right] _{ \mathcal{B} \mathcal{B}} P _{ \mathcal{F} \to \mathcal{B}}
           .\] 
         }

         \dfn{ Similar/ Conjugate Matrices :}{
                                               Two $n \times n$  matrices are said to be \underline{similar} or \underline{conjugate} if there exists an invertible matrix $ M $ such that
                                               \[
                                               B = M ^{-1} A M
                                               .\] 
                                               equivalently, 
                                               \[
                                               A = MB M^{-1}
                                               .\] 
         }

\rmk{ :}{
  The previous lemma implies that $ \left[ \phi  \right] _{ \mathcal{F} \mathcal{F}}$  and $ \left[  \phi  \right] _{ \mathcal{B} \mathcal{B}}$ are similar. In fact, two $n \times n$  matrices are similar if and only if they represent the same linear operator on $ \mathbb{R} ^{ n}$ with respect to different bases.
}

\dfn{Diagonal Matrices :}{
  An $n \times n$  matrix $ D = \left[ d_{ij} \right] $ with $ 1 \leq i \leq n$ and $ 1 \leq j \leq n$ is said to be a \underline{diagonal matrix} if $ d_{ij} = 0 $ for all $ i \neq j $. If 
  \[
  D = \begin{bmatrix}
      \lambda_1 & 0 & 0 & \dots  & 0 \\
      0 & \lambda_2 & 0 & \dots  & 0 \\
      \vdots & \vdots & \vdots & \ddots & \vdots \\
      0 & 0 & 0 & \dots  & \lambda_n\end{bmatrix}     \qquad  \text{ i.e. } d_{ii} = \lambda_i \quad d_{ij} = 0, \text{ for } i \neq j
  .\]         
  \[
  \text{ then } D ^{k} = \begin{bmatrix}
      \lambda_1 ^{k} & 0 & 0 & \dots  & 0 \\
      0 & \lambda_2 ^{k} & 0 & \dots  & 0 \\
      \vdots & \vdots & \vdots & \ddots & \vdots \\
    0 & 0 & 0 & \dots  & \lambda _n ^{k}\end{bmatrix} \qquad  \text{ (can prove by induction) }
  .\] 
}
 \dfn{Diagonalizable Matrices :}{
 An $n \times n$  matrix A is said to be \underline{diagonalizable} if $ A$ is simlar to a diagonal matrix $ D$, i.e. there exists an invertible matrix $ B$ such that $ D = B ^{-1} A B$
 }
 \nt{
 If $ A$ is diagonalizable, then computing powers of $ A$ is straightforward, since $ A = B D B ^{-1}$
 \[
 A^{k} = \left( BD B^{-1} \right) ^{k} =\left( BD B^{-1} \right) \left( BD B^{-1} \right) \ldots \left( BD B ^{-1} \right)  = B D^{k} B^{-1} 
 .\] 
 }
 Next we will investigate the circumstances under which a matrix is diagonalizable.\\
 \dfn{Eigenvectors and Eigenvalues :}{
  Suppose $ V$ is a vector space  and $ \phi : V \to V$ is a linear operator.
  \begin{enumerate}[label=(\roman*)]
    \item  A non-zero  vector  $ v \in V$ is said to be an \underline{eigenvector} of $ \phi $ if $ \phi  \left( \vec{ v}  \right) = \lambda \vec{ v} $ for some $ \lambda \in \mathbb{R}$
    \item   The scalar $ \lambda $ is called an \underline{eigenvalue} of $ \phi $ 
    \end{enumerate}
 }
 
 \ex{}{
 $ \phi : \mathbb{R} ^2 \to \mathbb{R} ^2$
 \[
 \phi  \begin{bmatrix}
 x\\
 y\\
 \end{bmatrix}
 = \begin{bmatrix}
 2 & 0\\
 0 & 3\\
 \end{bmatrix} \begin{bmatrix}
 x\\
 y\\
 \end{bmatrix}
 = \begin{bmatrix}
 2x\\
 3y\\
 \end{bmatrix}
 .\] 
 Clearly, $ \phi  \begin{bmatrix}
 1\\
 0\\
 \end{bmatrix}
 = 2 \begin{bmatrix}
 1\\
 0\\
 \end{bmatrix}
 $ and $ \phi \begin{bmatrix}
 0\\
 1\\
 \end{bmatrix}        = 3 \begin{bmatrix}
 0\\
 1\\
 \end{bmatrix}
 $ i.e. $ \begin{bmatrix}
 1\\
 0\\
 \end{bmatrix}
 $ is an eigenvector with eigenvalue $ \lambda =2$ and $  \begin{bmatrix}
 0\\
 1\\
 \end{bmatrix}
 $ is an eigenvector with eigenvalue $ \lambda = 3$. \\
 The effect that $ \phi $  has on $ \mathbb{R} ^2$ is easy to understand, it scales the $ x$-coordinate horizontally away from the $ y$-axis by a factor of $ 2$ and the $ y$-coordinate vertically away from the $ x$-axis by a factor of $ 3$.
 }
 
 \ex{}{
 $ \phi : \mathbb{R} ^2 \to \mathbb{R} ^2$, $ \phi $ a reflection across the x-axis given by 
 \[
 \phi \begin{bmatrix}
 x\\
 y\\
 \end{bmatrix}
 = \begin{bmatrix}
 1 & 0\\
 0 & -1\\
 \end{bmatrix} \begin{bmatrix}
 x\\
 y\\
 \end{bmatrix}
 = \begin{bmatrix}
 x\\
 -y\\
 \end{bmatrix}
 .\] 
 $ \begin{bmatrix}
 1\\
 0\\
 \end{bmatrix}
 $ is a eigenvector with eigenvalue $ \lambda = 1$.\\
 $ \begin{bmatrix}
 0\\
 1\\
 \end{bmatrix}
  $ is an eigenvector with eigenvalue $ \lambda = -1$.\\
 }
   The transformation that a linear operator $ \phi: V \to V$ imposes on a vector space $ V$  is easier to understand if $ \phi $ has eigenvectors.         \\

  \\
  \thm{Diagonalizable Matrices :}
  {
    Suppose $ A$ is an $n \times n$  matrix and $ B = \left[ \vec{ v_1} , \vec{ v_2} ,\ldots , \vec{ v_n}  \right]$ where $ \vec{ v_i } \in \mathbb{R} ^{n} \forall  i $.\\
    Then $ B^{-1}A B$ is a diagonal matrix $ \iff$  $ \left\{ \vec{ v_1} ,\ldots \vec{ v_n}  \right\} $ is a basis of $ \mathbb{R} ^{n}$ consisting of eigenvectors.
  \\
  Note that $ B \vec{ e_i } = \vec{ v_i} \forall  i$ ( $ \vec{ e_i} $ is the standard basis).\\
  \\
  \[
  B^{-1} A B = \begin{bmatrix}
      \lambda_1 & 0 & 0 & \dots  & 0 \\
      0 & \lambda_2 & 0 & \dots  & 0 \\
      \vdots & \vdots & \vdots & \ddots & \vdots \\
      0 & 0 & 0 & \dots  & \lambda_n\end{bmatrix}
  .\] 
  \[
   \iff \left( B^{-1} A B \right) \left( \vec{ e_i}  \right) = \lambda _i \vec{ e_i} 
  .\] 
  \[
  \iff A B \vec{ e_i} = B \lambda_i \vec{ e_i}
  .\] 
  \[
  \text{ i.e. } AB \left( \vec{ e_i}  \right) = \lambda_i \left( B \vec{ e_i}  \right) 
  .\] 
  \[
  A \left( B \vec{ e_i}  \right) = \lambda_i \left( B \vec{ e_i}  \right)
  .\] 
  \[
  \iff A \left( \vec{ v_i}  \right) = \lambda_i \left( \vec{ v_i}  \right)
  .\] 
  i.e. $ \vec{ v_i} $ is an eigenvector of $ A \forall  i$.\\
  Moreover, $ B$ is invertible if and only if $ \left\{ \vec{ v_1} ,\ldots \vec{ v_n}  \right\} $ is a basis of $ \mathbb{R} ^{n}$.
}
\\
  \\
  \section{Finding Eigenvalues and Eigenvectors}
    
  In order to find the eigenvectors we start by finding the eigenvalues: For an $n \times n$ marix $ A$, this involves finding $ \lambda$ such that $ A \vec{ v} = \lambda \vec{ v} $ for non-zero vectors $ \vec{ v}$.\\
 \[
  A \vec{ v} - \lambda \vec{ v} = \vec{ 0} 
 .\] 
 \[
 A \vec{ v} - \lambda I_n \vec{ v} = \vec{ 0}
 .\] 
  \[
\boxed{(A - \lambda I)\,\vec v = \mathbf 0}
\]
We wish to find non-zero solutions to this equation. Which exist if and only if the matrix $ A - \lambda I$ is  not invertible, i.e. iff $ \text{ det } \left( A - \lambda I \right) =0$ \\
$ \implies \lambda$ is an eigenvalue of $ A \iff$ $ \lambda$ is a root of $ \text{ det } \left( A - \lambda I \right)$
$
   \dfn{Characteristic Polynomial :}{
   The Polynomial $ \text{ det } \left( A - \lambda I  \right) $         is called the \underline{ Characteristic Polynomial of $ A$}  and is denoted by $ \chi _{A} \left( \lambda \right) $          .\\
   The roots of $ \chi _A \left( \lambda \right) $ are the eigenvalues of $ A$.\\
   Finding eigenvectors that have $ \lambda$ as an eigenvalue mearly involves solving $ A \vec{ v} = \lambda \vec{ v} $
   }


   \ex{}{
   \textit{Find the eigenvalues and corresponding eigenvectors of $ A = \begin{bmatrix}
   1 & 3\\
   4 & 2\\
   \end{bmatrix}$. \\
Is $ A$ diagonalizable?}\\
\\
\textbf{Solution:}\\
To find eigenvalues we solve $ \text{ det } \left( A - \lambda I \right) $
\[
\text{ i.e. }  \text{ det }  \left( \begin{bmatrix}
1 & 3\\
4 & 2\\
\end{bmatrix} - \begin{bmatrix}
\lambda & 0\\
0 & \lambda\\
\end{bmatrix} \right) = 0
.\] 
\[
\text{ det } \begin{bmatrix}
1-\lambda  & 3 \\
4 & 2- \lambda\\
\end{bmatrix} =0
.\] 
\begin{align*}
	\left( 1- \lambda \right) \left(  2- \lambda \right) \\
	&= 2 - 3 \lambda -10 =0\\
	&= \lambda^2 - 3\lambda -10 =0\\
	&= \left( \lambda-5 \right) \left( \lambda+2 \right)            =0\\
	& \lambda=5 \qquad \lambda=-2
.\end{align*}\\
\\
\underline{Eigenvectors:}\\
 \[
 \lambda = 5 : \left( A - 4 I \right) \vec{ v} = \vec{ 0} 
 .\] 
 \[
 \left[
 \begin{array}{cc;{2pt/2pt}c}  
 -4 & 3 & 0\\
 4 & -3 & 0\\
 \end{array}
 \right]    \to  \left[
 \begin{array}{cc;{2pt/2pt}c}  
 -4 & 3 & 0\\
 0 & 0 & 0\\
 \end{array}
 \right]
 .\] 
 \[
 -4x +3y =0 \qquad  \implies y = \frac{4}{3} x
 .\] 
  \\
  \\
  Eigenvectors with eigenvalue $ \lambda =5 $ \\
  \[
	  \left\{ x \begin{bmatrix}
	  3\\
	  4\\
	  \end{bmatrix}
	  \mid x \in \mathbb{R}, x \neq 0 \right\}
  .\] 
  \[
  \lambda=-2 \qquad  \left( A +2 I \right) \vec{ v}    = \vec{ 0} 
  .\] 
  \[
	  \left[
	  \begin{array}{cc;{2pt/2pt}c}  
	  3 & 3 & 0\\
	  4 & 4 & 0\\
	  \end{array}
	  \right] \to \left[
	  \begin{array}{cc;{2pt/2pt}c}  
	  1 & 1 & 0\\
	  0 & 0 & 0\\
	  \end{array}
	  \right]
  .\] 
  \[
  x+y =0 \qquad \implies y = -x 
  .\] 
     \\
     Eigenvectors with eigenvalue $ \lambda = -2$ :\\
     \[
	     \left\{ x \begin{bmatrix}
	     1\\
	     -1\\
	     \end{bmatrix}
	     \mid x \in \mathbb{R} , x \neq 0 \right\}
     .\] 
     $ A$ is diagonalizable since there exists a basis of $  \mathbb{R} ^2$ consisting of eigenvectors of $ A$.  \\
     \\
     Note: $ \mathcal{B} = \left\{ \begin{bmatrix}
     3\\
     4\\
     \end{bmatrix}
     , \begin{bmatrix}
     1\\
     -1\\
     \end{bmatrix}
      \right\}$, forms a basis for $   \mathbb{R} ^2$.\\
      The linear operatro $ \phi : \mathbb{R} ^2 \to \mathbb{R} ^2 $ defined by $ \phi  \left( \vec{ v}  \right) = A \vec{ v} $ where $ A = \begin{bmatrix}
      1 & 3\\
      4 & 2\\
      \end{bmatrix}$.\\
      \[
	      \left[ \phi  \right] _{ \mathcal{E} , \mathcal{E}} = \begin{bmatrix}
	      1 & 3\\
	      4 & 2\\
	      \end{bmatrix}
      .\] 
      \[
	      \left[ \phi  \right] _{ \mathcal{B}, \mathcal{B}}  = \begin{bmatrix}
	      5 & 0\\
	      0 & -2\\
	      \end{bmatrix} \text{ since }  \phi  \left( \vec{ b_1}  \right) = 5 \vec{ b_1} \quad \phi \left( \vec{ b_2}  \right) = -2 \vec{ b_2}  
      .\] 
      \[
      \text{ and, } P_{ \mathcal{B}\to \mathcal{E} } = \begin{bmatrix}
      3 & 1\\
      4 & -1\\
      \end{bmatrix}
      .\] 
      \[
      P _{ \mathcal{B} \to \mathcal{E}} =  \frac{1}{-7}\begin{bmatrix}
      -1 & -1\\
      -4 & 3\\
      \end{bmatrix}
      .\] 
 We can check that
 \[
 \begin{bmatrix}
 5 & 0\\
 0 & -2\\
 \end{bmatrix} = \frac{ 1  }{ -7 }\begin{bmatrix}
 -1 & -1\\
 -4 & 3\\
 \end{bmatrix} 
 \begin{bmatrix}
 1 & 3\\
 -4 & 2\\
 \end{bmatrix}
 \begin{bmatrix}
 3 & 1\\
 4 & -1\\
 \end{bmatrix}
 .\] 
 \[
	 \left[ \phi  \right]  _{ \mathcal{B} , \mathcal{B}} = P _{ \mathcal{E} \to\mathcal{B}}  \left[ \phi  \right] _{   \mathcal{E} , \mathcal{E}} P _{ \mathcal{B} \to \mathcal{E}}
 .\] 

   }
   
   \ex{}{
   Let $ A = \begin{bmatrix}
   1 & 1\\
   0 & 1\\
   \end{bmatrix}$. Find the eigenvalues and the corresponding eigenvectors for $ A$ and determine whether $ A$ is diagonalizable.\\
   \\
   \textbf{Solution:}\\
       Eigenvalues:      $ \text{ det } \left( A -  \lambda I \right) =0$\\
        \[
        \left( 1- \lambda \right) \left( 1-\lambda \right) 
        .\] 
	\[
	\lambda =1  \text{ is the only eigenvalue}
	.\] 
	Eigenvectors:\\
	\[
	\left[
	\begin{array}{cc;{2pt/2pt}c}  
	0 & 1 & 0\\
	0 & 0 & 0\\
	\end{array}
	\right] \qquad  x \text{ free, }  y=0
	.\] 
	\[
		U_1 = \left\{ \begin{bmatrix}
		x\\
		0\\
		\end{bmatrix}
		 \mid x \in \mathbb{R} \right\} 
	.\] 
	A is not diagonalizable since there does not exist a basis for $ \mathbb{R}^2 $ consisting of eigenvectors of $ A$.
   }
     \thm{}
     {
       Suppose $ \phi  : V \to V$ is a linear operator defined on the vector space $ V$. \\
All vectors $ \vec{ v} \in V$ that satisfy $ \phi  \left( \vec{ v}  \right) = \lambda \vec{ v} $ for some fixed $  \lambda \in \mathbb{R}$  along with $ \vec{ 0}_v $, form a subspace of $ V$. We'll use $ U_{\lambda}$ to denote this subspace.\\
     }
     \pf{Proof:}{
      \underline{addition:}\\
      Suppose $ \vec{ u_1} , \vec{ u_2}  \in U_{\lambda}$  i.e.  $ \phi  \left(  \vec{ u_1}  \right) = \lambda \vec{ u_1} $ and $ \phi  \left(  \vec{ u_2}  \right)= \lambda \vec{ u_2}    $.\\
      Then,
      \[
      \phi  \left(  \vec{ u_1} + \vec{ u_2}  \right) = \phi  \left(  \vec{ u_1}  \right) + \phi \left( \vec{ u_2}  \right) = \lamba \vec{ u_1} + \lambda \vec{  u_2} = \lambda \left(  \vec{ u_1} + \vec{ u_2}  \right)   \implies \vec{ u_1} + \vec{ u_2}  \in U_{ \lambda}
      .\] 
      \\
      \underline{Scalar multiplication:}\\
      Suppose $ \vec{ u} \in U _{ \lambda}, \alpha \in \mathbb{R}$ \\
      \[
      \phi \left( \alpha \vec{ u}  \right) = \alpha \phi  \left( \vec{ u}  \right) = \alpha \left(  \lambda \vec{ u}  \right) =\lambda \left( \alpha \vec{ u}  \right)         \qquad   \implies   \alpha \vec{ u} \in U _{ \lambda}
      .\] 
      by definition $ \vec{ 0}_V \in  U _{ \lambda} \qquad \implies U_{ \lambda} $  forms a subspace of $ V$.
     }
      We can prove this another way\\
      \pf{Proof:}{
       If $ I:V \to V$ is the identity mapping, $ I \left(  \vec{ v}  \right) = \vec{  v}  $, $ \forall  \vec{ v} \in V$ the $ U_{ \lambda} = ker \left( \phi  -I \right) $ which is a subspace of $ V$ as long as $ \phi - I $ is a linear transformation. $ \phi  -I : V \to V$
 \[
 \left( \phi - I \right) \left( \vec{ v}  \right)  = \phi  \left( \vec{ v}  \right) - I \left(  \vec{ v}  \right) 
 .\]      
 This method can be shown to preserve addition and scalar multiplication.
      }
      \thm{Upper/Lower Triangular Matrices}
      {
        Suppose $ A$ is an $n \times n$  upper (or lower) triangluar matrix, say 
	\[
	A = \begin{bmatrix}
		a_{11} & a_{12} & a_{13} & \dots  & a_{1n} \\
	0 & a_{22} & a_{23} & \dots  & a_{2n} \\
	\vdots & \vdots & \vdots & \ddots & \vdots \\
	0 & 0 & 0 & \dots  & a_{ nn}\end{bmatrix}
	.\]  then the eigenvalues of $ A$ would lie along the diagonal of $ A$.
      }
      \pf{Proof:}{
       \[
       \chi _{A} \left( \lambda \right) = \text{ det } \begin{bmatrix}
       		a_{11}-\lambda & a_{12} & a_{13} & \dots  & a_{1n} \
       	0 & a_{22}-\lambda & a_{23} & \dots  & a_{2n} \
       	\vdots & \vdots & \vdots & \ddots & \vdots \
       	0 & 0 & 0 & \dots  & a_{ nn} - \lambda\end{bmatrix}      
       .\] 
       \[
       \chi _{A} \left( \lambda \right) = \left( a_{11} - \lambda\right)  \left(  a_{22}-\lambda \right) \ldots \left( a_{ n n} -\lambda \right) 
       .\] 
       $ \implies $  solutions to $ \chi   _{A} \left( \lambda \right) =0 $ are $ \lambda = a_{11}, a_{22}, \ldots , a_{ n n}  $
      }

      \ex{}{
      Let $ A = \begin{bmatrix}
      \cos \theta &  -\sin \theta \\
      \sin \theta  &  \cos \theta \\
      \end{bmatrix}$
      \\
      We know that $ A$ rotatrs $ \mathbb{R} $ counterclockwise about the origin by an angle of $  \theta$.\\
      \textit{ Is $ A$ diagonalizable?}\\
      \\
      \textbf{Solution:}\\
      \\
      \[
      \text{ det } \begin{bmatrix}
      \cos \theta - \lambda  & - \sin \theta \\
      \sin\theta   & \cos \theta -\lambda\\
      \end{bmatrix}                        =0
      .\] 
      \[
      \left( \cos \theta- \lambda   \right) ^2 + \sin ^2 \theta  =0
      .\] 
 \[
 \cos ^2 \theta - 2 \lambda \cos \theta + \lambda ^2+ \sin ^2 \theta =0 \qquad \implies \lambda ^2 - 2 \lambda \cos \theta +1 =0 
 .\] 
 \[
 \lambda = \frac{  2 \cos \theta \pm \sqrt{ \left( 2 \cos \theta  \right) ^2 -4}   }{ 2 }
 .\] 
 \[
 \lambda = \frac{ 2 \cos \theta \pm \sqrt{ 4 \left(  \cos ^2 \theta -1 \right) }   }{ 2 }
 .\] 
 \[
 \lambda = \frac{  2 \cos \theta  \pm \sqrt{-4 \sin \theta }  }{  2}
 .\] 
 if $ \sin \theta \neq 0$ then $  \lambda \notin  \mathbb{R} $\\
 $ \implies$ no eigenvalues if $  \theta \in \mathbb{R} \backslash \left\{ k \pi \backslash k \in \mathbb{Z} \right\} $ \\
 If $ \theta = 2k \pi $, $  k \in \mathbb{Z}$ then $ A = \begin{bmatrix}
 1 & 0\\
 0 & 1\\
 \end{bmatrix}$ which is $ I_2$ the $2 \times 2$  identity, which has $ \lambda =1 $ as an eigenvalue and corresponding eigenvectors: $ \begin{bmatrix}
 1\\
 0\\
 \end{bmatrix}
 , \begin{bmatrix}
 0\\
 1\\
 \end{bmatrix}
 $ \\
 \\
 If $ \theta = \left( 2k +1  \right)  \pi$, $ k \in \mathbb{Z}$ then $ A = \begin{bmatrix}
 -1 & 0\\
 0 & -1\\
 \end{bmatrix}$
 Which has $ \lambda = -1 $ as an eigenvalue and corresponding eigenvectors $ \begin{bmatrix}
 1\\
 0\\
 \end{bmatrix}, \begin{bmatrix}
 0\\
 1\\
 \end{bmatrix}
 $ \\
 \\
 In fact the last two cases $ A$ is clearly diagonalizable since it is a diagonal matrix.
      }
      \mlem{}{
      If $ A$ and $ B$ are similar matrices then $ \chi _{A } \left( \lambda \right) = \chi _{ B} \left( \lambda \right) $}


      \pf{Proof:}{
       $ A,B$ are similar, hence there exists an invertible matrix $ M$ such that $ B = M ^{ -1} A M$ \\
       \begin{align*}
       	\chi _{B} \left( \lambda \right) = \text{ det } \left( B - \lambda I \right) = \text{ det } \left( M^{-1} A M - \lambda I \right) \\
	&= \text{ det } \left( M ^{-1}A M - M ^{-1} \left( \lambda I \right) M \right)\\
	&= \text{ det } \left( M^{-1} \right)  \text{ det }  \left( A - \lambda I \right) \text{ det } M\\
	&=  \left( \frac{1}{ \text{ det } M} \right) \left( \chi _A \left( \lambda \right)  \right) \text{ det } M\\
	\implies \chi _{B} \left( \lambda \right) = \chi _{A} \left( \lambda \right)
       .\end{align*}
      }
       \begin{corollary}[]
	       Let $ \phi : V \to V$ be a linear operator and let $ \mathcal{B}$ be a basis of $ V$ and let $ \left[ \phi  \right] _{ \mathcal{B} , \mathcal{B} } $  be a matrix of $ \phi $ with respect to $ \mathcal{B}$. We define $  \chi _{ \phi } \left(  \lambda \right) = \chi _{ \left[ \phi  \right]_{ \mathcal{B} \mathcal{B}}} \left( \lambda \right) $   and $  \chi _{ \phi } \left( \lambda \right) $ is independent of the choice of basis $ \mathcal{B}$.
       \end{corollary}
                   \pf{Proof:}{
			   Let $ \mathcal{F}$ be another basis for $ V$ and let $ \left[ \phi  \right] _{ \mathcal{F} \mathcal{F}}$ be the matrix of $ \phi $ with respect to $ \mathcal{F}$.\\
			   We know that $ \left[ \phi  \right] _{ \mathcal{F} \mathcal{F}}$ is similar to $ \left[ \phi  \right] _{ \mathcal{F} \mathcal{F}}$,
			   \[
				   \left[ \phi  \right]_{ \mathcal{B} \mathcal{B}} = P _{ \mathcal{F} \to \mathcal{B}} \left[ \phi  \right]_{ \mathcal{F} \mathcal{F}} P _{ \mathcal{B} \to \mathcal{F}}
			   .\] 
			   Hence, $ \chi _{ \left[ \phi  \right]_{ \mathcal{B} \mathcal{B}}} \left( \lambda \right) = \chi _{ \left[ \phi  \right] _{ \mathcal{F} \mathcal{F}}} \left( \lambda \right) $                  \\
			   This implies that the characteristic polynomial of a linear operator $ \phi : V \to V$ is independent of the choice of basis for $ V$.
                   }
                   
      
      
	   
      

     
   
         
         
         
















\end{document}          
