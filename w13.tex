 \documentclass{report}

\input{preamble}
\input{macros}
%--------------------------------------------------
% LIE ALGEBRAS
%--------------------------------------------------
\newcommand*{\kb}{\mathfrak{b}}  % Borel subalgebra
\newcommand*{\kg}{\mathfrak{g}}  % Lie algebra
\newcommand*{\kh}{\mathfrak{h}}  % Cartan subalgebra
\newcommand*{\kn}{\mathfrak{n}}  % Nilradical
\newcommand*{\ku}{\mathfrak{u}}  % Unipotent algebra
\newcommand*{\kz}{\mathfrak{z}}  % Center of algebra

%--------------------------------------------------
% HOMOLOGICAL ALGEBRA
%--------------------------------------------------
\DeclareMathOperator{\Ext}{Ext} % Ext functor
\DeclareMathOperator{\Tor}{Tor} % Tor functor

%--------------------------------------------------
% MATRIX & GROUP NOTATION
%--------------------------------------------------
\DeclareMathOperator{\GL}{GL} % General Linear Group
\DeclareMathOperator{\SL}{SL} % Special Linear Group
\newcommand*{\gl}{\operatorname{\mathfrak{gl}}} % General linear Lie algebra
\newcommand*{\sl}{\operatorname{\mathfrak{sl}}} % Special linear Lie algebra

%--------------------------------------------------
% NUMBER SETS
%--------------------------------------------------
\newcommand*{\RR}{\mathbb{R}}
\newcommand*{\NN}{\mathbb{N}}
\newcommand*{\ZZ}{\mathbb{Z}}
\newcommand*{\QQ}{\mathbb{Q}}
\newcommand*{\CC}{\mathbb{C}}
\newcommand*{\PP}{\mathbb{P}}
\newcommand*{\HH}{\mathbb{H}}
\newcommand*{\FF}{\mathbb{F}}
\newcommand*{\EE}{\mathbb{E}} % Expected Value

%--------------------------------------------------
% MATH SCRIPT, FRAKTUR, AND BOLD SYMBOLS
%--------------------------------------------------
\newcommand*{\mcA}{\mathcal{A}}
\newcommand*{\mcB}{\mathcal{B}}
\newcommand*{\mcC}{\mathcal{C}}
\newcommand*{\mcD}{\mathcal{D}}
\newcommand*{\mcE}{\mathcal{E}}
\newcommand*{\mcF}{\mathcal{F}}
\newcommand*{\mcG}{\mathcal{G}}
\newcommand*{\mcH}{\mathcal{H}}

\newcommand*{\mfA}{\mathfrak{A}}  \newcommand*{\mfB}{\mathfrak{B}}
\newcommand*{\mfC}{\mathfrak{C}}  \newcommand*{\mfD}{\mathfrak{D}}
\newcommand*{\mfE}{\mathfrak{E}}  \newcommand*{\mfF}{\mathfrak{F}}
\newcommand*{\mfG}{\mathfrak{G}}  \newcommand*{\mfH}{\mathfrak{H}}

\usepackage{bm} % Ensure bold math works correctly
\newcommand*{\bmA}{\bm{A}}
\newcommand*{\bmB}{\bm{B}}
\newcommand*{\bmC}{\bm{C}}
\newcommand*{\bmD}{\bm{D}}
\newcommand*{\bmE}{\bm{E}}
\newcommand*{\bmF}{\bm{F}}
\newcommand*{\bmG}{\bm{G}}
\newcommand*{\bmH}{\bm{H}}

%--------------------------------------------------
% FUNCTIONAL ANALYSIS & ALGEBRA
%--------------------------------------------------
\DeclareMathOperator{\Aut}{Aut} % Automorphism group
\DeclareMathOperator{\Inn}{Inn} % Inner automorphisms
\DeclareMathOperator{\Syl}{Syl} % Sylow subgroups
\DeclareMathOperator{\Gal}{Gal} % Galois group
\DeclareMathOperator{\sign}{sign} % Sign function


%\usepackage[tagged, highstructure]{accessibility}
\usepackage{tocloft}
\usepackage{arydshln}
\usetikzlibrary{arrows.meta, decorations.pathreplacing}
\usepackage{tikz-cd}



\begin{document}
\title{Linear Algebra I}
\author{Lecture Notes Provided by Dr.~Miriam Logan.}
\date{}
\maketitle
\tableofcontents
\newpage  

\section{Linear Operators}
         \dfn{Linear Operator :}{
         A \underline{linear operator} $ \phi $, is a transformation mapping from a vector space $ V $ to itself, i.e.  $ \phi : V \to V$
         }
         \mlem{}{
           Let $ \phi : V \to V$ be a linear operator and $ \mathcal{B}$ and $ \mathcal{F} $ be two bases for $ V $. Then 
           \[
             \left[ \phi  \right] _{ \mathcal{F} \mathcal{F}} = P _{ \mathcal{B} \to \mathcal{F}} \left[ \phi \right] _{ \mathcal{B} \mathcal{B}} P _{ \mathcal{F} \to \mathcal{B}}
           .\] 
           \[
              \left[ \phi  \right] _{ \mathcal{F} \mathcal{F}} = \left( P _{ \mathcal{F} \to \mathcal{B}} \right) ^{ -1} \left[ \phi \right] _{ \mathcal{B} \mathcal{B}} P _{ \mathcal{F} \to \mathcal{B}}
           .\] 
         }

         \dfn{ Similar/ Conjugate Matrices :}{
                                               Two $n \times n$  matrices are said to be \underline{similar} or \underline{conjugate} if there exists an invertible matrix $ M $ such that
                                               \[
                                               B = M ^{-1} A M
                                               .\] 
                                               equivalently, 
                                               \[
                                               A = MB M^{-1}
                                               .\] 
         }

\rmk{ :}{
  The previous lemma implies that $ \left[ \phi  \right] _{ \mathcal{F} \mathcal{F}}$  and $ \left[  \phi  \right] _{ \mathcal{B} \mathcal{B}}$ are similar. In fact, two $n \times n$  matrices are similar if and only if they represent the same linear operator on $ \mathbb{R} ^{ n}$ with respect to different bases.
}

\dfn{Diagonal Matrices :}{
  An $n \times n$  matrix $ D = \left[ d_{ij} \right] $ with $ 1 \leq i \leq n$ and $ 1 \leq j \leq n$ is said to be a \underline{diagonal matrix} if $ d_{ij} = 0 $ for all $ i \neq j $. If 
  \[
  D = \begin{bmatrix}
      \lambda_1 & 0 & 0 & \dots  & 0 \\
      0 & \lambda_2 & 0 & \dots  & 0 \\
      \vdots & \vdots & \vdots & \ddots & \vdots \\
      0 & 0 & 0 & \dots  & \lambda_n\end{bmatrix}     \qquad  \text{ i.e. } d_{ii} = \lambda_i \quad d_{ij} = 0, \text{ for } i \neq j
  .\]         
  \[
  \text{ then } D ^{k} = \begin{bmatrix}
      \lambda_1 ^{k} & 0 & 0 & \dots  & 0 \\
      0 & \lambda_2 ^{k} & 0 & \dots  & 0 \\
      \vdots & \vdots & \vdots & \ddots & \vdots \\
    0 & 0 & 0 & \dots  & \lambda _n ^{k}\end{bmatrix} \qquad  \text{ (can prove by induction) }
  .\] 
}
 \dfn{Diagonalizable Matrices :}{
 An $n \times n$  matrix A is said to be \underline{diagonalizable} if $ A$ is similar to a diagonal matrix $ D$, i.e. there exists an invertible matrix $ B$ such that $ D = B ^{-1} A B$
 }
 \nt{
 If $ A$ is diagonalizable, then computing powers of $ A$ is straightforward, since $ A = B D B ^{-1}$
 \[
 A^{k} = \left( BD B^{-1} \right) ^{k} =\left( BD B^{-1} \right) \left( BD B^{-1} \right) \ldots \left( BD B ^{-1} \right)  = B D^{k} B^{-1} 
 .\] 
 }
 Next we will investigate the circumstances under which a matrix is diagonalizable.\\
 \dfn{Eigenvectors and Eigenvalues :}{
  Suppose $ V$ is a vector space  and $ \phi : V \to V$ is a linear operator.
  \begin{enumerate}[label=(\roman*)]
    \item  A non-zero  vector  $ v \in V$ is said to be an \underline{eigenvector} of $ \phi $ if $ \phi  \left( \vec{ v}  \right) = \lambda \vec{ v} $ for some $ \lambda \in \mathbb{R}$
    \item   The scalar $ \lambda $ is called an \underline{eigenvalue} of $ \phi $ 
    \end{enumerate}
 }
 
 \ex{}{
 $ \phi : \mathbb{R} ^2 \to \mathbb{R} ^2$
 \[
 \phi  \begin{bmatrix}
 x\\
 y\\
 \end{bmatrix}
 = \begin{bmatrix}
 2 & 0\\
 0 & 3\\
 \end{bmatrix} \begin{bmatrix}
 x\\
 y\\
 \end{bmatrix}
 = \begin{bmatrix}
 2x\\
 3y\\
 \end{bmatrix}
 .\] 
 Clearly, $ \phi  \begin{bmatrix}
 1\\
 0\\
 \end{bmatrix}
 = 2 \begin{bmatrix}
 1\\
 0\\
 \end{bmatrix}
 $ and $ \phi \begin{bmatrix}
 0\\
 1\\
 \end{bmatrix}        = 3 \begin{bmatrix}
 0\\
 1\\
 \end{bmatrix}
 $ i.e. $ \begin{bmatrix}
 1\\
 0\\
 \end{bmatrix}
 $ is an eigenvector with eigenvalue $ \lambda =2$ and $  \begin{bmatrix}
 0\\
 1\\
 \end{bmatrix}
 $ is an eigenvector with eigenvalue $ \lambda = 3$. \\
 The effect that $ \phi $  has on $ \mathbb{R} ^2$ is easy to understand, it scales the $ x$-coordinate horizontally away from the $ y$-axis by a factor of $ 2$ and the $ y$-coordinate vertically away from the $ x$-axis by a factor of $ 3$.
 }
 
 \ex{}{
 $ \phi : \mathbb{R} ^2 \to \mathbb{R} ^2$, $ \phi $ a reflection across the x-axis given by 
 \[
 \phi \begin{bmatrix}
 x\\
 y\\
 \end{bmatrix}
 = \begin{bmatrix}
 1 & 0\\
 0 & -1\\
 \end{bmatrix} \begin{bmatrix}
 x\\
 y\\
 \end{bmatrix}
 = \begin{bmatrix}
 x\\
 -y\\
 \end{bmatrix}
 .\] 
 $ \begin{bmatrix}
 1\\
 0\\
 \end{bmatrix}
 $ is a eigenvector with eigenvalue $ \lambda = 1$.\\
 $ \begin{bmatrix}
 0\\
 1\\
 \end{bmatrix}
  $ is an eigenvector with eigenvalue $ \lambda = -1$.\\
 }
   The transformation that a linear operator $ \phi: V \to V$ imposes on a vector space $ V$  is easier to understand if $ \phi $ has eigenvectors.         \\

  \\
  \thm{Diagonalizable Matrices :}
  {
    Suppose $ A$ is an $n \times n$  matrix and $ B = \left[ \vec{ v_1} , \vec{ v_2} ,\ldots , \vec{ v_n}  \right]$ where $ \vec{ v_i } \in \mathbb{R} ^{n} \forall  i $.\\
    Then $ B^{-1}A B$ is a diagonal matrix $ \iff$  $ \left\{ \vec{ v_1} ,\ldots \vec{ v_n}  \right\} $ is a basis of $ \mathbb{R} ^{n}$ consisting of eigenvectors.
  \\
  Note that $ B \vec{ e_i } = \vec{ v_i} \forall  i$ ( $ \vec{ e_i} $ is the standard basis).\\
  \\
  \[
  B^{-1} A B = \begin{bmatrix}
      \lambda_1 & 0 & 0 & \dots  & 0 \\
      0 & \lambda_2 & 0 & \dots  & 0 \\
      \vdots & \vdots & \vdots & \ddots & \vdots \\
      0 & 0 & 0 & \dots  & \lambda_n\end{bmatrix}
  .\] 
  \[
   \iff \left( B^{-1} A B \right) \left( \vec{ e_i}  \right) = \lambda _i \vec{ e_i} 
  .\] 
  \[
  \iff A B \vec{ e_i} = B \lambda_i \vec{ e_i}
  .\] 
  \[
  \text{ i.e. } AB \left( \vec{ e_i}  \right) = \lambda_i \left( B \vec{ e_i}  \right) 
  .\] 
  \[
  A \left( B \vec{ e_i}  \right) = \lambda_i \left( B \vec{ e_i}  \right)
  .\] 
  \[
  \iff A \left( \vec{ v_i}  \right) = \lambda_i \left( \vec{ v_i}  \right)
  .\] 
  i.e. $ \vec{ v_i} $ is an eigenvector of $ A \forall  i$.\\
  Moreover, $ B$ is invertible if and only if $ \left\{ \vec{ v_1} ,\ldots \vec{ v_n}  \right\} $ is a basis of $ \mathbb{R} ^{n}$.
}
\\
  \\
  \section{Finding Eigenvalues and Eigenvectors}
    
  In order to find the eigenvectors we start by finding the eigenvalues: For an $n \times n$ matrix $ A$, this involves finding $ \lambda$ such that $ A \vec{ v} = \lambda \vec{ v} $ for non-zero vectors $ \vec{ v}$.\\
 \[
  A \vec{ v} - \lambda \vec{ v} = \vec{ 0} 
 .\] 
 \[
 A \vec{ v} - \lambda I_n \vec{ v} = \vec{ 0}
 .\] 
  \[
\boxed{(A - \lambda I)\,\vec v = \mathbf 0}
\]
We wish to find non-zero solutions to this equation. Which exist if and only if the matrix $ A - \lambda I$ is  not invertible, i.e. iff $ \text{ det } \left( A - \lambda I \right) =0$ \\
$ \implies \lambda$ is an eigenvalue of $ A \iff$ $ \lambda$ is a root of $ \text{ det } \left( A - \lambda I \right)$
$
   \dfn{Characteristic Polynomial :}{
   The Polynomial $ \text{ det } \left( A - \lambda I  \right) $         is called the \underline{ Characteristic Polynomial of $ A$}  and is denoted by $ \chi _{A} \left( \lambda \right) $          .\\
   The roots of $ \chi _A \left( \lambda \right) $ are the eigenvalues of $ A$.\\
   Finding eigenvectors that have $ \lambda$ as an eigenvalue mearly involves solving $ A \vec{ v} = \lambda \vec{ v} $
   }


   \ex{}{
   \textit{Find the eigenvalues and corresponding eigenvectors of $ A = \begin{bmatrix}
   1 & 3\\
   4 & 2\\
   \end{bmatrix}$. \\
Is $ A$ diagonalizable?}\\
\\
\textbf{Solution:}\\
To find eigenvalues we solve $ \text{ det } \left( A - \lambda I \right) $
\[
\text{ i.e. }  \text{ det }  \left( \begin{bmatrix}
1 & 3\\
4 & 2\\
\end{bmatrix} - \begin{bmatrix}
\lambda & 0\\
0 & \lambda\\
\end{bmatrix} \right) = 0
.\] 
\[
\text{ det } \begin{bmatrix}
1-\lambda  & 3 \\
4 & 2- \lambda\\
\end{bmatrix} =0
.\] 
\begin{align*}
 &\left( 1- \lambda \right) \left(  2- \lambda \right) -12\\
	&= 2 - 3 \lambda+ \lambda ^2 -12 =0\\
	&= \lambda^2 - 3\lambda -10 =0\\
	&= \left( \lambda-5 \right) \left( \lambda+2 \right)            =0\\
	& \lambda=5 \qquad \lambda=-2
.\end{align*}\\
\\
\underline{Eigenvectors:}\\
 \[
 \lambda = 5 : \left( A - 4 I \right) \vec{ v} = \vec{ 0} 
 .\] 
 \[
 \left[
 \begin{array}{cc;{2pt/2pt}c}  
 -4 & 3 & 0\\
 4 & -3 & 0\\
 \end{array}
 \right]    \to  \left[
 \begin{array}{cc;{2pt/2pt}c}  
 -4 & 3 & 0\\
 0 & 0 & 0\\
 \end{array}
 \right]
 .\] 
 \[
 -4x +3y =0 \qquad  \implies y = \frac{4}{3} x
 .\] 
  \\
  \\
  Eigenvectors with eigenvalue $ \lambda =5 $ \\
  \[
	  \left\{ x \begin{bmatrix}
	  3\\
	  4\\
	  \end{bmatrix}
	  \mid x \in \mathbb{R}, x \neq 0 \right\}
  .\] 
  \[
  \lambda=-2 \qquad  \left( A +2 I \right) \vec{ v}    = \vec{ 0} 
  .\] 
  \[
	  \left[
	  \begin{array}{cc;{2pt/2pt}c}  
	  3 & 3 & 0\\
	  4 & 4 & 0\\
	  \end{array}
	  \right] \to \left[
	  \begin{array}{cc;{2pt/2pt}c}  
	  1 & 1 & 0\\
	  0 & 0 & 0\\
	  \end{array}
	  \right]
  .\] 
  \[
  x+y =0 \qquad \implies y = -x 
  .\] 
     \\
     Eigenvectors with eigenvalue $ \lambda = -2$ :\\
     \[
	     \left\{ x \begin{bmatrix}
	     1\\
	     -1\\
	     \end{bmatrix}
	     \mid x \in \mathbb{R} , x \neq 0 \right\}
     .\] 
     $ A$ is diagonalizable since there exists a basis of $  \mathbb{R} ^2$ consisting of eigenvectors of $ A$.  \\
     \\
     Note: $ \mathcal{B} = \left\{ \begin{bmatrix}
     3\\
     4\\
     \end{bmatrix}
     , \begin{bmatrix}
     1\\
     -1\\
     \end{bmatrix}
      \right\}$, forms a basis for $   \mathbb{R} ^2$.\\
      The linear operator $ \phi : \mathbb{R} ^2 \to \mathbb{R} ^2 $ defined by $ \phi  \left( \vec{ v}  \right) = A \vec{ v} $ where $ A = \begin{bmatrix}
      1 & 3\\
      4 & 2\\
      \end{bmatrix}$.\\
      \[
	      \left[ \phi  \right] _{ \mathcal{E} , \mathcal{E}} = \begin{bmatrix}
	      1 & 3\\
	      4 & 2\\
	      \end{bmatrix}
      .\] 
      \[
	      \left[ \phi  \right] _{ \mathcal{B}, \mathcal{B}}  = \begin{bmatrix}
	      5 & 0\\
	      0 & -2\\
	      \end{bmatrix} \text{ since }  \phi  \left( \vec{ b_1}  \right) = 5 \vec{ b_1} \quad \phi \left( \vec{ b_2}  \right) = -2 \vec{ b_2}  
      .\] 
      \[
      \text{ and, } P_{ \mathcal{B}\to \mathcal{E} } = \begin{bmatrix}
      3 & 1\\
      4 & -1\\
      \end{bmatrix}
      .\] 
      \[
      P _{ \mathcal{B} \to \mathcal{E}} =  \frac{1}{-7}\begin{bmatrix}
      -1 & -1\\
      -4 & 3\\
      \end{bmatrix}
      .\] 
 We can check that
 \[
 \begin{bmatrix}
 5 & 0\\
 0 & -2\\
 \end{bmatrix} = \frac{ 1  }{ -7 }\begin{bmatrix}
 -1 & -1\\
 -4 & 3\\
 \end{bmatrix} 
 \begin{bmatrix}
 1 & 3\\
 -4 & 2\\
 \end{bmatrix}
 \begin{bmatrix}
 3 & 1\\
 4 & -1\\
 \end{bmatrix}
 .\] 
 \[
	 \left[ \phi  \right]  _{ \mathcal{B} , \mathcal{B}} = P _{ \mathcal{E} \to\mathcal{B}}  \left[ \phi  \right] _{   \mathcal{E} , \mathcal{E}} P _{ \mathcal{B} \to \mathcal{E}}
 .\] 

   }
   
   \ex{}{
   Let $ A = \begin{bmatrix}
   1 & 1\\
   0 & 1\\
   \end{bmatrix}$. Find the eigenvalues and the corresponding eigenvectors for $ A$ and determine whether $ A$ is diagonalizable.\\
   \\
   \textbf{Solution:}\\
       Eigenvalues:      $ \text{ det } \left( A -  \lambda I \right) =0$\\
        \[
        \left( 1- \lambda \right) \left( 1-\lambda \right) 
        .\] 
	\[
	\lambda =1  \text{ is the only eigenvalue}
	.\] 
	Eigenvectors:\\
	\[
	\left[
	\begin{array}{cc;{2pt/2pt}c}  
	0 & 1 & 0\\
	0 & 0 & 0\\
	\end{array}
	\right] \qquad  x \text{ free, }  y=0
	.\] 
	\[
		U_1 = \left\{ \begin{bmatrix}
		x\\
		0\\
		\end{bmatrix}
		 \mid x \in \mathbb{R} \right\} 
	.\] 
	A is not diagonalizable since there does not exist a basis for $ \mathbb{R}^2 $ consisting of eigenvectors of $ A$.
   }
     \thm{}
     {
       Suppose $ \phi  : V \to V$ is a linear operator defined on the vector space $ V$. \\
All vectors $ \vec{ v} \in V$ that satisfy $ \phi  \left( \vec{ v}  \right) = \lambda \vec{ v} $ for some fixed $  \lambda \in \mathbb{R}$  along with $ \vec{ 0}_v $, form a subspace of $ V$. We'll use $ U_{\lambda}$ to denote this subspace.\\
     }
     \pf{Proof:}{
      \underline{addition:}\\
      Suppose $ \vec{ u_1} , \vec{ u_2}  \in U_{\lambda}$  i.e.  $ \phi  \left(  \vec{ u_1}  \right) = \lambda \vec{ u_1} $ and $ \phi  \left(  \vec{ u_2}  \right)= \lambda \vec{ u_2}    $.\\
      Then,
      \[
      \phi  \left(  \vec{ u_1} + \vec{ u_2}  \right) = \phi  \left(  \vec{ u_1}  \right) + \phi \left( \vec{ u_2}  \right) = \lambda \vec{ u_1} + \lambda \vec{  u_2} = \lambda \left(  \vec{ u_1} + \vec{ u_2}  \right)   \implies \vec{ u_1} + \vec{ u_2}  \in U_{ \lambda}
      .\] 
      \\
      \underline{Scalar multiplication:}\\
      Suppose $ \vec{ u} \in U _{ \lambda}, \alpha \in \mathbb{R}$ \\
      \[
      \phi \left( \alpha \vec{ u}  \right) = \alpha \phi  \left( \vec{ u}  \right) = \alpha \left(  \lambda \vec{ u}  \right) =\lambda \left( \alpha \vec{ u}  \right)         \qquad   \implies   \alpha \vec{ u} \in U _{ \lambda}
      .\] 
      by definition $ \vec{ 0}_V \in  U _{ \lambda} \qquad \implies U_{ \lambda} $  forms a subspace of $ V$.
     }
      We can prove this another way\\
      \pf{Proof:}{
       If $ I:V \to V$ is the identity mapping, $ I \left(  \vec{ v}  \right) = \vec{  v}  $, $ \forall  \vec{ v} \in V$ the $ U_{ \lambda} = ker \left( \phi  -I \right) $ which is a subspace of $ V$ as long as $ \phi - I $ is a linear transformation. $ \phi  -I : V \to V$
 \[
 \left( \phi - I \right) \left( \vec{ v}  \right)  = \phi  \left( \vec{ v}  \right) - I \left(  \vec{ v}  \right) 
 .\]      
 This method can be shown to preserve addition and scalar multiplication.
      }
      \thm{Upper/Lower Triangular Matrices}
      {
        Suppose $ A$ is an $n \times n$  upper (or lower) triangular matrix, say 
	\[
	A = \begin{bmatrix}
		a_{11} & a_{12} & a_{13} & \dots  & a_{1n} \\
	0 & a_{22} & a_{23} & \dots  & a_{2n} \\
	\vdots & \vdots & \vdots & \ddots & \vdots \\
	0 & 0 & 0 & \dots  & a_{ nn}\end{bmatrix}
	.\]  then the eigenvalues of $ A$ would lie along the diagonal of $ A$.
      }
      \pf{Proof:}{
       \[
       \chi _{A} \left( \lambda \right) = \text{ det } \begin{bmatrix}
       		a_{11}-\lambda & a_{12} & a_{13} & \dots  & a_{1n} \
       	0 & a_{22}-\lambda & a_{23} & \dots  & a_{2n} \
       	\vdots & \vdots & \vdots & \ddots & \vdots \
       	0 & 0 & 0 & \dots  & a_{ nn} - \lambda\end{bmatrix}      
       .\] 
       \[
       \chi _{A} \left( \lambda \right) = \left( a_{11} - \lambda\right)  \left(  a_{22}-\lambda \right) \ldots \left( a_{ n n} -\lambda \right) 
       .\] 
       $ \implies $  solutions to $ \chi   _{A} \left( \lambda \right) =0 $ are $ \lambda = a_{11}, a_{22}, \ldots , a_{ n n}  $
      }

      \ex{}{
      Let $ A = \begin{bmatrix}
      \cos \theta &  -\sin \theta \\
      \sin \theta  &  \cos \theta \\
      \end{bmatrix}$
      \\
      We know that $ A$ rotates $ \mathbb{R} $ counterclockwise about the origin by an angle of $  \theta$.\\
      \textit{ Is $ A$ diagonalizable?}\\
      \\
      \textbf{Solution:}\\
      \\
      \[
      \text{ det } \begin{bmatrix}
      \cos \theta - \lambda  & - \sin \theta \\
      \sin\theta   & \cos \theta -\lambda\\
      \end{bmatrix}                        =0
      .\] 
      \[
      \left( \cos \theta- \lambda   \right) ^2 + \sin ^2 \theta  =0
      .\] 
 \[
 \cos ^2 \theta - 2 \lambda \cos \theta + \lambda ^2+ \sin ^2 \theta =0 \qquad \implies \lambda ^2 - 2 \lambda \cos \theta +1 =0 
 .\] 
 \[
 \lambda = \frac{  2 \cos \theta \pm \sqrt{ \left( 2 \cos \theta  \right) ^2 -4}   }{ 2 }
 .\] 
 \[
 \lambda = \frac{ 2 \cos \theta \pm \sqrt{ 4 \left(  \cos ^2 \theta -1 \right) }   }{ 2 }
 .\] 
 \[
 \lambda = \frac{  2 \cos \theta  \pm \sqrt{-4 \sin ^2\theta }  }{  2}
 .\] 
 if $ \sin \theta \neq 0$ then $  \lambda \notin  \mathbb{R} $\\
 $ \implies$ no eigenvalues if $  \theta \in \mathbb{R} \backslash \left\{ k \pi \backslash k \in \mathbb{Z} \right\} $ \\
 If $ \theta = 2k \pi $, $  k \in \mathbb{Z}$ then $ A = \begin{bmatrix}
 1 & 0\\
 0 & 1\\
 \end{bmatrix}$ which is $ I_2$ the $2 \times 2$  identity, which has $ \lambda =1 $ as an eigenvalue and corresponding eigenvectors: $ \begin{bmatrix}
 1\\
 0\\
 \end{bmatrix}
 , \begin{bmatrix}
 0\\
 1\\
 \end{bmatrix}
 $ \\
 \\
 If $ \theta = \left( 2k +1  \right)  \pi$, $ k \in \mathbb{Z}$ then $ A = \begin{bmatrix}
 -1 & 0\\
 0 & -1\\
 \end{bmatrix}$
 Which has $ \lambda = -1 $ as an eigenvalue and corresponding eigenvectors $ \begin{bmatrix}
 1\\
 0\\
 \end{bmatrix}, \begin{bmatrix}
 0\\
 1\\
 \end{bmatrix}
 $ \\
 \\
 In fact the last two cases $ A$ is clearly diagonalizable since it is a diagonal matrix.
      }
      \mlem{}{
      If $ A$ and $ B$ are similar matrices then $ \chi _{A } \left( \lambda \right) = \chi _{ B} \left( \lambda \right) $}


      \pf{Proof:}{
       $ A,B$ are similar, hence there exists an invertible matrix $ M$ such that $ B = M ^{ -1} A M$ \\
       \begin{align*}
       	\chi _{B} \left( \lambda \right) = \text{ det } \left( B - \lambda I \right) = \text{ det } \left( M^{-1} A M - \lambda I \right) \\
	&= \text{ det } \left( M ^{-1}A M - M ^{-1} \left( \lambda I \right) M \right)\\
	&= \text{ det } \left( M^{-1} \right)  \text{ det }  \left( A - \lambda I \right) \text{ det } M\\
	&=  \left( \frac{1}{ \text{ det } M} \right) \left( \chi _A \left( \lambda \right)  \right) \text{ det } M\\
	\implies \chi _{B} \left( \lambda \right) = \chi _{A} \left( \lambda \right)
       .\end{align*}
      }
       \begin{corollary}[]
	       Let $ \phi : V \to V$ be a linear operator and let $ \mathcal{B}$ be a basis of $ V$ and let $ \left[ \phi  \right] _{ \mathcal{B} , \mathcal{B} } $  be a matrix of $ \phi $ with respect to $ \mathcal{B}$. We define $  \chi _{ \phi } \left(  \lambda \right) = \chi _{ \left[ \phi  \right]_{ \mathcal{B} \mathcal{B}}} \left( \lambda \right) $   and $  \chi _{ \phi } \left( \lambda \right) $ is independent of the choice of basis $ \mathcal{B}$.
       \end{corollary}
                   \pf{Proof:}{
			   Let $ \mathcal{F}$ be another basis for $ V$ and let $ \left[ \phi  \right] _{ \mathcal{F} \mathcal{F}}$ be the matrix of $ \phi $ with respect to $ \mathcal{F}$.\\
			   We know that $ \left[ \phi  \right] _{ \mathcal{F} \mathcal{F}}$ is similar to $ \left[ \phi  \right] _{ \mathcal{F} \mathcal{F}}$,
			   \[
				   \left[ \phi  \right]_{ \mathcal{B} \mathcal{B}} = P _{ \mathcal{F} \to \mathcal{B}} \left[ \phi  \right]_{ \mathcal{F} \mathcal{F}} P _{ \mathcal{B} \to \mathcal{F}}
			   .\] 
			   Hence, $ \chi _{ \left[ \phi  \right]_{ \mathcal{B} \mathcal{B}}} \left( \lambda \right) = \chi _{ \left[ \phi  \right] _{ \mathcal{F} \mathcal{F}}} \left( \lambda \right) $                  \\
			   This implies that the characteristic polynomial of a linear operator $ \phi : V \to V$ is independent of the choice of basis for $ V$.
                   }
                   
      
      
	   
      

     
   
         
         
         
















\end{document}                              
