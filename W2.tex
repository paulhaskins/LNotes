\documentclass{report}

\input{preamble}
\input{macros}
%--------------------------------------------------
% LIE ALGEBRAS
%--------------------------------------------------
\newcommand*{\kb}{\mathfrak{b}}  % Borel subalgebra
\newcommand*{\kg}{\mathfrak{g}}  % Lie algebra
\newcommand*{\kh}{\mathfrak{h}}  % Cartan subalgebra
\newcommand*{\kn}{\mathfrak{n}}  % Nilradical
\newcommand*{\ku}{\mathfrak{u}}  % Unipotent algebra
\newcommand*{\kz}{\mathfrak{z}}  % Center of algebra

%--------------------------------------------------
% HOMOLOGICAL ALGEBRA
%--------------------------------------------------
\DeclareMathOperator{\Ext}{Ext} % Ext functor
\DeclareMathOperator{\Tor}{Tor} % Tor functor

%--------------------------------------------------
% MATRIX & GROUP NOTATION
%--------------------------------------------------
\DeclareMathOperator{\GL}{GL} % General Linear Group
\DeclareMathOperator{\SL}{SL} % Special Linear Group
\newcommand*{\gl}{\operatorname{\mathfrak{gl}}} % General linear Lie algebra
\newcommand*{\sl}{\operatorname{\mathfrak{sl}}} % Special linear Lie algebra

%--------------------------------------------------
% NUMBER SETS
%--------------------------------------------------
\newcommand*{\RR}{\mathbb{R}}
\newcommand*{\NN}{\mathbb{N}}
\newcommand*{\ZZ}{\mathbb{Z}}
\newcommand*{\QQ}{\mathbb{Q}}
\newcommand*{\CC}{\mathbb{C}}
\newcommand*{\PP}{\mathbb{P}}
\newcommand*{\HH}{\mathbb{H}}
\newcommand*{\FF}{\mathbb{F}}
\newcommand*{\EE}{\mathbb{E}} % Expected Value

%--------------------------------------------------
% MATH SCRIPT, FRAKTUR, AND BOLD SYMBOLS
%--------------------------------------------------
\newcommand*{\mcA}{\mathcal{A}}
\newcommand*{\mcB}{\mathcal{B}}
\newcommand*{\mcC}{\mathcal{C}}
\newcommand*{\mcD}{\mathcal{D}}
\newcommand*{\mcE}{\mathcal{E}}
\newcommand*{\mcF}{\mathcal{F}}
\newcommand*{\mcG}{\mathcal{G}}
\newcommand*{\mcH}{\mathcal{H}}

\newcommand*{\mfA}{\mathfrak{A}}  \newcommand*{\mfB}{\mathfrak{B}}
\newcommand*{\mfC}{\mathfrak{C}}  \newcommand*{\mfD}{\mathfrak{D}}
\newcommand*{\mfE}{\mathfrak{E}}  \newcommand*{\mfF}{\mathfrak{F}}
\newcommand*{\mfG}{\mathfrak{G}}  \newcommand*{\mfH}{\mathfrak{H}}

\usepackage{bm} % Ensure bold math works correctly
\newcommand*{\bmA}{\bm{A}}
\newcommand*{\bmB}{\bm{B}}
\newcommand*{\bmC}{\bm{C}}
\newcommand*{\bmD}{\bm{D}}
\newcommand*{\bmE}{\bm{E}}
\newcommand*{\bmF}{\bm{F}}
\newcommand*{\bmG}{\bm{G}}
\newcommand*{\bmH}{\bm{H}}

%--------------------------------------------------
% FUNCTIONAL ANALYSIS & ALGEBRA
%--------------------------------------------------
\DeclareMathOperator{\Aut}{Aut} % Automorphism group
\DeclareMathOperator{\Inn}{Inn} % Inner automorphisms
\DeclareMathOperator{\Syl}{Syl} % Sylow subgroups
\DeclareMathOperator{\Gal}{Gal} % Galois group
\DeclareMathOperator{\sign}{sign} % Sign function


%\usepackage[tagged, highstructure]{accessibility}
\usepackage{tocloft}
\usepackage{arydshln}
\usetikzlibrary{arrows.meta, decorations.pathreplacing}
\usepackage{tikz-cd}
\usepackage{polynom}
\usepackage{pifont}
\newcommand{\pistar}{{\zf\symbol{"4A}}}
% a tiny helper for a stretched phantom (for the underbrace)
\newcommand\mc[1]{\multicolumn{1}{c}{#1}}


\newlist{steplist}{enumerate}{1}
\setlist[steplist,1]{%
  label=\underline{\textbf{\arabic*:}}, % gives “1:” “2:” … (bold, underlined)
  labelsep=0.6em,                      % space after the label
  leftmargin=2.8em,                    % indent so text lines up neatly
  itemsep=1\baselineskip,              % vertical gap between steps
  topsep=0pt,                          % no extra space before/after list
  align=left
}

\begin{document}
\title{Linear Algebra I}
\author{Lecture Notes Provided by Dr.~Miriam Logan.}
\date{}
\maketitle
\tableofcontents
\newpage  


\thm{}{
    \label{thm:sec-\thesection.\arabic{theorem}?}
    Let $\vec{u} $, $\vec{v} $, $\vec{w} \in \mathbb{R}^{3} $ and $c\in \mathbb{R}$ we have

    \begin{enumerate}[label=(\roman*)]
      \item $\vec{v} \times \vec{w} = -  \vec{w} \times \vec{v} $
      \item $\vec{v} \times  \left( \vec{w} +\vec{u}  \right) = \vec{v} \times \vec{w} + \vec{v} \times  \vec{u}                  $
      \item $\vec{v} \times \left( c \vec{w}  \right) = c \left( \vec{v} \times  \vec{w}  \right) $ 
      \item $\vec{v} \times \vec{v} = \vec{0} $, $\|\vec{v} \times \vec{v} \|=0$, $\implies \vec{v} \times  \vec{v} =\vec{0} $.
      \item $\vec{u} \cdot \left( \vec{v} \times \vec{w}  \right) = - \vec{v} \cdot \left( \vec{u} \times \vec{w}  \right) $ 
      \item $\vec{u} \times \left( \vec{v} \times \vec{w}  \right) = \left( \vec{u} **\vec{v}  \right) \vec{w}  $
      \item Jacobi Identity: $\left( \vec{u} \times  \vec{v}  \right)\times \vec{w} + \left( \vec{v} \times \vec{w}  \right) \times  \vec{u}  + \left( \vec{w} \times \vec{u} \right) \times \vec{v}  = \vec{0}   $
    \end{enumerate}
}



 \pf{Proof:}{
  \begin{enumerate}[label=(\roman*)]
    \item 

      Let \[
\vec{v} = \langle v_1,v_2,v_3  \rangle \quad \vec{w} = \langle w_1,w_2,w_3  \rangle \] \[
\vec{v} \times \vec{w} = \langle v_2w_3-w_2v_3, v_3w_1-v_1w_3, v_1w_2-v_2w_1  \rangle  
%= \langle v_2\left( w_3+u_3 \right) -\left( w_2+u_2 \right) v_3, v_3 \left( w_1+u_1 \right) -v_1\left( w_3+u_3 \right)   \rangle 
.\] 
\[
=\langle -1 \left( w_2v_3-w_3v_2, -\left( w_3v_1-w_1v_3 \right) , -\left( w_1v_2-w_2v_1 \right)  \right)   \rangle 
.\] \[
= - \left( \vec{w} \times \vec{v}  \right) 
.\] 
    

    \item   \\
       $\vec{v} \times  \left( \vec{w} +\vec{u}  \right) = \langle v_1,v_2,v_3  \rangle \times  \left(  \langle w_1+u_1, w_2+u_2, w_3+u_3  \rangle  \right) $ 
    \[
    \langle v_2\left( w_3+u_3 \right) - \left( w_2+u_2 \right) v_3, v_3\left( w_1+u_1 \right) - v_1\left( w_3+u_3 \right) , v_1\left( w_2+u_2 \right) - v_2\left( w_1+u_1 \right)   \rangle 
    .\] \[
    = \langle v2w_3-w2v_3+v_2u_3-u2v_3, v3w_1-v_1w_3+v_3u_1-v_1u_3, v_1,w_2-v_2w_1+v_1u_2-v_2u_1  \rangle 
    .\] 
\[
= \langle v_2w_3-w_2v_3, v_3w_1-v_1w_3, v_1w_2-v_2w_1  \rangle + \langle v_2u_3-u_2v_3, v_3u_1-v_1u_3, v_1u_2-v_2u_1  \rangle 
.\] \[
= \vec{v} \times \vec{w} +\vec{ v} \times \vec{u} 
.\]
    \item 
      \[
\vec{v} \times \left( c\vec{w} \right) = \langle v_1,v_2,v_3  \rangle \times  \langle cw_1,cw_2,cw_3  \rangle 
.\] \[
= \langle c \left( v_2w_3 -w_2v_3 \right) , c\left( v_2w_1-v_1w_3 \right) , c\left( v_1w_2-v_2w_1 \right)   \rangle 
.\] \[
=c \left( \vec{v} \times  \vec{w}  \right) 
.\]
  \item 
    \[
\vec{v} \times \vec{v} = \langle v_2v_3- v_2v_3, v_3v_1- v_1v_3, v_1v_2-v_2v_1  \rangle = \langle 0,0,0  \rangle 
.\] 

Note the property we take for granted in multiplying real numbers: $\left( ab \right)c = a \left( bc  \right)   $ (associativity) is not satisfied for cross products i.e
\[
\left( \vec{u} \times  \vec{v}  \right) \times  \vec{w}  \neq \vec{u} \times  \left( \vec{v}  \times  \vec{w}  \right) 
.\] 
We won't prove the rest of the identities, since the proofs become quite convoluted for this early stage.\\
    \end{enumerate}
One observation worth making 
\[
\vec{ u} \times  \left(  \vec{ v} \times \vec{ w}  \right) = \left(  \vec{ u} \cdot  \vec{ w}  \right) \vec{ v} - \left(  \vec{ u} \cdot  \vec{ v}  \right) \vec{ w}
.\] 
Whileas,
\begin{align*}
  \left( \vec{ u} \times  \vec{ v}  \right) \times  \vec{ w} &= - \vec{ w} \times  \left( \vec{ u} \times  \vec{ v}  \right) - \left[ \left( \vec{ w} \cdot  \vec{ v}  \right) \vec{ u} - \left( \vec{ w} \cdot  \vec{ u}  \right) \vec{ v}  \right]  \\
  &= \left( \vec{ w} \cdot \vec{ u}  \right) \vec{ v} - \left( \vec{ w} \cdot \vec{ v}  \right) \vec{ u} \\
  &\neq \vec{ u} \times  \left( \vec{ v} \times  \vec{ w}  \right) 
.\end{align*}
i.e. a property of real numbers that we take for granted,  $ \left( a \cdot  b \right) \cdot c = a \left( b \cdot  c \right) \qquad  \forall  a,b,c \in \mathbb{R}$ associativity of multiplication is not satisfied for cross products.\\
  } 
  \ex{}{
  Find the angle in between the vectors $\vec{u} = \langle -2,5 \rangle$ and $\vec{v} = \langle 5, 12 \rangle$.\\
  \[
  \| \vec{ v} \|= \sqrt{ \left( -2 \right) ^2 + 5^2} = \sqrt{4+25} = \sqrt{29}
  .\] 
  \[
  \| \vec{ u} \|= \sqrt{ 5^2 + 12^2} = \sqrt{25+144} = \sqrt{169} = 13
  .\] 
  \[
  \vec{ u} \cdot  \vec{ v} = -10 + 60 = 50
  .\] 
  \[
  \vec{ u} \cdot  \vec{ v} = \| \vec{ u} \| \| \vec{ v} \| \cos \phi  = 13 \sqrt{29} \cos \phi \qquad  \implies 50 = 13 \sqrt{29} \cos \phi
  .\] 
  \[
  \frac{ 50  }{ 13 \sqrt{29} } = \cos \phi 
  .\] 
  \[
  \phi = \cos^{-1} \left( \frac{ 50  }{ 13 \sqrt{29} } \right)
  .\] 
  }
  \ex{}{
  Determine whether the following vectors are orthogonal
  \begin{enumerate} [label=(\alph*)]
    \item  $ \langle -5,3,7  \rangle $ and $ \langle 6, -8 ,2  \rangle $
    \item $ \langle 4,6  \rangle $ and $ \langle -3 ,2  \rangle $
    \item    $ \langle 1,0,1,0,1,0,1  \rangle $ and $ \langle 0,1,0,1,0,1,1  \rangle $
    \item $ \langle x,y,z  \rangle $ and $ \langle y, -x, 0  \rangle $
    \end{enumerate}
    \underline{Dot Product:}\\
    \begin{enumerate} [label=(\alph*)]
      \item $ -30 -24 +14 \neq 0 \implies $ not orthogonal
      \item $ -12 + 12 = 0 \implies$ orthogonal
      \item $ x - xy + 0 =0 \implies$  orthogonal 
      \end{enumerate}
      
      
    
  }
  

  \section{Volume of a Parallelepiped}
   \textit{A parallelepiped is a 3-dimensional figure formed by six parallelograms} \\
\textbf{Volume of a parallelepiped:} \\
\\
Let $\vec{u} $, $\vec{v} $, $\vec{w} \in \mathbb{R}^3$ \\
\\

XXX TEXIT  FIX

\\
\\
\begin{figure}
  
  \tdplotsetmaincoords{100}{75}{50}
\begin{tikzpicture}[
    tdplot_main_coords,
    line cap=round,   >=Stealth,
    thick,
    every node/.style={font=\small}
  ]

% ── fundamental vectors ───────────────────────────────────────────────
\coordinate (O) at (0,0,0);        % origin
\coordinate (U) at (4,0,0);        % \vec u       (along +x)
\coordinate (P) at (1.2,2,0);      % back-shift   (into +y)
\coordinate (W) at (0.8,1.0,3);    % \vec w       (slanted up/out)
\coordinate (V) at (2.3,0.8,0);    % \vec v       (in base plane)

% corners of the parallelepiped
\coordinate (A)  at (U);
\coordinate (B)  at (P);
\coordinate (C)  at ($(U)+(P)$);

\coordinate (Ow) at (W);
\coordinate (Aw) at ($(U)+(W)$);
\coordinate (Bw) at ($(P)+(W)$);
\coordinate (Cw) at ($(U)+(P)+(W)$);

% ── edges (solid = visible, dashed = hidden) ──────────────────────────
% base rectangle
\draw[blue!70!black]        (O)--(A)--(C);
\draw[blue!70!black,dashed] (C)--(B)--(O);

% verticals (W-edges)
\draw[blue!70!black]        (O)--(Ow)  (A)--(Aw)  (C)--(Cw);
\draw[blue!70!black,dashed] (B)--(Bw);

% top rectangle
\draw[blue!70!black]        (Ow)--(Aw)--(Cw)--(Bw)--cycle;

% ── main vectors ──────────────────────────────────────────────────────
\draw[->,very thick,blue!80!black]  (O) -- (U)
      node[below right=-2pt] {$\vec u$};

\draw[->,very thick,green!70!black] (O) -- (V)
      node[above right=-1pt] {$\vec v$};

\draw[->,very thick,magenta!70]     (O) -- (W)
      node[above left=-1pt] {$\vec w$};

% copy of w at the tip of v (helps visualise the height)
\draw[very thick,magenta!70,dashed,->] (V) -- ($(V)+(W)$);

% (optional) dashed trace of v across the base
\draw[blue!70!black,dashed]
      (V) .. controls +(0.8,0.1,0) and +(0,-0.2,0) .. (A);

% a vertical reference axis (normal to the base)
\draw[purple!80!black,very thick] (O) -- (0,0,4)
      node[above,align=center] {perpendicular\\[-2pt] to the base};

\end{tikzpicture}

\end{figure}
\\
\clearpage
\\
The three vectors  $\vec{u} $, $\vec{w} $ and $\vec{v} $ (assuming one doesn't lie on the plane defined by the other two.) determine a parallelpiped. \\
\\
A parallelpiped is an oblique prism and thus
the volume of a parallelpiped is given by the \textit{(area of base) ( $\perp$ height)}. If we call the perpendicular height $h$ then
\[
V_p = | \vec{v} \times \vec{w} | \cdot h
.\] 
The perpendicular height is the length of a projection of $u$ onto the direction perpendicular to the base. $\vec{v} \times \vec{w} $ is perpendicular to the base.\\
XXX TEXIT\\ 
  


\raggedcolumns
\begin{multicols}{2}


\break


Let $ \theta $ be the angle between $ \vec{ w} $ and $ \vec{ u} \times \vec{ v}  $  \\
\[
\cos \theta = \frac{ h  }{ \| \vec{ w} \| } \qquad \implies h = \| \vec{ w} \| \cos \theta
.\] 
\begin{align*}
  \text{ Hence the volume of a parallelpiped} &= \| \vec{ u} \times  \vec{ v} \|  \| \vec{ w} \| \cos \theta\\
  &= \| \vec{ u} \times  \vec{ v} \| \cdot \vec{ w} \\
.\end{align*}
by definition of the cross product
\end{multicols}


\ex{}{
  \textit{Find the volume of the parallelpiped determined by the vectors $\vec{u} = \langle 1,2,2  \rangle $, $\vec{v} = \langle 2,3,5  \rangle $ and $\vec{w} = \langle 3,0,1  \rangle $}\\
\textbf{Solution:} \[
|\vec{u} \cdot \left( \vec{v} \times \vec{w}  \right) | = \|\langle 1,2,2  \rangle \cdot \langle 3,13,-9  \rangle \| = | 3 + 26 -18| =11
.\] 


}

  \nt{
  Since the volume of the parallelpidped determined by $\vec{u} $, $\vec{v} $ and $ \vec{ w} $ is equal to $|\vec{u} \cdot \left( \vec{v} \times \vec{w}  \right) |$ it should be true that 
\[
|\vec{u} \cdot \left( \vec{v} \times \vec{w}  \right) | = |\vec{u} \cdot \left( \vec{w} \times \vec{v}  \right) | = \|\vec{v} \cdot \left( \vec{u} \times  \vec{w}  \right) \|= \|\vec{v} \cdot \left( \vec{w} \times \vec{u}  \right) \|= \|\vec{w} \cdot \left( \vec{v} \times \vec{u}  \right) \|= \|\vec{w} \cdot \left( \vec{u} \times \vec{v}  \right) \|
.\] 
These properties come from the following three facts
\begin{enumerate}[label=(\roman*)]
  \item $\vec{v} \times \vec{w} = - \vec{w} \times  \vec{v} $
  \item $\vec{u} \cdot  \left( \vec{v} \times  \vec{w} = -\vec{v} \cdot \left( \vec{u} \times \vec{w}  \right)  \right) $\\
  \item $\vec{u} \cdot  \vec{ v} = \vec{ v} \cdot  \vec{ u} $\\
\end{enumerate}
  }
  For example, to show 
  \[
  \| \left( \vec{ u} \times  \vec{ v}  \right) \cdot  \vec{ w} \|= \| \vec{ v} \times \vec{w} \| \cdot \vec{ u}
  .\]
  \begin{align*}
    \left( \vec{ u} \times  \vec{ v}  \right) \cdot  \vec{w } &= \vec{ w} \cdot  \left( \vec{ u} \times  \vec{ v}  \right) &= - \vec{ u} \cdot \left( \vec{ w} \times  \vec{ v}  \right)   \\
    &= - \vec{ u} \cdot  \left( - \vec{ v} \times  \vec{ w}  \right) &= \vec{ u} \cdot  \left( \vec{ v} \times  \vec{ w}  \right)   \\
    &= \left( \vec{ v}  \times  \vec{ w}  \right) \cdot  \vec{ u} 
  .\end{align*}
  Hence $ \| \left( \vec{ u} \times  \vec{ v}  \right) \cdot  \vec{ w} \|= \| \left( \vec{ v} \times  \vec{ w} \cdot  \vec{ u}  \right) \|$
  
  \section{Vectors in $\mathbb{R}^3$}
  A line in $\mathbb{R}^2 $ is completely determined by knowing 
\begin{itemize}
        \item The slope
        \item A point on the line
\end{itemize}
A line in $\mathbb{R}^3 $ is completely determined by knowing 
\begin{itemize}
        \item A direction vector
        \item A point on the line
\end{itemize}
Suppose a line $L$ in  $\mathbb{R}^3$ passes through the point $p_0$ and has the direction vector $\vec{v} $and we want to find the vector equation of $L$.\\
XXX TEXIT\\

\\

\begin{tikzpicture}[scale=0.55,>=Stealth,line cap=round,font=\small]

  % -------- colours ---------------------------------------------------
  \definecolor{ax}{RGB}{150,70,200}     % axis colour (purple)
  \definecolor{vecblue}{RGB}{40,60,160} % blue rays
  \definecolor{vecred}{RGB}{204, 45, 64}% vector v
  \definecolor{vecgreen}{RGB}{ 10,160, 90}% point P0

  % -------- axes ------------------------------------------------------
  \coordinate (O) at (0,0);
  \draw[ax,very thick,->] (-6,0) -- (12.5,0);      %  x-axis
  \draw[ax,very thick,->] (0,-2) -- (0,6.5);       %  y-axis
  \draw[ax,very thick,->] (O) -- (-6,-6);          %  extra diagonal
  \draw[ax,thick,dashed] (-6,0) -- (O);            %  x negative (dashed)
  \draw[ax,thick,dashed] (0,-2) -- (O);            %  y negative (dashed)

  % -------- basic vectors --------------------------------------------
  \coordinate (v)  at (3,1);          %  \vec v
  \coordinate (P0) at (2,2.5);        %  \vec{OP_0}

  % \vec v from the origin
  \draw[->,very thick,vecred] (O) -- (v)
        node[below right=-1pt] {$\vec v$};

  % \vec{OP_0}
  \draw[->,very thick,vecblue] (O) -- (P0)
        node[above left=-1pt] {$\vec{OP_0}$};

  % copy of v starting at P0 (illustrates OP_0+v)
  \draw[->,very thick,vecred] (P0) -- ($(P0)+(3,1)$)
        node[above=2pt] {$\vec v$};

  % dashed arc showing the angle between v and OP_0
  \draw[ax,dashed] (1.4,0.46) arc (18:51:1.55);

  % -------- the rays  OP_0 + k v  (k = −2.5, −1, 0, 1, 2, 3) ----------
  \foreach \x/\y/\txt/\pos in
    { -5.5/0  /{$\vec{OP_0}-2.5\vec v$}/left,
      -1  /1.5/{$\vec{OP_0}-\vec v$}/above left,
       2  /2.5/{$\vec{OP_0}$}/above left,
       5  /3.5/{$\vec{OP_0}+\vec v$}/above,
       8  /4.5/{$\vec{OP_0}+2\vec v$}/above,
      11  /5.5/{$\vec{OP_0}+3\vec v$}/above}
  {%
     \draw[->,very thick,vecblue] (O) -- (\x,\y)
           node[\pos] {\txt};
  }

  % -------- mark the point P_0 ---------------------------------------
  \fill[vecgreen] (P0) circle (3pt) node[above right=-1pt] {$P_0$};

\end{tikzpicture}
\\
$L$ passes through $p_0$ and has the same direction as $\vec{v} $. To get to any point on the line, you start at the origin, go to $p_0$ and then go any amount you like in the $\vec{v} $ direction (or $-\vec{v} $ ).\\
\\
Vector equation of $L$ :\\
All vectors $\vec{r}\left(t  \right)  $, whose tips lie on the line $L$ satisfy the vector equation 
\[
\vec{r} \left( t \right)  = \vec{Op_0} +t \vec{v} 
.\] As we vary $t$ the line is traced out by the tips of the vectors $\vec{Op_0} +t \vec{v} $.\\  


\section{Co-ordinates of Vector equation}
    

Suppose $P_0 = \left( x_0,y_0,z_0 \right) $ and $\vec{v} = \langle a,b,c  \rangle $, then the vector $\langle x,y,z  \rangle $ lies on $L$ if
\[
\langle x,y,z  \rangle = \langle x_0,y_0,z_0  \rangle _ t \langle a,b,c  \rangle 
.\] i.e
\[
\langle x,y,z  \rangle = \langle x_0+ta, y_0+tb, z_0+tc  \rangle 
.\] 
$\implies$ Parametric equations of a line through $\left( x_0,y_0,z_0 \right) $ in the direction of $\vec{v} = \langle  a,b,c  \rangle $ are given by 
\begin{align*}
        x=x_0+ta,\\
        y=y_0+tb\\
        z=z_0+tc
\end{align*}
This may look like a lot of unknowns but, remember, $x_0$, $y_0$, $z_0$, $a$, $b$, $c$ are all known.\\
\\
\nt{
Since there are an infinite number  of points on a line, and, because any scalar multiple of $ \vec{ v} = \langle a,b,c  \rangle $ series as a direction vector for a line that means that there are an infinite number of different looking vector equations for the same line.\\
}

\ex{}{
\textit{Find the vector equation of the line passing through $P \left( -8,1,4 \right) $ and $Q \left( 3,-2,4 \right) $}\\
\textbf{Solution:} TEXT XXX \\
direction vector $\vec{v} $ joining $P$ to $Q$ $= \langle 11,-3,0  \rangle $ \\
\begin{align*}
        \vec{r} \left( t \right) = \langle -8,1,4  \rangle + t \langle 11,-3,0  \rangle \\
        = \langle -8 +11t, 1-3t, 4  \rangle 
.\end{align*}
Note we could also have described the line as
 \begin{align*}
        \vec{s} \left( t \right) = \langle 3,-2,4  \rangle + t \langle 11,-3,0  \rangle \\
        = \langle 3+11t, -2-3t,4  \rangle 
.\end{align*}
}
\section{ Planes in $\mathbb{R}^3$}
A plane in $\mathbb{R}^3$ is completely determined by knowing a point that lies on that plane and a vector that is orthogonal to the plane.\\
\\
Suppose $P_0\left( x_0,y_0,z_0 \right) $ is a point on a plane with normal vector $\vec{n} = \langle a,b,c  \rangle $ \\
\\

\tdplotsetmaincoords{70}{25} % (elev, azimuth) – tweak view if desired

\begin{tikzpicture}[tdplot_main_coords,
                    line cap=round,
                    >=Stealth,
                    scale=0.9,
                    every node/.style={font=\small}]

% -----------------------------------------------------------------
% colours
% -----------------------------------------------------------------
\definecolor{axcol}{RGB}{25,50,140}      % axes
\definecolor{planecol}{RGB}{  5,160,140} % plane edges
\definecolor{p0col}{RGB}{ 30,140,255}    % P0 vector/point
\definecolor{pcol}{RGB}{210, 40, 80}     % P  vector/point
\definecolor{segcol}{RGB}{  0,160,110}   % P0–P segment
\definecolor{ncol}{RGB}{ 50, 60,180}     % normal arrow

% -----------------------------------------------------------------
% plane (choose four corners in the z=0 plane)
% -----------------------------------------------------------------
\coordinate (A) at (-4,-2,0);
\coordinate (B) at ( 5,-1,0);
\coordinate (C) at ( 6, 2,0);
\coordinate (D) at (-3, 1,0);

\draw[planecol, very thick] (A)--(B)--(C)--(D)--cycle;

% -----------------------------------------------------------------
% axes
% -----------------------------------------------------------------
\coordinate (O) at (0,0,0);
\draw[axcol,very thick,->] (O) -- (7,0,0) node[below right=-2pt] {$x$};
\draw[axcol,very thick,->] (O) -- (0,0,6) node[left=2pt] {$z$};
\draw[axcol,very thick,->] (O) -- (-5,-5,0) node[left] {$y$};     % back/left axis
% dashed negative parts
\draw[axcol,dashed,thick] (O) -- (-2,0,0);
\draw[axcol,dashed,thick] (O) -- (0,-2,0);
\draw[axcol,dashed,thick] (O) -- (0,0,-2);

% -----------------------------------------------------------------
% points & vectors
% -----------------------------------------------------------------
\coordinate (P0) at (2,1,1);          % blue point
\coordinate (P)  at (4,1.8,1.6);      % red point

% OP0
\draw[->,very thick,p0col] (O) -- (P0)
      node[above left=-2pt] {$\vec{OP_0}$};

% OP
\draw[->,very thick,pcol] (O) -- (P)
      node[above right=-2pt] {$\vec{OP}$};

% segment P0–P
\draw[very thick,segcol] (P0) -- (P);

% marks for the two points
\fill[p0col] (P0) circle (3pt)
      node[above=4pt] {$P_0(x_0,y_0,z_0)$};
\fill[pcol]  (P)  circle (3pt)
      node[right=3pt] {$P(x,y,z)$};

% (optional) dashed helper curve (projection or path)
\draw[axcol,dashed] (O) .. controls (1,0.3,0.3) .. (P0);

% -----------------------------------------------------------------
% normal vector (drawn at a convenient spot on the plane)
% -----------------------------------------------------------------
\coordinate (Nbase) at (5,0.6,0);        % foot on the plane
\draw[->,very thick,ncol] (Nbase) -- ++(0,0,3)
      node[right] {$\vec n\!{}_{\!\perp}$};

\end{tikzpicture}\\
\\
\\
\\
\tdplotsetmaincoords{68}{200}  %  (elev, azimuth) – adjust to taste

\begin{tikzpicture}[tdplot_main_coords,
                    line cap=round, >=Stealth,
                    every node/.style={font=\small},
                    scale=0.9]

%──────────────── colours ──────────────────────────────────────────────
\colorlet{axcol}    = {rgb:(0.1,0.25,0.55)} %  axes (deep navy)
\colorlet{planecol} = {rgb:(0.0,0.65,0.48)} %  plane outline (teal)
\colorlet{p0col}    = {rgb:(0.0,0.45,0.95)} %  P0 vector / dot (blue)
\colorlet{pcol}     = {rgb:(0.80,0.10,0.30)}%  P  vector / dot (red)
\colorlet{segcol}   = {rgb:(0.0,0.55,0.35)} %  P0–P connector  (green)
\colorlet{ncol}     = {rgb:(0.20,0.25,0.70)}%  normal arrow (purple-blue)
\colorlet{dashcol}  = {axcol}               %  dashed helpers (axes colour)

%──────────────── axes ─────────────────────────────────────────────────
\coordinate (O) at (0,0,0);
\draw[axcol,very thick,->] (O) -- (6.5,0,0) node[below right=-2pt] {$x$};
\draw[axcol,very thick,->] (O) -- (-4,-4,0) node[left=0pt] {$y$};
\draw[axcol,very thick,->] (O) -- (0,0,6)  node[left=2pt] {$z$};

% negative/dashed parts for a little depth cue
\draw[axcol,dashed,thick] (O) -- (-2,0,0);
\draw[axcol,dashed,thick] (O) -- (0,-2,0);
\draw[axcol,dashed,thick] (O) -- (0,0,-2);

%──────────────── tilted plane  (4 co-planar points) ───────────────────
\coordinate (A) at (-3.5,-2, 0.5);
\coordinate (B) at ( 5.5,-1, 1.2);
\coordinate (C) at ( 6.5, 3, 2.4);
\coordinate (D) at (-2.5, 3, 1.7);
\draw[planecol,very thick] (A)--(B)--(C)--(D)--cycle;

%──────────────── key points & vectors ─────────────────────────────────
\coordinate (P0) at (2.0,1.2,2.0);   %  blue   point
\coordinate (P)  at (4.2,2.0,2.8);   %  red    point

%  OP0  (blue)
\draw[->,very thick,p0col] (O) -- (P0)
      node[above left=-1pt] {$\vec{OP_0}$};

%  OP   (red)
\draw[->,very thick,pcol]  (O) -- (P)
      node[above right=-1pt] {$\vec{OP}$};

%  P0–P connector (green, no arrow head so you can tell it from OP)
\draw[very thick,segcol] (P0) -- (P);

% dot markers
\fill[p0col] (P0) circle (3pt) node[above=4pt] {$P_{0}(x_{0},y_{0},z_{0})$};
\fill[pcol]  (P)  circle (3pt) node[right=4pt] {$P(x,y,z)$};

%──────────────── helper dashed projections ────────────────────────────
\draw[dashcol,dashed] (O) .. controls (1,0.3,0) .. (P0);  % dashed arc
\draw[dashcol,dashed] (P0) -- ($(P0)-(0,0,P0_z)$);        % drop to x–y plane
\draw[dashcol,dashed] ($(P0)-(0,0,P0_z)$) -- (O);         % over to origin

%──────────────── outward normal vector  n_⊥ ───────────────────────────
%  (use two non-parallel edges of the quadrilateral to compute one)
\path let
   \p1 = ($(B)-(A)$),
   \p2 = ($(D)-(A)$),
   \n1 = {veclen(\x1,\y1,\z1)},
   \n2 = {veclen(\x2,\y2,\z2)}
 in
 {%
   % cross product for normal (scaled)
   \pgfmathsetmacro{\nx}{\y1*\z2 - \z1*\y2}
   \pgfmathsetmacro{\ny}{\z1*\x2 - \x1*\z2}
   \pgfmathsetmacro{\nz}{\x1*\y2 - \y1*\x2}
   \pgfmathsetmacro{\scale}{2.5/veclen(\nx,\ny,\nz)} % normalise length ~2.5
   \coordinate (Ndir) at (\scale*\nx,\scale*\ny,\scale*\nz);
 };

\coordinate (Nbase) at (3.8,0.7,1.25);      % nice spot on the plane
\draw[->,very thick,ncol] (Nbase) -- ++(Ndir)
      node[above right=-1pt] {$\vec n_{\!\perp}$};

\end{tikzpicture}\\
\\
\\
XXX TEXIT\\
\\
\\
  \textit{How do we describe every other point $P\left( x,y,z \right) $ that lies on that plane?} \\
\\
The vector $\vec{P_0P} $ lies on the plane and hence $\vec{P_0P} \perp \vec{n} $ i.e. $\vec{P_0P \cdot \vec{n} } =0$ 
\[
\vec{P_0P} = \langle x-x_0, y-y_0, z-z_0  \rangle 
.\] This implies that the equation of the plane is given by 
\begin{align*}
        \langle x-x_0,y-y_0,z-z_0  \rangle \cdot \langle a,b,c  \rangle =0\\
        a\left( x-x_0 \right) +b\left( y-y_0 \right) +c\left( z-z_0 \right) =0\\
        \text{i.e.  } ax+ by +cz = ax_0 + by_0 +cz_0
.\end{align*}
i.e. every point $ \left( x,y,z \right) $ in the plane containing $ \left( x_0 ,y_0 ,z_0 \right) $ with normal vector $ \vec{ n} = \langle a,b,c  \rangle $ satisfies the equation
\[
ax +by +cz = a x_0 +b y_0 +c z_0
.\] 

\nt{
The coefficients of $ x$, $ y$, $ z$ in the equation are the components of the normal vector $ \langle a,b,c  \rangle $
}




\ex{}{
\textit{Find the equation of the plane that passes through the point $\left( 1,-1,-1 \right) $ and is parallel to the plane $5x-y-z=6$}\\
\\
\textbf{Solution:} $\vec{n} = \langle  5,-1,-1  \rangle $ since the plane is parallel to $5x-y-z=6$.\\
\begin{align*}
        5 \left( x-1 \right) - \left( y+1 \right) - \left( z+1 \right) =0\\
        5x-y -z -5-1-1=0\\
        5x-y-z=7
.\end{align*}
The intersection of two non-parallel planes is a line. That means if we try to solve a system of two equations of the form 
\begin{align*}
        a_1x+b_1y+c_1z=d_1\\
        a_2x\+b_2y+c_2z=d_2
.\end{align*}
where $\langle a_1,b_1,c_1  \rangle $ is not proportional to $\langle a_2,b_2,c_2  \rangle $ then the solution set can be described parametrically as 
\begin{align*}
        x=x_0+at\\
        y=y_0+bt\\
        z=z_0+ct
.\end{align*}
}

\ex{}{
 \textit{Find the parametric equations of the line of intersection of the two planes} 
\begin{align*}
        x+y+z =1 \quad \vec{n_1} = \langle 1,1,1  \rangle \\
        x+2y+2z=1 \quad \vec{n_2} = \langle 1,2,2  \rangle 
.\end{align*}
\textbf{Solution:} The line $L$ lies on plane $1$  this implies the direction vector is perpendicular to $\vec{n_1} = \langle 1,1,1  \rangle $. $L$ also lies on plane $2$ which implies the direction vector is perpendicular to $\vec{n_2} = \langle 1,2,2  \rangle $.\\
Hence $\vec{v} $ can be expressed as $\vec{n_1} \times  \vec{n_2} $ 
\[
\vec{v} = \langle 2-2,1-2,2-1  \rangle = \langle 0,-1,1  \rangle 
.\] We need a point on this line. By observation we know that this line passes through the x-y plane which implies by setting $z-0$ in both plane equations and solving for $x$ and $y$ we will obtain a point on the line 
\begin{align*}
        x+y=1 & \mathrel{\hspace{1cm}} \to-x-y=-1\\
        x+2y=1 \mathrel{\hspace{1cm}} x+2y=1\\
        y=0 \implies x=1
.\end{align*}
$\left( 1,0,0 \right) $ lies on the line\\
Thus the parametric equation of the line of intersection is given by
\begin{align*}
        x=1+0t \qquad x=1\\
        y=0-t \qquad y=-t\\
        z=0+t \qquad z=t
.\end{align*}
}


 

\section{Two planes in $  \mathbb{R} ^3$}
 \begin{itemize}
        \item \textbf{(a)} be parallel and not intersect (e.g. have the same normal vector $\vec{n} = \langle a,b,c  \rangle $ but have a different constant for e.g $ax + by +cz =d_1$ and $ax+by+cz =d_2$
        \item \textbf{(b)}  be parallel and intersect (this occurs when they are the same plane e.g. $x+y+z=1$ and $2x +2y +2z =2$, (notice they only differ by an overall scalar).
        \item \textbf{(c)} intersect in a line
\end{itemize}
Note: Two planes can not intersect in a point (see justification later).\\
\\
Three planes in $\mathbb{R}^3 $ could:
\begin{itemize}
        \textbf{(a)} intersect in a point e.g. $xy$, $yz$ and $xz$ plane intersect at the origin $\left( 0,0,0 \right) $
        \item \textbf{(b)} intersect in a line, e.g. $xy$ plane and the $xz$ plane intersect in the x-axis
        \item intersect in a plane, e.g.  $xy$ plane and the $xy$ plane
        \item (d) not intersect at all, e.g. three parallel, non-intersecting planes $x=0$,  $x=1$, $x=2$
\end{itemize}
\\
\section{Three Planes in $ \mathbb{R} ^3$}
  
 \begin{itemize}
   \item Intersect in a point e.g. $xy$, $yz$ and $xz$ plane intersect at the origin $\left( 0,0,0 \right) $ \\
        \item  intersect in a line, e.g. $xy$ plane and the $xz$ plane intersect in the x-axis  \\
        \item intersect in a plane, e.g.  $xy$ plane and the $xy$ plane \\
        \item not intersect at all, e.g. three parallel, non-intersecting planes $x=0$,  $x=1$, $x=2$
\end{itemize}
\\
Given three linear equations in $\mathbb{R}^3$ e.g
\begin{align*}
        x-5y+4z=-2\\
        2x-7y+3z=1\\
        -2x +y +8z=0\\
.\end{align*}
How do we find the solutions to the system? i.e. the points of intersections of the three planes?\\
We will develope a technique but first we need some definitions.
\section{Linear Systems Definitions}



\dfn{Linear Equation :}{
       A \underline{linear equation}  in $n$ variables $x_1$, $x_2$, $x_3$, $\ldots$, $x_n$ is an equation of the form
\[
a_1x_1+a_2x_2+a_3x_3+\ldots+a_nx_n =b
.\] 
Where we have $a_1$,$a_2$, $a_3$, $\ldots$, $a_n$, $b \in \mathbb{R}$ are constants.

 
}
\dfn{ :}{
     A \underline{system of linear equations } in $n$ variables (also known as a linear system) is a collection of one, or more, linear equations involving $n$ variables $x_1$, $x_2$, $x_3$, $\ldots$, $x_n$.\\ 
        For example \begin{align*}
                x_1 -5x_2 +4x_3 = -2\\
                2x_1 -7x_2 +3x_3 =1\\
                -2x_1+x_2+8x_3 =0
.\end{align*}     
}

\dfn{Solution :}{
 A \underline{solution}  of a linear system in $n$ variables is a list of numbers ( $x_1$, $x_2$, $x_3$, $\ldots$, $x_n$) that satisfy every equation in the system.   
}

\dfn{Solution Set :}{
 The \underline{solution set}  of a linear system is the set of all possible solutions of the system.           
}

 We will introduce a second method of solving systems of equations\\
\newpage
\begin{multicols}{2}
        \textbf{Method 1:} 
  \begin{align*}
        x+3y=0\\
        2x+7y =-1\\
        \Big\downarrow\\
        -2x-6y=0\\
        2x+7y=-1\\
        \implies y=-1 \\
        \\
        x+3y =0\\
        x+3\left( -1 \right) =0\\
        \implies x = 3
.\end{align*}
\break
\textbf{Method 2:} 
  \begin{align*}
\left[
\begin{array}{cc;{2pt/2pt}c}  % solid | then dashed ; after col-2
  1 & 3 & 0\\
  2 & 7 & -1\\
\end{array}
\right]\\
\text{We want to remove 2x below the x. EQN 1 - 2 $\times $(EQN 2)}\\
\text{row 1 - $2 \times $ row 2}\Big\downarrow\\
\left[
\begin{array}{cc;{2pt/2pt}c}  % solid | then dashed ; after col-2
  1 & 3 & 0\\
  0 & 7 & -1\\
\end{array}
\right]\\
\text{$r_2 -3 r_1$}\Big\downarrow\\
\left[
\begin{array}{cc;{2pt/2pt}c}  % solid | then dashed ; after col-2
  1 & 0 & 3\\
  0 & 1 & -1\\
\end{array}
\right]
\end{align*}
\end{multicols}
\textit{ We will shortly give a list of row operations that can be performed that will not alter the solution set. For now, think carefully about each one:}\\
\ex{}{
Solve
\begin{align*}
  x-5y+4z &=-2\\
  2x -7y +3z &=1\\
  -2x +y +8z &=0\\
.\end{align*}
   \[
   \text{Augmented matrix:} \left[
   \begin{array}{ccc;{2pt/2pt}c}  
     1 & -5 & 4& -2\\
     2& -7&3 &1\\
     -2& 1 & 8&0\\
   \end{array}
   \right]
   .\] 
   \textit{ We are just pulling out the coefficients and arranging them in an array but keeping the structure. Goal is to introduce some zeros.}\\
   Want zeros under the $ 1$ in the top left position.
   \[
   \xrightarrow[ r_2- 2 r_1]{} \left[
   \begin{array}{ccc;{2pt/2pt}c}  
     1 & -5 & 4 & -2\\
     0 & 3 & -5 & 5\\
     -2 & 1 & 8 & 0\\
   \end{array}
   \right] \xrightarrow[ r_3 + 2 r_1]{}
   \left[
   \begin{array}{ccc;{2pt/2pt}c}  
     1 & -5 & 4 & -2\\
     0 & 3 & -5 & 5\\
     0 &  -9 & 16 &-4\\
   \end{array}
   \right]
   .\] 
   Nex I want to make the entry under the $ 3 $ a zero

   \[
   \xrightarrow[ r_3- 3 r_2]{}
   \left[
   \begin{array}{ccc;{2pt/2pt}c}  
     1 & -5 & 4 & 2\\
     0 & 3 & -5 & 5\\
     0 & 0 & 1 & 11\\
   \end{array}
   \right]
   .\] 
   We could stop at this point and say 
   \begin{align*}
     x - 5y +4z &= -2\\
     3y -5z =5\\
     z=11
   .\end{align*}
     and solve using back substitution


     \raggedcolumns
     \begin{multicols}{2}
     \begin{align*}
       3y - 5 \left( 11 \right) =5\\
       3y =60\\
       y=20
     .\end{align*}
     
     \break
      \begin{align*}
        x - 5 \left( 20 \right) +4 \left( 11 \right) = -2\\
        x - 56=-2\\
        x = 54
      .\end{align*}
     \end{multicols}
       or we could continue and isolate each variable in the augmented matrix:
       \raggedcolumns
       \begin{multicols}{2}
       



         \[
\left[
\begin{array}{rrr|r}
 1 &
 % --- circled –5 ---
 \tikz[baseline=(c1.base)]{
      \node[inner sep=1pt] (c1){\(-5\)};
      \draw[pink,thick] (c1.center) circle [radius=10pt];
 } &
 % --- circled 4 ---
 \tikz[baseline=(c2.base)]{
      \node[inner sep=1pt] (c2){\(4\)};
      \draw[pink,thick] (c2.center) circle [radius=10pt];
 } & -2 \\[6pt]
 0 & 3 &
 % --- circled –5 ---
 \tikz[baseline=(c3.base)]{
      \node[inner sep=1pt] (c3){\(-5\)};
      \draw[pink,thick] (c3.center) circle [radius=10pt];
 } &  5 \\[6pt]
 0 & 0 &  1 & 11
\end{array}
\right]
\]
       
       \break
       \\
       We want zeros in the circled positions
       \end{multicols}

       \[
       \xrightarrow[ r_3 + 5 r_2]{}
       \left[
       \begin{array}{ccc;{2pt/2pt}c}  
         1 & -5 & 4 & -2\\
         0 & 3 & 0 & 60\\
         0 & 0 & 1 & 11 \\
       \end{array}
       \right]       \xrightarrow[ r_1 - 4 r_3]{}
       \left[
       \begin{array}{ccc;{2pt/2pt}c}  
         1 & -5 & 0 & -46\\
         0 & 3 & 0 & 60\\
         0 & 0 & 1 & 11\\
       \end{array}
       \right]
       .\] 
       \[
       \xrightarrow[ r_1 + \frac{5}{3} r_2]{}
       \left[
       \begin{array}{ccc;{2pt/2pt}c}  
         1 & 0 & 0 & 54\\
         0 & 3 & 0 & 60\\
         0 & 0 & 1 & 11\\
       \end{array}
       \right]      \xrightarrow[ r_2 \times  \frac{1}{3}]{}  \left[
       \begin{array}{ccc;{2pt/2pt}c}  
         1 & 0 & 0 & 54\\
         0 & 1 & 0 & 20\\
         0 & 0 & 1 & 11\\
       \end{array}
       \right]
       .\] 
       Going back to the system of equations we get
       \begin{align*}
         x&=54\\
         y&=20\\
         z&=11
       .\end{align*}
       
}

  \section{Row Operations}
    

 \dfn{Elementary Row Operations :}{
  Row operations that result in an equivalent system of linear equations ( i.e.  a system that has the same solution set as the original) are called \underline{ elementary row operations}   
 }
 
  There are only three types of elementary row operations:
  \begin{enumerate}[label=(\arabic*).]  
    \item \textbf{Replacement}: Replace on row by the sum of itself and a multiple of another row\\
      e.g 
      \[
      r_2 \to r_2+5 r_3 \qquad  \text{ or } \qquad  r_3 \to r_3 - 2 r_1
      .\] 
    \item   \textbf{Scaling}: Multiplying all entries in a row by a non-zero constant\\
       e.g. 
       \[
       r_2 \to 4 r_2
       .\] 
     \item \textbf{Interchanging}: Interchanging two rows:\\
       e.g. 
       \[
         r_1 \leftrightarrow      r_2
       .\] 
  \end{enumerate}
  \section{ The row Reduction Algorithm}
  A nice algorithm exist for row reducing an augmented matrix to its simplest form ( for solving)- it is called the row reduciton algorithm.\\

  We wish to develop a method for solving systems of $m$ simultaneous linear
equations with $n$ unknowns
\[
\begin{aligned}
a_{11}x_{1}+a_{12}x_{2}+\dots +a_{1n}x_{n} &= b_{1}\\
a_{21}x_{1}+a_{22}x_{2}+\dots +a_{2n}x_{n} &= b_{2}\\
&\ \;\vdots \\[-2pt]
a_{m1}x_{1}+a_{m2}x_{2}+\dots +a_{mn}x_{n} &= b_{m}
\end{aligned}
\]
where $a_{ij}$ and $b_i$ are known constants $(1\le i\le m,\;1\le j\le n)$.  \\
 \\
The most common technique for solving simultaneous systems of linear
equations is \emph{Gauss–Jordan elimination}. This method, which involves
carefully eliminating one variable after another, is a typical procedure
used to solve systems of linear equations. We will formulate this recipe in
the form of an algorithm, that is, a sequence of instructions that a person
(or a computer) can perform mechanically, ending up with a solution to the
given system. For convenience, we will not carry around the symbols
representing the unknowns, and will encode the given system of linear
equations by an $m\times(n+1)$ augmented matrix.   \\
\dfn{Matrices :}{
 A matrix is a rectangular array of numbers arranged in rows and columns.\\
 An $ n \times  m$ array has $ n$ rows and $ m$ columns.
}
\dfn{Leading Entry :}{
A \underline{leading entry} of a row of a matrix is the leftmost non-zero entry in a non-zero row
}

\dfn{Echelon Form :}{
 A rectangular matrix is in \textbf{ echelon form } if it has the following properties:\\
 \begin{enumerate}[label=(\arabic*).]  
   \item All nonzero rows are above any rows of all zeros
   \item Each leading entry of a row is in a column to the right of the leading entry of the row above it.
   \item All entries in a column below a leadin entry are zeros
 \end{enumerate}
 
}
 \dfn{ Reduced Echelon Form :}{
    If a matrix is in echelon form and satisfies the following additional conditions, then it
is in \textbf{ reduced echelon form} 
\begin{enumerate}[label=(\arabic*).]  
  \item The leading entry in each nonzero row is $ 1$
  \item  Each leading $ 1$ is the only nonzero in its column
\end{enumerate}

 }
 

 \section{ The Row Reduction Algorithm}

 \begin{itemize}
\item To obtain an \emph{echelon form} of a matrix:

  \begin{steplist}
    \item Begin with the leftmost non-zero column.  
          This is a pivot column.  
          The pivot position is always at the top.

    \item Select a non-zero entry in the pivot column as a pivot.  
          If necessary, interchange rows to move this entry into the
          pivot position.

    \item Use row-replacement operations to create zeros in all positions
          below the pivot (use the pivot as your means of “clearing out”
          the column).

    \item Cover the row containing the pivot position and cover all rows,
          if any, above it.  
          Apply steps 1–3 to the sub-matrix that remains.  
          Repeat the process until there are no more non-zero rows to modify.
  \end{steplist}

\item To continue to the \emph{reduced echelon form} of the original matrix:

  \begin{steplist}[resume] % continue numbering: 5, 6, …
    \item Beginning with the rightmost pivot and working upward and to
          the left, create zeros above each pivot.

    \item If a pivot is not 1, make it 1 by a scaling operation.
  \end{steplist}
\end{itemize}  





    
  


























\end{document}       
