\documentclass{report}

%%%%%%%%%%%%%%%%%%%%%%%%%%%%%%%%%
% PACKAGE IMPORTS
%%%%%%%%%%%%%%%%%%%%%%%%%%%%%%%%%


\usepackage[tmargin=2cm,rmargin=1in,lmargin=1in,margin=0.85in,bmargin=2cm,footskip=.2in]{geometry}
\usepackage{amsmath,amsfonts,amsthm,amssymb,mathtools}
\usepackage[varbb]{newpxmath}
\usepackage{xfrac}
\usepackage[makeroom]{cancel}
\usepackage{bookmark}
\usepackage{enumitem}
\usepackage{hyperref,theoremref}
\hypersetup{
	pdftitle={Assignment},
	colorlinks=true, linkcolor=doc!90,
	bookmarksnumbered=true,
	bookmarksopen=true
}
\usepackage[most,many,breakable]{tcolorbox}
\usepackage{xcolor}
\usepackage{varwidth}
\usepackage{varwidth}
\usepackage{tocloft}
\usepackage{etoolbox}
\usepackage{derivative} %many derivativess partials
%\usepackage{authblk}
\usepackage{nameref}
\usepackage{multicol,array}
\usepackage{tikz-cd}
\usepackage[ruled,vlined,linesnumbered]{algorithm2e}
\usepackage{comment} % enables the use of multi-line comments (\ifx \fi) 
\usepackage{import}
\usepackage{xifthen}
\usepackage{pdfpages}
\usepackage{transparent}
\usepackage{verbatim}

\newcommand\mycommfont[1]{\footnotesize\ttfamily\textcolor{blue}{#1}}
\SetCommentSty{mycommfont}
\newcommand{\incfig}[1]{%
    \def\svgwidth{\columnwidth}
    \import{./figures/}{#1.pdf_tex}
}
\usepackage[tagged, highstructure]{accessibility}
\usepackage{tikzsymbols}
\renewcommand\qedsymbol{$\Laughey$}


%\usepackage{import}
%\usepackage{xifthen}
%\usepackage{pdfpages}
%\usepackage{transparent}


%%%%%%%%%%%%%%%%%%%%%%%%%%%%%%
% SELF MADE COLORS
%%%%%%%%%%%%%%%%%%%%%%%%%%%%%%



\definecolor{myg}{RGB}{56, 140, 70}
\definecolor{myb}{RGB}{45, 111, 177}
\definecolor{myr}{RGB}{199, 68, 64}
\definecolor{mytheorembg}{HTML}{F2F2F9}
\definecolor{mytheoremfr}{HTML}{00007B}
\definecolor{mylenmabg}{HTML}{FFFAF8}
\definecolor{mylenmafr}{HTML}{983b0f}
\definecolor{mypropbg}{HTML}{f2fbfc}
\definecolor{mypropfr}{HTML}{191971}
\definecolor{myexamplebg}{HTML}{F2FBF8}
\definecolor{myexamplefr}{HTML}{88D6D1}
\definecolor{myexampleti}{HTML}{2A7F7F}
\definecolor{mydefinitbg}{HTML}{E5E5FF}
\definecolor{mydefinitfr}{HTML}{3F3FA3}
\definecolor{notesgreen}{RGB}{0,162,0}
\definecolor{myp}{RGB}{197, 92, 212}
\definecolor{mygr}{HTML}{2C3338}
\definecolor{myred}{RGB}{127,0,0}
\definecolor{myyellow}{RGB}{169,121,69}
\definecolor{myexercisebg}{HTML}{F2FBF8}
\definecolor{myexercisefg}{HTML}{88D6D1}


%%%%%%%%%%%%%%%%%%%%%%%%%%%%
% TCOLORBOX SETUPS
%%%%%%%%%%%%%%%%%%%%%%%%%%%%

\setlength{\parindent}{1cm}
%================================
% THEOREM BOX
%================================

\tcbuselibrary{theorems,skins,hooks}
\newtcbtheorem[number within=section]{Theorem}{Theorem}
{%
	enhanced,
	breakable,
	colback = mytheorembg,
	frame hidden,
	boxrule = 0sp,
	borderline west = {2pt}{0pt}{mytheoremfr},
	sharp corners,
	detach title,
	before upper = \tcbtitle\par\smallskip,
	coltitle = mytheoremfr,
	fonttitle = \bfseries\sffamily,
	description font = \mdseries,
	separator sign none,
	segmentation style={solid, mytheoremfr},
}
{th}

\tcbuselibrary{theorems,skins,hooks}
\newtcbtheorem[number within=chapter]{theorem}{Theorem}
{%
	enhanced,
	breakable,
	colback = mytheorembg,
	frame hidden,
	boxrule = 0sp,
	borderline west = {2pt}{0pt}{mytheoremfr},
	sharp corners,
	detach title,
	before upper = \tcbtitle\par\smallskip,
	coltitle = mytheoremfr,
	fonttitle = \bfseries\sffamily,
	description font = \mdseries,
	separator sign none,
	segmentation style={solid, mytheoremfr},
}
{th}


\tcbuselibrary{theorems,skins,hooks}
\newtcolorbox{Theoremcon}
{%
	enhanced
	,breakable
	,colback = mytheorembg
	,frame hidden
	,boxrule = 0sp
	,borderline west = {2pt}{0pt}{mytheoremfr}
	,sharp corners
	,description font = \mdseries
	,separator sign none
}

%================================
% Corollery
%================================
\tcbuselibrary{theorems,skins,hooks}
\newtcbtheorem[number within=section]{Corollary}{Corollary}
{%
	enhanced
	,breakable
	,colback = myp!10
	,frame hidden
	,boxrule = 0sp
	,borderline west = {2pt}{0pt}{myp!85!black}
	,sharp corners
	,detach title
	,before upper = \tcbtitle\par\smallskip
	,coltitle = myp!85!black
	,fonttitle = \bfseries\sffamily
	,description font = \mdseries
	,separator sign none
	,segmentation style={solid, myp!85!black}
}
{th}
\tcbuselibrary{theorems,skins,hooks}
\newtcbtheorem[number within=chapter]{corollary}{Corollary}
{%
	enhanced
	,breakable
	,colback = myp!10
	,frame hidden
	,boxrule = 0sp
	,borderline west = {2pt}{0pt}{myp!85!black}
	,sharp corners
	,detach title
	,before upper = \tcbtitle\par\smallskip
	,coltitle = myp!85!black
	,fonttitle = \bfseries\sffamily
	,description font = \mdseries
	,separator sign none
	,segmentation style={solid, myp!85!black}
}
{th}


%================================
% LENMA
%================================

\tcbuselibrary{theorems,skins,hooks}
\newtcbtheorem[number within=section]{Lenma}{Lenma}
{%
	enhanced,
	breakable,
	colback = mylenmabg,
	frame hidden,
	boxrule = 0sp,
	borderline west = {2pt}{0pt}{mylenmafr},
	sharp corners,
	detach title,
	before upper = \tcbtitle\par\smallskip,
	coltitle = mylenmafr,
	fonttitle = \bfseries\sffamily,
	description font = \mdseries,
	separator sign none,
	segmentation style={solid, mylenmafr},
}
{th}

\tcbuselibrary{theorems,skins,hooks}
\newtcbtheorem[number within=chapter]{lenma}{Lenma}
{%
	enhanced,
	breakable,
	colback = mylenmabg,
	frame hidden,
	boxrule = 0sp,
	borderline west = {2pt}{0pt}{mylenmafr},
	sharp corners,
	detach title,
	before upper = \tcbtitle\par\smallskip,
	coltitle = mylenmafr,
	fonttitle = \bfseries\sffamily,
	description font = \mdseries,
	separator sign none,
	segmentation style={solid, mylenmafr},
}
{th}


%================================
% PROPOSITION
%================================

\tcbuselibrary{theorems,skins,hooks}
\newtcbtheorem[number within=section]{Prop}{Proposition}
{%
	enhanced,
	breakable,
	colback = mypropbg,
	frame hidden,
	boxrule = 0sp,
	borderline west = {2pt}{0pt}{mypropfr},
	sharp corners,
	detach title,
	before upper = \tcbtitle\par\smallskip,
	coltitle = mypropfr,
	fonttitle = \bfseries\sffamily,
	description font = \mdseries,
	separator sign none,
	segmentation style={solid, mypropfr},
}
{th}

\tcbuselibrary{theorems,skins,hooks}
\newtcbtheorem[number within=chapter]{prop}{Proposition}
{%
	enhanced,
	breakable,
	colback = mypropbg,
	frame hidden,
	boxrule = 0sp,
	borderline west = {2pt}{0pt}{mypropfr},
	sharp corners,
	detach title,
	before upper = \tcbtitle\par\smallskip,
	coltitle = mypropfr,
	fonttitle = \bfseries\sffamily,
	description font = \mdseries,
	separator sign none,
	segmentation style={solid, mypropfr},
}
{th}


%================================
% CLAIM
%================================

\tcbuselibrary{theorems,skins,hooks}
\newtcbtheorem[number within=section]{claim}{Claim}
{%
	enhanced
	,breakable
	,colback = myg!10
	,frame hidden
	,boxrule = 0sp
	,borderline west = {2pt}{0pt}{myg}
	,sharp corners
	,detach title
	,before upper = \tcbtitle\par\smallskip
	,coltitle = myg!85!black
	,fonttitle = \bfseries\sffamily
	,description font = \mdseries
	,separator sign none
	,segmentation style={solid, myg!85!black}
}
{th}



%================================
% Exercise
%================================

\tcbuselibrary{theorems,skins,hooks}
\newtcbtheorem[number within=section]{Exercise}{Exercise}
{%
	enhanced,
	breakable,
	colback = myexercisebg,
	frame hidden,
	boxrule = 0sp,
	borderline west = {2pt}{0pt}{myexercisefg},
	sharp corners,
	detach title,
	before upper = \tcbtitle\par\smallskip,
	coltitle = myexercisefg,
	fonttitle = \bfseries\sffamily,
	description font = \mdseries,
	separator sign none,
	segmentation style={solid, myexercisefg},
}
{th}

\tcbuselibrary{theorems,skins,hooks}
\newtcbtheorem[number within=chapter]{exercise}{Exercise}
{%
	enhanced,
	breakable,
	colback = myexercisebg,
	frame hidden,
	boxrule = 0sp,
	borderline west = {2pt}{0pt}{myexercisefg},
	sharp corners,
	detach title,
	before upper = \tcbtitle\par\smallskip,
	coltitle = myexercisefg,
	fonttitle = \bfseries\sffamily,
	description font = \mdseries,
	separator sign none,
	segmentation style={solid, myexercisefg},
}
{th}

%================================
% EXAMPLE BOX
%================================

\newtcbtheorem[number within=section]{Example}{Example}
{%
	colback = myexamplebg
	,breakable
	,colframe = myexamplefr
	,coltitle = myexampleti
	,boxrule = 1pt
	,sharp corners
	,detach title
	,before upper=\tcbtitle\par\smallskip
	,fonttitle = \bfseries
	,description font = \mdseries
	,separator sign none
	,description delimiters parenthesis
}
{ex}

\newtcbtheorem[number within=chapter]{example}{Example}
{%
	colback = myexamplebg
	,breakable
	,colframe = myexamplefr
	,coltitle = myexampleti
	,boxrule = 1pt
	,sharp corners
	,detach title
	,before upper=\tcbtitle\par\smallskip
	,fonttitle = \bfseries
	,description font = \mdseries
	,separator sign none
	,description delimiters parenthesis
}
{ex}

%================================
% DEFINITION BOX
%================================

\newtcbtheorem[number within=section]{Definition}{Definition}{enhanced,
	before skip=2mm,after skip=2mm, colback=red!5,colframe=red!80!black,boxrule=0.5mm,
	attach boxed title to top left={xshift=1cm,yshift*=1mm-\tcboxedtitleheight}, varwidth boxed title*=-3cm,
	boxed title style={frame code={
					\path[fill=tcbcolback]
					([yshift=-1mm,xshift=-1mm]frame.north west)
					arc[start angle=0,end angle=180,radius=1mm]
					([yshift=-1mm,xshift=1mm]frame.north east)
					arc[start angle=180,end angle=0,radius=1mm];
					\path[left color=tcbcolback!60!black,right color=tcbcolback!60!black,
						middle color=tcbcolback!80!black]
					([xshift=-2mm]frame.north west) -- ([xshift=2mm]frame.north east)
					[rounded corners=1mm]-- ([xshift=1mm,yshift=-1mm]frame.north east)
					-- (frame.south east) -- (frame.south west)
					-- ([xshift=-1mm,yshift=-1mm]frame.north west)
					[sharp corners]-- cycle;
				},interior engine=empty,
		},
	fonttitle=\bfseries,
	title={#2},#1}{def}
\newtcbtheorem[number within=chapter]{definition}{Definition}{enhanced,
	before skip=2mm,after skip=2mm, colback=red!5,colframe=red!80!black,boxrule=0.5mm,
	attach boxed title to top left={xshift=1cm,yshift*=1mm-\tcboxedtitleheight}, varwidth boxed title*=-3cm,
	boxed title style={frame code={
					\path[fill=tcbcolback]
					([yshift=-1mm,xshift=-1mm]frame.north west)
					arc[start angle=0,end angle=180,radius=1mm]
					([yshift=-1mm,xshift=1mm]frame.north east)
					arc[start angle=180,end angle=0,radius=1mm];
					\path[left color=tcbcolback!60!black,right color=tcbcolback!60!black,
						middle color=tcbcolback!80!black]
					([xshift=-2mm]frame.north west) -- ([xshift=2mm]frame.north east)
					[rounded corners=1mm]-- ([xshift=1mm,yshift=-1mm]frame.north east)
					-- (frame.south east) -- (frame.south west)
					-- ([xshift=-1mm,yshift=-1mm]frame.north west)
					[sharp corners]-- cycle;
				},interior engine=empty,
		},
	fonttitle=\bfseries,
	title={#2},#1}{def}



%================================
% Solution BOX
%================================

\makeatletter
\newtcbtheorem{question}{Question}{enhanced,
	breakable,
	colback=white,
	colframe=myb!80!black,
	attach boxed title to top left={yshift*=-\tcboxedtitleheight},
	fonttitle=\bfseries,
	title={#2},
	boxed title size=title,
	boxed title style={%
			sharp corners,
			rounded corners=northwest,
			colback=tcbcolframe,
			boxrule=0pt,
		},
	underlay boxed title={%
			\path[fill=tcbcolframe] (title.south west)--(title.south east)
			to[out=0, in=180] ([xshift=5mm]title.east)--
			(title.center-|frame.east)
			[rounded corners=\kvtcb@arc] |-
			(frame.north) -| cycle;
		},
	#1
}{def}
\makeatother

%================================
% SOLUTION BOX
%================================

\makeatletter
\newtcolorbox{solution}{enhanced,
	breakable,
	colback=white,
	colframe=myg!80!black,
	attach boxed title to top left={yshift*=-\tcboxedtitleheight},
	title=Solution,
	boxed title size=title,
	boxed title style={%
			sharp corners,
			rounded corners=northwest,
			colback=tcbcolframe,
			boxrule=0pt,
		},
	underlay boxed title={%
			\path[fill=tcbcolframe] (title.south west)--(title.south east)
			to[out=0, in=180] ([xshift=5mm]title.east)--
			(title.center-|frame.east)
			[rounded corners=\kvtcb@arc] |-
			(frame.north) -| cycle;
		},
}
\makeatother

%================================
% Question BOX
%================================

\makeatletter
\newtcbtheorem{qstion}{Question}{enhanced,
	breakable,
	colback=white,
	colframe=mygr,
	attach boxed title to top left={yshift*=-\tcboxedtitleheight},
	fonttitle=\bfseries,
	title={#2},
	boxed title size=title,
	boxed title style={%
			sharp corners,
			rounded corners=northwest,
			colback=tcbcolframe,
			boxrule=0pt,
		},
	underlay boxed title={%
			\path[fill=tcbcolframe] (title.south west)--(title.south east)
			to[out=0, in=180] ([xshift=5mm]title.east)--
			(title.center-|frame.east)
			[rounded corners=\kvtcb@arc] |-
			(frame.north) -| cycle;
		},
	#1
}{def}
\makeatother

\newtcbtheorem[number within=chapter]{wconc}{Wrong Concept}{
	breakable,
	enhanced,
	colback=white,
	colframe=myr,
	arc=0pt,
	outer arc=0pt,
	fonttitle=\bfseries\sffamily\large,
	colbacktitle=myr,
	attach boxed title to top left={},
	boxed title style={
			enhanced,
			skin=enhancedfirst jigsaw,
			arc=3pt,
			bottom=0pt,
			interior style={fill=myr}
		},
	#1
}{def}



%================================
% NOTE BOX
%================================

\usetikzlibrary{arrows,calc,shadows.blur}
\tcbuselibrary{skins}
\newtcolorbox{note}[1][]{%
	enhanced jigsaw,
	colback=gray!20!white,%
	colframe=gray!80!black,
	size=small,
	boxrule=1pt,
	title=\textbf{Note:-},
	halign title=flush center,
	coltitle=black,
	breakable,
	drop shadow=black!50!white,
	attach boxed title to top left={xshift=1cm,yshift=-\tcboxedtitleheight/2,yshifttext=-\tcboxedtitleheight/2},
	minipage boxed title=1.5cm,
	boxed title style={%
			colback=white,
			size=fbox,
			boxrule=1pt,
			boxsep=2pt,
			underlay={%
					\coordinate (dotA) at ($(interior.west) + (-0.5pt,0)$);
					\coordinate (dotB) at ($(interior.east) + (0.5pt,0)$);
					\begin{scope}
						\clip (interior.north west) rectangle ([xshift=3ex]interior.east);
						\filldraw [white, blur shadow={shadow opacity=60, shadow yshift=-.75ex}, rounded corners=2pt] (interior.north west) rectangle (interior.south east);
					\end{scope}
					\begin{scope}[gray!80!black]
						\fill (dotA) circle (2pt);
						\fill (dotB) circle (2pt);
					\end{scope}
				},
		},
	#1,
}

%%%%%%%%%%%%%%%%%%%%%%%%%%%%%%
% SELF MADE COMMANDS
%%%%%%%%%%%%%%%%%%%%%%%%%%%%%%


\newcommand{\thm}[2]{\begin{Theorem}{#1}{}#2\end{Theorem}}
\newcommand{\cor}[2]{\begin{Corollary}{#1}{}#2\end{Corollary}}
\newcommand{\mlenma}[2]{\begin{Lenma}{#1}{}#2\end{Lenma}}
\newcommand{\mprop}[2]{\begin{Prop}{#1}{}#2\end{Prop}}
\newcommand{\clm}[3]{\begin{claim}{#1}{#2}#3\end{claim}}
\newcommand{\wc}[2]{\begin{wconc}{#1}{}\setlength{\parindent}{1cm}#2\end{wconc}}
\newcommand{\thmcon}[1]{\begin{Theoremcon}{#1}\end{Theoremcon}}
\newcommand{\ex}[2]{\begin{Example}{#1}{}#2\end{Example}}
\newcommand{\dfn}[2]{\begin{Definition}[colbacktitle=red!75!black]{#1}{}#2\end{Definition}}
\newcommand{\dfnc}[2]{\begin{definition}[colbacktitle=red!75!black]{#1}{}#2\end{definition}}
\newcommand{\qs}[2]{\begin{question}{#1}{}#2\end{question}}
\newcommand{\pf}[2]{\begin{myproof}[#1]#2\end{myproof}}
\newcommand{\nt}[1]{\begin{note}#1\end{note}}

\newcommand*\circled[1]{\tikz[baseline=(char.base)]{
		\node[shape=circle,draw,inner sep=1pt] (char) {#1};}}
\newcommand\getcurrentref[1]{%
	\ifnumequal{\value{#1}}{0}
	{??}
	{\the\value{#1}}%
}
\newcommand{\getCurrentSectionNumber}{\getcurrentref{section}}
\newenvironment{myproof}[1][\proofname]{%
	\proof[\bfseries #1: ]%
}{\endproof}

\newcommand{\mclm}[2]{\begin{myclaim}[#1]#2\end{myclaim}}
\newenvironment{myclaim}[1][\claimname]{\proof[\bfseries #1: ]}{}

\newcounter{mylabelcounter}

\makeatletter
\newcommand{\setword}[2]{%
	\phantomsection
	#1\def\@currentlabel{\unexpanded{#1}}\label{#2}%
}
\makeatother




\tikzset{
	symbol/.style={
			draw=none,
			every to/.append style={
					edge node={node [sloped, allow upside down, auto=false]{$#1$}}}
		}
}


% deliminators
\DeclarePairedDelimiter{\abs}{\lvert}{\rvert}
\DeclarePairedDelimiter{\norm}{\lVert}{\rVert}

\DeclarePairedDelimiter{\ceil}{\lceil}{\rceil}
\DeclarePairedDelimiter{\floor}{\lfloor}{\rfloor}
\DeclarePairedDelimiter{\round}{\lfloor}{\rceil}

\newsavebox\diffdbox
\newcommand{\slantedromand}{{\mathpalette\makesl{d}}}
\newcommand{\makesl}[2]{%
\begingroup
\sbox{\diffdbox}{$\mathsurround=0pt#1\mathrm{#2}$}%
\pdfsave
\pdfsetmatrix{1 0 0.2 1}%
\rlap{\usebox{\diffdbox}}%
\pdfrestore
\hskip\wd\diffdbox
\endgroup
}
\newcommand{\dd}[1][]{\ensuremath{\mathop{}\!\ifstrempty{#1}{%
\slantedromand\@ifnextchar^{\hspace{0.2ex}}{\hspace{0.1ex}}}%
{\slantedromand\hspace{0.2ex}^{#1}}}}
\ProvideDocumentCommand\dv{o m g}{%
  \ensuremath{%
    \IfValueTF{#3}{%
      \IfNoValueTF{#1}{%
        \frac{\dd #2}{\dd #3}%
      }{%
        \frac{\dd^{#1} #2}{\dd #3^{#1}}%
      }%
    }{%
      \IfNoValueTF{#1}{%
        \frac{\dd}{\dd #2}%
      }{%
        \frac{\dd^{#1}}{\dd #2^{#1}}%
      }%
    }%
  }%
}
\providecommand*{\pdv}[3][]{\frac{\partial^{#1}#2}{\partial#3^{#1}}}
%  - others
\DeclareMathOperator{\Lap}{\mathcal{L}}
\DeclareMathOperator{\Var}{Var} % varience
\DeclareMathOperator{\Cov}{Cov} % covarience
\DeclareMathOperator{\E}{E} % expected

% Since the amsthm package isn't loaded

% I prefer the slanted \leq
\let\oldleq\leq % save them in case they're every wanted
\let\oldgeq\geq
\renewcommand{\leq}{\leqslant}
\renewcommand{\geq}{\geqslant}

% % redefine matrix env to allow for alignment, use r as default
% \renewcommand*\env@matrix[1][r]{\hskip -\arraycolsep
%     \let\@ifnextchar\new@ifnextchar
%     \array{*\c@MaxMatrixCols #1}}


%\usepackage{framed}
%\usepackage{titletoc}
%\usepackage{etoolbox}
%\usepackage{lmodern}


%\patchcmd{\tableofcontents}{\contentsname}{\sffamily\contentsname}{}{}

%\renewenvironment{leftbar}
%{\def\FrameCommand{\hspace{6em}%
%		{\color{myyellow}\vrule width 2pt depth 6pt}\hspace{1em}}%
%	\MakeFramed{\parshape 1 0cm \dimexpr\textwidth-6em\relax\FrameRestore}\vskip2pt%
%}
%{\endMakeFramed}

%\titlecontents{chapter}
%[0em]{\vspace*{2\baselineskip}}
%{\parbox{4.5em}{%
%		\hfill\Huge\sffamily\bfseries\color{myred}\thecontentspage}%
%	\vspace*{-2.3\baselineskip}\leftbar\textsc{\small\chaptername~\thecontentslabel}\\\sffamily}
%{}{\endleftbar}
%\titlecontents{section}
%[8.4em]
%{\sffamily\contentslabel{3em}}{}{}
%{\hspace{0.5em}\nobreak\itshape\color{myred}\contentspage}
%\titlecontents{subsection}
%[8.4em]
%{\sffamily\contentslabel{3em}}{}{}  
%{\hspace{0.5em}\nobreak\itshape\color{myred}\contentspage}



%%%%%%%%%%%%%%%%%%%%%%%%%%%%%%%%%%%%%%%%%%%
% TABLE OF CONTENTS
%%%%%%%%%%%%%%%%%%%%%%%%%%%%%%%%%%%%%%%%%%%

\usepackage{tikz}
\definecolor{doc}{RGB}{0,60,110}
\usepackage{titletoc}
\contentsmargin{0cm}
\titlecontents{chapter}[3.7pc]
{\addvspace{30pt}%
	\begin{tikzpicture}[remember picture, overlay]%
		\draw[fill=doc!60,draw=doc!60] (-7,-.1) rectangle (-0.9,.5);%
		\pgftext[left,x=-3.5cm,y=0.2cm]{\color{white}\Large\sc\bfseries Chapter\ \thecontentslabel};%
	\end{tikzpicture}\color{doc!60}\large\sc\bfseries}%
{}
{}
{\;\titlerule\;\large\sc\bfseries Page \thecontentspage
	\begin{tikzpicture}[remember picture, overlay]
		\draw[fill=doc!60,draw=doc!60] (2pt,0) rectangle (4,0.1pt);
	\end{tikzpicture}}%
\titlecontents{section}[3.7pc]
{\addvspace{2pt}}
{\contentslabel[\thecontentslabel]{2pc}}
{}
{\hfill\small \thecontentspage}
[]
\titlecontents*{subsection}[3.7pc]
{\addvspace{-1pt}\small}
{}
{}
{\ --- \small\thecontentspage}
[ \textbullet\ ][]

\makeatletter
\renewcommand{\tableofcontents}{%
	\chapter*{%
	  \vspace*{-20\p@}%
	  \begin{tikzpicture}[remember picture, overlay]%
		  \pgftext[right,x=15cm,y=0.2cm]{\color{doc!60}\Huge\sc\bfseries \contentsname};%
		  \draw[fill=doc!60,draw=doc!60] (13,-.75) rectangle (20,1);%
		  \clip (13,-.75) rectangle (20,1);
		  \pgftext[right,x=15cm,y=0.2cm]{\color{white}\Huge\sc\bfseries \contentsname};%
	  \end{tikzpicture}}%
	\@starttoc{toc}}
\makeatother


%From M275 "Topology" at SJSU
\newcommand{\id}{\mathrm{id}}
\newcommand{\taking}[1]{\xrightarrow{#1}}
\newcommand{\inv}{^{-1}}

%From M170 "Introduction to Graph Theory" at SJSU
\DeclareMathOperator{\diam}{diam}
\DeclareMathOperator{\ord}{ord}
\newcommand{\defeq}{\overset{\mathrm{def}}{=}}

%From the USAMO .tex files
\newcommand{\ts}{\textsuperscript}
\newcommand{\dg}{^\circ}
\newcommand{\ii}{\item}

% % From Math 55 and Math 145 at Harvard
% \newenvironment{subproof}[1][Proof]{%
% \begin{proof}[#1] \renewcommand{\qedsymbol}{$\blacksquare$}}%
% {\end{proof}}

\newcommand{\liff}{\leftrightarrow}
\newcommand{\lthen}{\rightarrow}
\newcommand{\opname}{\operatorname}
\newcommand{\surjto}{\twoheadrightarrow}
\newcommand{\injto}{\hookrightarrow}
\newcommand{\On}{\mathrm{On}} % ordinals
\DeclareMathOperator{\img}{im} % Image
\DeclareMathOperator{\Img}{Im} % Image
\DeclareMathOperator{\coker}{coker} % Cokernel
\DeclareMathOperator{\Coker}{Coker} % Cokernel
\DeclareMathOperator{\Ker}{Ker} % Kernel
\DeclareMathOperator{\rank}{rank}
\DeclareMathOperator{\Spec}{Spec} % spectrum
\DeclareMathOperator{\Tr}{Tr} % trace
\DeclareMathOperator{\pr}{pr} % projection
\DeclareMathOperator{\ext}{ext} % extension
\DeclareMathOperator{\pred}{pred} % predecessor
\DeclareMathOperator{\dom}{dom} % domain
\DeclareMathOperator{\ran}{ran} % range
\DeclareMathOperator{\Hom}{Hom} % homomorphism
\DeclareMathOperator{\Mor}{Mor} % morphisms
\DeclareMathOperator{\End}{End} % endomorphism

\newcommand{\eps}{\epsilon}
\newcommand{\veps}{\varepsilon}
\newcommand{\ol}{\overline}
\newcommand{\ul}{\underline}
\newcommand{\wt}{\widetilde}
\newcommand{\wh}{\widehat}
\newcommand{\vocab}[1]{\textbf{\color{blue} #1}}
\providecommand{\half}{\frac{1}{2}}
\newcommand{\dang}{\measuredangle} %% Directed angle
\newcommand{\ray}[1]{\overrightarrow{#1}}
\newcommand{\seg}[1]{\overline{#1}}
\newcommand{\arc}[1]{\wideparen{#1}}
\DeclareMathOperator{\cis}{cis}
\DeclareMathOperator*{\lcm}{lcm}
\DeclareMathOperator*{\argmin}{arg min}
\DeclareMathOperator*{\argmax}{arg max}
\newcommand{\cycsum}{\sum_{\mathrm{cyc}}}
\newcommand{\symsum}{\sum_{\mathrm{sym}}}
\newcommand{\cycprod}{\prod_{\mathrm{cyc}}}
\newcommand{\symprod}{\prod_{\mathrm{sym}}}
\newcommand{\Qed}{\begin{flushright}\qed\end{flushright}}
\newcommand{\parinn}{\setlength{\parindent}{1cm}}
\newcommand{\parinf}{\setlength{\parindent}{0cm}}
% \newcommand{\norm}{\|\cdot\|}
\newcommand{\inorm}{\norm_{\infty}}
\newcommand{\opensets}{\{V_{\alpha}\}_{\alpha\in I}}
\newcommand{\oset}{V_{\alpha}}
\newcommand{\opset}[1]{V_{\alpha_{#1}}}
\newcommand{\lub}{\text{lub}}
\newcommand{\del}[2]{\frac{\partial #1}{\partial #2}}
\newcommand{\Del}[3]{\frac{\partial^{#1} #2}{\partial^{#1} #3}}
\newcommand{\deld}[2]{\dfrac{\partial #1}{\partial #2}}
\newcommand{\Deld}[3]{\dfrac{\partial^{#1} #2}{\partial^{#1} #3}}
\newcommand{\lm}{\lambda}
\newcommand{\uin}{\mathbin{\rotatebox[origin=c]{90}{$\in$}}}
\newcommand{\usubset}{\mathbin{\rotatebox[origin=c]{90}{$\subset$}}}
\newcommand{\lt}{\left}
\newcommand{\rt}{\right}
\newcommand{\bs}[1]{\boldsymbol{#1}}
\newcommand{\exs}{\exists}
\newcommand{\st}{\strut}
\newcommand{\dps}[1]{\displaystyle{#1}}

\newcommand{\sol}{\setlength{\parindent}{0cm}\textbf{\textit{Solution:}}\setlength{\parindent}{1cm} }
\newcommand{\solve}[1]{\setlength{\parindent}{0cm}\textbf{\textit{Solution: }}\setlength{\parindent}{1cm}#1 \Qed}

%--------------------------------------------------
% LIE ALGEBRAS
%--------------------------------------------------
\newcommand*{\kb}{\mathfrak{b}}  % Borel subalgebra
\newcommand*{\kg}{\mathfrak{g}}  % Lie algebra
\newcommand*{\kh}{\mathfrak{h}}  % Cartan subalgebra
\newcommand*{\kn}{\mathfrak{n}}  % Nilradical
\newcommand*{\ku}{\mathfrak{u}}  % Unipotent algebra
\newcommand*{\kz}{\mathfrak{z}}  % Center of algebra

%--------------------------------------------------
% HOMOLOGICAL ALGEBRA
%--------------------------------------------------
\DeclareMathOperator{\Ext}{Ext} % Ext functor
\DeclareMathOperator{\Tor}{Tor} % Tor functor

%--------------------------------------------------
% MATRIX & GROUP NOTATION
%--------------------------------------------------
\DeclareMathOperator{\GL}{GL} % General Linear Group
\DeclareMathOperator{\SL}{SL} % Special Linear Group
\newcommand*{\gl}{\operatorname{\mathfrak{gl}}} % General linear Lie algebra
\newcommand*{\sl}{\operatorname{\mathfrak{sl}}} % Special linear Lie algebra

%--------------------------------------------------
% NUMBER SETS
%--------------------------------------------------
\newcommand*{\RR}{\mathbb{R}}
\newcommand*{\NN}{\mathbb{N}}
\newcommand*{\ZZ}{\mathbb{Z}}
\newcommand*{\QQ}{\mathbb{Q}}
\newcommand*{\CC}{\mathbb{C}}
\newcommand*{\PP}{\mathbb{P}}
\newcommand*{\HH}{\mathbb{H}}
\newcommand*{\FF}{\mathbb{F}}
\newcommand*{\EE}{\mathbb{E}} % Expected Value

%--------------------------------------------------
% MATH SCRIPT, FRAKTUR, AND BOLD SYMBOLS
%--------------------------------------------------
\newcommand*{\mcA}{\mathcal{A}}
\newcommand*{\mcB}{\mathcal{B}}
\newcommand*{\mcC}{\mathcal{C}}
\newcommand*{\mcD}{\mathcal{D}}
\newcommand*{\mcE}{\mathcal{E}}
\newcommand*{\mcF}{\mathcal{F}}
\newcommand*{\mcG}{\mathcal{G}}
\newcommand*{\mcH}{\mathcal{H}}

\newcommand*{\mfA}{\mathfrak{A}}  \newcommand*{\mfB}{\mathfrak{B}}
\newcommand*{\mfC}{\mathfrak{C}}  \newcommand*{\mfD}{\mathfrak{D}}
\newcommand*{\mfE}{\mathfrak{E}}  \newcommand*{\mfF}{\mathfrak{F}}
\newcommand*{\mfG}{\mathfrak{G}}  \newcommand*{\mfH}{\mathfrak{H}}

\usepackage{bm} % Ensure bold math works correctly
\newcommand*{\bmA}{\bm{A}}
\newcommand*{\bmB}{\bm{B}}
\newcommand*{\bmC}{\bm{C}}
\newcommand*{\bmD}{\bm{D}}
\newcommand*{\bmE}{\bm{E}}
\newcommand*{\bmF}{\bm{F}}
\newcommand*{\bmG}{\bm{G}}
\newcommand*{\bmH}{\bm{H}}

%--------------------------------------------------
% FUNCTIONAL ANALYSIS & ALGEBRA
%--------------------------------------------------
\DeclareMathOperator{\Aut}{Aut} % Automorphism group
\DeclareMathOperator{\Inn}{Inn} % Inner automorphisms
\DeclareMathOperator{\Syl}{Syl} % Sylow subgroups
\DeclareMathOperator{\Gal}{Gal} % Galois group
\DeclareMathOperator{\sign}{sign} % Sign function


%\usepackage[tagged, highstructure]{accessibility}
\usepackage{tocloft}
\usepackage{arydshln}
\usetikzlibrary{arrows.meta, decorations.pathreplacing}
\usepackage{tikz-cd}
\usepackage{polynom}
\usepackage{pifont}
\newcommand{\pistar}{{\zf\symbol{"4A}}}
% a tiny helper for a stretched phantom (for the underbrace)
\newcommand\mc[1]{\multicolumn{1}{c}{#1}}



\begin{document}
\title{Linear Algebra I}
\author{Lecture Notes Provided by Dr.~Miriam Logan.}
\date{}
\maketitle
\tableofcontents
\newpage  

\nt{
We will use $ V \left( \lambda \right) $ to denote the generalized eigenspace of a linear operator $ T: V \to V$ associated with $ \lambda$.
}
\nt{
It is also true $ Im \left( T - \lambda I  \right) ^{j}$ is a $ T$- invariant subspace of $ V$ - can be shown in a similar manner.
}
\section{Invariant Subspaces and Blocks}
 	
Suppose $ T: V \to V$ is a linear operator on the vector space $ V$, where dim $ V = 6$. Suppose $ T$ has three invariant subspaces $ U_1 = span \left\{ \vec{ v_1} , \vec{ v_2} , \vec{ v_3} \right\}$, $ U_2 = span \left\{ \vec{ v_4} , \vec{ v_5} \right\}$, and $ U_3 = span \left\{ \vec{ v_6} \right\}$, where $ \mathcal{F} = \left\{ \vec{ v_1} ,\vec{ v_2} ,\vec{ v_3} ,\vec{ v_4} ,\vec{ v_5} ,\vec{ v_6}  \right\} $ forms a basis for $ V$.
\[
	\left[ T \right]  _{ \mathcal{F} , \mathcal{F}}= \left[ \left[ T \left( \vec{ v_1}  \right)  \right]_{ \mathcal{F}} \left[ T \left( \vec{ v_2}  \right)  \right]  \ldots \left[ T \left( \vec{ v_6}  \right)  \right]\right] 
.\] 
Note: $ T \left( \vec{ v_1}  \right) , T \left( \vec{ v_2}  \right) , T \left( \vec{ v_3}  \right)  \in span \left\{ \vec{ v_1} ,\vec{ v_2 } , \vec{ v_3}  \right\} $ \\
$ \implies \left[ T \left( \vec{ v_1}  \right)  \right] _{ \mathcal{F}}, \left[ T \left( \vec{ v_2}  \right)  \right] , \left[ T \left( \vec{ v_3}  \right)  \right] _{\mathcal{F}}$        have zeros in their last three entries, i.e.  they have the form
\[
	\left[ T \left( \vec{ v_i}  \right)  \right] _{ \mathcal{F}} = \begin{bmatrix}
		\star\\
		 \star\\
		 \star\\
	0\\
	0\\
	0\\
	\end{bmatrix}
	   \qquad  1 \leq i \leq 3
.\] 
Similarly, $ T \left( \vec{ v_4}  \right) , T \left( \vec{ v_5}  \right)  \in span \left\{ \vec{ v_4} ,\vec{ v_5 }  \right\} $ \\
Thus their co-ordinate vectors have zeros the top three entries and the last entry, 
\[
	\left[ T \left( \vec{ v_i}  \right)  \right] _{ \mathcal{F}} = \begin{bmatrix}
		0\\
		 0\\
		 0\\
	\star\\
	\star\\
	0\\
	\end{bmatrix}
	   \qquad  4 \leq i \leq 5
.\] 
\[
T \left(  \vec{ v_6}  \right) \in span \left\{ \vec{ v_6}  \right\} \implies \left[ T \left( \vec{ v_6}  \right)  \right] _{ \mathcal{F}} = \lambda \vec{ v_6} \text{ for some } \lambda \in \mathbb{R}
.\] 
$ \implies \left[ T \right] _{ \mathcal{F} \mathcal{F}}$ takes the form
\[
	\left[ T \right] _{ \mathcal{F} , \mathcal{F}} = \begin{bmatrix}
		\star & \star & \star & 0 & 0 & 0\\
		\star & \star & \star & 0 & 0 & 0\\
		\star & \star & \star & 0 & 0 & 0\\
		0 & 0 & 0 & \star & \star & 0\\
		0 & 0 & 0 & \star & \star & 0\\
		0 & 0 & 0 & 0 & 0 & \star\\
	\end{bmatrix}
.\]
XXX  \\
\\
\\
We are building up to the fact that if $ T$ has invariant subspaces whose bases toghether form a basis for $ V$, call it $  \mathcal{F}$, then $ \left[ T \right] _{ \mathcal{F} \mathcal{F}}$ consists of blocks along the diagonal and zeros elsewhere.\\
We are breaking down $ V$ into subspaces of dimensions $ k$, where $ T$ acts line an operator on $ \mathbb{R} ^{ k}$.\\
\section{Sums and Direct Products of Vectorspaces}
 \dfn{ :}{
 Let $ V_1, V_2, \ldots , V_k$ be subspaces of a vector space $ V$. The  sum 
 \[
 \sum\limits_{i=1}^{k} V_i = V_1 + V_2 + \ldots + V_k 
 .\] 
 is defined as the set of vectors of the form $ \vec{ v_1} + \vec{ v_2} + \ldots \vec{ v_k} $ where $ \vec{ v_i} \in V_i $ for $ 1 \leq i \leq k$.\\
 \\
 The sum $ \sum\limits_{i=1}^{k} V_i$  is said to be direct if $ \vec{ 0} + \vec{ 0} + \ldots+ \vec{ 0} $   is the only way to write $ \vec{ 0} \in V$ as a sum of the form $ \vec{ v_1} + \vec{ v_2} + \ldots + \vec{ v_k} $ where $ \vec{ v_i} \in V_i$.\\
 If the sum is direct, it is denoted by
 \[
 \bigoplus\limits_{i=1}^{k} V_i = V_1 \oplus V_2 \oplus \ldots \oplus V_k
 .\] 
 }
 \mlem{}{
 $ \sum\limits_{i=1}^{k} V_i
 $ is a subspace of $ V$.
 }


 \pf{Proof:}{
  \underline{Zero Vector:} \\
  \[
  \vec{ 0} \in V_i \qquad  \forall  i
  .\] \[
  \implies   \underbrace{\vec{ 0} + \vec{ 0} + \ldots + \vec{ 0}}_{k \text{ times}} = \vec{ 0} \in \sum\limits_{i=1}^{k} V_i
  .\] 
  \underline{Closure under Addition:} \\
  Let $ \vec{ v_1} + \vec{ v_2} + \ldots + \vec{ v_k} , \vec{ w_1} + \vec{ w_2} + \ldots + \vec{ w_k} \in \sum\limits_{i=1}^{k} V_i$ 
  \[
	  \left( \vec{ v_1} + \vec{ v_2} + \ldots + \vec{ v_k}  \right) + \left( \vec{ w_1} + \vec{ w_2} + \ldots + \vec{ w_k}  \right) = \left( \vec{ v_1} + \vec{ w_1}  \right) + \left( \vec{ v_2} + \vec{ w_2}  \right) + \ldots + \left( \vec{ v_k} + \vec{ w_k}  \right) \in \sum\limits_{i=1}^{k} V_i \text{ since each } V_i \text{ is a subspace}
  .\] 
  \underline{Closure under Scalar Multiplication:} \\
  Let $ \lambda \in \mathbb{R}$, $ \vec{ v_1} + \vec{ v_2} + \ldots + \vec{ v_k} \in \sum\limits_{i=1}^{k} V_i$
  \[
	  \lambda \left( \vec{ v_1} + \vec{ v_2} + \ldots + \vec{ v_k}  \right) = \lambda \vec{ v_1} + \lambda \vec{ v_2} + \ldots + \lambda \vec{ v_k}  \in \sum\limits_{i=1}^{k} V_i \text{ since each } V_i \text{ is a subspace}
  .\] 
  \[
  \implies \sum\limits_{i=1}^{k} V_i \text{ is a subspace of } V
  .\] 
 }
 \ex{}{
 Let $ V_1 = span \left\{ \begin{bmatrix}
 1\\
 0\\
 \end{bmatrix}
  \right\} = \left\{ a \begin{bmatrix}
  1\\
  0\\
  \end{bmatrix}
\mid a \in \mathbb{R} \right\} = \left\{ \begin{bmatrix}
a\\
0\\
\end{bmatrix}
\mid a \in \mathbb{R} \right\}$\\
\[
 V_2 = span \left\{ \begin{bmatrix}
 0\\
 1\\
 \end{bmatrix}
  \right\} = \left\{ \begin{bmatrix}
  0\\
  b\\
  \end{bmatrix}
\mid b \in \mathbb{R} \right\}
.\] 
\[
V_1 + V_2 = \left\{ \begin{bmatrix}
a\\
0\\
\end{bmatrix}
+ \begin{bmatrix}
0\\
b\\
\end{bmatrix}
\mid a,b \in \mathbb{R} \right\} = \left\{ \begin{bmatrix}
a\\
b\\
\end{bmatrix}
\mid a,b \in \mathbb{R} \right\} = \mathbb{R} ^2
.\] 
\[
	\text{ Clearly  if } \begin{bmatrix}
	0\\
	0\\
	\end{bmatrix}
	= a \begin{bmatrix}
	1\\
	0\\
	\end{bmatrix}
	+ b \begin{bmatrix}
	0\\
	1\\
	\end{bmatrix}
	 \text{ then } a = 0 \text{ and } b = 0
.\] 
i.e.  the sum is direct: $ V_1 \oplus V_2 = \mathbb{R} ^2$\\
 }
 \ex{}{
	 Similarly,  if $ U_n = span \left\{  \vec{ e_1} ,\ldots , \vec{ e_n}  \right\}$ , and $ U_m = span \left\{ \vec{ e_n+1}, \vec{ e_n+2}, \ldots , \vec{ e_{n+m}}    \right\} $, $ U_n, U_m \subseteq \mathbb{R} ^{n+m}$. Then 
	 \[
	 \mathbb{R} ^{n_m} = U_n \oplus U_m 
	 .\] 
 }
 
 \ex{}{
 SSuppose $ \vec{ v_1} , \vec{ v_2} ,\ldots , \vec{ v_k}  \in V $ is a collection of non-zero vectors. Let 
 \[
	 V_i = \left\{ \lambda \vec{ v_i} \mid \lambda \in \mathbb{R} \right\}
 .\] 
Then $ V_1 + V_2 + \ldots + V_k = \left\{ \lambda_{1i} \vec{ v_1} + \lambda_{2i} \vec{ v_2} + \ldots + \lambda_{ki} \vec{ v_k} \mid \lambda _{1i}, \ldots , \lambda _{ki} \in \mathbb{R}\right\} $
\[
\text{ i.e. }   \sum\limits_{i=1}^{k} V_i = span \left\{ \vec{ v_1} , \vec{ v_2} ,\ldots , \vec{ v_k}  \right\}

.\]  
and this is direct if and only if $ \vec{ 0} = c_1 \vec{ v_1} + c_2 \vec{  v_2} + \ldots + c_k \vec{ v_k} $ implies $ c_i =0 \qquad  \forall  i$ i.e.  $ \iff \left\{ \vec{ v_1} ,\vec{ v_2} , \ldots , \vec{ v_k }  \right\} $  is a linearly independent set.\\
 }
 \mlem{}{
	 Suppose $ V_1, V_2, $ are two subspaces of the vector space $ V$. $ V_1 + V_2 $ is a direct $  \iff V_1 \cap  V_2 = { \vec{ 0} } $
 }
 \pf{Proof:}{
       $ \impliedby $ \\
       Suppose $ V_1 + V_2 $ is not a direct sum, then $ \exists  \quad \vec{ v_1} \in V_1$, $ \vec{ v_2} \in V_2$, $ \vec{ v_i} \neq \vec{ 0} $ such that $ \vec{ v_1} + \vec{ v_2} = \vec{ 0} $.\\
       Hence $ \vec{ v_1} = - \vec{  v_2} $ and so $ - \vec{ v_2} \in V_1$ ie $ \vec{ v_2} \in V_1$ and $ \vec{ v_1} \in V_2$ i.e.  $ \vec{ v_1} , \vec{ v_2} \in V_1 \cap V_2 $ 
       \[
       \text{ i.e. }   V_1 \cap V_2 \neq \left\{ \vec{ 0} \right\}
       .\]  \\
       \\
       \\
       $ \implies $ \\
       Suppose $ \exists  \vec{ w}  \in V_1 \cap V_2 $, $ \vec{ w} \neq \vec{ 0} $.\\
       Since $ V_1$, $ V_2$ are vector spaces, $ \lambda \vec{ w} \in V_1 \cap V_2$ $ \forall  \lambda \in \mathbb{R}$,\\
       Hence $ - \vec{ w} \in V_1 \cap V_2$
       \[
       \implies \vec{ w} + \left( - \vec{ w} \right) \in V_1 + V_2
       .\] 
       \[
       \text{ i.e. }   V_1 + V_2 \text{ is not a direct sum}
       .\] 
 }
  \nt{
  The above statement doesn't hold for more than two subspaces. There exists subspaces $ V_1, V_2, \ldots , V_k$ such that $ V_i \cap V_j = {\vec{ 0} }$ is not a direct sum.\\
  \\
  \raggedcolumns
  \begin{multicols}{2}
  For example,\\
  \begin{align*}
  	V_1 &= \left\{ \begin{bmatrix}
  	x\\
  	0\\
  	\end{bmatrix}
  	\mid x \in \mathbb{R} \right\} \\
	  	V_2 &= \left\{ \begin{bmatrix}
	  	0\\
	  	y\\
	  	\end{bmatrix}
	  	\mid y \in \mathbb{R} \right\} \\
			  	V_3 &= \left\{ \begin{bmatrix}
			  	x\\
			  	x\\
			  	\end{bmatrix}
			  	\mid x \in \mathbb{R} \right\}
  .\end{align*}
  
  \break
  
  \begin{tikzpicture}[>=stealth,line width=1.1pt]

  %--- 1. faint horizontal rulings -------------------------------------------
  \foreach \y in {-2,...,2}{
    \draw[gray!30] (-2,\y) -- (4.2,\y);
  }

  %--- 2. the three vectors ---------------------------------------------------
  % v1 : horizontal, pink
  \draw[->,magenta] (0,0) -- (4,0) node[right=4pt] {$\mathbf{v}_1$};

  % v2 : vertical, cyan
  \draw[->,cyan]    (0,0) -- (0,3) node[above=4pt] {$\mathbf{v}_2$};

  % v3 : diagonal, purple  (slope chosen to look like the sketch)
  \draw[->,violet]  (0,0) -- (3,2.25) node[above right=4pt] {$\mathbf{v}_3$};

\end{tikzpicture}
  \end{multicols}
  $ V_i \cap V_j = {\vec{ 0} } \qquad  \forall  i \neq j$ but the sum $ \sum\limits_{i=1}^{3} V_i
  $ is not a direct since:
  \[
  \begin{bmatrix}
  1\\
  0\\
  \end{bmatrix}
  + \begin{bmatrix}
  0\\
  1\\
  \end{bmatrix}
  + \begin{bmatrix}
  -1\\
  -1\\
  \end{bmatrix}
  = \begin{bmatrix}
  0\\
  0\\
  \end{bmatrix}
  .\] 
  }
  \mlem{}{
  
  	Let $ V_1, V_2, \ldots , V_k$ be subspaces of a vector space $ V$. The following are equivalent(TFAE):
	\begin{enumerate}[label=(\arabic*).]  
	\item $ \sum\limits_{i=1}^{k} V_i $ is a direct sum.
	\item  Every vector in $ \sum\limits_{i=1}^{k} V_i 
	$ can be expressed uniquely in the form $ \vec{ v_1} + \vec{ v_2} + \ldots + \vec{ v_k}  $ with $ \vec{ v_i} \in V_i$
	\end{enumerate}
}
\pf{Proof:}{
 $ (1) \implies (2)$ \\
 Suppose $ \vec{ v} = \sum\limits_{i=1}^{k} V_i
 $ and 
 \[
 \vec{ v} = \sum\limits_{i=1}^{k} \vec{ v_i} = \sum\limits_{i=1}^{k} \vec{ w_i} \qquad \vec{ v_i} , \vec{ w_i} \in V_i \qquad \text{ not all } \vec{ v_i} = \vec{ w_i}
 .\]
 Then $ \sum\limits_{i=1}^{k} \vec{ v_i} - \vec{ w_i} = \vec{ 0} $\\
 Since $ \sum\limits_{i=1}^{k} V_i $ is a direct sum, thus $ \vec{ v_i} - \vec{ w_i} = \vec{ 0} $  $ \forall i $, i.e. $ \vec{ v_i} = \vec{ w_i}  \forall  i$
 i.e.  $ \vec{ v} $ can be expressed uniquely as $ \sum\limits_{i=1}^{k} \vec{ v_i} $ \\
 $ (2) \implies (1)$ \\
 $ \vec{ 0} \in \sum\limits_{i=1}^{k} V_i ,$  $ \vec{ 0} = \vec{ 0} + \ldots + \vec{ 0} $ \\
 i.e. $ \vec{ 0} $ can not be expressed in any other manner $ \implies \sum\limits_{i=1}^{k} V_i $  is a direct sum.\\
}

\thm{}
{
Let $ V_1$, $ V_2$ be subspaces of a vector space $ V$. \[
dim \left( V_1 + V_2 \right) = dim \left( V_1 \right) + dim \left( V_2 \right) - dim \left( V_1 \cap V_2 \right)
.\] 
}  
\pf{Proof:}{
	Let $ \left\{ \vec{ b_1} , \vec{ b_2} , \ldots , \vec{ b_k}  \right\} $ be a basis for $ V_1 \cap  V_2 $. Extend this to a basis for $ V_1$ by adding vectors $ \left\{ \vec{ f_1} , \vec{ f_2} , \ldots , \vec{ f_{\ell}}$ and to a basis for $ V_2$ by adding $ \vec{ w_1} ,\ldots \vec{ w_m} $ so that 
		\[
	\left\{ \vec{ b_1} , \vec{ b_2} , \ldots , \vec{ b_k} , \vec{ f_1} , \ldots , \vec{ f_{\ell}} \text{ is a basis for } V_1 \right\}
		.\] 
		\[
	\left\{ \vec{ b_1} , \vec{ b_2} , \ldots , \vec{ b_k} , \vec{ w_1} , \ldots , \vec{ w_m} \text{ is a basis for } V_2 \right\}
		.\]
\underline{Claim:}\\
$ \left\{ \vec{ b_1} , \ldots, \vec{ b_k} , \vec{ f_1} ,\ldots , \vec{ f_{\ell }}, \vec{ w_1} ,\ldots, \vec{ w_m}  \right\} $ forms a basis for $ V_1 + V_2$\\
It certainly spans $ V_1 + V_2$ since it contains a basis for $ V_1$ and a basis for $ V_2$.\\
\textit{Assume we have a dependence relation:}\\
Suppose $ \exists  \alpha_i , \gamma_i , \beta _i \in \mathbb{R}$ such that
\[
\sum\limits_{i=1}^{k} \alpha_i \vec{ b_i} + \sum\limits_{i=1}^{\ell} \gamma_i \vec{ f_i} + \sum\limits_{i=1}^{m} \beta_i \vec{ w_i} = \vec{ 0}
.\] 
\[
\implies   \underbrace{ \sum\limits_{i=1}^{k} \alpha_i \vec{ b_i} + \sum\limits_{i=1}^{\ell } \gamma_i \vec{ f_i} 
 }_{ \in V_1 } = \underbrace{ \sum\limits_{i=1}^{m} \beta_i \vec{ w_i} }_{ \in V_2} 
.\] 
\[
\implies \vec{ v} = \sum\limits_{i=1}^{k} \alpha_i \vec{ b_i} + \sum\limits_{i=1}^{\ell } \gamma_i \vec{ f_i}  \in V_1 \cap V_2
.\] 
Hence $ \vec{ v} = \sum\limits_{i=1}^{k} \delta_i \vec{ b_i} $ for some $ \delta_i \in \mathbb{R}$, and so $ \vec{ v} = \sum\limits_{i=1}^{k} \delta_i \vec{ b_i} = - \sum\limits_{i=1}^{m} \beta_i \vec{ w_i} 
$ 
\\
\[
\text{ i.e. }  \sum\limits_{i=1}^{k} \delta_i \vec{ b_i} + \sum\limits_{i=1}^{m} \beta_i \vec{ w_i} = \vec{ 0}
.\] 
but $ \left\{ \vec{ b_1} , \ldots, \vec{ b_k} , \vec{ w_1} ,\ldots, \vec{ w_m}  \right\} $ forms a basis for $ V_2$ and so $ \delta_i =0 $, $ \beta_i =0 \forall i$
 i.e.  $ \vec{ v} = \vec{ 0} $ \\
 Hence $ \sum\limits_{i=1}^{k} \alpha_i \vec{ b_i} + \sum\limits_{i=1}^{\ell} \gamma_i \vec{ f_i} = \vec{ 0} $\\
 and since $ \left\{ \vec{ b_1} , \ldots, \vec{ b_k} , \vec{ f_1} ,\ldots, \vec{ f_{\ell } }  \right\} $ forms a basis for $ V_1$, we have $ \alpha_i =0$ and $ \gamma_i =0$ for all $ i$.\\
 \[
\implies \left\{ \vec{ b_1} , \ldots, \vec{ b_k} , \vec{ f_1} ,\ldots, \vec{ f_{\ell }}, \vec{ w_1} ,\ldots, \vec{ w_m}  \right\} \text{ forms a basis for } V_1 + V_2
 .\] 
 \begin{align*}
 	dim \left( V_1 + V_2 \right) &= k + \ell + m\\
	dim \left( V_1 \right) &= k + \ell \qquad  \implies dim \left( V_1 + V_2 \right) = dim \left( V_1 \right) + dim \left( V_2 \right) - dim \left( V_1 \cap V_2 \right)\\
	dim \left( V_1 \cap V_2 \right) &= k
 .\end{align*}

}

\section{External Direct  Sum}
\dfn{ :}{
Given Vector spaces $ V_1, V_2, \ldots , V_k$, we define their external direct sum:\\
\[
V = \bigoplus_{i=1}^{k} V_i = V_1 \oplus V_2 \oplus \ldots \oplus V_k
.\]           to be the set $ \left\{ \vec{ v_1} ,\vec{ v_2} \ldots , \vec{ v_k}  \mid \vec{ v_i} \in V_i \right\} $ equipped with the vector space structure:\\
\underline{Addition:} \\
\[
\left( \vec{ v_1} ,\vec{ v_2} \ldots, \vec{ v_k}  \right) + \left( \vec{ w_1} ,\vec{ w_2} \ldots, \vec{ w_k}  \right) = \left( \vec{ v_1} + \vec{ w_1} ,\vec{ v_2} + \vec{ w_2} ,\ldots, \vec{ v_k} + \vec{ w_k}  \right)
.\] 
\underline{Scalar Multiplication:} \\
\[
\lambda \left( \vec{ v_1} ,\vec{ v_2} \ldots, \vec{ v_k}  \right) = \left( \lambda \vec{ v_1} ,\lambda \vec{ v_2} ,\ldots, \lambda \vec{ v_k}  \right)
.\] 
Each vector space $ V_i$ can be considered as a subspace of $ V$ by identifying $ \vec{ v_i} \in V_i$ with $ \left( 0,0, \ldots , 0 , \vec{ v_i} ,\ldots,0 \right) $ \\
And so the external direct sum can be view as an internal direct sum.
}
 \section{Direct Sum of Matrices}
 Let $ T_i: V_i \to V_i$ be linear operators for $ 1 \leq i \leq k$.\\
 Let $ V = \bigoplus_{i=1} ^{k}$. We can define a linear operator $ T: V \to V$ by
 \[
 T = \bigoplus_{i=1} ^{k} T_i: V \to V \text{ where }
 .\] 
 \[
 T \left( \vec{ v_1} ,\vec{ v_2} ,\ldots , \vec{ v_k }  \right)  = \left(  T_1 \left( \vec{ v_1}  \right)  \right) , T_2 \left( \vec{ v_2}  \right) ,\ldots , T_k \left( \vec{ v_k}  \right)
 .\] 
 Let $ \mathcal{E} _i$ be a basis for $ V_i$ $ \forall  i \le i \le k$. The collection of vectors from all the $ \mathcal{E}_i$ 's form a basis for $ V$, which we denote by $ \mathcal{E}$.\\
 The matrix of $ T$ with respect to $ \mathcal{E}$ is 
 \[
  \left[ T \right] _{ \mathcal{E}, \mathcal{E}}= \bigoplus_{i=1} ^{k} \left[ T_i \right] _{ \mathcal{E}_i, \mathcal{E}_i} = \begin{bmatrix}
	  \left[ T_1 \right] _{ \mathcal{E}_1, \mathcal{E}_1} & 0 & 0 & \dots  & 0 \\
	  0 & \left[ T_2 \right] _{ \mathcal{E}_2 , \mathcal{E}_2} & 0 & \dots  & 0 \\
  \vdots & \vdots & \vdots & \ddots & \vdots \\
0 & 0 & 0 & \dots  & \left[ T_k \right] _{ \mathcal{E} _k , \mathcal{E} _k}\end{bmatrix}
 .\] 
 where each $ \left[ T_i \right] _{ \mathcal{E}_i, \mathcal{E}_i}$ is the matrix of $ T_i$ with respect to the basis $ \mathcal{E}_i$ and each $ \left[ T_i \right] _{ \mathcal{E} _i , \mathcal{E} _i} $ is a block of dimension $ dim \left( V_i \right) \times dim \left( V_i \right)$.\\
 	      \dfn{ Nilpotent Operators and the Index of an Operator :}{
 	           \begin{enumerate}[label=(\roman*)]
 	           \item A linear operator $ T: V \to V$ is said to be \underline{nilpotent} if  $ T ^{r}=0$ for some $ r \ge 1$. The minimal $ r$ with this property is called the \underline{index} of $ T$ 
 	           \item  Given a vector $ \vec{ 0} \neq \vec{ v} \in V$, we define its index (relative to $ T$) to be the minimal $ r \ge 1$ such that $ T ^{r} \vec{ v} = \vec{ 0}$. 
 	           \end{enumerate}
 	      }
\ex{}{
Suppose $ T: \mathbb{R} ^2 \to \mathbb{R} ^2 $ 
\[
T \begin{bmatrix}
x\\
y\\
\end{bmatrix}
= \begin{bmatrix}
0 & 1\\
0 & 0\\
\end{bmatrix}
.\] 
\[
\text{ Note: } T ^2 = \begin{bmatrix}
0 & 0\\
0 & 0\\
\end{bmatrix} \qquad  \text{ since } \begin{bmatrix}
0 & 1\\
0 & 0\\
\end{bmatrix} \begin{bmatrix}
0 & 1\\
0 & 0\\
\end{bmatrix} = \begin{bmatrix}
0 & 1\\
0 & 0\\
\end{bmatrix} \implies T \text{ is nilpotent with index } 2
.\] 
\\
\underline{Note:}  $ T \begin{bmatrix}
x\\
y\\
\end{bmatrix}
= \begin{bmatrix}
y\\
0\\
\end{bmatrix}
$.\\
Hence $ \begin{bmatrix}
x\\
0\\
\end{bmatrix}
$ has an index $ 1$ relative to $ T$ and $ \begin{bmatrix}
x\\
y\\
\end{bmatrix}
$ has an index $ 2$ $ \implies T ^2 \left(  \vec{ v}  \right) = \vec{ 0}  \qquad  \forall  \vec{ v} \in \mathbb{R} ^2$ \\
\\
More generally, it is true that if $ T: V \to V $ has index $ r$, there exists $ \vec{ v} \in V$   such that $ \vec{ v} $ has index $ r$ 
}
 \mlem{}{
 Suppose $ \vec{ v} \in V$ has index $ r$ relative to $ T$. Then the vectors $ \left\{ \vec{ v} , T \left( \vec{ v}  \right) , T ^2 \left( \vec{ v}  \right) , \ldots , T ^{r-1}\left( \vec{ v}  \right)  \right\} $ are linearly independent.\\

 }
 
 \pf{Proof:}{
	 Let $ U = span \left\{ \vec{ v_1} T \left( \vec{ v}  \right) , \ldots , T ^{r-1}\left( \vec{ v}  \right)  \right\}$ \\
	 Note: $ U$ is an invariant subspace of $ T$.\\
	 \\
	 On $ U$, $ T ^{r}=0$ since $ T ^{r} \left(  T ^{j} \left( \vec{ v}  \right)  \right) = T ^{j} \left( T ^{r} \left( \vec{ v}  \right)  \right) = \vec{ 0}  \qquad  \forall  j \ge 0$.\\
	 \\
Suppose $ \left\{ \vec{ v_1} T \left( \vec{ v}  \right) , \ldots , T ^{r}\left( \vec{ v}  \right)  \right\} $  is linearly dependent, i.e.  there exists  $ a_0, a_1, \ldots , a_{r-1}$, not all zero, such that
\[
\sum\limits_{i=0}^{r-1} a_i T ^{i} \left( \vec{ v}  \right) = \vec{ 0} \qquad  \text{ where }  T^{0} \left( \vec{ v}  \right) = I \left( \vec{ v}  \right) = \vec{ v} 
.\] 
Let $ j$ be the minimal integer such that $ a_j \neq 0$.\\
Dividing across by $ a_j$, we get:
\[
\sum\limits_{i=0}^{r-1} \frac{ -a_i  }{ a_j } T ^{i} \left( \vec{ v}  \right) = \vec{ 0}
.\]
\[
\text{ i.e.  } \sum\limits_{i=j}^{r-1} \frac{ -a_i  }{ a_j } T ^{i} \left( \vec{ v}  \right) = \vec{ 0} 
.\] 
\[
\text{ i.e.  } -T ^{j} \left( \vec{ v}  \right) + \sum\limits_{i=j+1}^{r-1} \frac{ -a_i  }{ a_j } T ^{i} \left( \vec{ v}  \right) = \vec{ 0} 
.\] 
\[
\text{ i.e.  } T ^{j} \left( \vec{ v}  \right) = \sum\limits_{i=j+1}^{r-1} \frac{ -a_i  }{ a_j } T ^{i} \left( \vec{ v}  \right)
.\] 
\[
	T ^{j} \left( \vec{ v}  \right) =  \sum\limits_{i=j+1}^{r-1} \frac{ -a_i  }{ a_j } T ^{j+1} \circ T ^{i-j-1} \left( \vec{ v}  \right) 
.\] 
By linearity  we have:
\[
T ^{j} \left( \vec{ v}  \right) =  T ^{j+1} \left( \sum\limits_{i=j+1}^{r-1} \frac{ -a_i  }{ a_j } T ^{i-j-1} \left( \vec{ v}  \right) \right)
.\] 
\[
\text{ Let } \vec{ w} = \sum\limits_{i=j+1}^{r-1} \frac{ -a_i  }{ a_j } T ^{i-j-1} \left( \vec{ v}  \right)
.\] 
Hence $ T ^{j} \left(  \vec{ v}  \right) = T ^{j+1} \left( \vec{ w}  \right) $ and ,
\[
T ^{r-1} \left( \vec{ v}  \right) = T ^{r-1-j} \circ T ^{j} \left( \vec{ v}  \right) = T ^{r-1-j} \circ T ^{j+1} \left( \vec{ w}  \right)
.\] 
\[
\text{ i.e. } T ^{r-1} \left( \vec{ v}  \right) = T ^{r-j} \left( \vec{ w}  \right)
.\] 
but, $ \vec{ w} \in span \left\{  \vec{ v} , T \left( \vec{ v} , \ldots , T ^{r-1}\left( \vec{ v}  \right)  \right) \right\} = U$ and $ T ^{r} =0$ on $ U \implies T ^{r} \left(  \vec{ w}  \right) =0$ i.e.  $ T ^{r-1} \left( \vec{ v}  \right) =0$, which contradicts the assumption that $ \vec{ v} $ has index $ r$.\\
$ \implies $ $ \left\{ \vec{ v} , T \left( \vec{ v}  \right) , T ^2 \left( \vec{ v}  \right) , \ldots , T ^{r-1}\left( \vec{ v}  \right)  \right\} $ is linearly independent.\\
 }
 \dfn{ :}{
	 Let $ T: V \to V$ be a linear operator. A vector $ \vec{ v} \in V$ is  called a \underline{ cyclic vector} relative to $ T$ (or T-cyclic) if the vectors $ \left\{ \vec{ v} , T \left( \vec{ v} ,\ldots \right)  \right\}$ span $ V$.\\
	 The vector space $ V$ is called cyclic relative to $ T$ if there exists a cyclic vector $ v \in V$ relative to $ T$.\\
	 \textbf{Example:}\\
	 $ T: \mathbb{R} ^2 \to \mathbb{R} ^2$, $ T \begin{bmatrix}
	 x\\
	 y\\
	 \end{bmatrix}
	  = \begin{bmatrix}
	  0 & 1\\
	  0 & 0\\
	  \end{bmatrix} \begin{bmatrix}
	  x\\
	  y\\
	  \end{bmatrix}
	  = \begin{bmatrix}
	  y\\
	  0\\
	  \end{bmatrix}
	  $ \\
	  $ \vec{ v} = \begin{bmatrix}
	  0\\
	  1\\
	  \end{bmatrix}
	  $ is a cyclic vector relative to $ T$ since $ \left\{  \vec{ v} , T \left( \vec{ v}  \right)  \right\} $ span $ \mathbb{R} ^2$ $ = \left\{  \begin{bmatrix}
	  0\\
	  1\\
	  \end{bmatrix}
	  , \begin{bmatrix}
	  1\\
	  0\\
	  \end{bmatrix}
	   \right\} $.\\
 }
 \nt{
 Given a vector $ \vec{ v} \in V$, consider the subspace $ U = span \left\{ \vec{ v} , T \left( \vec{ v}  \right) , T ^2 \left( \vec{ v}  \right) , \ldots  \right\}  \subseteq V$.\\
 $ U$ is a cyclic subspace of $ V$ , called the subspace generated by $ \vec{ v} $. \\
 $ U$ is $ T$- invariant     ( $ T \left( T ^{j} \left( \vec{ v}  \right)  \right) = T ^{j+1} \left( \vec{ v}  \right) \in U$) \\
 \\
 Let $ r \ge 1$ the maximal number such that the vectors $ \left\{ \vec{ v} , T \left( \vec{ v}  \right) , T ^2 \left( \vec{ v}  \right) , \ldots , T ^{r-1} \left( \vec{ v}  \right)  \right\} $ are linearly independent. Then the set $ \left\{ \vec{ v} , T \left( \vec{ v}  \right) ,\ldots , T ^{r-1} \left( \vec{ v}  \right)  \right\} $ forms a basis for $ U$.\\
 }
 \dfn{ :}{
	 Given $ T: V \to V$ and suppose $ \vec{ v} \in V$ is a cyclic vector of index $ r$. By the last lemma, the vectors $ \mathcal{F} = \left\{  T ^{r-1}\left( \vec{ v}  \right) , T ^{r-2}\left( \vec{ v}  \right) ,\ldots , T \left( \vec{ v}   \right) ,\vec{ v}  \right\} $ form a basis for $ V$. The matrix of $ T$ relative to $ \mathcal{F} $ has the form:
	 \[
		 \left[ T \right] _{ \mathcal{F} , \mathcal{F}} = \left[ \left[ T \left( T ^{r-1} \left( \vec{ v}  \right)  \right)  \right]_{ \mathcal{F}}  \left[ T \left( T ^{r-2} \left( \vec{ v}  \right)  \right)  \right] _{ \mathcal{F}}\ldots \left[ T \left( T \left( \vec{ v}  \right)  \right)  \right]_{ \mathcal{F}} \left[ T \left( \vec{ v}  \right)  \right] _{\mathcal{F}}\right] 
	 .\] 
	 \[
		 \left[ T \right] _{ \mathcal{F} , \mathcal{F}} = \begin{bmatrix}
		 0 & 1 & 0 & \dots  & 0 \\
		 0 & 0 & 1 & \dots  & 0 \\
		 \vdots & \vdots & \vdots & \ddots & \vdots \\
		 0 & 0 & 0 & \dots  & 1 \\
		 0 & 0 & 0 & \dots  & 0\end{bmatrix}
	 .\] 
 
 }
    \dfn{ :}{
    Suppose $ A \in M_{ n \times  n} \left(  \mathbb{C} \right) $. A vector $ \vec{ v} \in \mathbb{C} ^n$ is called a generalized eigenvector of rank $ m$ of the matrix $ A$ corresponding to the eigenvalue $ \lambda \in \mathbb{C}$ if $ \vec{ v} $ has index $ m$ relative to $ \left( A - \lambda I \right) $ i.e.  $ \left( A - \lambda I \right) ^{m} \left(  \vec{ v}  \right) = \vec{ 0} $ but $ \left( A - \lambda I  \right) ^{j } \vec{ v} \neq \vec{ 0} $  for any $ j < m$ i.e.  $ m$ is the smallest integer such that $ \vec{ v}  \in \ker \left( A - \lambda I \right) ^{m}$.\\
    }
    \nt{
    If $ \vec{ v_m} \in \mathbb{C} ^{n}$ is a generalized eigenvector of rank $ m$, then $ \left( A - \lambda I \right) ^{m } \vec{ v_m} = \vec{ 0} $.\\
    Hence $ \left(  A - \lambda I  \right) ^{m - j} \left( A - \lambda I  \right) ^{j} \vec{ v_m} = \vec{ 0}  $.\\
    i.e. $ \left( A - \lambda I \right) ^{j} \vec{ v_m} $ is also a generalized eigenvector of rank $ m - j$. This is true for all $ 0 \le j \le m-1$.\\
    Let 
    \begin{align*}
    	\vec{ v_{m-1}} &= \left( A - \lambda I \right) ^{m - 1} \vec{ v_m}  \qquad  \implies \vec{ v_{m-1}} \in \mathcal{N}   \\
	    	\vec{ v_{m-2}} &= \left( A - \lambda I \right) ^{m - 2} \vec{ v_m}  \qquad  \implies \vec{ v_{m-2}} \in \mathcal{N}   \\
			    	&\vdots \\
				    	\vec{ v_1} &= \left( A - \lambda I \right) \vec{ v_m}  \qquad  \implies \vec{ v_1} \in \mathcal{N}   \\
    .\end{align*}
    In general, $ \left( A - \lambda I  \right) ^{j} \left( A - \lambda I \right) ^{ m - j} \vec{ v_m} = \vec{ 0} $ \\
    \[
    \vec{ v_j} = \left( A - \lambda I \right) ^{m-j}, \qquad \vec{ v_j} \in \mathcal{N} \left( A - \lambda I \right) ^{j}
    .\] 
    \[
    \text{ Also, note } \vec{ v_j} = \left( A - \lambda I  \right) \vec{ v_{j+1}} 
    .\] 
    The set $ \left\{ \vec{ v_1} ,\vec{ v_2} ,\ldots, \vec{ v_j} ,\ldots , \vec{ v_m}  \right\} $ is called a Jordan chain.
    }
    \nt{
    \begin{enumerate}[label=(\arabic*).]  
    \item $ \vec{ v_j} = \left( A - \lambda I \right) ^{ m -j } \vec{ v_m} \in \mathcal{N} \left( A - \lambda I \right) ^{j}  $ \\
	    \[
	    \vec{ v_j} \notin \mathcal{N} \left( A - \lambda I \right) ^{k} \qquad  \forall k < j
	    .\] 
	    \[
	    \text{ i.e. } \vec{ v_j} \text{ has index } j \text{ relative to } \left( A - \lambda I \right) \qquad  \forall  1 \leq j \leq m-1
	    .\] 
    \item    $ \vec{ v_1} = \left( A - \lambda I  \right) ^{ m-1} \vec{ v_m} $ is an eigenvector
	    	\item    $ \vec{ v_m} $ has index $ m$ relative to $ \left( A - \lambda I \right) $\\
			$ \implies$ by the previous lemma, the set 
			\[
			\left\{ \left( A-\lambda I \right) ^{ m-1} \vec{ v_m} , \left( A - \lambda I \right) ^{ m-2} \vec{ v_m} ,\ldots, \left( A - \lambda I \right)  \vec{ v_m} ,\vec{ v_m}  \right\}
			.\] 
			is a linearly independent set.\\
		\item span $ \left\{ \vec{ v_1} , \vec{ v_2} , \ldots , \vec{ v_m}  \right\} =  span \left\{ \left( A - \lambda I  \right) ^{ m-1} \vec{ v_m} , \left( A - \lambda I  \right) ^{m-2} \vec{ v_m}, \ldots ,  \left( A -  \lambda I  \right) \vec{ v_m} , \vec{ v_m}     \right\} $ is a cyclic subspace of $ V$. It is clearly $ \left( A - \lambda I \right)  $ invariant.\\ It is also $ A$ invariant since,
			\[
			\vec{ v_j} = \left( A - \lambda I  \right) \vec{ v_{j+1}} \qquad  \forall  1 \leq j \leq m-1
			.\] 
			\[
			\text{ i.e. } \vec{ v_j} = A \vec{ v_{j+1}} - \lambda \vec{ v_{j+1}} 
			.\] 
			\[
			\vec{ v_j} + \lambda \vec{ v_{j+1}} = A \vec{ v_{j+1}}
			.\] 
			$ \implies$ each $ A \vec{ v_{j+1}} $ is a linear combination of precisely 2 vectors from the Jordan chain $ \forall  1 \leq j \leq m-1$ and $ A \vec{ v_1} = \lambda \vec{ v_1} $ 
			
		\item Putting (3) and (4) together, we get that $ \left\{ \vec{ v_1} , \vec{ v_2} ,\ldots , \vec{ v_m}  \right\}$ forms a basis for  span $ \left\{ \vec{ v_1} , \vec{ v_2} ,\ldots , \vec{ v_m}  \right\}$. We can extend this set to form a basis for $ \mathbb{C} ^{n}$. 
			\[
			\mathcal{F} = \left\{ \vec{ v_1} , \vec{ v_2} ,\ldots , \vec{ v_m} , \vec{ w_1} , \ldots , \vec{ w_{ \ell}} \right\} \qquad  \left( \ell + m = n \right) 
			.\] 
			and the matrix of $ A$ relative to $ \mathcal{F}$ takes the form:
                     \[
  \bigl[A\bigr]_{\mathcal F,\mathcal F}
  \;=\;
  \Bigl[
      [A(\vec v_1)]_{\mathcal F}\;
      [A(\vec v_2)]_{\mathcal F}\;\dots\;
      [A(\vec v_m)]_{\mathcal F}\;
      [A(\vec w_1)]_{\mathcal F}\;\dots\;
      [A(\vec w_\ell)]_{\mathcal F}
  \Bigr]
\]

\[
  =
  \left[
  \begin{array}{*{4}{c}|c}
    \lambda & 1        & 0        & \cdots &                \\[-2pt]
    0       & \lambda  & 1        & \ddots &                \\[-2pt]
    \vdots  & \ddots   & \ddots   & \ddots & \raisebox{-5.5ex}{\Huge$0$}\\[-2pt]
    0       & \cdots   & 0        & \lambda&                \\ \cdashline{1-4}
    0       & \cdots   & 0        & 0      &
      \bigl[\;A(\vec w_1)\;\dots\;A(\vec w_\ell)\;\bigr]_{\mathcal F}
  \end{array}
  \right]
  \qquad
  \underbrace{\hspace{4.5cm}}_{\text{$m$ columns}}
\]
    \end{enumerate}
    
    }
    
   XXX THIS FILE CHECK ALOT XXX 
 

 

 
 	      
	
   
 
 
         




     
\end{document}         
