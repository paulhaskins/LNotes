\documentclass{report}

%%%%%%%%%%%%%%%%%%%%%%%%%%%%%%%%%
% PACKAGE IMPORTS
%%%%%%%%%%%%%%%%%%%%%%%%%%%%%%%%%


\usepackage[tmargin=2cm,rmargin=1in,lmargin=1in,margin=0.85in,bmargin=2cm,footskip=.2in]{geometry}
\usepackage{amsmath,amsfonts,amsthm,amssymb,mathtools}
\usepackage[varbb]{newpxmath}
\usepackage{xfrac}
\usepackage[makeroom]{cancel}
\usepackage{bookmark}
\usepackage{enumitem}
\usepackage{hyperref,theoremref}
\hypersetup{
	pdftitle={Assignment},
	colorlinks=true, linkcolor=doc!90,
	bookmarksnumbered=true,
	bookmarksopen=true
}
\usepackage[most,many,breakable]{tcolorbox}
\usepackage{xcolor}
\usepackage{varwidth}
\usepackage{varwidth}
\usepackage{tocloft}
\usepackage{etoolbox}
\usepackage{derivative} %many derivativess partials
%\usepackage{authblk}
\usepackage{nameref}
\usepackage{multicol,array}
\usepackage{tikz-cd}
\usepackage[ruled,vlined,linesnumbered]{algorithm2e}
\usepackage{comment} % enables the use of multi-line comments (\ifx \fi) 
\usepackage{import}
\usepackage{xifthen}
\usepackage{pdfpages}
\usepackage{transparent}
\usepackage{verbatim}

\newcommand\mycommfont[1]{\footnotesize\ttfamily\textcolor{blue}{#1}}
\SetCommentSty{mycommfont}
\newcommand{\incfig}[1]{%
    \def\svgwidth{\columnwidth}
    \import{./figures/}{#1.pdf_tex}
}
\usepackage[tagged, highstructure]{accessibility}
\usepackage{tikzsymbols}
\renewcommand\qedsymbol{$\Laughey$}


%\usepackage{import}
%\usepackage{xifthen}
%\usepackage{pdfpages}
%\usepackage{transparent}


%%%%%%%%%%%%%%%%%%%%%%%%%%%%%%
% SELF MADE COLORS
%%%%%%%%%%%%%%%%%%%%%%%%%%%%%%



\definecolor{myg}{RGB}{56, 140, 70}
\definecolor{myb}{RGB}{45, 111, 177}
\definecolor{myr}{RGB}{199, 68, 64}
\definecolor{mytheorembg}{HTML}{F2F2F9}
\definecolor{mytheoremfr}{HTML}{00007B}
\definecolor{mylenmabg}{HTML}{FFFAF8}
\definecolor{mylenmafr}{HTML}{983b0f}
\definecolor{mypropbg}{HTML}{f2fbfc}
\definecolor{mypropfr}{HTML}{191971}
\definecolor{myexamplebg}{HTML}{F2FBF8}
\definecolor{myexamplefr}{HTML}{88D6D1}
\definecolor{myexampleti}{HTML}{2A7F7F}
\definecolor{mydefinitbg}{HTML}{E5E5FF}
\definecolor{mydefinitfr}{HTML}{3F3FA3}
\definecolor{notesgreen}{RGB}{0,162,0}
\definecolor{myp}{RGB}{197, 92, 212}
\definecolor{mygr}{HTML}{2C3338}
\definecolor{myred}{RGB}{127,0,0}
\definecolor{myyellow}{RGB}{169,121,69}
\definecolor{myexercisebg}{HTML}{F2FBF8}
\definecolor{myexercisefg}{HTML}{88D6D1}


%%%%%%%%%%%%%%%%%%%%%%%%%%%%
% TCOLORBOX SETUPS
%%%%%%%%%%%%%%%%%%%%%%%%%%%%

\setlength{\parindent}{1cm}
%================================
% THEOREM BOX
%================================

\tcbuselibrary{theorems,skins,hooks}
\newtcbtheorem[number within=section]{Theorem}{Theorem}
{%
	enhanced,
	breakable,
	colback = mytheorembg,
	frame hidden,
	boxrule = 0sp,
	borderline west = {2pt}{0pt}{mytheoremfr},
	sharp corners,
	detach title,
	before upper = \tcbtitle\par\smallskip,
	coltitle = mytheoremfr,
	fonttitle = \bfseries\sffamily,
	description font = \mdseries,
	separator sign none,
	segmentation style={solid, mytheoremfr},
}
{th}

\tcbuselibrary{theorems,skins,hooks}
\newtcbtheorem[number within=chapter]{theorem}{Theorem}
{%
	enhanced,
	breakable,
	colback = mytheorembg,
	frame hidden,
	boxrule = 0sp,
	borderline west = {2pt}{0pt}{mytheoremfr},
	sharp corners,
	detach title,
	before upper = \tcbtitle\par\smallskip,
	coltitle = mytheoremfr,
	fonttitle = \bfseries\sffamily,
	description font = \mdseries,
	separator sign none,
	segmentation style={solid, mytheoremfr},
}
{th}


\tcbuselibrary{theorems,skins,hooks}
\newtcolorbox{Theoremcon}
{%
	enhanced
	,breakable
	,colback = mytheorembg
	,frame hidden
	,boxrule = 0sp
	,borderline west = {2pt}{0pt}{mytheoremfr}
	,sharp corners
	,description font = \mdseries
	,separator sign none
}

%================================
% Corollery
%================================
\tcbuselibrary{theorems,skins,hooks}
\newtcbtheorem[number within=section]{Corollary}{Corollary}
{%
	enhanced
	,breakable
	,colback = myp!10
	,frame hidden
	,boxrule = 0sp
	,borderline west = {2pt}{0pt}{myp!85!black}
	,sharp corners
	,detach title
	,before upper = \tcbtitle\par\smallskip
	,coltitle = myp!85!black
	,fonttitle = \bfseries\sffamily
	,description font = \mdseries
	,separator sign none
	,segmentation style={solid, myp!85!black}
}
{th}
\tcbuselibrary{theorems,skins,hooks}
\newtcbtheorem[number within=chapter]{corollary}{Corollary}
{%
	enhanced
	,breakable
	,colback = myp!10
	,frame hidden
	,boxrule = 0sp
	,borderline west = {2pt}{0pt}{myp!85!black}
	,sharp corners
	,detach title
	,before upper = \tcbtitle\par\smallskip
	,coltitle = myp!85!black
	,fonttitle = \bfseries\sffamily
	,description font = \mdseries
	,separator sign none
	,segmentation style={solid, myp!85!black}
}
{th}


%================================
% LENMA
%================================

\tcbuselibrary{theorems,skins,hooks}
\newtcbtheorem[number within=section]{Lenma}{Lenma}
{%
	enhanced,
	breakable,
	colback = mylenmabg,
	frame hidden,
	boxrule = 0sp,
	borderline west = {2pt}{0pt}{mylenmafr},
	sharp corners,
	detach title,
	before upper = \tcbtitle\par\smallskip,
	coltitle = mylenmafr,
	fonttitle = \bfseries\sffamily,
	description font = \mdseries,
	separator sign none,
	segmentation style={solid, mylenmafr},
}
{th}

\tcbuselibrary{theorems,skins,hooks}
\newtcbtheorem[number within=chapter]{lenma}{Lenma}
{%
	enhanced,
	breakable,
	colback = mylenmabg,
	frame hidden,
	boxrule = 0sp,
	borderline west = {2pt}{0pt}{mylenmafr},
	sharp corners,
	detach title,
	before upper = \tcbtitle\par\smallskip,
	coltitle = mylenmafr,
	fonttitle = \bfseries\sffamily,
	description font = \mdseries,
	separator sign none,
	segmentation style={solid, mylenmafr},
}
{th}


%================================
% PROPOSITION
%================================

\tcbuselibrary{theorems,skins,hooks}
\newtcbtheorem[number within=section]{Prop}{Proposition}
{%
	enhanced,
	breakable,
	colback = mypropbg,
	frame hidden,
	boxrule = 0sp,
	borderline west = {2pt}{0pt}{mypropfr},
	sharp corners,
	detach title,
	before upper = \tcbtitle\par\smallskip,
	coltitle = mypropfr,
	fonttitle = \bfseries\sffamily,
	description font = \mdseries,
	separator sign none,
	segmentation style={solid, mypropfr},
}
{th}

\tcbuselibrary{theorems,skins,hooks}
\newtcbtheorem[number within=chapter]{prop}{Proposition}
{%
	enhanced,
	breakable,
	colback = mypropbg,
	frame hidden,
	boxrule = 0sp,
	borderline west = {2pt}{0pt}{mypropfr},
	sharp corners,
	detach title,
	before upper = \tcbtitle\par\smallskip,
	coltitle = mypropfr,
	fonttitle = \bfseries\sffamily,
	description font = \mdseries,
	separator sign none,
	segmentation style={solid, mypropfr},
}
{th}


%================================
% CLAIM
%================================

\tcbuselibrary{theorems,skins,hooks}
\newtcbtheorem[number within=section]{claim}{Claim}
{%
	enhanced
	,breakable
	,colback = myg!10
	,frame hidden
	,boxrule = 0sp
	,borderline west = {2pt}{0pt}{myg}
	,sharp corners
	,detach title
	,before upper = \tcbtitle\par\smallskip
	,coltitle = myg!85!black
	,fonttitle = \bfseries\sffamily
	,description font = \mdseries
	,separator sign none
	,segmentation style={solid, myg!85!black}
}
{th}



%================================
% Exercise
%================================

\tcbuselibrary{theorems,skins,hooks}
\newtcbtheorem[number within=section]{Exercise}{Exercise}
{%
	enhanced,
	breakable,
	colback = myexercisebg,
	frame hidden,
	boxrule = 0sp,
	borderline west = {2pt}{0pt}{myexercisefg},
	sharp corners,
	detach title,
	before upper = \tcbtitle\par\smallskip,
	coltitle = myexercisefg,
	fonttitle = \bfseries\sffamily,
	description font = \mdseries,
	separator sign none,
	segmentation style={solid, myexercisefg},
}
{th}

\tcbuselibrary{theorems,skins,hooks}
\newtcbtheorem[number within=chapter]{exercise}{Exercise}
{%
	enhanced,
	breakable,
	colback = myexercisebg,
	frame hidden,
	boxrule = 0sp,
	borderline west = {2pt}{0pt}{myexercisefg},
	sharp corners,
	detach title,
	before upper = \tcbtitle\par\smallskip,
	coltitle = myexercisefg,
	fonttitle = \bfseries\sffamily,
	description font = \mdseries,
	separator sign none,
	segmentation style={solid, myexercisefg},
}
{th}

%================================
% EXAMPLE BOX
%================================

\newtcbtheorem[number within=section]{Example}{Example}
{%
	colback = myexamplebg
	,breakable
	,colframe = myexamplefr
	,coltitle = myexampleti
	,boxrule = 1pt
	,sharp corners
	,detach title
	,before upper=\tcbtitle\par\smallskip
	,fonttitle = \bfseries
	,description font = \mdseries
	,separator sign none
	,description delimiters parenthesis
}
{ex}

\newtcbtheorem[number within=chapter]{example}{Example}
{%
	colback = myexamplebg
	,breakable
	,colframe = myexamplefr
	,coltitle = myexampleti
	,boxrule = 1pt
	,sharp corners
	,detach title
	,before upper=\tcbtitle\par\smallskip
	,fonttitle = \bfseries
	,description font = \mdseries
	,separator sign none
	,description delimiters parenthesis
}
{ex}

%================================
% DEFINITION BOX
%================================

\newtcbtheorem[number within=section]{Definition}{Definition}{enhanced,
	before skip=2mm,after skip=2mm, colback=red!5,colframe=red!80!black,boxrule=0.5mm,
	attach boxed title to top left={xshift=1cm,yshift*=1mm-\tcboxedtitleheight}, varwidth boxed title*=-3cm,
	boxed title style={frame code={
					\path[fill=tcbcolback]
					([yshift=-1mm,xshift=-1mm]frame.north west)
					arc[start angle=0,end angle=180,radius=1mm]
					([yshift=-1mm,xshift=1mm]frame.north east)
					arc[start angle=180,end angle=0,radius=1mm];
					\path[left color=tcbcolback!60!black,right color=tcbcolback!60!black,
						middle color=tcbcolback!80!black]
					([xshift=-2mm]frame.north west) -- ([xshift=2mm]frame.north east)
					[rounded corners=1mm]-- ([xshift=1mm,yshift=-1mm]frame.north east)
					-- (frame.south east) -- (frame.south west)
					-- ([xshift=-1mm,yshift=-1mm]frame.north west)
					[sharp corners]-- cycle;
				},interior engine=empty,
		},
	fonttitle=\bfseries,
	title={#2},#1}{def}
\newtcbtheorem[number within=chapter]{definition}{Definition}{enhanced,
	before skip=2mm,after skip=2mm, colback=red!5,colframe=red!80!black,boxrule=0.5mm,
	attach boxed title to top left={xshift=1cm,yshift*=1mm-\tcboxedtitleheight}, varwidth boxed title*=-3cm,
	boxed title style={frame code={
					\path[fill=tcbcolback]
					([yshift=-1mm,xshift=-1mm]frame.north west)
					arc[start angle=0,end angle=180,radius=1mm]
					([yshift=-1mm,xshift=1mm]frame.north east)
					arc[start angle=180,end angle=0,radius=1mm];
					\path[left color=tcbcolback!60!black,right color=tcbcolback!60!black,
						middle color=tcbcolback!80!black]
					([xshift=-2mm]frame.north west) -- ([xshift=2mm]frame.north east)
					[rounded corners=1mm]-- ([xshift=1mm,yshift=-1mm]frame.north east)
					-- (frame.south east) -- (frame.south west)
					-- ([xshift=-1mm,yshift=-1mm]frame.north west)
					[sharp corners]-- cycle;
				},interior engine=empty,
		},
	fonttitle=\bfseries,
	title={#2},#1}{def}



%================================
% Solution BOX
%================================

\makeatletter
\newtcbtheorem{question}{Question}{enhanced,
	breakable,
	colback=white,
	colframe=myb!80!black,
	attach boxed title to top left={yshift*=-\tcboxedtitleheight},
	fonttitle=\bfseries,
	title={#2},
	boxed title size=title,
	boxed title style={%
			sharp corners,
			rounded corners=northwest,
			colback=tcbcolframe,
			boxrule=0pt,
		},
	underlay boxed title={%
			\path[fill=tcbcolframe] (title.south west)--(title.south east)
			to[out=0, in=180] ([xshift=5mm]title.east)--
			(title.center-|frame.east)
			[rounded corners=\kvtcb@arc] |-
			(frame.north) -| cycle;
		},
	#1
}{def}
\makeatother

%================================
% SOLUTION BOX
%================================

\makeatletter
\newtcolorbox{solution}{enhanced,
	breakable,
	colback=white,
	colframe=myg!80!black,
	attach boxed title to top left={yshift*=-\tcboxedtitleheight},
	title=Solution,
	boxed title size=title,
	boxed title style={%
			sharp corners,
			rounded corners=northwest,
			colback=tcbcolframe,
			boxrule=0pt,
		},
	underlay boxed title={%
			\path[fill=tcbcolframe] (title.south west)--(title.south east)
			to[out=0, in=180] ([xshift=5mm]title.east)--
			(title.center-|frame.east)
			[rounded corners=\kvtcb@arc] |-
			(frame.north) -| cycle;
		},
}
\makeatother

%================================
% Question BOX
%================================

\makeatletter
\newtcbtheorem{qstion}{Question}{enhanced,
	breakable,
	colback=white,
	colframe=mygr,
	attach boxed title to top left={yshift*=-\tcboxedtitleheight},
	fonttitle=\bfseries,
	title={#2},
	boxed title size=title,
	boxed title style={%
			sharp corners,
			rounded corners=northwest,
			colback=tcbcolframe,
			boxrule=0pt,
		},
	underlay boxed title={%
			\path[fill=tcbcolframe] (title.south west)--(title.south east)
			to[out=0, in=180] ([xshift=5mm]title.east)--
			(title.center-|frame.east)
			[rounded corners=\kvtcb@arc] |-
			(frame.north) -| cycle;
		},
	#1
}{def}
\makeatother

\newtcbtheorem[number within=chapter]{wconc}{Wrong Concept}{
	breakable,
	enhanced,
	colback=white,
	colframe=myr,
	arc=0pt,
	outer arc=0pt,
	fonttitle=\bfseries\sffamily\large,
	colbacktitle=myr,
	attach boxed title to top left={},
	boxed title style={
			enhanced,
			skin=enhancedfirst jigsaw,
			arc=3pt,
			bottom=0pt,
			interior style={fill=myr}
		},
	#1
}{def}



%================================
% NOTE BOX
%================================

\usetikzlibrary{arrows,calc,shadows.blur}
\tcbuselibrary{skins}
\newtcolorbox{note}[1][]{%
	enhanced jigsaw,
	colback=gray!20!white,%
	colframe=gray!80!black,
	size=small,
	boxrule=1pt,
	title=\textbf{Note:-},
	halign title=flush center,
	coltitle=black,
	breakable,
	drop shadow=black!50!white,
	attach boxed title to top left={xshift=1cm,yshift=-\tcboxedtitleheight/2,yshifttext=-\tcboxedtitleheight/2},
	minipage boxed title=1.5cm,
	boxed title style={%
			colback=white,
			size=fbox,
			boxrule=1pt,
			boxsep=2pt,
			underlay={%
					\coordinate (dotA) at ($(interior.west) + (-0.5pt,0)$);
					\coordinate (dotB) at ($(interior.east) + (0.5pt,0)$);
					\begin{scope}
						\clip (interior.north west) rectangle ([xshift=3ex]interior.east);
						\filldraw [white, blur shadow={shadow opacity=60, shadow yshift=-.75ex}, rounded corners=2pt] (interior.north west) rectangle (interior.south east);
					\end{scope}
					\begin{scope}[gray!80!black]
						\fill (dotA) circle (2pt);
						\fill (dotB) circle (2pt);
					\end{scope}
				},
		},
	#1,
}

%%%%%%%%%%%%%%%%%%%%%%%%%%%%%%
% SELF MADE COMMANDS
%%%%%%%%%%%%%%%%%%%%%%%%%%%%%%


\newcommand{\thm}[2]{\begin{Theorem}{#1}{}#2\end{Theorem}}
\newcommand{\cor}[2]{\begin{Corollary}{#1}{}#2\end{Corollary}}
\newcommand{\mlenma}[2]{\begin{Lenma}{#1}{}#2\end{Lenma}}
\newcommand{\mprop}[2]{\begin{Prop}{#1}{}#2\end{Prop}}
\newcommand{\clm}[3]{\begin{claim}{#1}{#2}#3\end{claim}}
\newcommand{\wc}[2]{\begin{wconc}{#1}{}\setlength{\parindent}{1cm}#2\end{wconc}}
\newcommand{\thmcon}[1]{\begin{Theoremcon}{#1}\end{Theoremcon}}
\newcommand{\ex}[2]{\begin{Example}{#1}{}#2\end{Example}}
\newcommand{\dfn}[2]{\begin{Definition}[colbacktitle=red!75!black]{#1}{}#2\end{Definition}}
\newcommand{\dfnc}[2]{\begin{definition}[colbacktitle=red!75!black]{#1}{}#2\end{definition}}
\newcommand{\qs}[2]{\begin{question}{#1}{}#2\end{question}}
\newcommand{\pf}[2]{\begin{myproof}[#1]#2\end{myproof}}
\newcommand{\nt}[1]{\begin{note}#1\end{note}}

\newcommand*\circled[1]{\tikz[baseline=(char.base)]{
		\node[shape=circle,draw,inner sep=1pt] (char) {#1};}}
\newcommand\getcurrentref[1]{%
	\ifnumequal{\value{#1}}{0}
	{??}
	{\the\value{#1}}%
}
\newcommand{\getCurrentSectionNumber}{\getcurrentref{section}}
\newenvironment{myproof}[1][\proofname]{%
	\proof[\bfseries #1: ]%
}{\endproof}

\newcommand{\mclm}[2]{\begin{myclaim}[#1]#2\end{myclaim}}
\newenvironment{myclaim}[1][\claimname]{\proof[\bfseries #1: ]}{}

\newcounter{mylabelcounter}

\makeatletter
\newcommand{\setword}[2]{%
	\phantomsection
	#1\def\@currentlabel{\unexpanded{#1}}\label{#2}%
}
\makeatother




\tikzset{
	symbol/.style={
			draw=none,
			every to/.append style={
					edge node={node [sloped, allow upside down, auto=false]{$#1$}}}
		}
}


% deliminators
\DeclarePairedDelimiter{\abs}{\lvert}{\rvert}
\DeclarePairedDelimiter{\norm}{\lVert}{\rVert}

\DeclarePairedDelimiter{\ceil}{\lceil}{\rceil}
\DeclarePairedDelimiter{\floor}{\lfloor}{\rfloor}
\DeclarePairedDelimiter{\round}{\lfloor}{\rceil}

\newsavebox\diffdbox
\newcommand{\slantedromand}{{\mathpalette\makesl{d}}}
\newcommand{\makesl}[2]{%
\begingroup
\sbox{\diffdbox}{$\mathsurround=0pt#1\mathrm{#2}$}%
\pdfsave
\pdfsetmatrix{1 0 0.2 1}%
\rlap{\usebox{\diffdbox}}%
\pdfrestore
\hskip\wd\diffdbox
\endgroup
}
\newcommand{\dd}[1][]{\ensuremath{\mathop{}\!\ifstrempty{#1}{%
\slantedromand\@ifnextchar^{\hspace{0.2ex}}{\hspace{0.1ex}}}%
{\slantedromand\hspace{0.2ex}^{#1}}}}
\ProvideDocumentCommand\dv{o m g}{%
  \ensuremath{%
    \IfValueTF{#3}{%
      \IfNoValueTF{#1}{%
        \frac{\dd #2}{\dd #3}%
      }{%
        \frac{\dd^{#1} #2}{\dd #3^{#1}}%
      }%
    }{%
      \IfNoValueTF{#1}{%
        \frac{\dd}{\dd #2}%
      }{%
        \frac{\dd^{#1}}{\dd #2^{#1}}%
      }%
    }%
  }%
}
\providecommand*{\pdv}[3][]{\frac{\partial^{#1}#2}{\partial#3^{#1}}}
%  - others
\DeclareMathOperator{\Lap}{\mathcal{L}}
\DeclareMathOperator{\Var}{Var} % varience
\DeclareMathOperator{\Cov}{Cov} % covarience
\DeclareMathOperator{\E}{E} % expected

% Since the amsthm package isn't loaded

% I prefer the slanted \leq
\let\oldleq\leq % save them in case they're every wanted
\let\oldgeq\geq
\renewcommand{\leq}{\leqslant}
\renewcommand{\geq}{\geqslant}

% % redefine matrix env to allow for alignment, use r as default
% \renewcommand*\env@matrix[1][r]{\hskip -\arraycolsep
%     \let\@ifnextchar\new@ifnextchar
%     \array{*\c@MaxMatrixCols #1}}


%\usepackage{framed}
%\usepackage{titletoc}
%\usepackage{etoolbox}
%\usepackage{lmodern}


%\patchcmd{\tableofcontents}{\contentsname}{\sffamily\contentsname}{}{}

%\renewenvironment{leftbar}
%{\def\FrameCommand{\hspace{6em}%
%		{\color{myyellow}\vrule width 2pt depth 6pt}\hspace{1em}}%
%	\MakeFramed{\parshape 1 0cm \dimexpr\textwidth-6em\relax\FrameRestore}\vskip2pt%
%}
%{\endMakeFramed}

%\titlecontents{chapter}
%[0em]{\vspace*{2\baselineskip}}
%{\parbox{4.5em}{%
%		\hfill\Huge\sffamily\bfseries\color{myred}\thecontentspage}%
%	\vspace*{-2.3\baselineskip}\leftbar\textsc{\small\chaptername~\thecontentslabel}\\\sffamily}
%{}{\endleftbar}
%\titlecontents{section}
%[8.4em]
%{\sffamily\contentslabel{3em}}{}{}
%{\hspace{0.5em}\nobreak\itshape\color{myred}\contentspage}
%\titlecontents{subsection}
%[8.4em]
%{\sffamily\contentslabel{3em}}{}{}  
%{\hspace{0.5em}\nobreak\itshape\color{myred}\contentspage}



%%%%%%%%%%%%%%%%%%%%%%%%%%%%%%%%%%%%%%%%%%%
% TABLE OF CONTENTS
%%%%%%%%%%%%%%%%%%%%%%%%%%%%%%%%%%%%%%%%%%%

\usepackage{tikz}
\definecolor{doc}{RGB}{0,60,110}
\usepackage{titletoc}
\contentsmargin{0cm}
\titlecontents{chapter}[3.7pc]
{\addvspace{30pt}%
	\begin{tikzpicture}[remember picture, overlay]%
		\draw[fill=doc!60,draw=doc!60] (-7,-.1) rectangle (-0.9,.5);%
		\pgftext[left,x=-3.5cm,y=0.2cm]{\color{white}\Large\sc\bfseries Chapter\ \thecontentslabel};%
	\end{tikzpicture}\color{doc!60}\large\sc\bfseries}%
{}
{}
{\;\titlerule\;\large\sc\bfseries Page \thecontentspage
	\begin{tikzpicture}[remember picture, overlay]
		\draw[fill=doc!60,draw=doc!60] (2pt,0) rectangle (4,0.1pt);
	\end{tikzpicture}}%
\titlecontents{section}[3.7pc]
{\addvspace{2pt}}
{\contentslabel[\thecontentslabel]{2pc}}
{}
{\hfill\small \thecontentspage}
[]
\titlecontents*{subsection}[3.7pc]
{\addvspace{-1pt}\small}
{}
{}
{\ --- \small\thecontentspage}
[ \textbullet\ ][]

\makeatletter
\renewcommand{\tableofcontents}{%
	\chapter*{%
	  \vspace*{-20\p@}%
	  \begin{tikzpicture}[remember picture, overlay]%
		  \pgftext[right,x=15cm,y=0.2cm]{\color{doc!60}\Huge\sc\bfseries \contentsname};%
		  \draw[fill=doc!60,draw=doc!60] (13,-.75) rectangle (20,1);%
		  \clip (13,-.75) rectangle (20,1);
		  \pgftext[right,x=15cm,y=0.2cm]{\color{white}\Huge\sc\bfseries \contentsname};%
	  \end{tikzpicture}}%
	\@starttoc{toc}}
\makeatother


%From M275 "Topology" at SJSU
\newcommand{\id}{\mathrm{id}}
\newcommand{\taking}[1]{\xrightarrow{#1}}
\newcommand{\inv}{^{-1}}

%From M170 "Introduction to Graph Theory" at SJSU
\DeclareMathOperator{\diam}{diam}
\DeclareMathOperator{\ord}{ord}
\newcommand{\defeq}{\overset{\mathrm{def}}{=}}

%From the USAMO .tex files
\newcommand{\ts}{\textsuperscript}
\newcommand{\dg}{^\circ}
\newcommand{\ii}{\item}

% % From Math 55 and Math 145 at Harvard
% \newenvironment{subproof}[1][Proof]{%
% \begin{proof}[#1] \renewcommand{\qedsymbol}{$\blacksquare$}}%
% {\end{proof}}

\newcommand{\liff}{\leftrightarrow}
\newcommand{\lthen}{\rightarrow}
\newcommand{\opname}{\operatorname}
\newcommand{\surjto}{\twoheadrightarrow}
\newcommand{\injto}{\hookrightarrow}
\newcommand{\On}{\mathrm{On}} % ordinals
\DeclareMathOperator{\img}{im} % Image
\DeclareMathOperator{\Img}{Im} % Image
\DeclareMathOperator{\coker}{coker} % Cokernel
\DeclareMathOperator{\Coker}{Coker} % Cokernel
\DeclareMathOperator{\Ker}{Ker} % Kernel
\DeclareMathOperator{\rank}{rank}
\DeclareMathOperator{\Spec}{Spec} % spectrum
\DeclareMathOperator{\Tr}{Tr} % trace
\DeclareMathOperator{\pr}{pr} % projection
\DeclareMathOperator{\ext}{ext} % extension
\DeclareMathOperator{\pred}{pred} % predecessor
\DeclareMathOperator{\dom}{dom} % domain
\DeclareMathOperator{\ran}{ran} % range
\DeclareMathOperator{\Hom}{Hom} % homomorphism
\DeclareMathOperator{\Mor}{Mor} % morphisms
\DeclareMathOperator{\End}{End} % endomorphism

\newcommand{\eps}{\epsilon}
\newcommand{\veps}{\varepsilon}
\newcommand{\ol}{\overline}
\newcommand{\ul}{\underline}
\newcommand{\wt}{\widetilde}
\newcommand{\wh}{\widehat}
\newcommand{\vocab}[1]{\textbf{\color{blue} #1}}
\providecommand{\half}{\frac{1}{2}}
\newcommand{\dang}{\measuredangle} %% Directed angle
\newcommand{\ray}[1]{\overrightarrow{#1}}
\newcommand{\seg}[1]{\overline{#1}}
\newcommand{\arc}[1]{\wideparen{#1}}
\DeclareMathOperator{\cis}{cis}
\DeclareMathOperator*{\lcm}{lcm}
\DeclareMathOperator*{\argmin}{arg min}
\DeclareMathOperator*{\argmax}{arg max}
\newcommand{\cycsum}{\sum_{\mathrm{cyc}}}
\newcommand{\symsum}{\sum_{\mathrm{sym}}}
\newcommand{\cycprod}{\prod_{\mathrm{cyc}}}
\newcommand{\symprod}{\prod_{\mathrm{sym}}}
\newcommand{\Qed}{\begin{flushright}\qed\end{flushright}}
\newcommand{\parinn}{\setlength{\parindent}{1cm}}
\newcommand{\parinf}{\setlength{\parindent}{0cm}}
% \newcommand{\norm}{\|\cdot\|}
\newcommand{\inorm}{\norm_{\infty}}
\newcommand{\opensets}{\{V_{\alpha}\}_{\alpha\in I}}
\newcommand{\oset}{V_{\alpha}}
\newcommand{\opset}[1]{V_{\alpha_{#1}}}
\newcommand{\lub}{\text{lub}}
\newcommand{\del}[2]{\frac{\partial #1}{\partial #2}}
\newcommand{\Del}[3]{\frac{\partial^{#1} #2}{\partial^{#1} #3}}
\newcommand{\deld}[2]{\dfrac{\partial #1}{\partial #2}}
\newcommand{\Deld}[3]{\dfrac{\partial^{#1} #2}{\partial^{#1} #3}}
\newcommand{\lm}{\lambda}
\newcommand{\uin}{\mathbin{\rotatebox[origin=c]{90}{$\in$}}}
\newcommand{\usubset}{\mathbin{\rotatebox[origin=c]{90}{$\subset$}}}
\newcommand{\lt}{\left}
\newcommand{\rt}{\right}
\newcommand{\bs}[1]{\boldsymbol{#1}}
\newcommand{\exs}{\exists}
\newcommand{\st}{\strut}
\newcommand{\dps}[1]{\displaystyle{#1}}

\newcommand{\sol}{\setlength{\parindent}{0cm}\textbf{\textit{Solution:}}\setlength{\parindent}{1cm} }
\newcommand{\solve}[1]{\setlength{\parindent}{0cm}\textbf{\textit{Solution: }}\setlength{\parindent}{1cm}#1 \Qed}

%--------------------------------------------------
% LIE ALGEBRAS
%--------------------------------------------------
\newcommand*{\kb}{\mathfrak{b}}  % Borel subalgebra
\newcommand*{\kg}{\mathfrak{g}}  % Lie algebra
\newcommand*{\kh}{\mathfrak{h}}  % Cartan subalgebra
\newcommand*{\kn}{\mathfrak{n}}  % Nilradical
\newcommand*{\ku}{\mathfrak{u}}  % Unipotent algebra
\newcommand*{\kz}{\mathfrak{z}}  % Center of algebra

%--------------------------------------------------
% HOMOLOGICAL ALGEBRA
%--------------------------------------------------
\DeclareMathOperator{\Ext}{Ext} % Ext functor
\DeclareMathOperator{\Tor}{Tor} % Tor functor

%--------------------------------------------------
% MATRIX & GROUP NOTATION
%--------------------------------------------------
\DeclareMathOperator{\GL}{GL} % General Linear Group
\DeclareMathOperator{\SL}{SL} % Special Linear Group
\newcommand*{\gl}{\operatorname{\mathfrak{gl}}} % General linear Lie algebra
\newcommand*{\sl}{\operatorname{\mathfrak{sl}}} % Special linear Lie algebra

%--------------------------------------------------
% NUMBER SETS
%--------------------------------------------------
\newcommand*{\RR}{\mathbb{R}}
\newcommand*{\NN}{\mathbb{N}}
\newcommand*{\ZZ}{\mathbb{Z}}
\newcommand*{\QQ}{\mathbb{Q}}
\newcommand*{\CC}{\mathbb{C}}
\newcommand*{\PP}{\mathbb{P}}
\newcommand*{\HH}{\mathbb{H}}
\newcommand*{\FF}{\mathbb{F}}
\newcommand*{\EE}{\mathbb{E}} % Expected Value

%--------------------------------------------------
% MATH SCRIPT, FRAKTUR, AND BOLD SYMBOLS
%--------------------------------------------------
\newcommand*{\mcA}{\mathcal{A}}
\newcommand*{\mcB}{\mathcal{B}}
\newcommand*{\mcC}{\mathcal{C}}
\newcommand*{\mcD}{\mathcal{D}}
\newcommand*{\mcE}{\mathcal{E}}
\newcommand*{\mcF}{\mathcal{F}}
\newcommand*{\mcG}{\mathcal{G}}
\newcommand*{\mcH}{\mathcal{H}}

\newcommand*{\mfA}{\mathfrak{A}}  \newcommand*{\mfB}{\mathfrak{B}}
\newcommand*{\mfC}{\mathfrak{C}}  \newcommand*{\mfD}{\mathfrak{D}}
\newcommand*{\mfE}{\mathfrak{E}}  \newcommand*{\mfF}{\mathfrak{F}}
\newcommand*{\mfG}{\mathfrak{G}}  \newcommand*{\mfH}{\mathfrak{H}}

\usepackage{bm} % Ensure bold math works correctly
\newcommand*{\bmA}{\bm{A}}
\newcommand*{\bmB}{\bm{B}}
\newcommand*{\bmC}{\bm{C}}
\newcommand*{\bmD}{\bm{D}}
\newcommand*{\bmE}{\bm{E}}
\newcommand*{\bmF}{\bm{F}}
\newcommand*{\bmG}{\bm{G}}
\newcommand*{\bmH}{\bm{H}}

%--------------------------------------------------
% FUNCTIONAL ANALYSIS & ALGEBRA
%--------------------------------------------------
\DeclareMathOperator{\Aut}{Aut} % Automorphism group
\DeclareMathOperator{\Inn}{Inn} % Inner automorphisms
\DeclareMathOperator{\Syl}{Syl} % Sylow subgroups
\DeclareMathOperator{\Gal}{Gal} % Galois group
\DeclareMathOperator{\sign}{sign} % Sign function


%\usepackage[tagged, highstructure]{accessibility}
\usepackage{tocloft}
\usepackage{arydshln}
\usetikzlibrary{arrows.meta, decorations.pathreplacing}
\usepackage{tikz-cd}
\usepackage{polynom}
\usepackage{pifont}
\newcommand{\pistar}{{\zf\symbol{"4A}}}
% a tiny helper for a stretched phantom (for the underbrace)
\newcommand\mc[1]{\multicolumn{1}{c}{#1}}



\begin{document}
\title{Linear Algebra I}
\author{Lecture Notes Provided by Dr.~Miriam Logan.}
\date{}
\maketitle
\tableofcontents
\newpage  
         \ex{}{
          Let $ \mathcal{P} _1 \left[ x \right] $ be the set of polynomials of degree at most 1.\\
          Let $ \langle f,g  \rangle = \int_{0}^{1} f\left( x \right) g \left( x \right) dx $ \\
          $ \langle ,  \rangle $ is an inner product defined on $ \mathcal{P} _1 \left[ x \right] \times  \mathcal{P} _1 \left[ x \right] $ (already prvoen)\\
          \\
          $ \mathcal{B} = \left\{ 1 , x \right\} $ is not orthogonal :\\
          \[
          \langle 1,x  \rangle = \int_{0}^{1} x dx = \frac{ x^2  }{ 2 } \Big|_{0}^{1} = \frac{1}{2}
          .\] 
          Use the Gram-Schmidt orthogonalization process  to orthogonalize $  \mathcal{B}$.\\
          Let $ \vec{ w_1} =1$
          \begin{align*}
           \vec{ w_2} &= x - \frac{ \langle 1,x \rangle }{ \langle 1,1 \rangle } 1\\
           &= x - \frac{ \int_{0}^{1} x dx   }{ \int_{0}^{1} 1 dx } 1  \\
           &= x - \frac{ \frac{1}{2} }{ 1 } 1 \\
           &= x - \frac{1}{2}
          .\end{align*}   
          $ \left\{ 1, x - \frac{1}{2} \right\}$  is an orthogonal basis for $ \mathcal{P} _1 \left[ x \right] $.\\
          To get an orthonormal basis
          \begin{align*}
           \| 1\| &= \sqrt{ \langle 1,1 \rangle } = \sqrt{ \int_{0}^{1} 1 dx } = \sqrt{ 1 } = 1\\
           \|x - \frac{1}{2}\|  &= \sqrt{ \langle x - \frac{1}{2}, x - \frac{1}{2} \rangle } = \sqrt{ \int_{0}^{1} \left( x - \frac{1}{2} \right)^2 dx }\\
           &= \sqrt{ \int_{0}^{1} \left( x^2 - x + \frac{1}{4} \right) dx }= \sqrt{ \frac{ x^3  }{ 3 } - \frac{ x^2  }{ 2 } + \frac{ x  }{ 4 } \Big|_{0}^{1} }\\
           &= \sqrt{ \frac{1}{3} - \frac{1}{2} + \frac{1}{4} } = \sqrt{ \frac{4 - 6 + 3}{12} } = \sqrt{ \frac{1}{12} } 
          .\end{align*}
          \[
          \frac{  \vec{ w_2}   }{ \| \vec{ w_2} \| } = \frac{ x - \frac{1}{2} }{ \sqrt{ \frac{1}{12} } } = \sqrt{ 12 } \left( x - \frac{1}{2} \right) 
          .\] 
          \[
          \text{ orthonormal basis } = \left\{ 1, \sqrt{ 12 } \left( x - \frac{1}{2} \right) \right\}
          .\] 
         }
         \dfn{ :}{
         Suppose $ \langle ,  \rangle : V \times  V \to \mathbb{R}$ is a symmetric bilinear form defined on a real vector space $ V$. \\
         Two vectors $ \vec{ u}, \vec{ v} \in V$ are called orthogonal if $ \langle \vec{ u}, \vec{ v} \rangle = 0$.\\
         }
         \section{Orthogonal basis for symmetric bilinear forms}
         \thm{}
         {
         Suppose $ \langle ,  \rangle $ is a bilinear symmetric form defined on a vector space $ V$. There exists an orthogonal basis 
         \[
         \left\{ \vec{ w_1}, \ldots, \vec{ w_n} \right\} \text{ of } V \qquad  \left(  \text{ i.e. } \langle \vec{ v_i} , \vec{ v_j}\rangle =0 \qquad  \forall  i \neq j  \right) 
         .\] 
         }

         \pf{Proof:}{
         (by induction)\\
         $ n=1$. If $ V$ is one dimensional the statement is true.\\
         \\
         Assume for an $ \left( n-1 \right) $ dimensional vector space $ V$ there exists an orthogonal basis.\\
         \\
         \underline{Case 1:}\\
         $ \langle \vec{ u} ,\vec{ v}   \rangle =0$ $ \forall  \vec{ u} , \vec{ v} \in V$ in which case all bases are orthogonal.\\
         \\
         \underline{Case 2:}\\
         Suppose there exists $ \vec{ u} , \vec{ v} \in V$  such that $ \langle \vec{ u} ,\vec{ v}   \rangle \neq 0$.\\
         Then
         \[
         \langle \vec{ u} + \vec{ v} , \vec{ u} + \vec{ v}   \rangle = \langle \vec{ u} , \vec{ u}   \rangle + 2 \langle \vec{ u} , \vec{ v}  \rangle + \langle \vec{ v} , \vec{ v}   \rangle
         .\] 
         and so one out of $  \langle \vec{ u} + \vec{ v} , \vec{ u} + \vec{ v}   \rangle $, $ \langle \vec{ u}  , \vec{ u}   \rangle $ and  $ \langle \vec{ v} , \vec{ v}   \rangle $ are non-zero.\\
         \[
         \implies \exists  \quad \vec{ v_1} \in V \text{ such that } \langle \vec{ v_1} , \vec{ v_1}   \rangle \neq 0
         .\] 
         Let $ u = span  { \vec{ v_1} }$ and 
         \[
          W = U ^{\perp} = \left\{ \vec{ w} \in V \mid \langle \vec{ w} , \vec{ v_1}   \rangle =0 \right\}
         .\] 
         \underline{Claim:} $ V = U \bigoplus_{} W$ \\
         If this is the case then $ \left\{ \vec{ v_1} , \vec{ v_2} ,\ldots , \vec{ v_n}  \right\} $  is orthogonal basis for $ V$ where $ \left\{ \vec{ v_2} ,\ldots , \vec{ v_n}  \right\} $ is an orthogonal basis for $ W$.\\
         To check $ V = U \bigoplus_{} W$ \\
         Let $ \vec{ v} \in V$, $ \vec{ v} = \underbrace{ \frac{ \langle \vec{ v_1} ,\vec{ v_1}   \rangle   }{ \langle \vec{ v_1} ,\vec{ v_1}   \rangle  } \vec{ v_1}  }_{  \in U} +  \underbrace{ \vec{ v} - \frac{ \langle \vec{ v} , \vec{ v_1}   \rangle   }{ \langle \vec{ v_1} , \vec{ v_1} \rangle  } \vec{ v_1}  }_{ \in U ^{\perp} = W }$\\
         since 
         \begin{align*}
          \langle \vec{ v_1} ,\vec{ v} - \frac{ \langle \vec{ v} , \vec{ v_1}   \rangle   }{ \langle \vec{ v_1} , \vec{ v_1}   \rangle   } \vec{ v_1}   \rangle &= \langle \vec{ v_1} ,\vec{ v}   \rangle - \frac{ \langle \vec{ v} , \vec{ v_1}   \rangle   }{ \langle \vec{ v_1} , \vec{ v_1}   \rangle   }  \langle \vec{ v_1} ,\vec{ v_1}   \rangle  \\
          &= \langle \vec{ v_1} ,\vec{ v}   \rangle - \langle \vec{ v} , \vec{ v_1}   \rangle =0
         .\end{align*}
         Linear cominations are unique $  \iff U \cap W = \{ \vec{ 0} \}$ \\
         Suppose $ \vec{ v} \in U \cap  W$
         \[
         \implies \vec{ v} \in U, \vec{ v} = c \vec{ v_1} \text{ where } c \in \mathbb{R}
         .\] 
         \[
         \vec{ v} \in W \implies \langle \vec{ v} , \vec{ v_1}   \rangle =0
         .\] 
         \[
         \langle c \vec{ v_1} , \vec{ v_1}   \rangle =0 \implies c \langle \vec{ v_1} , \vec{ v_1}   \rangle =0
         .\] 
         \[
         c=0 \qquad  \text{ or } \qquad  \langle \vec{ v_1} , \vec{ v_1}   \rangle =0
         .\] 
         \[
         \text{ Recall } \langle \vec{ v_1} , \vec{ v_1}   \rangle \neq 0 \qquad  \implies c=0 \text{ and } \vec{ v} = 0 \left( \vec{ v_1}  \right) = \vec{ 0} 
         .\] 
         \[
         \text{ i.e.  } U \cap W = \{ \vec{ 0} \} \qquad  \implies V = U \bigoplus_{} W
         .\] 
         
         }
         \thm{}
         {
         Suppose $ A$ is a real $n \times n$  symmetric matrix. Then there exists an invertible matrix $ B$ such that $ B ^{T}A B$ is a diagonal matrix.\\
         }
         \pf{Proof:}{
         If $ A$ is real and symmetric it defineds a symmetric bilinear form on $ \mathbb{R} ^{n}$.\\
         \[
         \langle \vec{ x} , \vec{ y}   \rangle = \left( \vec{ x}  \right) ^{ T} A \vec{ y} \qquad  \forall  \vec{ x} , \vec{ y} \in \mathbb{R} ^{n}
         .\] 
         We know there exists an orthogonal basis for this symmetric bilinear form $  \mathcal{B}= \left\{ \vec{ v_1} ,\ldots , \vec{ v_n}  \right\}  $ with $ \langle \vec{ v_i} , \vec{ v_j}   \rangle =0 \qquad  \forall  i \neq j$.

          \[
 \text{ Let } B
  = \begin{bmatrix}
      \vline           & \vline           & \dots  & \vline \\
      \vec v_{1}  & \vec v_{2}  & \dots  & \vec v_{n} \\[4pt]
      \vline           & \vline           & \dots  & \vline
    \end{bmatrix}
\]

        then the matrix of the bilinear form with respect to $ \mathcal{B}$ is $ B ^{T}A B$\\
        Note: $ \left( B ^{T}A B \right) _{ i j}= \langle \vec{ v_i} , \vec{ v_j}   \rangle  =0 $ if $ i \neq j$ \\
        $ \implies B ^{T}A B$ is a diagonal matrix.\\
         }

 \textit{ We will prove next week that for a symmetric matrix $ A$ with distinct eigenvalues, the eigenvectors of $ A$ are orthogonal.}\\
         
 \\
 \ex{}{
 Let $ A = \begin{bmatrix}
 6 & 2\\
 2 & 3\\
 \end{bmatrix}$. Find a baisis $ \mathcal{B}$ of $ \mathbb{R} $ of $ \mathbb{R} ^2$ such that $ B^{T}A B$ is diagonal.\\
 \begin{align*}
  \chi _A \left( \lambda \right) &= \left( 6-\lambda \right)  \left( 3 - \lambda \right) -4 =0 \\
  &= 18 - 9 \lambda + \lambda ^2 -4 =0 \\
  &= \lambda ^2 - 9 \lambda + 14 =0\\
  &= \left( \lambda - 7 \right) \left( \lambda - 2 \right) =0 \\
  & \implies \lambda _1 = 7, \lambda _2 = 2

 .\end{align*}
 \[
 \lambda=7: \qquad  \left[
 \begin{array}{cc;{2pt/2pt}c}  
   -1 & 2 & 0\\
   2 & -4 & 0\\
 \end{array}
\right] \qquad  x = 2y \qquad  \text{ Let } \vec{ v_1} =  \begin{bmatrix}
2\\
1\\
\end{bmatrix}
 .]
 \[
 \lambda=2: \qquad  \left[
 \begin{array}{cc;{2pt/2pt}c}  
   4 & 2 & 0\\
   2 & 1 & 0\\
 \end{array}
 \right] \qquad  -2x =y \qquad  \text{ Let } \vec{ v_2} =  \begin{bmatrix}
 1\\
 -2\\
 \end{bmatrix}
 .\] 
 Let $ \mathcal{F} = \left\{ \begin{bmatrix}
 2\\
 1\\
 \end{bmatrix}
 , \begin{bmatrix}
 1\\
 -2\\
 \end{bmatrix}
  \right\} $. Clearly $ \mathcal{F}$ is orthogonal .\\
  Let $ B = \begin{bmatrix}
  2 & 1\\
  1 & -2\\
  \end{bmatrix}$ \\

  \begin{align*}
   B ^{T} A B &= \begin{bmatrix}
   2 & 1\\
   1 & -2\\
   \end{bmatrix} \begin{bmatrix}
   6 & 2\\
   2 & 3\\
   \end{bmatrix} \begin{bmatrix}
   2 & 1\\
   1 & -2\\
   \end{bmatrix}\\
   &= \begin{bmatrix}
   14 & 7\\
   2 & -4\\
   \end{bmatrix} \begin{bmatrix}
   2 & 1\\
   1 & -2\\
   \end{bmatrix}= \begin{bmatrix}
   35 & 0\\
   0 & 10\\
   \end{bmatrix}
  .\end{align*}
  Recall: $ B^{-1} A B = \begin{bmatrix}
  7 & 0\\
  0 & 2\\
  \end{bmatrix}$.\\
  In general, if $ \langle ,  \rangle : \mathbb{R} ^{n} \times  \mathbb{R} ^{n} \to \mathbb{R}$ is a bilinear form defined by $ \langle \vec{ x} ,\vec{ y}   \rangle = \left( \vec{ x}  \right) ^{T} A \vec{ y} $ for some $ A \in M_{n\times n} \left( \mathbb{R} \right) $ and $ \mathcal{F} = \left\{ \vec{ v_1} , \ldots , \vec{ v_n}  \right\} $ is a basis consisting of eigenvectors of $ A$ , then
  \begin{align*}
   \langle \vec{ x} ,\vec{ y}   \rangle = \sum\limits_{1 \leq i , j \leq n}^{}  x_i y_j \langle \vec{ v_i} ,\vec{ v_j}   \rangle \\
   &= \sum\limits_{i , j } ^{}  x_i y_j \left( \vec{ v_i}  \right) ^{T} A \vec{ v_j} \\
   &= \sum\limits_{i , j } ^{}  x_i y_j \lambda _i \left( \vec{ v_i}  \right) ^{T} \vec{ v_j} \\
  .\end{align*}
  Therfore, in the example above, with 
  \[
  \vec{ v_1} = \begin{bmatrix}
   2\\
  1\\
  \end{bmatrix}
  , \qquad  \vec{ v_2} = \begin{bmatrix}
  1\\
  -2\\
  \end{bmatrix}
      \qquad  \lambda_1 = 7, \lambda_2 = 2
  .\] 
  \begin{align*}
   \left( B ^{T}A B \right) _{ 1 1} &= \langle \vec{ v_1} , \vec{ v_1}   \rangle = \lambda_1 \left( 2^2+1^2 \right)  = 7 \left( 5 \right) = 35\\
   \left( B ^{T}A B \right) _{ 1 2} &= \langle \vec{ v_1} , \vec{ v_2}   \rangle = 0 = \left( B^{T}A B \right) _{ 2 1}\\
   \left( B ^{T}A B \right) _{ 2 2} &= \langle \vec{ v_2} , \vec{ v_2}   \rangle = \lambda_2 \left( 1^2+(-2)^2 \right)  = 2 \left( 5 \right) = 10
  .\end{align*}
  Motivation for defining and equivalent for on $ \mathbb{C} ^{n} \times  \mathbb{C} ^{n}$.\\
 }
 
    Next we wish to define an equivalent inner product on $ \mathbb{C} ^{n} \times  \mathbb{C} ^{n}$. The inner product $ \langle \vec{ x}, \vec{ y}   \rangle = \left( \vec{ x}  \right) ^{T} \vec{ y} $  defined on $ \mathbb{R} ^{n}$ won't work on $ \mathbb{C} ^{n}$ since 
    \[
    \langle \vec{ x} , \vec{ x}   \rangle = \sum\limits_{i=1}^{n} x_i ^2 \qquad  \text{ could be negative.}
    
    .\] 
    A natural dot product to defind on $ \mathbb{C} ^{n}$ is $ \langle ,  \rangle : \mathbb{C} ^{n} \times  \mathbb{C} ^{n} \to \mathbb{C}$
    \[
    \langle \vec{ z} ,\vec{ w}   \rangle = \sum\limits_{i=1}^{n} \overline{ z_i} w_i = \overline{ \vec{ z}  } ^{T} \vec{ w} 
    .\] 
    \[
    \text{ since }  \langle \vec{ z} , \vec{ z}   \rangle = \sum\limits_{i=1}^{n} \overline{ z_i} z_i = \sum\limits_{i=1}^{n} |z_i|^2 \geq 0 \qquad  \forall  \vec{ z} \in \mathbb{C} ^{n}
    .\] 
    \dfn{ :}{
    A sesquilinear form defined on a complex vector space $ V$  is a map 
    \[
    f : V \times  V \to \mathbb{C}
    .\] 
    such that
    \begin{enumerate}[label=(\arabic*).]  
      \item $ f$ is conjugate linear in the first variable, i.e. $ \forall  \lambda \in \mathbb{C}$, $ \vec{ v_1} , \vec{ v_2} , \vec{ w} \in \mathbb{C} ^{n}$
       \[
       f \left( \lambda \vec{ v_1} + \vec{ v_2} , \vec{ w}  \right) = \overline{ \lambda} f \left( \vec{ v_1} , \vec{ w}  \right) + f \left( \vec{ v_2} , \vec{ w}  \right)
       .\] 
      \item $f$ is linear in the second variable,
       \[
       f \left( \vec{ w} , \lambda \vec{ v_1} + \vec{ v_2}  \right) = \lambda f \left( \vec{ w_1} , \vec{ v_1}  \right) + f \left( \vec{ w_1} , \vec{ v_2}  \right) 
       .\] 
    \end{enumerate}
    
    }
    \ex{}{
    The dot product on $ \mathbb{C} ^{n}$ defined as 
    \[
    \langle \vec{ x} , \vec{ y}   \rangle  = \overline{ \vec{ x}  } ^{T} \vec{ y} = \sum\limits_{i=1}^{n} \overline{ x_i} y_i
    .\] 
    is a sesquilinear:\\
    Let $ \lambda \in \mathbb{C}$, $ \vec{ x} , \vec{ y} , \vec{ u} \in \mathbb{C} ^{n}$ \\
 \begin{align*}
  \langle \lambda \vec{ x} + \vec{ u} , \vec{ y}   \rangle &= \overline{ \left( \lambda \vec{ x} + \vec{ u}  \right) } ^{T} \vec{ y} \\
  &= \overline{ \lambda \left( \vec{ x}  \right) ^{T} + \vec{ u} ^{T} } \vec{ y} \\
  &= \left( \lambda \overline{ \left( \vec{ x}  \right) ^{T}} + \overline{\left( \vec{ u}  \right) ^{T}} \right) \left( \vec{ y}  \right) \\
  &= \overline{ \lambda} \overline{ \left( \vec{ x}  \right) ^{T}} \vec{ y} + \overline{ \left( \vec{ u}  \right) ^{T}} \vec{ y} \\
  &= \overline{ \lambda} \langle \vec{ x} , \vec{ y}   \rangle + \langle \vec{ u} , \vec{ y}   \rangle\\
  \langle \vec{ y} , \lambda \vec{ x} + \vec{ u}   \rangle &= \overline{ \vec{ y}  } ^{T} \left( \lambda \vec{ x} + \vec{ u}  \right)\\
  &= \overline{ \vec{ y}  } ^{T} \left( \lambda \vec{ x}  \right) + \overline{ \vec{ y}  } ^{T} \vec{ u}\\
  &= \lambda \overline{ \vec{ y}  } ^{T} \vec{ x} + \overline{ \vec{ y}  } ^{T} \vec{ u}\\
  &= \lambda \langle \vec{ y} , \vec{ x}   \rangle + \langle \vec{ y} , \vec{ u}   \rangle
 .\end{align*}
    }
    In a similar mannar to the real case, a sesquilinear form can be defined using an $n \times n$  matrix with complex entries aftere fixing a basis $ \mathcal{B} = \left\{  \vec{ b_1} , \ldots , \vec{ b_n}  \right\} $  for $ \mathbb{C} ^{n}$.\\
    \[
    \text{ If } \vec{ x} , \vec{ y} \in \mathbb{C} ^{n}, \qquad  \vec{ x} = \sum\limits_{i=1}^{n} x_i \vec{ b_i} , \qquad  \vec{ y} = \sum\limits_{i=1}^{n} y_i \vec{ b_i}\qquad \forall x_i, y_i \in \mathbb{C}
    .\] 
    
    \begin{align*}
     f \left( \vec{ x} , \vec{ y}  \right) &= f \left( \sum\limits_{i=1}^{n} x_i \vec{ b_i} , \sum\limits_{j=1}^{n} y_j \vec{ b_j}  \right)\\
    &= \sum\limits_{1 \leq i , j \leq n}^{n} \overline{ x_i} y_j f \left( \vec{ b_i} , \vec{ b_j}  \right)= \overline{ \left( \vec{ x}  \right) ^{T} } A \left(\vec{ y}\right) 
    .\end{align*}
    where $ A$ is $n \times n$  matrix whose $ (i,j) $ entry is $ f \left( \vec{ b_i} , \vec{ b_j}  \right)$.\\
    \underline{Notation:} We can use $ \vec{ x} ^{*}$ to denote $ \overline{ \left( \vec{ x}  \right) ^{T}}$ \\
    \mlem{}{
    Suppose $ A$ and $ B$ are $ m \times  k$ and $ k \times  n$ matrices respectively. Then
    \[
     \overline{ AB} = \overline{ A} \overline{ B}
    .\] 
   }
   
   \pf{Proof:}{
      \begin{align*}
       \left( \overline{AB} \right)_{ i j} = \overline{  \left( A B \right) _{ i j}} &= \overline{ \left( \sum\limits_{\ell =1}^{k} a_{ i \ell} b_{ \ell j} \right) } = \sum\limits_{\ell =1}^{k} \overline{ a_{ i \ell} b_{ \ell j}}\\
       &= \sum\limits_{\ell =1}^{k} \overline{ a_{ i \ell}} \overline{ b_{ \ell j}}\\
       &=  \left( \overline{A}  \overline{B}\right) _{ i j}\\
       &\implies \overline{AB} = \overline{A} \overline{B}
      .\end{align*}
   }
   \nt{
   The equivalent of a symmetric bilinear form for complex vector spaces is a Hermitian form.
   }
   Suppose $ V$ is a complex vector space.
   \dfn{Hermitian Form :}{
   A \underline{Hermitian form} is a squesilinear form $ f: V \times  V \to \mathbb{C}$ that satisfies
   \[
   f \left( \vec{ u} ,\vec{ v}  \right) = \overline{ f \left( \vec{ v} ,\vec{ u}  \right)} \qquad  \text{( conjugate symmetry)}
   .\] 
   }
   \thm{}
   {
   Suppose $ f : V \times  V \to \mathbb{C}$ is a sesquilinear form defined on a complex vector space $ V$.  Suppose $ A$ is the matrix of $ f$ with respect to the basis $ \mathcal{B} = \left\{  \vec{ b_1} , \ldots , \vec{ b_n}  \right\} $. Then $ f$ is a Hermitian form if and only if $ A ^{*} = A$  i.e.  $ \overline{A ^{T}} = A$
   }
   \pf{Proof:}{
    \begin{align*}
     f \left( \vec{ u} ,\vec{ v}  \right) &= \overline{f \left( \vec{ v} , \vec{ u}  \right) }\\
     \iff  \vec{ u} ^{*} A \vec{ v} &= \overline{ \vec{ v} ^{*} A \vec{ u} }\\
     \text{ i.e. } \vec{ u} ^{*} A \vec{ v} &= \overline{ \overline{\left( \vec{ v} ^{T} \right) }} \overline{A} \overline{ \vec{ u}}\\
     \vec{ u} ^{*} A \vec{ v} &= \vec{ v} ^{T} \overline{A} \overline{ \vec{ u}}    \\
     \sum\limits_{1 \leq i , j \leq n}^{} \overline{ u_i} v_j \left( A \right) _{ i j} &=  \sum\limits_{1 \leq i , j \leq n}^{} v_i \overline{ u_j} \left( \overline{A} \right) _{ i j}\\
     \text{ which happens } \iff A_{ij} &= \overline{A_{ji}} \\
     \text{ i.e. } A &= \overline{A ^{T}} \\
     A &= A ^{*} 
    .\end{align*}
   }
   \dfn{Hermitian Matrices :}{
   $ A \in M _{ n \times n} \left( \mathbb{C} \right) $ is said to be \underline{Hermitian} if $ A ^{*} = A$ 
   }
   \nt{
   A matrix with real entries, $ A \in M _{ n \times  n } \left( \mathbb{R} \right) $ also defines a sesquilinear form on $ \mathbb{C} ^{n}  \times  \mathbb{C} ^{n} $ 
   \[
   \langle \vec{ x} ,\vec{ y}   \rangle = \overline{ \vec{ x}  } ^{T} A \vec{ y} 
   .\] 
   The form is Hermitian if  $ \overline{A ^{T}}= A$ i.e.  $ A ^{T}= A$ \\
   $ \implies$ a real symmetric matrix defines a bilinear form on $ \mathbb{C} ^{n}  \times  \mathbb{R} ^{n}$ and a Hermitian sesquilinear form on $ \mathbb{C} ^{n}  \times  \mathbb{C} ^{n}$.
   }
   \dfn{ :}{
   An \underline{inner product} $ f : V \times  V \to \mathbb{C}$ on a complex vector space $ V$ is a \underline{Hermitian form} that is positive definite, i.e. $ f \left( \vec{ v} , \vec{ v}  \right) \ge 0 \qquad  \forall  \vec{ v} \in V$ with equality if and only if $ \vec{ v} = \vec{ 0}$.\\
   }
   \ex{}{
     Dot product on $ \mathbb{C} ^{n}$
     \[
     \langle \vec{ x} , \vec{ y}   \rangle = \sum\limits_{i=1}^{n} \overline{x_i} y_i 
     .\] 
     \begin{itemize}
      \item Already show to be a sesquilinear form.
      \item Hermitian?\\
       \[
        \langle \vec{ x} ,\vec{ y}   \rangle = \sum\limits_{i=1}^{n} \overline{x_i} \overline{  \overline{y_i}} = \sum\limits_{i=1}^{n}  \overline{ \overline{y_i} x_i} = \overline{ \sum\limits_{i=1}^{n} \overline{y_i} x_i} = \overline{ \langle \vec{ y} ,\vec{ x}   \rangle }
       .\] 
      \item Positive definite?\\
       \begin{align*}
        \langle \vec{ x} ,\vec{ x}   \rangle &= \sum\limits_{i=1}^{n} | x_i | ^2 \ge 0 \text{ with equality}\\
        \iff x_i &= 0 \qquad  \forall  i \\
       \text{ i.e. } \vec{ x} &= \vec{ 0}
        \implies \text{ The dot product on } \mathbb{C} ^{n} \text{ is an inner product.}
       .\end{align*}
     \end{itemize}
     The matrix of the dot product on $ \mathbb{C} ^{n}$ with respect to the standard basis on $ \mathbb{C} ^{n}, \left\{ \vec{ e_1} , \ldots \vec{ e_n}  \right\} $ is $ A$ where $ A_{ij} = \langle  \vec{ e_i} , \vec{ e_j}   \rangle = \begin{cases}
      0 & \text{if } i\neq j \\
      1 &  \text{if } i=j
     \end{cases}
    \qquad  \implies A = I_n
    $
   }
   As before, with the inner product defined on $ \mathbb{C} ^{n}$, we can define length, $ \forall  \vec{ x} \in \mathbb{C} ^{n}$
   \[
   \| \vec{ x} \|= \sqrt{ \langle \vec{ x} ,\vec{ x}   \rangle } 
   .\] 
   and, the angle between two vectors $ \vec{ x} , \vec{ y} \in \mathbb{C} ^{n}$ as $ \theta $ where $ \langle \vec{ x} ,\vec{ y}   \rangle = \| \vec{ x} \| \| \vec{ y} \| \cos \theta$
   \\
   \mlem{}{
    Suppose $ \langle ,  \rangle : \mathbb{C} ^{n }  \times  \mathbb{C} ^{n} \to \mathbb{C}$   is the dot prodcut on $ \mathbb{C} ^{n}$. Then 
    \begin{enumerate}[label=(\roman*)]
      \item $ \langle \vec{ x} , A \vec{ y}   \rangle = \langle A ^{*} \vec{ x} , \vec{ y}   \rangle $ and 
      \item   $ \langle A \vec{ x} , \vec{ y}   \rangle = \langle \vec{ x} , A ^{*} \vec{ y}   \rangle $
      \end{enumerate}
   }
   \pf{Proof:}{
    \begin{align*}
     \langle \vec{ x} , A \vec{ y}   \rangle &= \left( \vec{ x}  \right) ^{*} A \vec{ y} = \overline{ \left( \vec{ x}  \right) ^{T}} A \vec{ y} \\
     &= \overline{ \left( \vec{ x} ^{T} \right) \left( A ^{T} \right) ^{ T} \vec{ y} \\
      &= \overline{ \left( \vec{ x}  \right) ^{T}} \overline{ \overline{ \left( A^{T} \right) ^{T}}} \vec{ y} \\
      &= \overline{ \left( \vec{ x}  \right) ^{T}} \overline{ \left(  \overline{ \left(  A ^{T} \right) } \right) ^{ T}} \vec{ y } \qquad  \text{ note }  \overline{ B ^{T}} = \overline{ B} ^{T}\\
      &= \overline{ \left( \vec{ x}  \right) ^{T} \left(  \overline{\left( A^{T} \right) } \right) ^{T} } \vec{ y} \\
      &= \left( A ^{*} \vec{ x}  \right) ^{*} \vec{ y} \\
      &= \langle A ^{*} \vec{ x} , \vec{ y}   \rangle
    .\end{align*}
    (ii )  can be shown in a similar manner.\\
   }
   \nt{
   These formulas also hold for the standard inner product on $ \mathbb{R} ^{n}$,\\
   \[
   \langle ,  \rangle : \mathbb{R} ^{n} \times  \mathbb{R} ^{n} \to \mathbb{R}
   .\] 
   \begin{enumerate}[label=(\roman*)]
     \item $ \langle \vec{ x} , A \vec{ y}   \rangle = \langle A ^{T} \vec{ x} , \vec{ y}   \rangle $ 
     \item $ \langle A \vec{ x} , \vec{ y}   \rangle = \langle \vec{ x} , A ^{T} \vec{ y}   \rangle $
     \end{enumerate}
     Same proof as above without complex conjugation.\\
   }
   \thm{}
   {
   Suppose $ A \in M _{ n \times  n} \left( \mathbb{C} \right) $. If $ A ^{*}= A$ ( i.e. $ A$ is Hermitian)  then all eigenvalues of $ A$ are real.\\
   }

   \pf{Proof:}{
   Suppose $ \lambda$ is an eigenvalue of $ A$ with corresponding eigenvector $ \vec{ v} $, i.e.  $ A \vec{ v} = \lambda \vec{ v}  $ \\
   \begin{align*}
    \lambda \langle \vec{ v} , \vec{ v}   \rangle   &= \langle \vec{ v} , \lambda \vec{ v}  \rangle = \langle \vec{ v} , A \vec{ v}   \rangle \\
    &= \langle A ^{*} \vec{ v} , \vec{ v}   \rangle = \langle A \vec{ v} , \vec{ v}   \rangle = \langle \lambda \vec{ v} , \vec{ v}   \rangle\\
    &\implies \lambda \langle \vec{ v} ,\vec{ v}   \rangle = \overline{ \lambda} \langle \vec{ v} ,\vec{ v}   \rangle\\
   .\end{align*}
   Hence $ \lambda = \overline{\lambda}$  i.e.  $  \lambda \in \mathbb{R}$
   }
   \nt{
   The theorem above also holds if $ A$ is a real symmetric matrix, i.e.  $ A ^{T} = A$.\\
   Since $ A$ determines a hermitian sesquilinear form of $ \mathbb{C} ^{n}$ , using the same proof as above, we get that $ \lambda = \overline{\lambda}$ for all eigenvalues $ \lambda $ of $ A$ i.e.  all eigenvalues of $ A$ are real.\\
   }
   \ex{}{
   \[
   A = \begin{bmatrix}
   1 & 2\\
   2 & 4\\
   \end{bmatrix}
   .\] 
   Eigenvalues\\
   \begin{align*}
    \left( 1-\lambda \right) \left( 4 - \lambda \right) &= 0\\
    \lambda^2 - 5 \lambda = 0\\
    \lambda \left( \lambda - 5 \right) &= 0\\
    \lambda_1 &= 0, \lambda_2 = 5\\
   .\end{align*}
   \[
   \lambda=0 \to \vec{ v_1} = \begin{bmatrix}
   -2\\
   1\\
   \end{bmatrix}
   \qquad  \lambda=5 \to \vec{ v_2} = \begin{bmatrix}
   1\\
   2\\
   \end{bmatrix}
   .\] 
   }
   \thm{Orthogonal Eigenvectors :}
   {
   Suppose $ A ^{*} = A$. Suppose $ A \vec{ v_1} = \lambda_1 \vec{ v_1} $ and $ A \vec{ v_2} = \lambda_2 \vec{ v_2} $ with $  \lambda_1 \neq \lambda_2$.\\
   Then $ \langle \vec{ v_1} , \vec{ v_2}   \rangle = 0$ i.e.  $ \vec{ v_1}$ and $ \vec{ v_2}$ are orthogonal.\\
   }

   \pf{Proof:}{
     \begin{align*}
      \lambda_2 \langle \vec{ v_1} , \vec{ v_2}   \rangle &= \langle \vec{ v_1} , A \vec{ v_2}   \rangle \\
      &= \langle A ^{*} \vec{ v_1} , \vec{ v_2}   \rangle \\
      &= \langle A \vec{ v_1} , \vec{ v_2}   \rangle \\
      &= \langle \lambda_1 \vec{ v_1} , \vec{ v_2}   \rangle \\
      &=  \overline{\lambda_1} \langle \vec{ v_1} , \vec{ v_2}   \rangle \qquad  \text{ since } \lambda_1 , \lambda_2 \in \mathbb{R}\\
      \implies \left( \lambda_2 - \lambda_1 \right) \langle \vec{ v_1} , \vec{ v_2}   \rangle &= 0\\
      \text{ since } \lambda_2 \neq \lambda_1 \implies \langle \vec{ v_1} , \vec{ v_2}   \rangle = 0
     .\end{align*}

   }
   Corollary: \\
   If $ A$ is a real symmetric matrix the eigenvectors corresponding to distinct eigenvalues are orthogonal.\\
   \ex{}{
   \[
   A = \begin{bmatrix}
   1 & 2 & 1\\
   2 & 0 & 2\\
   1 & 2 & 1\\
   \end{bmatrix}
   .\] 
   Find $ B$ such that $ B ^{T}A B = \begin{bmatrix}
   0 & 0 & 0\\
   0 & 4 & 0\\
   0 & 0 & -2\\
   \end{bmatrix}$\\
   \\
   \[
   B = \begin{bmatrix}
   - \frac{1}{ \sqrt{2} } & \frac{1}{ \sqrt{3} }  & \frac{1}{ \sqrt{6} }\\
   0 & \frac{1}{\sqrt{3} } & - \frac{ 2  }{ \sqrt{6}  } \\
   \frac{1}{ \sqrt{2} }& \frac{1}{\sqrt{3} } &  \frac{1}{ \sqrt{6} }\\
   \end{bmatrix}  \qquad  \left( B ^{T}A B \right) _{ ij} =0 \forall  i \neq j
   .\] 
   \[
   \left( B ^{T}A B \right) _{ ii} = \lambda_i \left( \| \vec{ v_i} \| ^2 \right)  = \lambda_i
   .\] 
   The eigenvalues of $ A$ are $ 0, 4, -2$ with corresponding eigenvectors \[
   \begin{bmatrix}
   -1\\
   0\\
   1\\
   \end{bmatrix}
    , \qquad  \begin{bmatrix}
    1\\
    1\\
    1\\
    \end{bmatrix}
     , \qquad  \begin{bmatrix}
     1\\
     -2\\
     1\\
     \end{bmatrix}
   .\]
   The eigenvectors corresponding to distinct eigenvalues are linearly independent and orthogonal, and hence the three of them form an orthogonal basis for $ \mathbb{R} ^{3}$.\\
   An orthonormal basis is given by:
   \[
   \mathcal{F} = \left\{ \frac{1}{ \sqrt{2} }  \begin{bmatrix}
   -1\\
   0\\
   1\\
   \end{bmatrix}
    . \frac{1}{ \sqrt{3} } \begin{bmatrix}
    1\\
    1\\
    1\\
    \end{bmatrix}
     , \frac{1}{ \sqrt{6} }  \begin{bmatrix}
     1\\
     -2\\
     1\\
     \end{bmatrix}  \right\} 
   .\] 
   }

   \ex{}{
   \[
   A = \begin{bmatrix}
   3 & 2\\
   2 & 6\\
   \end{bmatrix}
   .\] 
   Find $ B$ such that $ B ^{T}A B $ is a diagonal matrix.\\
   Eigenvalues:\\
   \begin{align*}
   \left( 3 - \lambda \right) \left( 6 - \lambda \right) - 4 &= 0\\
   \lambda^2 - 9 \lambda + 14 &= 0\\
   \left( \lambda-7 \right) \left( \lambda - 2 \right) &= 0\\
   \lambda_1 &= 7, \lambda_2 = 2\\
  \text{ orthogonal eigenvectors} \vec{ v_1} &= \begin{bmatrix}
   1\\
   2\\
   \end{bmatrix}
    \qquad  \vec{ v_2} = \begin{bmatrix}
    2\\
    -1\\
    \end{bmatrix}\\
   \text{ orthonormal eigenvectors} \frac{1}{ \sqrt{5} }\begin{bmatrix}
   1\\
   2\\
   \end{bmatrix}
   , \qquad  \frac{1}{ \sqrt{5} } \begin{bmatrix}
   2\\
   -1\\
   \end{bmatrix}
   .\end{align*}
   Let 
   \begin{align*}
    B &= \frac{1}{ \sqrt{5} } \begin{bmatrix}
    1 & 2\\
    2 & -1\\
    \end{bmatrix}\\
    B ^{T} &= \frac{1}{ \sqrt{5} } \begin{bmatrix}
    1 & 2\\
    2 & -1\\
    \end{bmatrix}\\
    B ^{T}A B &= \begin{bmatrix}
    7 & 0\\
    0 & 2\\
    \end{bmatrix}
   .\end{align*}
   }
   \ex{}{
   Let $ A = \begin{bmatrix}
   0 & 1 & 1\\
   1 & 0 & 1\\
   1 & 1 & 0\\
   \end{bmatrix}$ Find $ B$ such that $ B ^{T}A B$ is a diagonal matrix.\\
   \begin{align*}
    \chi _A \left( \lambda \right) &= - \lambda \left( \lambda ^2 -1 \right) -1 \left( -\lambda -1 \right) +1 \left( 1+\lambda \right) \\
    &= - \lambda \left( \lambda+1 \right) \left( \lambda-1 \right) + \left( 1+\lambda \right) \\
    &= \left( \lambda+1 \right) \left( - \lambda ^2 + \lambda +2 \right) \\
    &= \left( -1 \right) \left( \lambda+1 \right) \left( \lambda-2 \right) \left( \lambda+1 \right) \\
    &= - \left( \lambda+1 \right) ^2 \left( \lambda-2 \right)\\
   .\end{align*}
   \[
   \lambda = -1: \mathcal{N} \left( A +I \right) : \left[
   \begin{array}{ccc;{2pt/2pt}c}  
     1 & 1 & 1 & 0\\
     1 & 1 & 1 & 0\\
     1 & 1 & 1 & 0\\
   \end{array}
   \right] \to \left[
   \begin{array}{ccc;{2pt/2pt}c}  
     1 & 1 & 1 & 0\\
     0 & 0 & 0 & 0\\
     0 & 0 & 0 & 0 \\
   \end{array}
   \right]
   .\] 
   \[
   dim \mathcal{N} \left( A +I \right) =  2
   .\] 
   \[
   \text{ Let } \vec{ v_1} = \begin{bmatrix}
   1\\
   0\\
   -1\\
   \end{bmatrix}
     \qquad  \vec{ v_2} =  \begin{bmatrix}
     1\\
     -1\\
     0\\
     \end{bmatrix}
   .\] 
   \[
   \lambda=2 : \mathcal{N} \left( A -2I \right) : \left[
   \begin{array}{ccc;{2pt/2pt}c}  
     -2 & 1 & 1 & 0\\
     1  & -2  & 1 & 0\\
     1 & 1 & -2 & 0\\
   \end{array}
   \right]            \to \left[
   \begin{array}{ccc;{2pt/2pt}c}  
     1 & 1 & -2 & 0\\
     0 & -3 & 3 & 0\\
     0 & 3 & -3 & 0\\
   \end{array}
   \right]   \to \left[
   \begin{array}{ccc;{2pt/2pt}c}  
     1 & 0 & -1 & 0\\
     0 & 1 & -1 & 0\\
     0 & 0 & 0 & 0\\
   \end{array}
   \right]
   .\]
   $ x = z \qquad  y = z$\\
   Let $ \vec{ v_3} = \begin{bmatrix}
   1\\
   1\\
   1\\
   \end{bmatrix}
    $ \\
    Note: $ \langle \vec{v_1 }, \vec{ v_3}   \rangle =0 $, $ \qquad  \langle \vec{ v_2} , \vec{ v_3}   \rangle =0$ but $ \langle \vec{ v_1} , \vec{ v_2}   \rangle  \neq 0$ \\
    $ \implies$ Apply Gram-Schmidt process to $ \vec{ v_1} , \vec{ v_2}$ to orthogonalize them.\\
    Let $ \vec{ w_1} = \vec{ v_1} $\\
    \[
    \vec{ w_2} = \vec{ v_2} - \frac{ \langle \vec{ v_2} , \vec{ w_1}   \rangle }{ \langle \vec{ w_1} , \vec{ w_1}   \rangle } \vec{ w_1} 
    .\] 
    \[
    \vec{ w_2} = \begin{bmatrix}
    1\\
    -1\\
    0\\
    \end{bmatrix}
     = \frac{1}{2} \begin{bmatrix}
     1\\
     0\\
     -1\\
     \end{bmatrix}
      =  \begin{bmatrix}
       \frac{1}{2}\\
      -1\\
      \frac{1}{2}\\
      \end{bmatrix}
    .\] 
    \[
    \text{ Orthonormal basis } \mathcal{F} = \left\{ \frac{1}{ \sqrt{2} }  \begin{bmatrix}
    1\\
    0\\
    -1\\
    \end{bmatrix}
     , \frac{1}{ \sqrt{6} } \begin{bmatrix}
     1\\
     -2\\
     1\\
     \end{bmatrix}
      , \frac{1}{ \sqrt{3} } \begin{bmatrix}
      1\\
      1\\
      1\\
      \end{bmatrix}
    \right\}
    .\] 
    Let $ B = \begin{bmatrix}
    \frac{1}{ \sqrt{2} } & \frac{1}{ \sqrt{6} } & \frac{1}{ \sqrt{3} }\\
     0 & - \frac{ 2  }{  \sqrt{6}  } & \frac{1}{ \sqrt{3} } \\
     - \frac{1}{ \sqrt{2} }& \frac{1}{ \sqrt{6} }  & \frac{1}{ \sqrt{3} } \\
    \end{bmatrix}$
    \[
    B ^{T}A B = \begin{bmatrix}
    -1 & 0 & 0\\
    0 & -1 & 0\\
    0 & 0 & 2\\
    \end{bmatrix}
    .\] 
   }
   \section{Orthogonal Matrices}
   \dfn{Orthogonal Matrix :}{
     Suppose $ B \in M_{n \times  n}$. $ B$ is said to be \underline{orthogonal} if $ B ^{T} = B ^{-1}$ or $ B^{T}B = I_n$
   }
   \thm{}
   {
   $ B \in M _{ n \times  n}\left( \mathbb{R} \right) $ is orthogonal if and only if the columns of $ B$ are orthonormal 
   }
   \pf{Proof:}{
   $ B ^{T}B = I_n$ \\
   $ \iff \left( B ^{T} B \right) _{ ij} = begin{cases}
    1 & \text{if } i =j \\
    0 & \text{ if }  i \neq j
   \end{cases}
   $    
   \[
    \iff \left( i ^{ \text{th}} \text{row of } B ^{T} \right) \cdot  \left( j ^{ \text{ th}} \text{ column of } B\right)  = begin{cases}
     1 & \text{if } i= j \\
     0 & \text{if } i \neq j
    \end{cases}
    
   .\] 
   \[
    \iff \left( i ^{ \text{ th}} \text{ column of } B \right)  \cdot  \left( j ^{ \text{ th}} \text{ column of } B \right) = begin{cases}
     1 & \text{if } i = j \\
     0 & \text{if } i \neq j
    \end{cases}
    
   .\] 
   \[
    \iff \text{ The columns of } B \text{ are orthonormal.}
   .\] 
   }
   
   \underline{Fact:}\\
   The determinant of an orthogonal matrix $ A \in M _{ n \times  n} \left( \mathbb{R} \right) $ is either $ 1$ or $ -1$.\\
   \pf{Proof:}{
    \[
    A ^{T} A = I_n 
    .\] 
    \[
    \implies \text{ det } \left( A^{T}A \right) = \text{ det } I_n 
    .\]
    \[
    \text{ det } \left( A^{T} \right) \text{ det } A = 1
    .\] 
    \[
    \implies \left(  \text{ det } A \right) ^2 =1 \text{ since } \text{ det } A = \text{ det } \left( A^{T} \right)
    .\]
    \[
    \implies \text{ det } A = \pm 1
    .\] 
   }
   
   \mlem{}{
   Orthogonal matrices preserve the dot product on $  \mathbb{R} ^{n}$
  }
   \pf{Proof:}{
    Suppose $ A$ is an orthogonal matrix, $ \vec{ x} , \vec{ y} \in \mathbb{R} ^{n}$. $ \vec{ x} , \vec{ y } \in \mathbb{R} ^{n}$, $ \langle A \vec{ x} , A \vec{ y}   \rangle = \langle A ^{T} A \vec{ x} , \vec{ y}   \rangle = \langle \vec{ x} , \vec{ y}   \rangle $\\
    In particular, 
    \begin{align*}
     \langle A \vec{ x} , A \vec{ x}   \rangle &= \langle  \vec{ x}  , \vec{ x}  \rangle \\
     \|A \vec{ x} \|^2 &= \| \vec{ x} \| ^2\\
     \implies \| A \vec{ x} \| &= \| \vec{ x} \|\\
    .\end{align*}
   }
   
   
   
    
   
   
   

   
   
   
   
   
   
   
   
   
   
   
   
   
   
   
    
    
         
         
         
          
         
         

















\end{document}  
