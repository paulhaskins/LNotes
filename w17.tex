\documentclass{report}

%%%%%%%%%%%%%%%%%%%%%%%%%%%%%%%%%
% PACKAGE IMPORTS
%%%%%%%%%%%%%%%%%%%%%%%%%%%%%%%%%


\usepackage[tmargin=2cm,rmargin=1in,lmargin=1in,margin=0.85in,bmargin=2cm,footskip=.2in]{geometry}
\usepackage{amsmath,amsfonts,amsthm,amssymb,mathtools}
\usepackage[varbb]{newpxmath}
\usepackage{xfrac}
\usepackage[makeroom]{cancel}
\usepackage{bookmark}
\usepackage{enumitem}
\usepackage{hyperref,theoremref}
\hypersetup{
	pdftitle={Assignment},
	colorlinks=true, linkcolor=doc!90,
	bookmarksnumbered=true,
	bookmarksopen=true
}
\usepackage[most,many,breakable]{tcolorbox}
\usepackage{xcolor}
\usepackage{varwidth}
\usepackage{varwidth}
\usepackage{tocloft}
\usepackage{etoolbox}
\usepackage{derivative} %many derivativess partials
%\usepackage{authblk}
\usepackage{nameref}
\usepackage{multicol,array}
\usepackage{tikz-cd}
\usepackage[ruled,vlined,linesnumbered]{algorithm2e}
\usepackage{comment} % enables the use of multi-line comments (\ifx \fi) 
\usepackage{import}
\usepackage{xifthen}
\usepackage{pdfpages}
\usepackage{transparent}
\usepackage{verbatim}

\newcommand\mycommfont[1]{\footnotesize\ttfamily\textcolor{blue}{#1}}
\SetCommentSty{mycommfont}
\newcommand{\incfig}[1]{%
    \def\svgwidth{\columnwidth}
    \import{./figures/}{#1.pdf_tex}
}
\usepackage[tagged, highstructure]{accessibility}
\usepackage{tikzsymbols}
\renewcommand\qedsymbol{$\Laughey$}


%\usepackage{import}
%\usepackage{xifthen}
%\usepackage{pdfpages}
%\usepackage{transparent}


%%%%%%%%%%%%%%%%%%%%%%%%%%%%%%
% SELF MADE COLORS
%%%%%%%%%%%%%%%%%%%%%%%%%%%%%%



\definecolor{myg}{RGB}{56, 140, 70}
\definecolor{myb}{RGB}{45, 111, 177}
\definecolor{myr}{RGB}{199, 68, 64}
\definecolor{mytheorembg}{HTML}{F2F2F9}
\definecolor{mytheoremfr}{HTML}{00007B}
\definecolor{mylenmabg}{HTML}{FFFAF8}
\definecolor{mylenmafr}{HTML}{983b0f}
\definecolor{mypropbg}{HTML}{f2fbfc}
\definecolor{mypropfr}{HTML}{191971}
\definecolor{myexamplebg}{HTML}{F2FBF8}
\definecolor{myexamplefr}{HTML}{88D6D1}
\definecolor{myexampleti}{HTML}{2A7F7F}
\definecolor{mydefinitbg}{HTML}{E5E5FF}
\definecolor{mydefinitfr}{HTML}{3F3FA3}
\definecolor{notesgreen}{RGB}{0,162,0}
\definecolor{myp}{RGB}{197, 92, 212}
\definecolor{mygr}{HTML}{2C3338}
\definecolor{myred}{RGB}{127,0,0}
\definecolor{myyellow}{RGB}{169,121,69}
\definecolor{myexercisebg}{HTML}{F2FBF8}
\definecolor{myexercisefg}{HTML}{88D6D1}


%%%%%%%%%%%%%%%%%%%%%%%%%%%%
% TCOLORBOX SETUPS
%%%%%%%%%%%%%%%%%%%%%%%%%%%%

\setlength{\parindent}{1cm}
%================================
% THEOREM BOX
%================================

\tcbuselibrary{theorems,skins,hooks}
\newtcbtheorem[number within=section]{Theorem}{Theorem}
{%
	enhanced,
	breakable,
	colback = mytheorembg,
	frame hidden,
	boxrule = 0sp,
	borderline west = {2pt}{0pt}{mytheoremfr},
	sharp corners,
	detach title,
	before upper = \tcbtitle\par\smallskip,
	coltitle = mytheoremfr,
	fonttitle = \bfseries\sffamily,
	description font = \mdseries,
	separator sign none,
	segmentation style={solid, mytheoremfr},
}
{th}

\tcbuselibrary{theorems,skins,hooks}
\newtcbtheorem[number within=chapter]{theorem}{Theorem}
{%
	enhanced,
	breakable,
	colback = mytheorembg,
	frame hidden,
	boxrule = 0sp,
	borderline west = {2pt}{0pt}{mytheoremfr},
	sharp corners,
	detach title,
	before upper = \tcbtitle\par\smallskip,
	coltitle = mytheoremfr,
	fonttitle = \bfseries\sffamily,
	description font = \mdseries,
	separator sign none,
	segmentation style={solid, mytheoremfr},
}
{th}


\tcbuselibrary{theorems,skins,hooks}
\newtcolorbox{Theoremcon}
{%
	enhanced
	,breakable
	,colback = mytheorembg
	,frame hidden
	,boxrule = 0sp
	,borderline west = {2pt}{0pt}{mytheoremfr}
	,sharp corners
	,description font = \mdseries
	,separator sign none
}

%================================
% Corollery
%================================
\tcbuselibrary{theorems,skins,hooks}
\newtcbtheorem[number within=section]{Corollary}{Corollary}
{%
	enhanced
	,breakable
	,colback = myp!10
	,frame hidden
	,boxrule = 0sp
	,borderline west = {2pt}{0pt}{myp!85!black}
	,sharp corners
	,detach title
	,before upper = \tcbtitle\par\smallskip
	,coltitle = myp!85!black
	,fonttitle = \bfseries\sffamily
	,description font = \mdseries
	,separator sign none
	,segmentation style={solid, myp!85!black}
}
{th}
\tcbuselibrary{theorems,skins,hooks}
\newtcbtheorem[number within=chapter]{corollary}{Corollary}
{%
	enhanced
	,breakable
	,colback = myp!10
	,frame hidden
	,boxrule = 0sp
	,borderline west = {2pt}{0pt}{myp!85!black}
	,sharp corners
	,detach title
	,before upper = \tcbtitle\par\smallskip
	,coltitle = myp!85!black
	,fonttitle = \bfseries\sffamily
	,description font = \mdseries
	,separator sign none
	,segmentation style={solid, myp!85!black}
}
{th}


%================================
% LENMA
%================================

\tcbuselibrary{theorems,skins,hooks}
\newtcbtheorem[number within=section]{Lenma}{Lenma}
{%
	enhanced,
	breakable,
	colback = mylenmabg,
	frame hidden,
	boxrule = 0sp,
	borderline west = {2pt}{0pt}{mylenmafr},
	sharp corners,
	detach title,
	before upper = \tcbtitle\par\smallskip,
	coltitle = mylenmafr,
	fonttitle = \bfseries\sffamily,
	description font = \mdseries,
	separator sign none,
	segmentation style={solid, mylenmafr},
}
{th}

\tcbuselibrary{theorems,skins,hooks}
\newtcbtheorem[number within=chapter]{lenma}{Lenma}
{%
	enhanced,
	breakable,
	colback = mylenmabg,
	frame hidden,
	boxrule = 0sp,
	borderline west = {2pt}{0pt}{mylenmafr},
	sharp corners,
	detach title,
	before upper = \tcbtitle\par\smallskip,
	coltitle = mylenmafr,
	fonttitle = \bfseries\sffamily,
	description font = \mdseries,
	separator sign none,
	segmentation style={solid, mylenmafr},
}
{th}


%================================
% PROPOSITION
%================================

\tcbuselibrary{theorems,skins,hooks}
\newtcbtheorem[number within=section]{Prop}{Proposition}
{%
	enhanced,
	breakable,
	colback = mypropbg,
	frame hidden,
	boxrule = 0sp,
	borderline west = {2pt}{0pt}{mypropfr},
	sharp corners,
	detach title,
	before upper = \tcbtitle\par\smallskip,
	coltitle = mypropfr,
	fonttitle = \bfseries\sffamily,
	description font = \mdseries,
	separator sign none,
	segmentation style={solid, mypropfr},
}
{th}

\tcbuselibrary{theorems,skins,hooks}
\newtcbtheorem[number within=chapter]{prop}{Proposition}
{%
	enhanced,
	breakable,
	colback = mypropbg,
	frame hidden,
	boxrule = 0sp,
	borderline west = {2pt}{0pt}{mypropfr},
	sharp corners,
	detach title,
	before upper = \tcbtitle\par\smallskip,
	coltitle = mypropfr,
	fonttitle = \bfseries\sffamily,
	description font = \mdseries,
	separator sign none,
	segmentation style={solid, mypropfr},
}
{th}


%================================
% CLAIM
%================================

\tcbuselibrary{theorems,skins,hooks}
\newtcbtheorem[number within=section]{claim}{Claim}
{%
	enhanced
	,breakable
	,colback = myg!10
	,frame hidden
	,boxrule = 0sp
	,borderline west = {2pt}{0pt}{myg}
	,sharp corners
	,detach title
	,before upper = \tcbtitle\par\smallskip
	,coltitle = myg!85!black
	,fonttitle = \bfseries\sffamily
	,description font = \mdseries
	,separator sign none
	,segmentation style={solid, myg!85!black}
}
{th}



%================================
% Exercise
%================================

\tcbuselibrary{theorems,skins,hooks}
\newtcbtheorem[number within=section]{Exercise}{Exercise}
{%
	enhanced,
	breakable,
	colback = myexercisebg,
	frame hidden,
	boxrule = 0sp,
	borderline west = {2pt}{0pt}{myexercisefg},
	sharp corners,
	detach title,
	before upper = \tcbtitle\par\smallskip,
	coltitle = myexercisefg,
	fonttitle = \bfseries\sffamily,
	description font = \mdseries,
	separator sign none,
	segmentation style={solid, myexercisefg},
}
{th}

\tcbuselibrary{theorems,skins,hooks}
\newtcbtheorem[number within=chapter]{exercise}{Exercise}
{%
	enhanced,
	breakable,
	colback = myexercisebg,
	frame hidden,
	boxrule = 0sp,
	borderline west = {2pt}{0pt}{myexercisefg},
	sharp corners,
	detach title,
	before upper = \tcbtitle\par\smallskip,
	coltitle = myexercisefg,
	fonttitle = \bfseries\sffamily,
	description font = \mdseries,
	separator sign none,
	segmentation style={solid, myexercisefg},
}
{th}

%================================
% EXAMPLE BOX
%================================

\newtcbtheorem[number within=section]{Example}{Example}
{%
	colback = myexamplebg
	,breakable
	,colframe = myexamplefr
	,coltitle = myexampleti
	,boxrule = 1pt
	,sharp corners
	,detach title
	,before upper=\tcbtitle\par\smallskip
	,fonttitle = \bfseries
	,description font = \mdseries
	,separator sign none
	,description delimiters parenthesis
}
{ex}

\newtcbtheorem[number within=chapter]{example}{Example}
{%
	colback = myexamplebg
	,breakable
	,colframe = myexamplefr
	,coltitle = myexampleti
	,boxrule = 1pt
	,sharp corners
	,detach title
	,before upper=\tcbtitle\par\smallskip
	,fonttitle = \bfseries
	,description font = \mdseries
	,separator sign none
	,description delimiters parenthesis
}
{ex}

%================================
% DEFINITION BOX
%================================

\newtcbtheorem[number within=section]{Definition}{Definition}{enhanced,
	before skip=2mm,after skip=2mm, colback=red!5,colframe=red!80!black,boxrule=0.5mm,
	attach boxed title to top left={xshift=1cm,yshift*=1mm-\tcboxedtitleheight}, varwidth boxed title*=-3cm,
	boxed title style={frame code={
					\path[fill=tcbcolback]
					([yshift=-1mm,xshift=-1mm]frame.north west)
					arc[start angle=0,end angle=180,radius=1mm]
					([yshift=-1mm,xshift=1mm]frame.north east)
					arc[start angle=180,end angle=0,radius=1mm];
					\path[left color=tcbcolback!60!black,right color=tcbcolback!60!black,
						middle color=tcbcolback!80!black]
					([xshift=-2mm]frame.north west) -- ([xshift=2mm]frame.north east)
					[rounded corners=1mm]-- ([xshift=1mm,yshift=-1mm]frame.north east)
					-- (frame.south east) -- (frame.south west)
					-- ([xshift=-1mm,yshift=-1mm]frame.north west)
					[sharp corners]-- cycle;
				},interior engine=empty,
		},
	fonttitle=\bfseries,
	title={#2},#1}{def}
\newtcbtheorem[number within=chapter]{definition}{Definition}{enhanced,
	before skip=2mm,after skip=2mm, colback=red!5,colframe=red!80!black,boxrule=0.5mm,
	attach boxed title to top left={xshift=1cm,yshift*=1mm-\tcboxedtitleheight}, varwidth boxed title*=-3cm,
	boxed title style={frame code={
					\path[fill=tcbcolback]
					([yshift=-1mm,xshift=-1mm]frame.north west)
					arc[start angle=0,end angle=180,radius=1mm]
					([yshift=-1mm,xshift=1mm]frame.north east)
					arc[start angle=180,end angle=0,radius=1mm];
					\path[left color=tcbcolback!60!black,right color=tcbcolback!60!black,
						middle color=tcbcolback!80!black]
					([xshift=-2mm]frame.north west) -- ([xshift=2mm]frame.north east)
					[rounded corners=1mm]-- ([xshift=1mm,yshift=-1mm]frame.north east)
					-- (frame.south east) -- (frame.south west)
					-- ([xshift=-1mm,yshift=-1mm]frame.north west)
					[sharp corners]-- cycle;
				},interior engine=empty,
		},
	fonttitle=\bfseries,
	title={#2},#1}{def}



%================================
% Solution BOX
%================================

\makeatletter
\newtcbtheorem{question}{Question}{enhanced,
	breakable,
	colback=white,
	colframe=myb!80!black,
	attach boxed title to top left={yshift*=-\tcboxedtitleheight},
	fonttitle=\bfseries,
	title={#2},
	boxed title size=title,
	boxed title style={%
			sharp corners,
			rounded corners=northwest,
			colback=tcbcolframe,
			boxrule=0pt,
		},
	underlay boxed title={%
			\path[fill=tcbcolframe] (title.south west)--(title.south east)
			to[out=0, in=180] ([xshift=5mm]title.east)--
			(title.center-|frame.east)
			[rounded corners=\kvtcb@arc] |-
			(frame.north) -| cycle;
		},
	#1
}{def}
\makeatother

%================================
% SOLUTION BOX
%================================

\makeatletter
\newtcolorbox{solution}{enhanced,
	breakable,
	colback=white,
	colframe=myg!80!black,
	attach boxed title to top left={yshift*=-\tcboxedtitleheight},
	title=Solution,
	boxed title size=title,
	boxed title style={%
			sharp corners,
			rounded corners=northwest,
			colback=tcbcolframe,
			boxrule=0pt,
		},
	underlay boxed title={%
			\path[fill=tcbcolframe] (title.south west)--(title.south east)
			to[out=0, in=180] ([xshift=5mm]title.east)--
			(title.center-|frame.east)
			[rounded corners=\kvtcb@arc] |-
			(frame.north) -| cycle;
		},
}
\makeatother

%================================
% Question BOX
%================================

\makeatletter
\newtcbtheorem{qstion}{Question}{enhanced,
	breakable,
	colback=white,
	colframe=mygr,
	attach boxed title to top left={yshift*=-\tcboxedtitleheight},
	fonttitle=\bfseries,
	title={#2},
	boxed title size=title,
	boxed title style={%
			sharp corners,
			rounded corners=northwest,
			colback=tcbcolframe,
			boxrule=0pt,
		},
	underlay boxed title={%
			\path[fill=tcbcolframe] (title.south west)--(title.south east)
			to[out=0, in=180] ([xshift=5mm]title.east)--
			(title.center-|frame.east)
			[rounded corners=\kvtcb@arc] |-
			(frame.north) -| cycle;
		},
	#1
}{def}
\makeatother

\newtcbtheorem[number within=chapter]{wconc}{Wrong Concept}{
	breakable,
	enhanced,
	colback=white,
	colframe=myr,
	arc=0pt,
	outer arc=0pt,
	fonttitle=\bfseries\sffamily\large,
	colbacktitle=myr,
	attach boxed title to top left={},
	boxed title style={
			enhanced,
			skin=enhancedfirst jigsaw,
			arc=3pt,
			bottom=0pt,
			interior style={fill=myr}
		},
	#1
}{def}



%================================
% NOTE BOX
%================================

\usetikzlibrary{arrows,calc,shadows.blur}
\tcbuselibrary{skins}
\newtcolorbox{note}[1][]{%
	enhanced jigsaw,
	colback=gray!20!white,%
	colframe=gray!80!black,
	size=small,
	boxrule=1pt,
	title=\textbf{Note:-},
	halign title=flush center,
	coltitle=black,
	breakable,
	drop shadow=black!50!white,
	attach boxed title to top left={xshift=1cm,yshift=-\tcboxedtitleheight/2,yshifttext=-\tcboxedtitleheight/2},
	minipage boxed title=1.5cm,
	boxed title style={%
			colback=white,
			size=fbox,
			boxrule=1pt,
			boxsep=2pt,
			underlay={%
					\coordinate (dotA) at ($(interior.west) + (-0.5pt,0)$);
					\coordinate (dotB) at ($(interior.east) + (0.5pt,0)$);
					\begin{scope}
						\clip (interior.north west) rectangle ([xshift=3ex]interior.east);
						\filldraw [white, blur shadow={shadow opacity=60, shadow yshift=-.75ex}, rounded corners=2pt] (interior.north west) rectangle (interior.south east);
					\end{scope}
					\begin{scope}[gray!80!black]
						\fill (dotA) circle (2pt);
						\fill (dotB) circle (2pt);
					\end{scope}
				},
		},
	#1,
}

%%%%%%%%%%%%%%%%%%%%%%%%%%%%%%
% SELF MADE COMMANDS
%%%%%%%%%%%%%%%%%%%%%%%%%%%%%%


\newcommand{\thm}[2]{\begin{Theorem}{#1}{}#2\end{Theorem}}
\newcommand{\cor}[2]{\begin{Corollary}{#1}{}#2\end{Corollary}}
\newcommand{\mlenma}[2]{\begin{Lenma}{#1}{}#2\end{Lenma}}
\newcommand{\mprop}[2]{\begin{Prop}{#1}{}#2\end{Prop}}
\newcommand{\clm}[3]{\begin{claim}{#1}{#2}#3\end{claim}}
\newcommand{\wc}[2]{\begin{wconc}{#1}{}\setlength{\parindent}{1cm}#2\end{wconc}}
\newcommand{\thmcon}[1]{\begin{Theoremcon}{#1}\end{Theoremcon}}
\newcommand{\ex}[2]{\begin{Example}{#1}{}#2\end{Example}}
\newcommand{\dfn}[2]{\begin{Definition}[colbacktitle=red!75!black]{#1}{}#2\end{Definition}}
\newcommand{\dfnc}[2]{\begin{definition}[colbacktitle=red!75!black]{#1}{}#2\end{definition}}
\newcommand{\qs}[2]{\begin{question}{#1}{}#2\end{question}}
\newcommand{\pf}[2]{\begin{myproof}[#1]#2\end{myproof}}
\newcommand{\nt}[1]{\begin{note}#1\end{note}}

\newcommand*\circled[1]{\tikz[baseline=(char.base)]{
		\node[shape=circle,draw,inner sep=1pt] (char) {#1};}}
\newcommand\getcurrentref[1]{%
	\ifnumequal{\value{#1}}{0}
	{??}
	{\the\value{#1}}%
}
\newcommand{\getCurrentSectionNumber}{\getcurrentref{section}}
\newenvironment{myproof}[1][\proofname]{%
	\proof[\bfseries #1: ]%
}{\endproof}

\newcommand{\mclm}[2]{\begin{myclaim}[#1]#2\end{myclaim}}
\newenvironment{myclaim}[1][\claimname]{\proof[\bfseries #1: ]}{}

\newcounter{mylabelcounter}

\makeatletter
\newcommand{\setword}[2]{%
	\phantomsection
	#1\def\@currentlabel{\unexpanded{#1}}\label{#2}%
}
\makeatother




\tikzset{
	symbol/.style={
			draw=none,
			every to/.append style={
					edge node={node [sloped, allow upside down, auto=false]{$#1$}}}
		}
}


% deliminators
\DeclarePairedDelimiter{\abs}{\lvert}{\rvert}
\DeclarePairedDelimiter{\norm}{\lVert}{\rVert}

\DeclarePairedDelimiter{\ceil}{\lceil}{\rceil}
\DeclarePairedDelimiter{\floor}{\lfloor}{\rfloor}
\DeclarePairedDelimiter{\round}{\lfloor}{\rceil}

\newsavebox\diffdbox
\newcommand{\slantedromand}{{\mathpalette\makesl{d}}}
\newcommand{\makesl}[2]{%
\begingroup
\sbox{\diffdbox}{$\mathsurround=0pt#1\mathrm{#2}$}%
\pdfsave
\pdfsetmatrix{1 0 0.2 1}%
\rlap{\usebox{\diffdbox}}%
\pdfrestore
\hskip\wd\diffdbox
\endgroup
}
\newcommand{\dd}[1][]{\ensuremath{\mathop{}\!\ifstrempty{#1}{%
\slantedromand\@ifnextchar^{\hspace{0.2ex}}{\hspace{0.1ex}}}%
{\slantedromand\hspace{0.2ex}^{#1}}}}
\ProvideDocumentCommand\dv{o m g}{%
  \ensuremath{%
    \IfValueTF{#3}{%
      \IfNoValueTF{#1}{%
        \frac{\dd #2}{\dd #3}%
      }{%
        \frac{\dd^{#1} #2}{\dd #3^{#1}}%
      }%
    }{%
      \IfNoValueTF{#1}{%
        \frac{\dd}{\dd #2}%
      }{%
        \frac{\dd^{#1}}{\dd #2^{#1}}%
      }%
    }%
  }%
}
\providecommand*{\pdv}[3][]{\frac{\partial^{#1}#2}{\partial#3^{#1}}}
%  - others
\DeclareMathOperator{\Lap}{\mathcal{L}}
\DeclareMathOperator{\Var}{Var} % varience
\DeclareMathOperator{\Cov}{Cov} % covarience
\DeclareMathOperator{\E}{E} % expected

% Since the amsthm package isn't loaded

% I prefer the slanted \leq
\let\oldleq\leq % save them in case they're every wanted
\let\oldgeq\geq
\renewcommand{\leq}{\leqslant}
\renewcommand{\geq}{\geqslant}

% % redefine matrix env to allow for alignment, use r as default
% \renewcommand*\env@matrix[1][r]{\hskip -\arraycolsep
%     \let\@ifnextchar\new@ifnextchar
%     \array{*\c@MaxMatrixCols #1}}


%\usepackage{framed}
%\usepackage{titletoc}
%\usepackage{etoolbox}
%\usepackage{lmodern}


%\patchcmd{\tableofcontents}{\contentsname}{\sffamily\contentsname}{}{}

%\renewenvironment{leftbar}
%{\def\FrameCommand{\hspace{6em}%
%		{\color{myyellow}\vrule width 2pt depth 6pt}\hspace{1em}}%
%	\MakeFramed{\parshape 1 0cm \dimexpr\textwidth-6em\relax\FrameRestore}\vskip2pt%
%}
%{\endMakeFramed}

%\titlecontents{chapter}
%[0em]{\vspace*{2\baselineskip}}
%{\parbox{4.5em}{%
%		\hfill\Huge\sffamily\bfseries\color{myred}\thecontentspage}%
%	\vspace*{-2.3\baselineskip}\leftbar\textsc{\small\chaptername~\thecontentslabel}\\\sffamily}
%{}{\endleftbar}
%\titlecontents{section}
%[8.4em]
%{\sffamily\contentslabel{3em}}{}{}
%{\hspace{0.5em}\nobreak\itshape\color{myred}\contentspage}
%\titlecontents{subsection}
%[8.4em]
%{\sffamily\contentslabel{3em}}{}{}  
%{\hspace{0.5em}\nobreak\itshape\color{myred}\contentspage}



%%%%%%%%%%%%%%%%%%%%%%%%%%%%%%%%%%%%%%%%%%%
% TABLE OF CONTENTS
%%%%%%%%%%%%%%%%%%%%%%%%%%%%%%%%%%%%%%%%%%%

\usepackage{tikz}
\definecolor{doc}{RGB}{0,60,110}
\usepackage{titletoc}
\contentsmargin{0cm}
\titlecontents{chapter}[3.7pc]
{\addvspace{30pt}%
	\begin{tikzpicture}[remember picture, overlay]%
		\draw[fill=doc!60,draw=doc!60] (-7,-.1) rectangle (-0.9,.5);%
		\pgftext[left,x=-3.5cm,y=0.2cm]{\color{white}\Large\sc\bfseries Chapter\ \thecontentslabel};%
	\end{tikzpicture}\color{doc!60}\large\sc\bfseries}%
{}
{}
{\;\titlerule\;\large\sc\bfseries Page \thecontentspage
	\begin{tikzpicture}[remember picture, overlay]
		\draw[fill=doc!60,draw=doc!60] (2pt,0) rectangle (4,0.1pt);
	\end{tikzpicture}}%
\titlecontents{section}[3.7pc]
{\addvspace{2pt}}
{\contentslabel[\thecontentslabel]{2pc}}
{}
{\hfill\small \thecontentspage}
[]
\titlecontents*{subsection}[3.7pc]
{\addvspace{-1pt}\small}
{}
{}
{\ --- \small\thecontentspage}
[ \textbullet\ ][]

\makeatletter
\renewcommand{\tableofcontents}{%
	\chapter*{%
	  \vspace*{-20\p@}%
	  \begin{tikzpicture}[remember picture, overlay]%
		  \pgftext[right,x=15cm,y=0.2cm]{\color{doc!60}\Huge\sc\bfseries \contentsname};%
		  \draw[fill=doc!60,draw=doc!60] (13,-.75) rectangle (20,1);%
		  \clip (13,-.75) rectangle (20,1);
		  \pgftext[right,x=15cm,y=0.2cm]{\color{white}\Huge\sc\bfseries \contentsname};%
	  \end{tikzpicture}}%
	\@starttoc{toc}}
\makeatother


%From M275 "Topology" at SJSU
\newcommand{\id}{\mathrm{id}}
\newcommand{\taking}[1]{\xrightarrow{#1}}
\newcommand{\inv}{^{-1}}

%From M170 "Introduction to Graph Theory" at SJSU
\DeclareMathOperator{\diam}{diam}
\DeclareMathOperator{\ord}{ord}
\newcommand{\defeq}{\overset{\mathrm{def}}{=}}

%From the USAMO .tex files
\newcommand{\ts}{\textsuperscript}
\newcommand{\dg}{^\circ}
\newcommand{\ii}{\item}

% % From Math 55 and Math 145 at Harvard
% \newenvironment{subproof}[1][Proof]{%
% \begin{proof}[#1] \renewcommand{\qedsymbol}{$\blacksquare$}}%
% {\end{proof}}

\newcommand{\liff}{\leftrightarrow}
\newcommand{\lthen}{\rightarrow}
\newcommand{\opname}{\operatorname}
\newcommand{\surjto}{\twoheadrightarrow}
\newcommand{\injto}{\hookrightarrow}
\newcommand{\On}{\mathrm{On}} % ordinals
\DeclareMathOperator{\img}{im} % Image
\DeclareMathOperator{\Img}{Im} % Image
\DeclareMathOperator{\coker}{coker} % Cokernel
\DeclareMathOperator{\Coker}{Coker} % Cokernel
\DeclareMathOperator{\Ker}{Ker} % Kernel
\DeclareMathOperator{\rank}{rank}
\DeclareMathOperator{\Spec}{Spec} % spectrum
\DeclareMathOperator{\Tr}{Tr} % trace
\DeclareMathOperator{\pr}{pr} % projection
\DeclareMathOperator{\ext}{ext} % extension
\DeclareMathOperator{\pred}{pred} % predecessor
\DeclareMathOperator{\dom}{dom} % domain
\DeclareMathOperator{\ran}{ran} % range
\DeclareMathOperator{\Hom}{Hom} % homomorphism
\DeclareMathOperator{\Mor}{Mor} % morphisms
\DeclareMathOperator{\End}{End} % endomorphism

\newcommand{\eps}{\epsilon}
\newcommand{\veps}{\varepsilon}
\newcommand{\ol}{\overline}
\newcommand{\ul}{\underline}
\newcommand{\wt}{\widetilde}
\newcommand{\wh}{\widehat}
\newcommand{\vocab}[1]{\textbf{\color{blue} #1}}
\providecommand{\half}{\frac{1}{2}}
\newcommand{\dang}{\measuredangle} %% Directed angle
\newcommand{\ray}[1]{\overrightarrow{#1}}
\newcommand{\seg}[1]{\overline{#1}}
\newcommand{\arc}[1]{\wideparen{#1}}
\DeclareMathOperator{\cis}{cis}
\DeclareMathOperator*{\lcm}{lcm}
\DeclareMathOperator*{\argmin}{arg min}
\DeclareMathOperator*{\argmax}{arg max}
\newcommand{\cycsum}{\sum_{\mathrm{cyc}}}
\newcommand{\symsum}{\sum_{\mathrm{sym}}}
\newcommand{\cycprod}{\prod_{\mathrm{cyc}}}
\newcommand{\symprod}{\prod_{\mathrm{sym}}}
\newcommand{\Qed}{\begin{flushright}\qed\end{flushright}}
\newcommand{\parinn}{\setlength{\parindent}{1cm}}
\newcommand{\parinf}{\setlength{\parindent}{0cm}}
% \newcommand{\norm}{\|\cdot\|}
\newcommand{\inorm}{\norm_{\infty}}
\newcommand{\opensets}{\{V_{\alpha}\}_{\alpha\in I}}
\newcommand{\oset}{V_{\alpha}}
\newcommand{\opset}[1]{V_{\alpha_{#1}}}
\newcommand{\lub}{\text{lub}}
\newcommand{\del}[2]{\frac{\partial #1}{\partial #2}}
\newcommand{\Del}[3]{\frac{\partial^{#1} #2}{\partial^{#1} #3}}
\newcommand{\deld}[2]{\dfrac{\partial #1}{\partial #2}}
\newcommand{\Deld}[3]{\dfrac{\partial^{#1} #2}{\partial^{#1} #3}}
\newcommand{\lm}{\lambda}
\newcommand{\uin}{\mathbin{\rotatebox[origin=c]{90}{$\in$}}}
\newcommand{\usubset}{\mathbin{\rotatebox[origin=c]{90}{$\subset$}}}
\newcommand{\lt}{\left}
\newcommand{\rt}{\right}
\newcommand{\bs}[1]{\boldsymbol{#1}}
\newcommand{\exs}{\exists}
\newcommand{\st}{\strut}
\newcommand{\dps}[1]{\displaystyle{#1}}

\newcommand{\sol}{\setlength{\parindent}{0cm}\textbf{\textit{Solution:}}\setlength{\parindent}{1cm} }
\newcommand{\solve}[1]{\setlength{\parindent}{0cm}\textbf{\textit{Solution: }}\setlength{\parindent}{1cm}#1 \Qed}

%--------------------------------------------------
% LIE ALGEBRAS
%--------------------------------------------------
\newcommand*{\kb}{\mathfrak{b}}  % Borel subalgebra
\newcommand*{\kg}{\mathfrak{g}}  % Lie algebra
\newcommand*{\kh}{\mathfrak{h}}  % Cartan subalgebra
\newcommand*{\kn}{\mathfrak{n}}  % Nilradical
\newcommand*{\ku}{\mathfrak{u}}  % Unipotent algebra
\newcommand*{\kz}{\mathfrak{z}}  % Center of algebra

%--------------------------------------------------
% HOMOLOGICAL ALGEBRA
%--------------------------------------------------
\DeclareMathOperator{\Ext}{Ext} % Ext functor
\DeclareMathOperator{\Tor}{Tor} % Tor functor

%--------------------------------------------------
% MATRIX & GROUP NOTATION
%--------------------------------------------------
\DeclareMathOperator{\GL}{GL} % General Linear Group
\DeclareMathOperator{\SL}{SL} % Special Linear Group
\newcommand*{\gl}{\operatorname{\mathfrak{gl}}} % General linear Lie algebra
\newcommand*{\sl}{\operatorname{\mathfrak{sl}}} % Special linear Lie algebra

%--------------------------------------------------
% NUMBER SETS
%--------------------------------------------------
\newcommand*{\RR}{\mathbb{R}}
\newcommand*{\NN}{\mathbb{N}}
\newcommand*{\ZZ}{\mathbb{Z}}
\newcommand*{\QQ}{\mathbb{Q}}
\newcommand*{\CC}{\mathbb{C}}
\newcommand*{\PP}{\mathbb{P}}
\newcommand*{\HH}{\mathbb{H}}
\newcommand*{\FF}{\mathbb{F}}
\newcommand*{\EE}{\mathbb{E}} % Expected Value

%--------------------------------------------------
% MATH SCRIPT, FRAKTUR, AND BOLD SYMBOLS
%--------------------------------------------------
\newcommand*{\mcA}{\mathcal{A}}
\newcommand*{\mcB}{\mathcal{B}}
\newcommand*{\mcC}{\mathcal{C}}
\newcommand*{\mcD}{\mathcal{D}}
\newcommand*{\mcE}{\mathcal{E}}
\newcommand*{\mcF}{\mathcal{F}}
\newcommand*{\mcG}{\mathcal{G}}
\newcommand*{\mcH}{\mathcal{H}}

\newcommand*{\mfA}{\mathfrak{A}}  \newcommand*{\mfB}{\mathfrak{B}}
\newcommand*{\mfC}{\mathfrak{C}}  \newcommand*{\mfD}{\mathfrak{D}}
\newcommand*{\mfE}{\mathfrak{E}}  \newcommand*{\mfF}{\mathfrak{F}}
\newcommand*{\mfG}{\mathfrak{G}}  \newcommand*{\mfH}{\mathfrak{H}}

\usepackage{bm} % Ensure bold math works correctly
\newcommand*{\bmA}{\bm{A}}
\newcommand*{\bmB}{\bm{B}}
\newcommand*{\bmC}{\bm{C}}
\newcommand*{\bmD}{\bm{D}}
\newcommand*{\bmE}{\bm{E}}
\newcommand*{\bmF}{\bm{F}}
\newcommand*{\bmG}{\bm{G}}
\newcommand*{\bmH}{\bm{H}}

%--------------------------------------------------
% FUNCTIONAL ANALYSIS & ALGEBRA
%--------------------------------------------------
\DeclareMathOperator{\Aut}{Aut} % Automorphism group
\DeclareMathOperator{\Inn}{Inn} % Inner automorphisms
\DeclareMathOperator{\Syl}{Syl} % Sylow subgroups
\DeclareMathOperator{\Gal}{Gal} % Galois group
\DeclareMathOperator{\sign}{sign} % Sign function


%\usepackage[tagged, highstructure]{accessibility}
\usepackage{tocloft}
\usepackage{arydshln}
\usetikzlibrary{arrows.meta, decorations.pathreplacing}
\usepackage{tikz-cd}
\usepackage{polynom}
\usepackage{pifont}
\newcommand{\pistar}{{\zf\symbol{"4A}}}
% a tiny helper for a stretched phantom (for the underbrace)
\newcommand\mc[1]{\multicolumn{1}{c}{#1}}



\begin{document}
\title{Linear Algebra I}
\author{Lecture Notes Provided by Dr.~Miriam Logan.}
\date{}
\maketitle
\tableofcontents
\newpage  
\section{Jordan Blocks}
\thm{}
{
Suppose $ J = J _k \left( \lambda \right) = \begin{bmatrix}
  \lambda & 1       & 0       & \cdots & 0 \\[2pt]
  0       & \lambda & 1       & \ddots & \vdots \\[2pt]
  \vdots  & 0       & \lambda & \ddots & 0 \\[2pt]
  \vdots  &         & \ddots  & \ddots & 1 \\[2pt]
  0       & \cdots  & 0       & 0      & \lambda
\end{bmatrix}$ \\
Then 
\begin{enumerate}[label=(\arabic*).]  
\item $ J$ has one eigenvalue and one linearly independent eigenvector
\item  $ \left( J - \lambda I \right) ^{ i}$ has ones $ i$ steps above the main diagonal in the $ i+1 ,i+2,\ldots, k$  columns and zeros elsewhere, $ \forall 1 \leq 1 \leq k-1$ $ \left( J - \lambda I \right) ^{l} $ is the zero matrix $ \forall  l \ge k$ 
\item $ \mathcal{N} \left( J - \lambda I \right) ^{i} = span \left\{ \vec{ e}_1, \vec{ e_2} , \ldots , \vec{ e_i}  \right\} $ 
	for each $ i $, $ 1 \leq i \leq k$ (where $ \vec{ e_i} =$ $i^{\text{th}}$ standard basis vectorin $ \mathbb{C} ^{k}$ )
\end{enumerate}
\pf{Proof:}{
(3)\\
In $ \left( J - \lambda I \right) ^{i}$ the first $ i$ columns consists of zeros, all other columns contain one $ 1$ ( pivot entry)\\
$ \implies x_1, \ldots x_i$ are free variables and $ \left\{ \vec{ e_1} , \ldots , \vec{ e_i}  \right\} $  span $ \mathcal{N} \left(  J - \lambda I  \right) ^{i}$.\\
Clearly this set is linearly independent, $ \implies \left\{ \vec{ e_1} , \ldots , \vec{ e_i }  \right\} $ forms a basis for $ \mathcal{N} \left(  J - \lambda I  \right) ^{i}$.\\

}


}
 \underline{Total Number of Jordan Blocks:} \\
 Let $ J$ be the Jordan form of an  $n \times n$  matrix $ A$.\\
 \[
 dim \left( \mathcal{N} \left( A - \lambda I \right)  \right) = dim \left( \mathcal{N} \left( J - \lambda I \right)  \right) \text{ (by similarity)}
 \] 
 which is equal to the number of linearly independent eigenvectors corresponding to $ \lambda$, the number of Jordan chains corresponding to $ \lambda$ and the number of Jordan blocks corresponding to $ \lambda$.\\
 \\
 \underline{Number of blocks of size $ \ge i$ :} \\
 Consider the difference $ dim \left( \mathcal{N} \left( A - \lambda I \right) ^{i} \right) - dim \left( \mathcal{N} \left( A - \lambda I \right) ^{i-1} \right) $. By similarity this equals $ dim \left( \mathcal{N} \left( J -  \lambda I \right)^{i}  \right) - dim \left( \mathcal{N} \left( J - \lambda I  \right)^{i-1}  \right)  $.\\
 A block of size $ j \times  j$ where $ j \le i-1$ contributes one to this difference. Thus the number of blocks of size $ j \times  j$ where $ j \ge i$ is equal to  $ dim \left( \mathcal{N} \left( A - \lambda I \right) ^{i} \right) - dim \left( \mathcal{N} \left( A - \lambda I \right) ^{ i -1} \right) $

\\
\thm{}
{
Two matrices $ A_1$ and $ A_2$ are similar if and only if they have the same Jordan form up to a permutation of the Jordan blocks.\\
}
\pf{Proof:}{
 $ \impliedby$ \\
 Suppose $ A_1$ and $ A_2$ have the same Jordan Normal Form. Then $ \exists  B_1, B_2$ such that 
 \[
 B_1 ^{-1} A_1 B_1 = J \qquad  \text{ and }  \qquad B_2 ^{-1} A_2 B_2 = J
 .\] 
 \[
  \text{ i.e. } B_1^{-1} A_1 B_1 = B_2 ^{-1} A_2 B_2 \qquad  \implies B_2 B_1 ^{-1} A_1 B_1  B_2 ^{-1} = A_2
 .\] 
 \[
 \left( B_1 B_2 ^{-1} \right) ^{-1} A_1 B_1 B_2 ^{-1} = A_2 \qquad \implies A_1 \text{ and } A_2 \text{ are similar.}
 .\] \\
 \\
 \\
 $ \implies$ \\
 \textit{Want to show that if $ A_1$ and $ A_2$ have the same eigenvalues and for each eigenvalue $ \lambda$, that the number of Jordan blocks associated with $ \lambda$ are equal and that the number of blocks of a particular size is equal.}\\
 Conversly, if $ A_1$ and $ A_2$ are similar, matrices then $ \left( A_1 - \lambda I  \right) ^{i} $ and $ \left( A_2 - \lambda I \right) ^{i}$ are similar $ \forall  \lambda , i$.
 \[
 \implies dim \left( \mathcal{N} \left( A_1 - \lambda I \right) ^{i} \right) = dim \left( \mathcal{N} \left( A_2 - \lambda I \right) ^{i} \right)
 .\]
  \[
  \implies dim \left( \mathcal{N} \left( A_1 - \lambda I \right)  \right)   = dim \left( \mathcal{N} \left( A_2- \lambda I \right)  \right) = \text{ the number of Jordan blocks corresponding to } \lambda
  .\] 
   and $ dim \left( \mathcal{N} \left( A_1 - \lambda I \right) ^{i} \right) - dim \left(  \mathcal{N} \left( A_1 - \lambda I \right) ^{i-1} \right) = dim \left( \mathcal{N} \left( A_2 - \lambda I \right) ^{i} \right) - dim \left( \mathcal{N} \left( A_2 - \lambda I \right) ^{i-1} \right) $ \\
   $ \implies A_1$ and $ A_2$ have the same number of Jordan blocks of size $ j \times j$ for each $ j \ge i$.\\
   $ \implies$ the Jordan Normal Forms of $ A_1$ and $ A_2$ are the same up to a permutation of the Jordan blocks.\\
}
\underline{Multiplying Matrices with square blcoks along the diagonal:}\\
\[
\text{ Suppose } A = \begin{bmatrix}
B & 0\\
0 & C\\
\end{bmatrix} \qquad  \text{ and} \qquad  D = \begin{bmatrix}
E & 0\\
0 & F\\
\end{bmatrix}
.\]  where $ B$ and $ E$ are $ k \times  k$ matrices and $ C$ and $ F$ are $ m \times  m$ matrices and the zeros are appropriately sized.\\
\[
AD = \begin{bmatrix}
BE & 0\\
0 & CF\\
\end{bmatrix}
.\] 
             \section{Powers of a Matrix}
 Suppose $ A$ is an $ n \times n$ matrix and $ J$ is its Jordan normal form, $ A = B ^{-1}J B$.\\
 We know $ A ^{n}= B ^{-1}J ^{n}B$.\\
 Hence to find $ A ^{n}$ we need to find $ J ^{n}$.\\
 Suppose $ J = \begin{bmatrix}
 J_1 & 0\\
 0 & J_2\\
 \end{bmatrix}$ where $ J_1$ and $ J_2$ are Jordan blocks.\\
 Using the result above it can be shown by induction that 
 \[
 J ^{n}= \begin{bmatrix}
 J_1 ^{n} & 0\\
 0 & J_2 ^{n}\\
 \end{bmatrix}
 .\] 
 Similarly, by induction it can be shown that
 \[
  \left[
\begin{array}{cccc}
J_{1} &        &        & 0 \\[2pt]
      & J_{2}  & \ddots &   \\[-2pt]
      &        & \ddots &   \\[2pt]
 0    &        &        & J_{k}
\end{array}
\right]^{\!n}
\;=\;
\left[
\begin{array}{cccc}
J_{1}^{\,n} &          &         & 0 \\[2pt]
            & J_{2}^{\,n} & \ddots  &   \\[-2pt]
            &            & \ddots  &   \\[2pt]
 0          &            &         & J_{k}^{\,n}
\end{array}
\right].
 .\] 
 It remains to compute $ \left( J_i \right) ^{n}$  for a Jordan block $ J_i$.\\
 Suppose $ J_i $ is and $ l \times  l$ Jordan block,
   \[
J_i(\lambda)=
\begin{bmatrix}
 \lambda & 1       & 0       & \cdots & 0 \\[2pt]
 0       & \lambda & 1       & \ddots & \vdots \\[2pt]
 \vdots  & 0       & \lambda & \ddots & 0 \\[2pt]
 \vdots  &         & \ddots  & \ddots & 1 \\[2pt]
 0       & \cdots  & 0       & 0      & \lambda
\end{bmatrix}
\]



 if $ \lambda -0$ then $ \left( J_i \right) ^{n}$ is and $ l \times  l$ matrix with ones $ n$ steps above the diagonal and zeros elsewhere, for $ n \leq l -1$. If $ n \ge l$ then $ \left( J_i \right) ^{n}$ is the zero matrix.\\
 If $ \lambda \neq 0 $ we use the fact that for matrices that commute $ \left( AB = BA \right) $
 \[
 \left( A+B \right) ^{n} = \sum_{k=0}^{n} \binom{n}{k} A^{n-k} B^{k}
 .\] 
 \[
 \implies J^{n} = \left( J - \lambda I + \lambda I \right) ^{n} = \sum_{k=0}^{n} \binom{n}{k} \left( J - \lambda I \right) ^{n-k} \left( \lambda I \right) ^{k}
 .\] 
 \textbf{Note:} $ \left( J - \lambda I \right) $ and $ \left( \lambda I \right) $ commute since $ \left( J - \lambda I \right) \left( \lambda I \right) = \lambda \left( J - \lambda I  \right) \left( I \right) = \lambda \left( I \right) \left( J - \lambda I  \right) = \left( \lambda I \right) \left( J- \lambda I \right) $ \\
 \\
 Now, $ \left( J -  \lambda I \right)^{k} $ is the $ l \times  l$ matrix with ones $ k$ steps above the diagonal and zeros elsewhere,
 \[
 \left( \lambda I \right) ^{ n-k} = \begin{bmatrix}
 \lambda ^{n-k} & 0 & 0 & \dots  & 0 \\
 0 & \lambda ^{n-k} & 0 & \dots  & 0 \\
 \vdots & \vdots & \vdots & \ddots & \vdots \\
 0 & 0 & 0 & \dots  &  \lambda ^{n-k}\end{bmatrix}    = \lambda ^{n-k} I
 .\] 
 \[
 \implies J ^{n} = \sum_{k=0}^{n} \binom{n}{k} \left( J - \lambda I \right) ^{k} \lambda ^{n-k} I
 .\] 
 \[
 J ^{n} =  \sum_{k=0}^{n} \binom{n}{k} \lambda ^{n-k} \left( J - \lambda I \right) ^{k} 
 .\] 
 Hence, $ J ^{n}= \lambda ^{n} I + \binom{n}{1} \lambda ^{ n-1} \left( J - \lambda I \right) + \binom{n}{2} \lambda ^{ n-2} \left( J - \lambda I \right) ^2 + \ldots + \binom{n}{k} \lambda ^{n-k} \left( J - \lambda I \right) ^{k}+ \ldots + \binom{n}{n} \left( J - \lambda I \right) ^{n} \qquad $.   $ \left( J - \lambda I \right) ^{k }=0$ if $ k \ge l$ \\
  \[
J^{n}=%
\begin{pmatrix}
\lambda^{n} & \binom{n}{1}\lambda^{\,n-1} & \binom{n}{2}\lambda^{\,n-2} & \cdots & \binom{n}{\ell-1}\lambda^{\,n-(\ell-1)}\\[4pt]
0          & \lambda^{n}                  & \binom{n}{1}\lambda^{\,n-1} & \ddots & \vdots\\[4pt]
0          & 0                           & \lambda^{n}                  & \ddots & \binom{n}{2}\lambda^{\,n-2}\\[4pt]
\vdots     & \vdots                      & \ddots                       & \ddots & \binom{n}{1}\lambda^{\,n-1}\\[4pt]
0          & 0                           & \cdots                       & 0      & \lambda^{n}
\end{pmatrix}
\]
\[
\implies J ^{n} \text{ has } \lambda ^{n } \text{ on the main diagonal, } 
.\] 
\[
\text{ } \binom{n}{1} \lambda ^{ n-1} \text{ one step above the main diagonal, } 
.\] 
\[
\binom{n}{2} \lambda ^{ n-2} \text{ two steps above the main diagonal, }
.\] 
and so on,
 \[
\binom{n}{k} \lambda ^{ n-k} \text{ $ k$ steps above the main diagonal, } \forall 1 \leq k \leq l-1
 .\] 
 \ex{}{
 \[
 \begin{bmatrix}
 \lambda & 1 & 0\\
 0 & \lambda & 1\\
 0 & 0 & \lambda\\
 \end{bmatrix} ^{4}  = \begin{bmatrix}
 \lambda ^{4} & 4 \lambda ^3 & 6 \lambda^2\\
 0 & \lambda^{4} & 4 \lambda ^3\\
 0 & 0 & \lambda ^{4}\\
 \end{bmatrix}
 .\] 
 }
 \ex{}{
  \[
\left(
\begin{array}{cc|cc|c}
2 & 1 & 0 & 0 & 0 \\[4pt]
0 & 2 & 0 & 0 & 0 \\ \hline
0 & 0 & 3 & 1 & 0 \\[4pt]
0 & 0 & 0 & 3 & 0 \\ \hline
0 & 0 & 0 & 0 & 4
\end{array}
\right)^{n}
\;=\;
\left(
\begin{array}{cc|cc|c}
2^{n} & n\,2^{\,n-1} & 0 & 0 & 0 \\[4pt]
0 & 2^{n} & 0 & 0 & 0 \\ \hline
0 & 0 & 3^{n} & n\,3^{\,n-1} & 0 \\[4pt]
0 & 0 & 0 & 3^{n} & 0 \\ \hline
0 & 0 & 0 & 0 & 4^{n}
\end{array}
\right).
\]

 }
 
   \ex{}{
   Let $ A = \begin{bmatrix}
   8 & -9\\
   4 & -4\\
   \end{bmatrix}$, \textit{Find } $ A ^{n}$ \\
   \begin{align*}
    \chi_A \left( \lambda \right) = \left( 8 - \lambda \right) \left(  -4 -\lambda \right) +36 &=0\\
    \lambda ^2 - 4 \lambda - 4 &= 0\\
    \left( \lambda - 2 \right) ^2 &= 0\\
    \lambda &= 2 \text{ is the only eigenvalue.}\\
   .\end{align*}
   \[
   \mathcal{N} \left( A - 2 I \right) : \left[
   \begin{array}{cc;{2pt/2pt}c}  
     6 & -9 & 0\\
     4 & -6 & 0\\
   \end{array}
   \right] \to \left[
   \begin{array}{cc;{2pt/2pt}c}  
     2 & -3 & 0\\
     0 & 0 & 0\\
   \end{array}
   \right]
   .\] 
   \[
    dim \left( \mathcal{N} \left( A - 2I \right)  \right) =1    \qquad  \begin{aligned}
\mathcal{N}_{1}:&
\;
\begin{array}{c}
\bullet\\[-0.25em]
|\\[-0.25em]
\bullet
\end{array}
\quad
\text{1 Jordan chain}\\[1em]
\mathcal{N}_{2}:&\;
\end{aligned}
XXX FIX
   .\] 
   $ \implies$ Jordan Normal Form of $ A = \begin{bmatrix}
   2 & 1\\
   0 & 2\\
   \end{bmatrix}$\\
   Jordan basis : $ \mathcal{N} \left(  A - 2I \right) ^2 = \mathbb{R} ^2$.\\
   Let $ \vec{ v_2} = \begin{bmatrix}
   1\\
   0\\
   \end{bmatrix}
    $, $ \vec{ v_1} = \left( A - \lambda I \right) \vec{ v_2} = \begin{bmatrix}
    6\\
    4\\
    \end{bmatrix}
     $             \\
     Let $ \mathcal{B} = \left\{ \begin{bmatrix}
     6\\
     4\\
     \end{bmatrix}
     , \begin{bmatrix}
     1\\
     0\\
     \end{bmatrix}
      \right\} $ \\
      \[
      XXX
      .\] 
      \[
         \begin{bmatrix}
         8 & -9\\
         4 & -4\\
         \end{bmatrix}= \begin{bmatrix}
         6 & 1\\
         4 & 0\\
         \end{bmatrix} \begin{bmatrix}
         2 & 1\\
         0 & 2\\
         \end{bmatrix} \begin{bmatrix}
         0 & -1\\
         -4 & 6\\
         \end{bmatrix} \left( - \frac{1}{4} \right) 
      .\] 
      \[
      \implies A ^{n} = \begin{bmatrix}
      6 & 1\\
      4 & 0\\
      \end{bmatrix} \begin{bmatrix}
      2^{n} & n 2 ^{n-1}\\
      0 & 2 ^{n}\\
      \end{bmatrix} \begin{bmatrix}
      0 & -1\\
      -4 & 6\\
      \end{bmatrix} \left( - \frac{1}{4} \right) 
      .\] 
      \[
      A ^{n} =  \begin{bmatrix}
      6 \left( 2 ^{n} \right)  & 6 \left(  n 2 ^{n-1} \right) + 2 ^{n}\\
       4 \left( 2 ^{n} \right) & 4 \left( n 2 ^{n-1} \right) \\
      \end{bmatrix}       \begin{bmatrix}
      0 & -1\\
      -4 & 6\\
      \end{bmatrix}
      .\] 
      \[
      A ^{n} = \begin{bmatrix}
      -24 \left( n 2 ^{n-1} \right) -4 \left( 2^{n} \right)  & 36 \left( n 2 ^{n-1} \right) \\
       -16 \left( n 2 ^{n-1} \right) & -4 \left( 2^{n} \right) + 24 \left( n 2 ^{n-1} \right) \\
      \end{bmatrix}
      .\] 
   }
   \section{Matrices and Polynomials}
 Suppose $ A$, $ C$ are similar $n \times n$  matrices. Hence $ \exists $ an invertible $n \times n$  matric $ B$ such that $ C = B ^{-1} A B$.\\
 \\
 \[
  \text{ Suppose }       f\left( x \right) = \sum\limits_{k=0}^{n} c_k x ^{k} \qquad  c_k \in \mathbb{C}
  
 .\] 
 \underline{Fact 1:} \\
 \[
 f \left( A \right) =0 \iff f \left(  C \right) =0
 .\] 
     \pf{Proof:}{
      \begin{align*}
       f \left( C \right) &= \sum\limits_{k=0}^{n} c_k C^{k}\\
       &= \sum\limits_{k=0}^{n} c_k \left( B ^{-1}A B  \right) ^{k}\\
       &= \sum\limits_{k=0}^{n} c_k \left( B ^{-1}A^{k}B \right) \\
       &= B^{-1} \left( \sum\limits_{k=0}^{n} c_k A ^{k}
        \right) B\\
        &= B ^{-1} f \left( A \right) B\\
        &=0
      .\end{align*}
     }
 \underline{Fact 2:} \\
 If $ f \left( A \right) =0$ then $ f \left( \lambda \right) =0$ $ \forall $ eigenvalues $ \lambda$ of $ A$.\\
 \pf{Proof:}{
 Suppose $ \vec{ v} $ is an eigenvector corresponding to the eigenvalue $ \lambda$.\\
 \begin{align*}
  0 \vec{ v} &= f \left( A \right)  \vec{ v} = \left(  \sum\limits_{k=0}^{n} c_k A ^{k}
   \right) \vec{ v}\\
   \vec{ 0} &= c_0 \vec{ v} +c_1 A \vec{ v} + c_2 A^2 \vec{ v} + \ldots + c_n A ^{n}\vec{ v}\\
   \vec{ 0} &= \left( c_0 +c_1 \lambda + c_2 \lambda^2 + \ldots + c_n \lambda^{n} \right) \vec{ v} \\
   \vec{ 0 } &= f \left( \lambda \right) \vec{ v} 
 .\end{align*}
 Since $ \vec{ v}  \neq \vec{ 0}  \implies f \left( \lambda \right) =0$.
 }
 \thm{Cayley Hamilton Theorem}
 {
 Suppose $ A$ is an $n \times n$  matrix, and $ \chi _a \left( x \right) = \text{ det } \left( A - x I \right) $ is the charactersistic polynomial of $ A$. $ \chi _a \left( A \right) =0$.\\

 }
 
  \pf{Proof:}{
   We will first show this to be true where $ A$ is a direct sum of Jordan blocks,\\
   \underline{Case 1:} \\
   \[
   A = \begin{bmatrix}
       J_1 & 0 & 0 & \dots  & 0 \\
       0 & J_2 & 0 & \dots  & 0 \\
       \vdots & \vdots & \vdots & \ddots & \vdots \\
       0 & 0 & 0 & \dots  & J_m\end{bmatrix}
   .\] 
   Proof using induction on $ m$ :\\
     $ m =1$ \\
     If    \[
A \;=\; J_{k}(\lambda)\;=\;
\begin{pmatrix}
\lambda & 1       &        &        & 0\\
        & \lambda & \ddots &        &  \\
        &         & \ddots & 1      &  \\
        &         &        & \lambda & 1\\
0       &         &        &        & \lambda
\end{pmatrix}
\]




     then $ \chi _A \left( x\right)= \left(  \lambda - x  \right) ^{k}= \left( -1 \right) ^{k} \left( x - \lambda \right) ^{k}$.\\
     We know that $ \left( A - \lambda I \right) ^{k}=0$
     $ \implies \chi _A \left( A \right) = \left( -1 \right) ^{k} \left( A - \lambda I \right) ^{k}=  0 $.\\
     $ \implies A$ satisfies its characteristic polynomial.\\
     Assume statmenet true for $ m-1$ and
     \[
     \text{ let } A = \begin{bmatrix}
       J_1 & 0 & 0 & \dots  & 0 \\
       0 & J_2 & 0 & \dots  & 0 \\
       \vdots & \vdots & \vdots & \ddots & \vdots \\
       0 & 0 & 0 & \dots  & J_m\end{bmatrix}     
     .\] 
     \[
     \chi _A \left( x \right) = \text{ det } \begin{bmatrix}
       J_1- xI & 0 & 0 & \dots  & 0 \\
       0 & J_2-xI & 0 & \dots  & 0 \\
       \vdots & \vdots & \vdots & \ddots & \vdots \\
       0 & 0 & 0 & \dots  & J_m- xI\end{bmatrix}     
     .\] 

        \[
\det\!\Bigl(
\underbrace{%
\begin{pmatrix}
\begin{array}{c|c|c}
J_{1}-xI & 0 & 0 \\ \hline
0        & J_{m-1}-xI & 0 \\ \hline
0        & I          & 0
\end{array}
\end{pmatrix}}_{\displaystyle M}\;
\underbrace{%
\begin{pmatrix}
\begin{array}{c|c|c}
0 & I & 0 \\ \hline
0 & 0 & 0 \\ \hline
0 & 0 & J_{n}-xI
\end{array}
\end{pmatrix}}_{\displaystyle N}
\Bigr)
\]
 XXX\\




     where $ B$ is the square matrix with $ m-1$ Jordan blocks along the diagonal.\\
     By induction $ \chi _B \left( B \right) =0$
     \[
     \chi _A \left( A \right) = \sum c_k A^{k}= \sum  c_k  \left(
\begin{array}{c|c}
\begin{array}{cccc}
J_{1} &        &        &        \\[4pt]
J_{2} & \ddots &        &        \\[4pt]
      & \ddots & \ddots &        \\[4pt]
      &        & J_{m-1}&
\end{array}
& 0 \\ \hline
0 & J_{m}
\end{array}
\right)^{\,k}
     .\] 
     XXX\\
        XXXXXX\\

        $ \chi _B \left( B \right) =0$ by assumption and $ \chi _{J_m} \left( J_m \right) =0$ from $ m =1$.\\
        $ \implies \chi _A \left( A \right) =0$     \
        \\
        \\
        \\
     \underline{Case 2:} \\
     Suppose $ A \in M_{n \times  n} \left( \mathbb{C} \right) $ and $ J$ is the Jordan Normal Form of $ A$.\\
     The by above, $ \chi _J \left( J \right) =0$. But, since $ A$ is simliar to $ J$
     \[
     \chi _A \left( \lambda \right) = \chi _J \left( \lambda \right) 
     .\] 
     $ \implies \chi _A \left( J \right) =0$ and by fact 1 above,\\
     \[
     \chi _A \left( A \right) =0
     .\] 
     i.e. $ A$ satisfies its own characteristic Polynomial.
  }
  \dfn{ :}{
  Suppose $ A \in M _{ n \times  n} \left( \mathbb{C} \right) $ . We define the minimal polynomial of $ A$, $ m \left( x \right) $ to be the monic polynomial of least degree satisfied by $ A$.\\
  (a monic polynomial is a polynomial in which the coefficient of the highest power of $ x$ is $ 1$).\\
  \textit{We already know that $ A$ satisifies some polynomial, $ \chi _A \left( x \right) $. The minimal polynomial could be equal to the characteristic polynomial or it could have lower degree}\\
  
  }
  \underline{Fact 1:}\\
  The minimal polynomial, $m \left( x \right) $ of an $n \times n$  matrix $ A$ divides $ \chi _A \left( x \right) $, and $ m \left( \lambda_i \right) =0 $ $ \forall $ eigenvalues $ \lambda_i$ of $ A$.\\
  \\
  \pf{Proof:}{
    We know $ \chi _A \left( A \right) =0$ and $ m \left( A \right) =0$.\\
    Dividing $ \chi _A \left( x \right) $ by $ m\left( x \right) $ we get: 
    \[
    \chi _A \left( x \right) = m \left( x \right) q \left( x \right) + r \left( x \right) \qquad \qquad \star
    .\]         where $ r \left( x \right) $ is the remainder.\\
    Either $ r \left( x \right) =0$ or $ deg \left( r \left( x \right)  \right) < deg \left( m \left( x \right)  \right) $.\\
    Evaluating $ \star$ at $ x = A$ we get:
    \[
    \chi _A = m \left( A \right) q \left( A \right) + r \left( A \right) 
    .\] 
    i.e.  $ 0 = r \left( A \right) $ \\
    $ \implies m \left( x \right) $ divides $ \chi _a \left( x \right) $ .\\
    \\
    \\
    For the second part, suppose $ \lambda$ is an eigenvalue of $ A$ with eigenvector $ \vec{ v} $.
    \begin{align*}
     A \vec{ v} &= \lambda \vec{ v}\\
     \implies A ^{ k} \vec{ v} &= \lambda ^{k} \vec{ v} \\
     \implies m \left( A \right) \vec{ v} &= m \left( \lambda \right) \vec{ v}\\
     \vec{ 0} &= m \left( \lambda \right) \vec{ v}\\
     \implies m \left( \lambda \right) &= 0 \text{ since } \vec{ v} \neq \vec{ 0} 
    .\end{align*}
  }
  \ex{}{
  Let 
  \[
  A \;=\;
\left(
\begin{array}{c:cc}
2 & 0 & 0 \\ \hdashline               % horizontal dashed line
0 & 2 & 1 \\
0 & 0 & 2
\end{array}
\right)
  .\] 


  \textit{What is the minimal polynomial of $ A$?} \\
  \[
  \chi _A \left( x \right) = \left( 2-x \right) ^3 = \left( -1 \right) ^{3} \left( x-2 \right) ^3 = -1 \left( x-2 \right) ^3
  .\] 
  $  m \left( x \right)  $ divides $ \chi _A \left( x \right) \implies m \left( x \right) = \left( x-2 \right) ^i$ for $ i= 1,2,$ or $ 3$.
  \[
  A -2I \;=\;
\left(
\begin{array}{c:cc}
0 & 0 & 0 \\ \hdashline               % horizontal dashed line
0 & 0 & 1 \\
0 & 0 & 0
\end{array}
\right)    \neq 0 \qquad  \implies m \left( x \right) \neq \left( x-2 \right)
  .\] 
  \[
  \left( A - 2I \right) ^2 = \text{ zero matrix } \implies m \left( x \right) = \left( x-2 \right) ^2
  .\] 
  \textit{Clearly the degree of the minimal polynomial is related to the size of the largest Jordan blocks - We need to take a high enough power to make the ones above the diagonal zero.}\\
  }
  \ex{}{
  Let \[
  A \;=\;
\left(
\begin{array}{c:cc}
2 & 0 & 0 \\ \hdashline               % horizontal dashed line
0 & 2 & 1 \\
0 & 0 & 2
\end{array}
\right)
  .\] 
  Use the minimal polynomial of $ A$ to find $ A ^{-1}$.\\
  \begin{align*}
   m \left( x \right) &= \left( x-2 \right) ^2\\
   m \left( A \right) = \left( A - 2I \right) ^2 &= 0\\
   A^2 - 4A + 4I &= 0\\
   A^2 - 4A &= -4I\\
   \frac{1}{4} \left( 4A- A^2 \right) &= I\\
   A - \frac{1}{4} A^2 &= I\\
   A \left( I - \frac{1}{4}A \right) =I\\
   \implies A^{-1} &= I - \frac{1}{4}A\\
   A^{-1} = \begin{bmatrix}
   1 & 0 & 0\\
   0 & 1 & 0\\
   0 & 0 & 1\\
   \end{bmatrix} - \frac{1}{4} \begin{bmatrix}
   2 & 0 & 0\\
   0 & 2 & 1\\
   0 & 0 & 2\\
   \end{bmatrix}\\
   A^{-1} = \begin{bmatrix}
   \frac{1}{2} & 0 & 0\\
    0& \frac{1}{2} & - \frac{1}{4}\\
   0 & 0 & \frac{1}{2}\\
   \end{bmatrix}
  .\end{align*}
  }
  \ex{Computing Powers}{
      Suppose $ A \in M _{ n\times  n} \left( \mathbb{C} \right) $ and $ m \left( x \right) - x \left( x-2 \right) $ is the minimal polynomial of $ A$.\\
      \\
      Then 
      \begin{align*}
       A \left( A -2I \right) &= 0 \\
       \implies A ^2 - 2 A =0\\
       A^2 = 2A \\
       A ^3 = 2 A^2 = 2 \left( 2A \right) = 2^2 A \\
       A^{4} = 2^2 A ^2 = 2^2 \left( 2A \right) = 2^3 A\\
       \text{ By induction it can be shown that}\\
       A^{k}= 2 ^{k-1}A \qquad  \forall k \in \mathbb{N}
      .\end{align*}

  }
  \thm{}
  {
  Suppose $ A$ is an $n \times n$  matrix with one eigenvalue $ \lambda$, $  m \left( x \right) = \left( x-\lambda  \right) ^{k} $ where $ k$ is the size of the largest Jordan block associated with $ \lambda$.
  }
  \pf{Proof:}{
   Suppose 
   \[
   J = \begin{bmatrix}
     J_1 & 0 & \dots & 0 \\
     0 & J_2 & \dots & 0 \\
     \vdots & \vdots & \ddots & \vdots \\
     0 & 0 & \dots & J_p
   \end{bmatrix}
   
   .\] is the Jordan Normal Form of $ A$ with $ J_i$ a Jordan block of size $ k_i \times  k_i$
   \[
   m \left( J \right) = \sum\limits_{k}^{} c_k J ^{k} = \sum\limits_{k}^{} c_k \begin{bmatrix}
     J_1 ^{k} & 0 & \dots & 0 \\
     0 & J_2 ^{k} & \dots & 0 \\
     \vdots & \vdots & \ddots & \vdots \\
     0 & 0 & \dots & J_p ^{k}
   \end{bmatrix}          = \begin{bmatrix}
     m \left( J_1 \right)  & 0 & \dots & 0 \\
     0 & m \left( J_2 \right)  & \dots & 0 \\
     \vdots & \vdots & \ddots & \vdots \\
     0 & 0 & \dots & m \left( J_p \right) 
   \end{bmatrix}
   
   .\] 
   \[
   m \left( A \right) =0 \qquad  \implies m \left( J \right) =0
   .\] 
   \[
   \text{ Now, } m \left( J \right) =0 \iff m \left( J_i \right) =0 \qquad \forall  i 
   .\] 

      \[
J_{i}\;=\;
\left(
\begin{array}{cccccc}
\lambda & 1      &        &        &        & 0\\
        & \lambda& 1      &        &        &  \\
        &        & \ddots & \ddots &        &  \\
        &        &        & \ddots & 1      &  \\
        &        &        &        & \lambda & 1\\
0       &        &        &        &        & \lambda
\end{array}
\right)
\;\left.\vphantom{\begin{array}{c} \\ \\ \\ \\ \\ \\ \end{array}}\right\}\!k_{i}
\]

% (Optional) If you prefer a vertical arrow instead of a brace, replace the last line with:
% \quad\raisebox{5.5ex}{\(\big\uparrow\)}\;k_{i}



   For each block $ J_i$, $ \left( J_i - \lambda I \right) ^{k_i} =0$ and $ \left( J_i - \lambda I \right) ^{l} \neq 0 $ if $ l < k_i$.\\
   Hence in order for $ m \left( J_i \right) =0 \qquad  \forall  i$, $ m \left( x \right) - \left( x - \lambda \right) ^{k}$ where $ k$ is the size of the largest Jordan block.
  }

  \textbf{Corraly}\\
  \\XXX\\
  \\
  \\
  Suppose $ A \in M _{ n \times  n}\left( \mathbb{C} \right) $ with eigenvalues $ \lambda_1 , \ldots , \lambda_p$
  \[
  m \left( x \right) = \left( x - \lambda_1 \right) ^{k_1} \left( x - \lambda_2 \right) ^{ k_2} \ldots  \left( x - \lambda_p \right) ^{k_p}
  .\] 
  with $ k_i$ being the size of the largest Jordan block associated with $ \lambda_i$
   XXX END CORALLy
   \\
   \\
   \\
   \pf{Proof:}{
   Suppose $ J$ is the Jordan Normal Form of $ A$ 
   \[
   J = \begin{bmatrix}
     B_1 & 0 & \dots & 0 \\
     0 & B_2 & \dots & 0 \\
     \vdots & \vdots & \ddots & \vdots \\
     0 & 0 & \dots & B_p
   \end{bmatrix}
   
   .\] 
   where $ B_i = l_i\times l_i $  matrix consisting of all Jordan blocks associated with $ \lambda_i$.\\
   \[
   m \left( J \right) = \begin{bmatrix}
     m \left( B_1 \right)  & 0 & \dots & 0 \\
     0 & m \left( B_2 \right)  & \dots & 0 \\
     \vdots & \vdots & \ddots & \vdots \\
     0 & 0 & \dots & m \left( B_p \right) 
   \end{bmatrix}
   
   .\] 
   \[
   m \left( J \right) =0 \iff m \left( B_i \right) =0 \qquad \forall  i
   .\] 
   From the previous theorem we know that $ m \left( B_i \right) =0 \iff \left( x - \lambda _i \right) ^{k_i}$ is a factor of $ m \left( x \right) $ where $ k_i$ is the size of the largest Jordan block associated with $ \lambda_i$.\\
   $ \implies $ each $ B_i$ contributes a factor $ \left( x- \lambda_i \right) ^{k_i}$ to $ m \left( x \right) $.\\
  $ \implies m\left( x \right) = \left( x- \lambda_1 \right) ^{k_1} \left( x- \lambda_2 \right) ^{k_2}\ldots \left( x- \lambda_p \right) ^{k_p}$
   }
    \ex{}{
        Suppose $ A$ is a $ 4 \times  4$ matrix with a Jordan normal form,
                    \[
J \;=\;
\left(
\begin{array}{cc:cc}
2 & 1 & 0 & 0 \\ \hdashline           % ─── upper-left 2×2 Jordan block
0 & 2 & 0 & 0 \\ \hdashline
0 & 0 & 2 & 0 \\ \hdashline           % ─── individual 1×1 blocks
0 & 0 & 0 & 1
\end{array}
\right)
\]

        \[
        \chi _A \left( x \right) = \left( 2-x  \right) ^3 \left( 1-x \right) 
        .\] 
        \[
        m \left( x \right) = \left( 2-x \right) ^2 \left( x-1 \right) 
        .\] 
    }
     \ex{}{
     Suppose $ \chi _A \left( x \right) = - \left( x-1 \right) ^2 \left( x-3 \right) ^3$ and $ m \left( x \right) = \left( x-1 \right) \left( x-3 \right) ^2$ \\
     What is the Jordan Normal Form of $ A$?\\
     $  \lambda =1 $ is a double eigenvalue,\\
     size of the largest Jordan Blocks associated with $ \lambda =1$ is $ 1 \times  1$ \\
     \\
     $ \lambda =3 $  is a triple eigenvalue,\\
     size of largest Jordan Block associated with $ \lambda =3$ is $ 2 \times  2$
        \[
J \;=\;
\left(
\begin{array}{c:c:cc:c}   % ← dashed vertical lines after cols 1, 2, 4
1 & 0 & 0 & 0 & 0 \\ \hdashline            % ─── first 1×1 block
0 & 1 & 0 & 0 & 0 \\ \hdashline            % ─── second 1×1 block
0 & 0 & 3 & 1 & 0 \\                       % ┐
0 & 0 & 0 & 3 & 0 \\ \hdashline            % │  2×2 Jordan block (λ=3)
0 & 0 & 0 & 0 & 3                          % ┘  final 1×1 block (λ=3)
\end{array}
\right)
\]

     }
     
  
      \ex{}{
      Suppose $ A$ is a square matrix with $ \chi_A = x ^{4} \left( x-1 \right) ^2$ and $ m \left( x \right) = x^2 \left( x-1 \right) ^2$.\\
      Suppose also that $ dim \left(  \mathcal{C} \left( A \right)  \right) =4$ \\
      \textit{Find the Jordan Normal Form of $ A$.} \\
      \\
      \\
      \raggedcolumns
      \begin{multicols}{2}
      Eigenvalues: $ \lambda =0$ \\
      Largest Jordan block associated with $ \lambda =0$ is $ 2 \times 2$
      
      \break
        $ \lambda =1$ \\ largest Jordan block associated with $ \lambda =1$ is $  2 \times  2$
      \end{multicols}
       Thus the degree of $ \chi _A \left( x \right) $ is $ 6$.\\
       $ \implies A $ is $ 6 \times  6$ matrix.\\
       \[
       dim \left(  \mathcal{C} \left( A \right)  \right) =4 \qquad  \implies dim \left( \mathcal{N} \left( A \right)  \right) =2
       .\] 
       $ \implies 2 $ linearly independent eigenvectors associated with $ \lambda =0$.\\
       $ \implies 2 $ Jordan chains of length $ 2$.\\
       $ \implies 2 $  Jordan blocks of dimension $ 2 \times 2$
       \\
       \[
       \text{ JNF of } A = \left(
\begin{array}{cc:ccc:cc}
0 & 1 & 0 & 0 & 0 & 0 & 0\\
0 & 0 & 0 & 0 & 0 & 0 & 0\\ \hdashline
0 & 0 & 0 & 1 & 0 & 0 & 0\\
0 & 0 & 0 & 0 & 1 & 0 & 0\\
0 & 0 & 0 & 0 & 0 & 0 & 0\\ \hdashline
0 & 0 & 0 & 0 & 0 & 1 & 1\\
0 & 0 & 0 & 0 & 0 & 0 & 1
\end{array}
\right)
       .\] 

       XXX remove row
      }
      
  
  
  




















\end{document}       
