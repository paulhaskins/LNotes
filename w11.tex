\documentclass{report}

%%%%%%%%%%%%%%%%%%%%%%%%%%%%%%%%%
% PACKAGE IMPORTS
%%%%%%%%%%%%%%%%%%%%%%%%%%%%%%%%%


\usepackage[tmargin=2cm,rmargin=1in,lmargin=1in,margin=0.85in,bmargin=2cm,footskip=.2in]{geometry}
\usepackage{amsmath,amsfonts,amsthm,amssymb,mathtools}
\usepackage[varbb]{newpxmath}
\usepackage{xfrac}
\usepackage[makeroom]{cancel}
\usepackage{bookmark}
\usepackage{enumitem}
\usepackage{hyperref,theoremref}
\hypersetup{
	pdftitle={Assignment},
	colorlinks=true, linkcolor=doc!90,
	bookmarksnumbered=true,
	bookmarksopen=true
}
\usepackage[most,many,breakable]{tcolorbox}
\usepackage{xcolor}
\usepackage{varwidth}
\usepackage{varwidth}
\usepackage{tocloft}
\usepackage{etoolbox}
\usepackage{derivative} %many derivativess partials
%\usepackage{authblk}
\usepackage{nameref}
\usepackage{multicol,array}
\usepackage{tikz-cd}
\usepackage[ruled,vlined,linesnumbered]{algorithm2e}
\usepackage{comment} % enables the use of multi-line comments (\ifx \fi) 
\usepackage{import}
\usepackage{xifthen}
\usepackage{pdfpages}
\usepackage{transparent}
\usepackage{verbatim}

\newcommand\mycommfont[1]{\footnotesize\ttfamily\textcolor{blue}{#1}}
\SetCommentSty{mycommfont}
\newcommand{\incfig}[1]{%
    \def\svgwidth{\columnwidth}
    \import{./figures/}{#1.pdf_tex}
}
\usepackage[tagged, highstructure]{accessibility}
\usepackage{tikzsymbols}
\renewcommand\qedsymbol{$\Laughey$}


%\usepackage{import}
%\usepackage{xifthen}
%\usepackage{pdfpages}
%\usepackage{transparent}


%%%%%%%%%%%%%%%%%%%%%%%%%%%%%%
% SELF MADE COLORS
%%%%%%%%%%%%%%%%%%%%%%%%%%%%%%



\definecolor{myg}{RGB}{56, 140, 70}
\definecolor{myb}{RGB}{45, 111, 177}
\definecolor{myr}{RGB}{199, 68, 64}
\definecolor{mytheorembg}{HTML}{F2F2F9}
\definecolor{mytheoremfr}{HTML}{00007B}
\definecolor{mylenmabg}{HTML}{FFFAF8}
\definecolor{mylenmafr}{HTML}{983b0f}
\definecolor{mypropbg}{HTML}{f2fbfc}
\definecolor{mypropfr}{HTML}{191971}
\definecolor{myexamplebg}{HTML}{F2FBF8}
\definecolor{myexamplefr}{HTML}{88D6D1}
\definecolor{myexampleti}{HTML}{2A7F7F}
\definecolor{mydefinitbg}{HTML}{E5E5FF}
\definecolor{mydefinitfr}{HTML}{3F3FA3}
\definecolor{notesgreen}{RGB}{0,162,0}
\definecolor{myp}{RGB}{197, 92, 212}
\definecolor{mygr}{HTML}{2C3338}
\definecolor{myred}{RGB}{127,0,0}
\definecolor{myyellow}{RGB}{169,121,69}
\definecolor{myexercisebg}{HTML}{F2FBF8}
\definecolor{myexercisefg}{HTML}{88D6D1}


%%%%%%%%%%%%%%%%%%%%%%%%%%%%
% TCOLORBOX SETUPS
%%%%%%%%%%%%%%%%%%%%%%%%%%%%

\setlength{\parindent}{1cm}
%================================
% THEOREM BOX
%================================

\tcbuselibrary{theorems,skins,hooks}
\newtcbtheorem[number within=section]{Theorem}{Theorem}
{%
	enhanced,
	breakable,
	colback = mytheorembg,
	frame hidden,
	boxrule = 0sp,
	borderline west = {2pt}{0pt}{mytheoremfr},
	sharp corners,
	detach title,
	before upper = \tcbtitle\par\smallskip,
	coltitle = mytheoremfr,
	fonttitle = \bfseries\sffamily,
	description font = \mdseries,
	separator sign none,
	segmentation style={solid, mytheoremfr},
}
{th}

\tcbuselibrary{theorems,skins,hooks}
\newtcbtheorem[number within=chapter]{theorem}{Theorem}
{%
	enhanced,
	breakable,
	colback = mytheorembg,
	frame hidden,
	boxrule = 0sp,
	borderline west = {2pt}{0pt}{mytheoremfr},
	sharp corners,
	detach title,
	before upper = \tcbtitle\par\smallskip,
	coltitle = mytheoremfr,
	fonttitle = \bfseries\sffamily,
	description font = \mdseries,
	separator sign none,
	segmentation style={solid, mytheoremfr},
}
{th}


\tcbuselibrary{theorems,skins,hooks}
\newtcolorbox{Theoremcon}
{%
	enhanced
	,breakable
	,colback = mytheorembg
	,frame hidden
	,boxrule = 0sp
	,borderline west = {2pt}{0pt}{mytheoremfr}
	,sharp corners
	,description font = \mdseries
	,separator sign none
}

%================================
% Corollery
%================================
\tcbuselibrary{theorems,skins,hooks}
\newtcbtheorem[number within=section]{Corollary}{Corollary}
{%
	enhanced
	,breakable
	,colback = myp!10
	,frame hidden
	,boxrule = 0sp
	,borderline west = {2pt}{0pt}{myp!85!black}
	,sharp corners
	,detach title
	,before upper = \tcbtitle\par\smallskip
	,coltitle = myp!85!black
	,fonttitle = \bfseries\sffamily
	,description font = \mdseries
	,separator sign none
	,segmentation style={solid, myp!85!black}
}
{th}
\tcbuselibrary{theorems,skins,hooks}
\newtcbtheorem[number within=chapter]{corollary}{Corollary}
{%
	enhanced
	,breakable
	,colback = myp!10
	,frame hidden
	,boxrule = 0sp
	,borderline west = {2pt}{0pt}{myp!85!black}
	,sharp corners
	,detach title
	,before upper = \tcbtitle\par\smallskip
	,coltitle = myp!85!black
	,fonttitle = \bfseries\sffamily
	,description font = \mdseries
	,separator sign none
	,segmentation style={solid, myp!85!black}
}
{th}


%================================
% LENMA
%================================

\tcbuselibrary{theorems,skins,hooks}
\newtcbtheorem[number within=section]{Lenma}{Lenma}
{%
	enhanced,
	breakable,
	colback = mylenmabg,
	frame hidden,
	boxrule = 0sp,
	borderline west = {2pt}{0pt}{mylenmafr},
	sharp corners,
	detach title,
	before upper = \tcbtitle\par\smallskip,
	coltitle = mylenmafr,
	fonttitle = \bfseries\sffamily,
	description font = \mdseries,
	separator sign none,
	segmentation style={solid, mylenmafr},
}
{th}

\tcbuselibrary{theorems,skins,hooks}
\newtcbtheorem[number within=chapter]{lenma}{Lenma}
{%
	enhanced,
	breakable,
	colback = mylenmabg,
	frame hidden,
	boxrule = 0sp,
	borderline west = {2pt}{0pt}{mylenmafr},
	sharp corners,
	detach title,
	before upper = \tcbtitle\par\smallskip,
	coltitle = mylenmafr,
	fonttitle = \bfseries\sffamily,
	description font = \mdseries,
	separator sign none,
	segmentation style={solid, mylenmafr},
}
{th}


%================================
% PROPOSITION
%================================

\tcbuselibrary{theorems,skins,hooks}
\newtcbtheorem[number within=section]{Prop}{Proposition}
{%
	enhanced,
	breakable,
	colback = mypropbg,
	frame hidden,
	boxrule = 0sp,
	borderline west = {2pt}{0pt}{mypropfr},
	sharp corners,
	detach title,
	before upper = \tcbtitle\par\smallskip,
	coltitle = mypropfr,
	fonttitle = \bfseries\sffamily,
	description font = \mdseries,
	separator sign none,
	segmentation style={solid, mypropfr},
}
{th}

\tcbuselibrary{theorems,skins,hooks}
\newtcbtheorem[number within=chapter]{prop}{Proposition}
{%
	enhanced,
	breakable,
	colback = mypropbg,
	frame hidden,
	boxrule = 0sp,
	borderline west = {2pt}{0pt}{mypropfr},
	sharp corners,
	detach title,
	before upper = \tcbtitle\par\smallskip,
	coltitle = mypropfr,
	fonttitle = \bfseries\sffamily,
	description font = \mdseries,
	separator sign none,
	segmentation style={solid, mypropfr},
}
{th}


%================================
% CLAIM
%================================

\tcbuselibrary{theorems,skins,hooks}
\newtcbtheorem[number within=section]{claim}{Claim}
{%
	enhanced
	,breakable
	,colback = myg!10
	,frame hidden
	,boxrule = 0sp
	,borderline west = {2pt}{0pt}{myg}
	,sharp corners
	,detach title
	,before upper = \tcbtitle\par\smallskip
	,coltitle = myg!85!black
	,fonttitle = \bfseries\sffamily
	,description font = \mdseries
	,separator sign none
	,segmentation style={solid, myg!85!black}
}
{th}



%================================
% Exercise
%================================

\tcbuselibrary{theorems,skins,hooks}
\newtcbtheorem[number within=section]{Exercise}{Exercise}
{%
	enhanced,
	breakable,
	colback = myexercisebg,
	frame hidden,
	boxrule = 0sp,
	borderline west = {2pt}{0pt}{myexercisefg},
	sharp corners,
	detach title,
	before upper = \tcbtitle\par\smallskip,
	coltitle = myexercisefg,
	fonttitle = \bfseries\sffamily,
	description font = \mdseries,
	separator sign none,
	segmentation style={solid, myexercisefg},
}
{th}

\tcbuselibrary{theorems,skins,hooks}
\newtcbtheorem[number within=chapter]{exercise}{Exercise}
{%
	enhanced,
	breakable,
	colback = myexercisebg,
	frame hidden,
	boxrule = 0sp,
	borderline west = {2pt}{0pt}{myexercisefg},
	sharp corners,
	detach title,
	before upper = \tcbtitle\par\smallskip,
	coltitle = myexercisefg,
	fonttitle = \bfseries\sffamily,
	description font = \mdseries,
	separator sign none,
	segmentation style={solid, myexercisefg},
}
{th}

%================================
% EXAMPLE BOX
%================================

\newtcbtheorem[number within=section]{Example}{Example}
{%
	colback = myexamplebg
	,breakable
	,colframe = myexamplefr
	,coltitle = myexampleti
	,boxrule = 1pt
	,sharp corners
	,detach title
	,before upper=\tcbtitle\par\smallskip
	,fonttitle = \bfseries
	,description font = \mdseries
	,separator sign none
	,description delimiters parenthesis
}
{ex}

\newtcbtheorem[number within=chapter]{example}{Example}
{%
	colback = myexamplebg
	,breakable
	,colframe = myexamplefr
	,coltitle = myexampleti
	,boxrule = 1pt
	,sharp corners
	,detach title
	,before upper=\tcbtitle\par\smallskip
	,fonttitle = \bfseries
	,description font = \mdseries
	,separator sign none
	,description delimiters parenthesis
}
{ex}

%================================
% DEFINITION BOX
%================================

\newtcbtheorem[number within=section]{Definition}{Definition}{enhanced,
	before skip=2mm,after skip=2mm, colback=red!5,colframe=red!80!black,boxrule=0.5mm,
	attach boxed title to top left={xshift=1cm,yshift*=1mm-\tcboxedtitleheight}, varwidth boxed title*=-3cm,
	boxed title style={frame code={
					\path[fill=tcbcolback]
					([yshift=-1mm,xshift=-1mm]frame.north west)
					arc[start angle=0,end angle=180,radius=1mm]
					([yshift=-1mm,xshift=1mm]frame.north east)
					arc[start angle=180,end angle=0,radius=1mm];
					\path[left color=tcbcolback!60!black,right color=tcbcolback!60!black,
						middle color=tcbcolback!80!black]
					([xshift=-2mm]frame.north west) -- ([xshift=2mm]frame.north east)
					[rounded corners=1mm]-- ([xshift=1mm,yshift=-1mm]frame.north east)
					-- (frame.south east) -- (frame.south west)
					-- ([xshift=-1mm,yshift=-1mm]frame.north west)
					[sharp corners]-- cycle;
				},interior engine=empty,
		},
	fonttitle=\bfseries,
	title={#2},#1}{def}
\newtcbtheorem[number within=chapter]{definition}{Definition}{enhanced,
	before skip=2mm,after skip=2mm, colback=red!5,colframe=red!80!black,boxrule=0.5mm,
	attach boxed title to top left={xshift=1cm,yshift*=1mm-\tcboxedtitleheight}, varwidth boxed title*=-3cm,
	boxed title style={frame code={
					\path[fill=tcbcolback]
					([yshift=-1mm,xshift=-1mm]frame.north west)
					arc[start angle=0,end angle=180,radius=1mm]
					([yshift=-1mm,xshift=1mm]frame.north east)
					arc[start angle=180,end angle=0,radius=1mm];
					\path[left color=tcbcolback!60!black,right color=tcbcolback!60!black,
						middle color=tcbcolback!80!black]
					([xshift=-2mm]frame.north west) -- ([xshift=2mm]frame.north east)
					[rounded corners=1mm]-- ([xshift=1mm,yshift=-1mm]frame.north east)
					-- (frame.south east) -- (frame.south west)
					-- ([xshift=-1mm,yshift=-1mm]frame.north west)
					[sharp corners]-- cycle;
				},interior engine=empty,
		},
	fonttitle=\bfseries,
	title={#2},#1}{def}



%================================
% Solution BOX
%================================

\makeatletter
\newtcbtheorem{question}{Question}{enhanced,
	breakable,
	colback=white,
	colframe=myb!80!black,
	attach boxed title to top left={yshift*=-\tcboxedtitleheight},
	fonttitle=\bfseries,
	title={#2},
	boxed title size=title,
	boxed title style={%
			sharp corners,
			rounded corners=northwest,
			colback=tcbcolframe,
			boxrule=0pt,
		},
	underlay boxed title={%
			\path[fill=tcbcolframe] (title.south west)--(title.south east)
			to[out=0, in=180] ([xshift=5mm]title.east)--
			(title.center-|frame.east)
			[rounded corners=\kvtcb@arc] |-
			(frame.north) -| cycle;
		},
	#1
}{def}
\makeatother

%================================
% SOLUTION BOX
%================================

\makeatletter
\newtcolorbox{solution}{enhanced,
	breakable,
	colback=white,
	colframe=myg!80!black,
	attach boxed title to top left={yshift*=-\tcboxedtitleheight},
	title=Solution,
	boxed title size=title,
	boxed title style={%
			sharp corners,
			rounded corners=northwest,
			colback=tcbcolframe,
			boxrule=0pt,
		},
	underlay boxed title={%
			\path[fill=tcbcolframe] (title.south west)--(title.south east)
			to[out=0, in=180] ([xshift=5mm]title.east)--
			(title.center-|frame.east)
			[rounded corners=\kvtcb@arc] |-
			(frame.north) -| cycle;
		},
}
\makeatother

%================================
% Question BOX
%================================

\makeatletter
\newtcbtheorem{qstion}{Question}{enhanced,
	breakable,
	colback=white,
	colframe=mygr,
	attach boxed title to top left={yshift*=-\tcboxedtitleheight},
	fonttitle=\bfseries,
	title={#2},
	boxed title size=title,
	boxed title style={%
			sharp corners,
			rounded corners=northwest,
			colback=tcbcolframe,
			boxrule=0pt,
		},
	underlay boxed title={%
			\path[fill=tcbcolframe] (title.south west)--(title.south east)
			to[out=0, in=180] ([xshift=5mm]title.east)--
			(title.center-|frame.east)
			[rounded corners=\kvtcb@arc] |-
			(frame.north) -| cycle;
		},
	#1
}{def}
\makeatother

\newtcbtheorem[number within=chapter]{wconc}{Wrong Concept}{
	breakable,
	enhanced,
	colback=white,
	colframe=myr,
	arc=0pt,
	outer arc=0pt,
	fonttitle=\bfseries\sffamily\large,
	colbacktitle=myr,
	attach boxed title to top left={},
	boxed title style={
			enhanced,
			skin=enhancedfirst jigsaw,
			arc=3pt,
			bottom=0pt,
			interior style={fill=myr}
		},
	#1
}{def}



%================================
% NOTE BOX
%================================

\usetikzlibrary{arrows,calc,shadows.blur}
\tcbuselibrary{skins}
\newtcolorbox{note}[1][]{%
	enhanced jigsaw,
	colback=gray!20!white,%
	colframe=gray!80!black,
	size=small,
	boxrule=1pt,
	title=\textbf{Note:-},
	halign title=flush center,
	coltitle=black,
	breakable,
	drop shadow=black!50!white,
	attach boxed title to top left={xshift=1cm,yshift=-\tcboxedtitleheight/2,yshifttext=-\tcboxedtitleheight/2},
	minipage boxed title=1.5cm,
	boxed title style={%
			colback=white,
			size=fbox,
			boxrule=1pt,
			boxsep=2pt,
			underlay={%
					\coordinate (dotA) at ($(interior.west) + (-0.5pt,0)$);
					\coordinate (dotB) at ($(interior.east) + (0.5pt,0)$);
					\begin{scope}
						\clip (interior.north west) rectangle ([xshift=3ex]interior.east);
						\filldraw [white, blur shadow={shadow opacity=60, shadow yshift=-.75ex}, rounded corners=2pt] (interior.north west) rectangle (interior.south east);
					\end{scope}
					\begin{scope}[gray!80!black]
						\fill (dotA) circle (2pt);
						\fill (dotB) circle (2pt);
					\end{scope}
				},
		},
	#1,
}

%%%%%%%%%%%%%%%%%%%%%%%%%%%%%%
% SELF MADE COMMANDS
%%%%%%%%%%%%%%%%%%%%%%%%%%%%%%


\newcommand{\thm}[2]{\begin{Theorem}{#1}{}#2\end{Theorem}}
\newcommand{\cor}[2]{\begin{Corollary}{#1}{}#2\end{Corollary}}
\newcommand{\mlenma}[2]{\begin{Lenma}{#1}{}#2\end{Lenma}}
\newcommand{\mprop}[2]{\begin{Prop}{#1}{}#2\end{Prop}}
\newcommand{\clm}[3]{\begin{claim}{#1}{#2}#3\end{claim}}
\newcommand{\wc}[2]{\begin{wconc}{#1}{}\setlength{\parindent}{1cm}#2\end{wconc}}
\newcommand{\thmcon}[1]{\begin{Theoremcon}{#1}\end{Theoremcon}}
\newcommand{\ex}[2]{\begin{Example}{#1}{}#2\end{Example}}
\newcommand{\dfn}[2]{\begin{Definition}[colbacktitle=red!75!black]{#1}{}#2\end{Definition}}
\newcommand{\dfnc}[2]{\begin{definition}[colbacktitle=red!75!black]{#1}{}#2\end{definition}}
\newcommand{\qs}[2]{\begin{question}{#1}{}#2\end{question}}
\newcommand{\pf}[2]{\begin{myproof}[#1]#2\end{myproof}}
\newcommand{\nt}[1]{\begin{note}#1\end{note}}

\newcommand*\circled[1]{\tikz[baseline=(char.base)]{
		\node[shape=circle,draw,inner sep=1pt] (char) {#1};}}
\newcommand\getcurrentref[1]{%
	\ifnumequal{\value{#1}}{0}
	{??}
	{\the\value{#1}}%
}
\newcommand{\getCurrentSectionNumber}{\getcurrentref{section}}
\newenvironment{myproof}[1][\proofname]{%
	\proof[\bfseries #1: ]%
}{\endproof}

\newcommand{\mclm}[2]{\begin{myclaim}[#1]#2\end{myclaim}}
\newenvironment{myclaim}[1][\claimname]{\proof[\bfseries #1: ]}{}

\newcounter{mylabelcounter}

\makeatletter
\newcommand{\setword}[2]{%
	\phantomsection
	#1\def\@currentlabel{\unexpanded{#1}}\label{#2}%
}
\makeatother




\tikzset{
	symbol/.style={
			draw=none,
			every to/.append style={
					edge node={node [sloped, allow upside down, auto=false]{$#1$}}}
		}
}


% deliminators
\DeclarePairedDelimiter{\abs}{\lvert}{\rvert}
\DeclarePairedDelimiter{\norm}{\lVert}{\rVert}

\DeclarePairedDelimiter{\ceil}{\lceil}{\rceil}
\DeclarePairedDelimiter{\floor}{\lfloor}{\rfloor}
\DeclarePairedDelimiter{\round}{\lfloor}{\rceil}

\newsavebox\diffdbox
\newcommand{\slantedromand}{{\mathpalette\makesl{d}}}
\newcommand{\makesl}[2]{%
\begingroup
\sbox{\diffdbox}{$\mathsurround=0pt#1\mathrm{#2}$}%
\pdfsave
\pdfsetmatrix{1 0 0.2 1}%
\rlap{\usebox{\diffdbox}}%
\pdfrestore
\hskip\wd\diffdbox
\endgroup
}
\newcommand{\dd}[1][]{\ensuremath{\mathop{}\!\ifstrempty{#1}{%
\slantedromand\@ifnextchar^{\hspace{0.2ex}}{\hspace{0.1ex}}}%
{\slantedromand\hspace{0.2ex}^{#1}}}}
\ProvideDocumentCommand\dv{o m g}{%
  \ensuremath{%
    \IfValueTF{#3}{%
      \IfNoValueTF{#1}{%
        \frac{\dd #2}{\dd #3}%
      }{%
        \frac{\dd^{#1} #2}{\dd #3^{#1}}%
      }%
    }{%
      \IfNoValueTF{#1}{%
        \frac{\dd}{\dd #2}%
      }{%
        \frac{\dd^{#1}}{\dd #2^{#1}}%
      }%
    }%
  }%
}
\providecommand*{\pdv}[3][]{\frac{\partial^{#1}#2}{\partial#3^{#1}}}
%  - others
\DeclareMathOperator{\Lap}{\mathcal{L}}
\DeclareMathOperator{\Var}{Var} % varience
\DeclareMathOperator{\Cov}{Cov} % covarience
\DeclareMathOperator{\E}{E} % expected

% Since the amsthm package isn't loaded

% I prefer the slanted \leq
\let\oldleq\leq % save them in case they're every wanted
\let\oldgeq\geq
\renewcommand{\leq}{\leqslant}
\renewcommand{\geq}{\geqslant}

% % redefine matrix env to allow for alignment, use r as default
% \renewcommand*\env@matrix[1][r]{\hskip -\arraycolsep
%     \let\@ifnextchar\new@ifnextchar
%     \array{*\c@MaxMatrixCols #1}}


%\usepackage{framed}
%\usepackage{titletoc}
%\usepackage{etoolbox}
%\usepackage{lmodern}


%\patchcmd{\tableofcontents}{\contentsname}{\sffamily\contentsname}{}{}

%\renewenvironment{leftbar}
%{\def\FrameCommand{\hspace{6em}%
%		{\color{myyellow}\vrule width 2pt depth 6pt}\hspace{1em}}%
%	\MakeFramed{\parshape 1 0cm \dimexpr\textwidth-6em\relax\FrameRestore}\vskip2pt%
%}
%{\endMakeFramed}

%\titlecontents{chapter}
%[0em]{\vspace*{2\baselineskip}}
%{\parbox{4.5em}{%
%		\hfill\Huge\sffamily\bfseries\color{myred}\thecontentspage}%
%	\vspace*{-2.3\baselineskip}\leftbar\textsc{\small\chaptername~\thecontentslabel}\\\sffamily}
%{}{\endleftbar}
%\titlecontents{section}
%[8.4em]
%{\sffamily\contentslabel{3em}}{}{}
%{\hspace{0.5em}\nobreak\itshape\color{myred}\contentspage}
%\titlecontents{subsection}
%[8.4em]
%{\sffamily\contentslabel{3em}}{}{}  
%{\hspace{0.5em}\nobreak\itshape\color{myred}\contentspage}



%%%%%%%%%%%%%%%%%%%%%%%%%%%%%%%%%%%%%%%%%%%
% TABLE OF CONTENTS
%%%%%%%%%%%%%%%%%%%%%%%%%%%%%%%%%%%%%%%%%%%

\usepackage{tikz}
\definecolor{doc}{RGB}{0,60,110}
\usepackage{titletoc}
\contentsmargin{0cm}
\titlecontents{chapter}[3.7pc]
{\addvspace{30pt}%
	\begin{tikzpicture}[remember picture, overlay]%
		\draw[fill=doc!60,draw=doc!60] (-7,-.1) rectangle (-0.9,.5);%
		\pgftext[left,x=-3.5cm,y=0.2cm]{\color{white}\Large\sc\bfseries Chapter\ \thecontentslabel};%
	\end{tikzpicture}\color{doc!60}\large\sc\bfseries}%
{}
{}
{\;\titlerule\;\large\sc\bfseries Page \thecontentspage
	\begin{tikzpicture}[remember picture, overlay]
		\draw[fill=doc!60,draw=doc!60] (2pt,0) rectangle (4,0.1pt);
	\end{tikzpicture}}%
\titlecontents{section}[3.7pc]
{\addvspace{2pt}}
{\contentslabel[\thecontentslabel]{2pc}}
{}
{\hfill\small \thecontentspage}
[]
\titlecontents*{subsection}[3.7pc]
{\addvspace{-1pt}\small}
{}
{}
{\ --- \small\thecontentspage}
[ \textbullet\ ][]

\makeatletter
\renewcommand{\tableofcontents}{%
	\chapter*{%
	  \vspace*{-20\p@}%
	  \begin{tikzpicture}[remember picture, overlay]%
		  \pgftext[right,x=15cm,y=0.2cm]{\color{doc!60}\Huge\sc\bfseries \contentsname};%
		  \draw[fill=doc!60,draw=doc!60] (13,-.75) rectangle (20,1);%
		  \clip (13,-.75) rectangle (20,1);
		  \pgftext[right,x=15cm,y=0.2cm]{\color{white}\Huge\sc\bfseries \contentsname};%
	  \end{tikzpicture}}%
	\@starttoc{toc}}
\makeatother


%From M275 "Topology" at SJSU
\newcommand{\id}{\mathrm{id}}
\newcommand{\taking}[1]{\xrightarrow{#1}}
\newcommand{\inv}{^{-1}}

%From M170 "Introduction to Graph Theory" at SJSU
\DeclareMathOperator{\diam}{diam}
\DeclareMathOperator{\ord}{ord}
\newcommand{\defeq}{\overset{\mathrm{def}}{=}}

%From the USAMO .tex files
\newcommand{\ts}{\textsuperscript}
\newcommand{\dg}{^\circ}
\newcommand{\ii}{\item}

% % From Math 55 and Math 145 at Harvard
% \newenvironment{subproof}[1][Proof]{%
% \begin{proof}[#1] \renewcommand{\qedsymbol}{$\blacksquare$}}%
% {\end{proof}}

\newcommand{\liff}{\leftrightarrow}
\newcommand{\lthen}{\rightarrow}
\newcommand{\opname}{\operatorname}
\newcommand{\surjto}{\twoheadrightarrow}
\newcommand{\injto}{\hookrightarrow}
\newcommand{\On}{\mathrm{On}} % ordinals
\DeclareMathOperator{\img}{im} % Image
\DeclareMathOperator{\Img}{Im} % Image
\DeclareMathOperator{\coker}{coker} % Cokernel
\DeclareMathOperator{\Coker}{Coker} % Cokernel
\DeclareMathOperator{\Ker}{Ker} % Kernel
\DeclareMathOperator{\rank}{rank}
\DeclareMathOperator{\Spec}{Spec} % spectrum
\DeclareMathOperator{\Tr}{Tr} % trace
\DeclareMathOperator{\pr}{pr} % projection
\DeclareMathOperator{\ext}{ext} % extension
\DeclareMathOperator{\pred}{pred} % predecessor
\DeclareMathOperator{\dom}{dom} % domain
\DeclareMathOperator{\ran}{ran} % range
\DeclareMathOperator{\Hom}{Hom} % homomorphism
\DeclareMathOperator{\Mor}{Mor} % morphisms
\DeclareMathOperator{\End}{End} % endomorphism

\newcommand{\eps}{\epsilon}
\newcommand{\veps}{\varepsilon}
\newcommand{\ol}{\overline}
\newcommand{\ul}{\underline}
\newcommand{\wt}{\widetilde}
\newcommand{\wh}{\widehat}
\newcommand{\vocab}[1]{\textbf{\color{blue} #1}}
\providecommand{\half}{\frac{1}{2}}
\newcommand{\dang}{\measuredangle} %% Directed angle
\newcommand{\ray}[1]{\overrightarrow{#1}}
\newcommand{\seg}[1]{\overline{#1}}
\newcommand{\arc}[1]{\wideparen{#1}}
\DeclareMathOperator{\cis}{cis}
\DeclareMathOperator*{\lcm}{lcm}
\DeclareMathOperator*{\argmin}{arg min}
\DeclareMathOperator*{\argmax}{arg max}
\newcommand{\cycsum}{\sum_{\mathrm{cyc}}}
\newcommand{\symsum}{\sum_{\mathrm{sym}}}
\newcommand{\cycprod}{\prod_{\mathrm{cyc}}}
\newcommand{\symprod}{\prod_{\mathrm{sym}}}
\newcommand{\Qed}{\begin{flushright}\qed\end{flushright}}
\newcommand{\parinn}{\setlength{\parindent}{1cm}}
\newcommand{\parinf}{\setlength{\parindent}{0cm}}
% \newcommand{\norm}{\|\cdot\|}
\newcommand{\inorm}{\norm_{\infty}}
\newcommand{\opensets}{\{V_{\alpha}\}_{\alpha\in I}}
\newcommand{\oset}{V_{\alpha}}
\newcommand{\opset}[1]{V_{\alpha_{#1}}}
\newcommand{\lub}{\text{lub}}
\newcommand{\del}[2]{\frac{\partial #1}{\partial #2}}
\newcommand{\Del}[3]{\frac{\partial^{#1} #2}{\partial^{#1} #3}}
\newcommand{\deld}[2]{\dfrac{\partial #1}{\partial #2}}
\newcommand{\Deld}[3]{\dfrac{\partial^{#1} #2}{\partial^{#1} #3}}
\newcommand{\lm}{\lambda}
\newcommand{\uin}{\mathbin{\rotatebox[origin=c]{90}{$\in$}}}
\newcommand{\usubset}{\mathbin{\rotatebox[origin=c]{90}{$\subset$}}}
\newcommand{\lt}{\left}
\newcommand{\rt}{\right}
\newcommand{\bs}[1]{\boldsymbol{#1}}
\newcommand{\exs}{\exists}
\newcommand{\st}{\strut}
\newcommand{\dps}[1]{\displaystyle{#1}}

\newcommand{\sol}{\setlength{\parindent}{0cm}\textbf{\textit{Solution:}}\setlength{\parindent}{1cm} }
\newcommand{\solve}[1]{\setlength{\parindent}{0cm}\textbf{\textit{Solution: }}\setlength{\parindent}{1cm}#1 \Qed}

%--------------------------------------------------
% LIE ALGEBRAS
%--------------------------------------------------
\newcommand*{\kb}{\mathfrak{b}}  % Borel subalgebra
\newcommand*{\kg}{\mathfrak{g}}  % Lie algebra
\newcommand*{\kh}{\mathfrak{h}}  % Cartan subalgebra
\newcommand*{\kn}{\mathfrak{n}}  % Nilradical
\newcommand*{\ku}{\mathfrak{u}}  % Unipotent algebra
\newcommand*{\kz}{\mathfrak{z}}  % Center of algebra

%--------------------------------------------------
% HOMOLOGICAL ALGEBRA
%--------------------------------------------------
\DeclareMathOperator{\Ext}{Ext} % Ext functor
\DeclareMathOperator{\Tor}{Tor} % Tor functor

%--------------------------------------------------
% MATRIX & GROUP NOTATION
%--------------------------------------------------
\DeclareMathOperator{\GL}{GL} % General Linear Group
\DeclareMathOperator{\SL}{SL} % Special Linear Group
\newcommand*{\gl}{\operatorname{\mathfrak{gl}}} % General linear Lie algebra
\newcommand*{\sl}{\operatorname{\mathfrak{sl}}} % Special linear Lie algebra

%--------------------------------------------------
% NUMBER SETS
%--------------------------------------------------
\newcommand*{\RR}{\mathbb{R}}
\newcommand*{\NN}{\mathbb{N}}
\newcommand*{\ZZ}{\mathbb{Z}}
\newcommand*{\QQ}{\mathbb{Q}}
\newcommand*{\CC}{\mathbb{C}}
\newcommand*{\PP}{\mathbb{P}}
\newcommand*{\HH}{\mathbb{H}}
\newcommand*{\FF}{\mathbb{F}}
\newcommand*{\EE}{\mathbb{E}} % Expected Value

%--------------------------------------------------
% MATH SCRIPT, FRAKTUR, AND BOLD SYMBOLS
%--------------------------------------------------
\newcommand*{\mcA}{\mathcal{A}}
\newcommand*{\mcB}{\mathcal{B}}
\newcommand*{\mcC}{\mathcal{C}}
\newcommand*{\mcD}{\mathcal{D}}
\newcommand*{\mcE}{\mathcal{E}}
\newcommand*{\mcF}{\mathcal{F}}
\newcommand*{\mcG}{\mathcal{G}}
\newcommand*{\mcH}{\mathcal{H}}

\newcommand*{\mfA}{\mathfrak{A}}  \newcommand*{\mfB}{\mathfrak{B}}
\newcommand*{\mfC}{\mathfrak{C}}  \newcommand*{\mfD}{\mathfrak{D}}
\newcommand*{\mfE}{\mathfrak{E}}  \newcommand*{\mfF}{\mathfrak{F}}
\newcommand*{\mfG}{\mathfrak{G}}  \newcommand*{\mfH}{\mathfrak{H}}

\usepackage{bm} % Ensure bold math works correctly
\newcommand*{\bmA}{\bm{A}}
\newcommand*{\bmB}{\bm{B}}
\newcommand*{\bmC}{\bm{C}}
\newcommand*{\bmD}{\bm{D}}
\newcommand*{\bmE}{\bm{E}}
\newcommand*{\bmF}{\bm{F}}
\newcommand*{\bmG}{\bm{G}}
\newcommand*{\bmH}{\bm{H}}

%--------------------------------------------------
% FUNCTIONAL ANALYSIS & ALGEBRA
%--------------------------------------------------
\DeclareMathOperator{\Aut}{Aut} % Automorphism group
\DeclareMathOperator{\Inn}{Inn} % Inner automorphisms
\DeclareMathOperator{\Syl}{Syl} % Sylow subgroups
\DeclareMathOperator{\Gal}{Gal} % Galois group
\DeclareMathOperator{\sign}{sign} % Sign function


%\usepackage[tagged, highstructure]{accessibility}
\usepackage{tocloft}
\usepackage{arydshln}
\usetikzlibrary{arrows.meta, decorations.pathreplacing}




\begin{document}
\title{Linear Algebra I}
\author{Lecture Notes Provided by Dr.~Miriam Logan.}
\date{}
\maketitle
\tableofcontents
\newpage  
  \section{Kernel and Image}
  \ex{}{
  \textit{ Find the kernel and image of $ T: \mathbb{R} ^2 \to \mathbb{R} ^2$
   \[
   T \begin{bmatrix}
   x\\
   y\\
   \end{bmatrix}
   = \begin{bmatrix}
   0 & 1\\
   1 & 0\\
   \end{bmatrix} \begin{bmatrix}
   x\\
   y\\
   \end{bmatrix}
   = \begin{bmatrix}
   y\\
   x\\
   \end{bmatrix}
   \]
   \textbf{ Solution:}\\
   \begin{align*}
	   \text{ ker } T  = \left\{ \begin{bmatrix}
	   x\\
	   y \\
	   \end{bmatrix}
	   \in \mathbb{R} ^2 \mid T \begin{bmatrix}
	   x\\
	   y\\
	   \end{bmatrix}         
	   = \begin{bmatrix}
	   0\\
	   0\\
	   \end{bmatrix}          \right\}
	    \right\} \\
	    & = \left\{ \begin{bmatrix}
	    x\\
	    y\\
	    \end{bmatrix}
	    \in \mathbb{R} ^2 \mid \begin{bmatrix}
	    y\\
	    x\\
	    \end{bmatrix}
	    = \begin{bmatrix}
	    0\\
	    0\\
	    \end{bmatrix}
	     \right\}\\
	     &= \left\{ \begin{bmatrix}
	     x\\
	     y\\
	     \end{bmatrix}
	     \in \mathbb{R} ^2 \mid \begin{bmatrix}
	     y\\
	     x\\
	     \end{bmatrix}
	     = \begin{bmatrix}
	     0\\
	     0\\
	     \end{bmatrix}  \right\}\\
	     & = \left\{  \begin{bmatrix}
	     x\\
	     y\\
	     \end{bmatrix}
	     \in \mathbb{R} ^2 \mid x=0 , y=0 \right\} 
	      \right\}                       \\
	      & = \left\{ \begin{bmatrix}
	      0\\
	      0\\
	      \end{bmatrix}
	       \right\} 
   .\end{align*}
  }
  \begin{align*}
	  Im T =  \left\{ \begin{bmatrix}
	  x\\
	  y\\
	  \end{bmatrix}
	  \in \mathbb{R} ^2 \mid T \begin{bmatrix}
	  u\\
	  v\\
	  \end{bmatrix}
	  = \begin{bmatrix}
	  x\\
	  y\\
	  \end{bmatrix}
	   \text{ for some }  \begin{bmatrix}
	   u\\
	   v\\
	   \end{bmatrix}
	    \in \mathbb{R} ^2\right\} \\
	    & = \left\{ \begin{bmatrix}
	    x\\
	    y\\
	    \end{bmatrix}
	    \in \mathbb{R} ^2 \mid \begin{bmatrix}
	    v\\
	    u\\
	    \end{bmatrix}
	    = \begin{bmatrix}
	    x\\
	    y\\
	    \end{bmatrix}
	    , \quad u,v \in \mathbb{R} \right\}  \\
	    & =  \mathbb{R} ^2\\
  .\end{align*}
    
}

\nt{
Suppose $ A \in M_{m \times  n}$  and let $ T: \mathbb{R} ^n \to \mathbb{R} ^m$ be defined as $ T \left( \vec{ x}  \right) = A \vec{ x}  \forall  \vec{ x} \in \mathbb{R} ^{n}$ .
\textit{Recall: $ A$ is the $m \times  n $ matrix whose $i^{\text{th}}$ column is $ T \left( \vec{ e_i}  \right) $
}
\\
\begin{align*}
 ker T &= \left\{ \vec{ x} \in \mathbb{R} ^n \mid T \left( \vec{ x}  \right) = \vec{ 0 } \right\} \\
 &= \left\{ \vec{ x} \in \mathbb{R} ^n \mid A \vec{ x} = \vec{ 0 } \right\}\\
 &= \mathcal{N} \left( A \right) \text{ (the null space of $ A$)}\\
.\end{align*}
\begin{align*}
 Im T &= \left\{ \vec{ y} \in \mathbb{R} ^m \mid \vec{ y} = T \left( \vec{ x}  \right) \text{ for some } \vec{ x} \in \mathbb{R} ^n \right\}\\
 &= \left\{ \vec{ y} \in \mathbb{R} ^m \mid \vec{ y} = A \vec{ x}  \text{ for some } \vec{ x} \in \mathbb{R} ^n \right\}\\
 &= \left\{ \vec{ y} \in \mathbb{R} ^m \mid \vec{ y} \text{ lies in the span of the columns of } A \right\}\\
 &= \mathcal{C} \left( A \right) \text{ (the column space of $ A$)}\\
.\end{align*} }
 \\
 \\
 \thm{}
 {
 Suppose $ T: V \to W$ is a linear transformation between the vector spaces $ V$ and $ W$.\\
 \begin{enumerate}[label=(\roman*)]
 \item ker T is a subspace of $ V$.
 \item Im T is a subspace of $ W$.
 \item dim (ker T) + dim (Im T) = dim (V).
 \end{enumerate}
 }
 
 \pf{Proof:}{
 (i)\\
 Since $ T \left(  \vec{ 0}_v \right) = \vec{ 0}_w \implies \vec{ 0}_v \in ker T  $ \\
 \begin{align*}
 	T \left(  \vec{ v_1} + \vec{ v_2}  \right) = T \left(  \vec{ v_1}  \right) + T \left(  \vec{ v_2}  \right) \\
	 	& = \vec{ 0} + \vec{ 0}\\
		& = \vec{ 0} 
 .\end{align*}
 $ \implies \vec{ v_1} + \vec{ v_2} \in  ker T $  i.e. ker T is closed under addition.\\
 \\
 Let $ \vec{ v} \in ker T$, $ \lambda \in \mathbb{R}$ i.e.  $ T \left(  \vec{ v}  \right) = \vec{ 0}_w $\\
 \[
 T \left( \lambda \vec{ v}  \right) = \lambda T \left(  \vec{ v}  \right) = \lambda \vec{ 0}= \vec{ 0}
 .\] 
 $ \implies \lambda \vec{ v} \in  ker T$  i.e.  ker T is closed under scalar multiplication.\\
 Hence ker T is a subspace of $ V$.\\
 \\
 (ii)\\
 Since $ T \left(  \vec{ 0}_v \right) = \vec{ 0}_w \implies \vec{ 0}_w \in Im T  $ \\
 Let $ \vec{ w_1}, \vec{ w_2} \in Im T$\\
 Then there exists $ \vec{ v_1}, \vec{ v_2} \in V$ such that $ T \left(  \vec{ v_1}  \right) = \vec{ w_1}$ and $ T \left(  \vec{ v_2}  \right) = \vec{ w_2}$\\
 \[
 \vec{ w_1} + \vec{ w_2} = T \left(  \vec{ v_1}  \right) + T \left(  \vec{ v_2}  \right) = T \left(  \vec{ v_1} + \vec{ v_2}  \right)
 .\] 
 Hence $ \vec{ w_1} + \vec{ w_2} \in Im T$  i.e. Im T is closed under addition.\\
 \\
 Let $ \vec{ w} \in Im T$, i.e.  $ \exists  \vec{ v}  \in V$ such that $ T \left(  \vec{ v}  \right) = \vec{ w}$\\
 \[
 \lambda \vec{ w} = \lambda T \left(  \vec{ v}  \right) = T \left(  \lambda \vec{ v}  \right)
 .\] 
 $ \implies w \in Im T $ i.e. Im T is closed under scalar multiplication.\\
 \\
 Hence Im T is a subspace of $ W$.\\
 \\
 \\
 (iii)\\
 In the case where $ V = \mathbb{R} ^{n}$ and $ W = \mathbb{R} ^{m}$ there exists $ A \in M_{m \times n}$  such that $ T \left(  \vec{ v}  \right) = A \vec{ v}  \forall  \vec{ x} \in \mathbb{R} ^{n}$ and hence, $ dim \left(  ker T \right) = dim \left( \mathcal{N} \left( A \right)  \right)  $ and $ dim \left(  Im T \right) = dim \left( \mathcal{C} \left( A \right) \right)  $ and so  $  dim \left(  ker T \right) + dim \left( Im T \right) = n = dim V$
 }
 
 We can consider now the more general case where $ V$ and $ W$ are arbitrary vector spaces $ T: V \to W$ linear transformation.\\
 Let $ \left\{ \vec{ u_1} , \vec{ u_2} , \ldots \vec{ u_k}  \right\}$ be a basis for $ ker T$. If span $ \left\{ \vec{ u_1} , \vec{ u_2} , \ldots \vec{ u_k}  \right\} \neq V $ we can extend this to a basis for $ V$ by adding vectors $ \vec{ b_1} , \vec{ b_2} , \ldots , \vec{ b_{\ell}}$ that lie outside span $ \left\{ \vec{ u_1} , \vec{ u_2} , \ldots \vec{ u_k}  \right\}$ such  that the set $ \left\{ \vec{ u_1} , \vec{ u_2} , \ldots \vec{ u_k} , \vec{ b_1} , \vec{ b_2} , \ldots , \vec{ b_{\ell}}  \right\}$ is a basis for $ V$.\\
 Then we can write any vector $ \vec{ v}  \in V$ as a linear combination of the basis vectors, i.e. $ \exists  \alpha _1, \alpha_2 , \ldots \alpha_k$ $ \gamma_1, \gamma_2, \ldots ,\gamma_{\ell} \in \mathbb{R}$ such that
 \[
 \vec{ v} = \alpha_1 \vec{ u_1} + \alpha_2 \vec{ u_2} + \ldots + \alpha_k \vec{ u_k} + \gamma_1 \vec{ b_1} + \gamma_2 \vec{ b_2} + \ldots + \gamma_{\ell} \vec{ b_{\ell}}
 .\] 
 By the linearity 
 \[
 T \left(  \vec{ v}  \right) = \alpha_1 T \left( \vec{ u_1}  \right) + \alpha_2 T \left( \vec{ u_2}  \right) + \ldots + \alpha_k T \left( \vec{ u_k}  \right) + \gamma_1 T \left( \vec{ b_1}  \right) + \gamma_2 T \left( \vec{ b_2}  \right) + \ldots + \gamma_{\ell} T \left( \vec{ b_{\ell}}  \right)
 .\] 
  \[
  T \left( \vec{ v}  \right) = \gamma_1 T \left( \vec{ b_1}  \right) + \gamma_2 T \left( \vec{ b_2}  \right) + \ldots + \gamma_{\ell} T \left( \vec{ b_{\ell}}  \right)
  .\] 
  Hence every vector $ T \left( \vec{ v}  \right) \in Im T$ can be written as a linear combination of the vectors $ T \left( \vec{ b_1}  \right) , T \left( \vec{ b_2}  \right) , \ldots , T \left( \vec{ b_{\ell}}  \right)$.\\
  \[
  \mathcal{F} = \left\{ T \left( \vec{ b_1}  \right) , T \left( \vec{ b_2}  \right) , \ldots , T \left( \vec{ b_{\ell}}  \right)  \right\} \text{ spans } Im T
  .\] 
\textit{Is \(\mathcal{F}\) linearly independent?}\\
Suppose, for the sake of contradiction (FTSOC), that \(\mathcal{F}\) is linearly dependent.  
Then there exist scalars \(c_1,c_2,\ldots,c_{\ell}\in\mathbb{R}\), not all zero, such that
\[
  c_1\,T(\vec b_1)+c_2\,T(\vec b_2)+\dots+c_{\ell}\,T(\vec b_{\ell})=\vec 0 .
\]
By the linearity of \(T\),
\[
  T\!\bigl(c_1\vec b_1+c_2\vec b_2+\dots+c_{\ell}\vec b_{\ell}\bigr)=\vec 0 .
\]
Hence \(c_1\vec b_1+c_2\vec b_2+\dots+c_{\ell}\vec b_{\ell}\in\ker T\).\\[6pt]
\(\{\vec u_1,\dots,\vec u_k\}\) forms a basis for \(\ker T\)\(\;\Rightarrow\;\) there exist scalars $ \lambda_1, \lambda_2 ,\ldots , \lambda_k \in \mathbb{R} $ such that
\[
	\sum\limits_{i=1}^{\ell } c_i \vec b_i = = \sum\limits_{j=1}^{k } \lambda_j \vec{u_j}
.\] 
Hence $ \sum\limits_{i=1}^{\ell}  c_i \vec{ b_i} - \sum\limits_{j=1}^{k} \lambda_j \vec{ u_j} = \vec{ 0}$\\
i.e. we have a linear combination of $ \left\{ \vec{ u_1} , \ldots , \vec{ u_k} , \vec{ b_1} , \ldots \vec{ b_{\ell}}  \right\}$ equal to the zero without all coefficients being zero which contradicts the fact that $ \left\{ \vec{ u_1} , \ldots , \vec{ u_k} , \vec{ b_1} , \ldots \vec{ b_{\ell}}  \right\}$ is a basis for $ V$.\\
Hence $ \mathcal{F} = \left\{ T \left( \vec{ b_1}  \right) , \ldots , T \left( \vec{ b_{\ell}}  \right)  \right\} $ forms a basis for $ Im T$.\\
$ \implies dim \left(  Im T \right) = \ell , \qquad dim \left(  ker T \right) =k$ and $ l+k =n$\\
$ \implies dim \left( Im T \right) + dim \left( ker T \right) = dim V$   \\
\\
 \dfn{Rank and Nullity:}{
                       Suppose $ T: V \to W$ is a linear transformation between the vector spaces $ V$ and $ W$.\\
		       \begin{enumerate}[label=(\roman*)]
		       \item $dim (Im T)$  is called the rank of $ T$ and is denoted by $ rk \left( T \right) $
		       \item         $ dim \left(  ker T \right) $ is called the nullity of $ T$ and is denoted by null $ \left( T \right) $
		       \end{enumerate}
		                  The previous theorem says:
				  \[
				  rk \left( T \right) + null \left( T \right) = dim \left( V \right) 
				  .\] 
				  This is known as the rank-nullity theorem.
 }
 \ex{}{
 Suppose $ I : V \to V$ is the identity mapping on $ V$.\\
 $ I \left(  \vec{  x}  \right) = \vec{ x}  \forall  \vec{ x} \in V$ \\
 \[
	 ker I = { \vec{ 0} } \qquad null I =0
 .\] 
 \[
 Im I = V \qquad rk I = dim V
 .\] 
 \[
 null I + rk I = dim V
 .\] 
 }
 \ex{}{
 Suppose $ T: \mathbb{R} ^2 \to \mathbb{R} ^2$ is defined as 
 $ T \begin{bmatrix}
 x\\
 y\\
 \end{bmatrix}
 = \begin{bmatrix}
 x\\
 0\\
 \end{bmatrix}
 $  i.e.  $ T$ projects $ \mathbb{R} ^2$ orthogonally onto the x-axis\\
 \begin{align*}
	 ker T = \left\{ \begin{bmatrix}
	 x\\
	 y\\
	 \end{bmatrix}
	 \in \mathbb{R} ^2 \mid \begin{bmatrix}
	 x\\
	 0\\
	 \end{bmatrix}
	 = \begin{bmatrix}
	 0\\
	 0\\
	 \end{bmatrix}
	  \right\} \\
	  & = \left\{ \begin{bmatrix}
	  0\\
	  y\\
	  \end{bmatrix}
	  \in \mathbb{R} ^2 \right\} = \text{ span } \begin{bmatrix}
	  0\\
	  1\\
	  \end{bmatrix}        \\
	  &= dim  ker T = 1 = null T
 .\end{align*}
 \begin{align*}
	 Im T = \left\{ \begin{bmatrix}
	 x\\
	 y\\
	 \end{bmatrix}
	 \in \mathbb{R} ^2 \mid \begin{bmatrix}
	 x\\
	 y\\
	 \end{bmatrix}
	 = T \begin{bmatrix}
	 u\\
	 v\\
	 \end{bmatrix}
	 \text{ for some } u ,v \in \mathbb{R} \right\}\\
	 &= \left\{ \begin{bmatrix}
	 x\\
	 y\\
	 \end{bmatrix}
	 \in \mathbb{R} ^2 \mid \begin{bmatrix}
	 x\\
	 y\\
	 \end{bmatrix}
	 = \begin{bmatrix}
	 u\\
	 0\\
	 \end{bmatrix}
	  \text{ for some } u \in \mathbb{R} \right\} \\
   &= \left\{ \begin{bmatrix}
   x\\
   0\\
   \end{bmatrix}
    \mid x \in \mathbb{R}\right\}= \text{ span } \begin{bmatrix}
    1\\
    0\\
    \end{bmatrix}
 .\end{align*}
 \[
 dim \left(  Im T \right) = 1 = rk T
 .\] 
 \[
 null T + rk T = dim \left(  \mathbb{R} ^2 \right) = 2
 .\] 
 }
 \ex{}{
  Consider the linear transformation $ D: \mathcal{P}_k \left[ x \right]  \to \mathcal{P}_{k} \left[ x \right]$ defined by $ D \left( p \left( x \right)  \right) = p ' \left( x \right) $ \\
$ ker D = \left\{ a_0 \in \mathcal{P} _k \left[ x \right] \right\} $, $ null D =1$.\\
$ Im D = \mathcal{P} _{k-1}\left[ x \right]$ , $ rk D = k$.\\
$ null D + rk D = k+1 = dim \left(  \mathcal{P}_k \left[ x \right] \right) $
 }
  \section{Injective, Surjective and Bijective Linear Transformations}
  Suppose $ V$ and $ W$ are vector spaces.\\
  \dfn{Injective Transformation:}{
  
   A linear transformation $ T: V \to W$ is called injective (or one-to-one) if and only if $ T \left( \vec{ u}  \right) = T \left(  \vec{ w}  \right) $ implies $ \vec{ u} = \vec{ w} $ i.e. $ T$ maps distinct vectors to distinct vectors.\\
   An injective linear transformation is also called a \underline{ monomorphism}.\\ 
  }
  \dfn{Surjective Transformation :}{
                               A linear transformation $ T: V \to W$ is said to be \underline{surjective} if and only if $ Im T = W$. \\
                               A surjective linear transformation is also called an \underline{epimorphism}.\\
  }
   \dfn{Bijective Transformation :}{
   A linear transformation that is both injective and surjective is called \underline{bijective} or a \underline{isomorphism}.\\
   }
   \mlem{}{
    Let $ T: V \to W $ be a linear transformation between the vector spaces $ V$ and $ W$.\\
    \[
    T \text{ is injective } \iff ker T = \left\{ \vec{ 0} \right\}
    .\] 
   }
  \pf{Proof:}{
  ( $ \Rightarrow$)\\
  Suppose $ T$ is injective and $ \vec{ v}  \in ker T$.\\
  Then $ T \left(  \vec{ v}  \right) = \vec{ 0} $ $ \implies \vec{ v} = \vec{ 0} $ since $ T$ is injective.\\
  $ \implies ker T = \left\{ \vec{ 0}  \right\}$
  }
   
  \pf{Proof:}{
  ( $ \Leftarrow$)\\
           Suppose $ ker T = \left\{ \vec{ 0}  \right\}$ and let $ \vec{ v} , \vec{ u}  \in V$ such that $ T \left( \vec{ u}  \right) = T \left( \vec{ v}  \right) $, then $ T \left( \vec{ u}  \right) - T \left( \vec{ v}  \right) = \vec{ 0} $.\\
           $ T \left( \vec{ u} - \vec{ v}  \right) = \vec{ 0} $ $ \implies \vec{ u}  - \vec{ v}  \in ker T = { \vec{ 0} }$ \\
           Hence $ \vec{ u}  - \vec{ v}  = \vec{ 0} $ $ \implies \vec{ u} = \vec{ v} $.\\
           Hence $ T$ is injective.
  }
 \ex{}{
 Suppose $ T: \mathbb{R} ^{5} \to \mathbb{R} ^{3}$ is defined as $ T \left( \vec{ x}  \right) = A \vec{ x} $ where
 \[
 A = \begin{bmatrix}
 1 & 2 & 1 & 4 & -1\\
 -1 & -1 & -4 & -2 & 3\\
 2 & 4 & 3 & 17 & 5\\
 \end{bmatrix}
 .\] 
 \textit{(i) Is $ T$ injective?}\\
 \textit{(ii) Is T surjective?}\\
              \\
              \textbf{Solution:}\\
              \begin{enumerate}[label=(\roman*)]
                \item 
                 $ T$ is injective $ \iff ker T = \vec{  {0}}$ i.e.  $ \mathcal{N } \left( A \right) = \left\{ \vec{ 0}  \right\}$.\\
                 $ \iff $ columns of $ A$ are linearly independent.\\
                 Since the columns of $ A$ lie in $ \mathbb{R} ^3$, and there are five of them they cannot be linearly independent.\\
                 Hence $ T$ is not injective.\\
                \item 
                 $ T $ is surjective $ \iff Im T = \mathbb{R} ^3$ i.e.  $ \mathcal{C} \left( A \right) = \mathbb{R} ^3$.\\
                 i.e.  if there is a pivot in every row of $ A$.\\
                 \[
                      \begin{bmatrix}
 1 & 2 & 1 & 4 & -1\\
 -1 & -1 & -4 & -2 & 3\\
 2 & 4 & 3 & 17 & 5\\
 \end{bmatrix}          \xrightarrow[ r_2 + r_1]{ r_3 -2 r_1} \begin{bmatrix}
 1 & 2 & 1 & 4 & -1\\
 0 & 1 & -3 & 2 & 2\\
 0 & 0 & 1 & 9 & 7\\
 \end{bmatrix}
                 .\] 
                 pivot entry in every row of $ A$ $ \implies T $ is surjective.\\
                \end{enumerate}
 }
  
 \ex{}{
  Suppose $ T: \mathcal{P}_2 \left[ x \right]$  \to $ M_{ 2 \times 2}$ is defined as
  \[
  T \left( ax^2 +bx +c \right) = \begin{bmatrix}
  a+b & a+c\\
  b-c & b+c\\
  \end{bmatrix} \qquad a,b,c \in \mathbb{R}
  .\] 
  We can check that $ T$ is linear transformation between the two vector spaces.\\
  \textit{Is $ T$ \begin{enumerate}[label=(\roman*)]
    \item injective?
    \item  surjective?
    \item bijective?
     \item Find a basis for $ Im T$ 
   \end{enumerate} }
      \begin{enumerate}[label=(\roman*)]
        \item 
         \[
        ker T = \left\{ a x^2 + bx + c \in \mathcal{P} _2 \left[ x \right] \mid a+b =0 , a+c =0 , b-c =0 , b+c =0\right\}
         .\] 
         which gives rise to a linear system in $ a,b,c$ with augmented matrix
         \[
\left[
\begin{array}{ccc|c}
 1 & 1 &  0 & 0\\
 1 & 0 &  1 & 0\\
 0 & 1 & -1 & 0\\
 0 & 1 &  1 & 0
\end{array}
\right]
\;\xrightarrow[\text{row reduce}]{\text{}}\;
\left[
\begin{array}{ccc|c}
 1 &  1 & 0 & 0\\
 0 & -1 & 1 & 0\\
 0 &  0 & 2 & 0\\
 0 &  0 & 0 & 0
\end{array}
\right].
\]
Pivot in every column $ \implies$ there is a unique solution to the system, namely $ a=b=c=0$.\\
         Hence $ ker T = \left\{ 0 \right\}$ i.e. $ T$ is injective.\\
        \item           Note: the rank-nullity theorem gives us that 
         \begin{align*}
          rk T + null T = dim \left( \mathcal{P} _2 \left[ x \right] \right)\\
          rk T + 0 = 3\\
          rk T = 3\\
          \text{ ie. } dim \left( Im T \right) = 3\\
          \text{ but } dim \left( M_{2 \times 2} \right) = 4\\
          \implies Im T \neq M_{2 \times 2} 
         .\end{align*}
         Hence $ T$ is not surjective.\\
        \item   Since $ T$ is not surjective, it thus is not bijective.\\
        \item  
         \begin{align*}
          Im T = \left\{ \begin{bmatrix}
          a+b & a+c\\
           b-c& b+c \\
          \end{bmatrix} \mid a, b,c \in \mathbb{R} \right\}                                    \\
          & = \left\{ a \begin{bmatrix}
          1 & 1\\
          0 & 0\\
          \end{bmatrix} + b \begin{bmatrix}
          1 & 0\\
          1 & 1\\
          \end{bmatrix} + c \begin{bmatrix}
          0 & 1\\
          1 & 1\\
          \end{bmatrix} \mid a,b,c \in \mathbb{R} \right\}  \\
         .\end{align*}
         Hence $ \mathcal{F} = \left\{ \begin{bmatrix}
         1 & 1\\
         0 & 0\\
         \end{bmatrix}, \begin{bmatrix}
         1 & 0\\
         1 & 1\\
         \end{bmatrix}, \begin{bmatrix}
         0 & 1\\
         -1 & 1\\
         \end{bmatrix} \right\}$  spans $ Im T$.\\
         $ dim \left(  Im T \right) = 3 \implies \mathcal{F} $ forms a basis for $ Im T$.\\
        \end{enumerate}
 }

     \mlem{}{
      For any linear transformation $ T: V \to W$ we have that $ rk \left( T \right) \leq min \left\{ dim V, dim W \right\}$\\
     }
      \pf{Proof:}{
       $ Im T$ is a subspace of $ W$ and hence $ dim \left(  Im T \right)= rk T \leq dim W$.\\
       On the other hand,  $ rk T + null T = dim V$ and $ null T \geq 0$ $ \implies rk T \leq dim V$.\\
       Hence $ rk T \leq min \left\{ dim V, dim W \right\}$
      }
      
     \mlem{}{
     Let $ T: V \to W$ be an injective linear transformation, and suppose $ dim V = dim W$ then $ T$ is an isomorphism.\\
    }
    \pf{Proof:}{
     Since $ T$ is injective, $ ker T = \left\{ \vec{ 0}  \right\}$ i.e.  $ null T = 0$. \\
     Hence $ rk T = dim V = dim W$\\
     i.e. $ dim \left( Im T \right) = dim W$ and, $ Im T$ is a subspace of $ W$, hence $ Im T = W$.\\
     i.e. $ T$ is surjective and hence $ T$ is an isomorphism.\\
    }
    
    \mlem{}{
     Let $ T: V \to W$ be a surjective linear transformation, and suppose $ dim V = dim W$ then $ T$ is an isomorphism.\\
    }
    \pf{Proof:}{
     $ T$ is surjective, hence $ Im T = W$.\\
     i.e. $ rk T = dim \left( Im T \right) = dim W = dim V$\\
     By the rank-nullity theorem, $ null T + rk T = dim V$\\
     but $ rk T = dim V$  i.e.  $ ker T = { \vec{ 0} }$ \\
     $ \implies T$ is injective and so $ T$ is an isomorphism.\\
    }
    \hline
    Give an example of a linear transformation $ T: \mathbb{R} ^2 \to \mathbb{R} ^2$ such that 
    \begin{enumerate}[label=(\roman*)]
      \item Null T =0
      \item Null T =2
      \item Rk T = 2
      \item Rk T = 0
      \end{enumerate}
       \textbf{Note:} Since $ dim \mathbb{R}^2 =2$ and $ rk T + null T = 2$ if $ rk T =2 $ then $ null T = 0$ i.e. the same linear transformation will work for both (i) and (iii) and if $ rk T =0$ then $ null T = 2$ i.e. the same linear transformation will work for both (ii) and (iv).\\
       \\
       \textit{ (i) and (iii) }\\
       $ T : \mathbb{R} ^2 \to \mathbb{R} ^2$ need $ ker T = { \vec{ 0} }$ and $ Im T = \mathbb{R} ^2$ if $ T \left(  \vec{ x}  \right) = A \vec{ x} $ for some $ 2 \times  2$  matrix $ A$ then we require tht $ W \left( A \right) = {\vec{ 0} } \qquad  \mathcal{C} \left( A \right) = \mathbb{R} ^2$ which will happen if there is a pivot in every column (or row) of $ A$  i.e.  if $ A$ is invertible.\\
    
       \[
       \text{\textbf{Example:} } A = \begin{bmatrix}
       1 & 0\\
       0 & 1\\
       \end{bmatrix} \quad \text{ or } A =  \begin{bmatrix}
       1 & 3\\
       2 & 4\\
       \end{bmatrix}
       .\] 
       \\
       \textit{ (ii) and (iv) }\\
       \\
       $ T: \mathbb{R} ^2 \to \mathbb{R} ^2$, $ T \left(  \vec{ x}  \right) = A \vec{ x} $. We require that $ ker T =  \mathcal{N} \left( A \right) = \mathbb{R} ^2 $ and $ Im T = \mathcal{C} \left( A \right) = { \vec{ 0} }$\\
       The only transformation that satisfies this is the zero transformation, $ A = \begin{bmatrix}
       0 & 0\\
       0& 0\\
       \end{bmatrix}$
       \section{Co-ordinate Vectors }
       Suppose $ V$ is a vector space with of dimension $ n$ and $ \mathcal{B} = \left\{ \vec{ b_1} ,\ldots , \vec{ b_n}  \right\} $ is a basis for $ V$.\\
       Let $ \vec{ v} \in V$. There exists a unique set of scalars $ \lambda_1, \lambda_2, \ldots , \lambda_n \in \mathbb{R}$ such that $ V = \lambda_1 \vec{ v_1} + \lambda_2 \vec{ v_2} + \ldots + \lambda _n \vec{ v_n} $.\\
       We define \[
        \left[ \vec{ v}  \right]_{ \mathcal{B}}  = \begin{bmatrix}
        \lambda_1\\
        \lambda_2\\
        \vdots\\
        \lambda_n\\
        \end{bmatrix} _{ \mathcal{B}} 
       .\] 
       to be the co-ordinate vector of $ \vec{ v} $ with respect to the basis $ \mathcal{B}$.\\
       Note: $ \left[ \vec{ v}  \right]_{ \mathcal{B}} \in \mathbb{R} ^{n}$ and, the co-ordinates are completely determined by the choice of basis $ \mathcal{B}$.\\
   \mlem{}{
   Suppose $ \mathcal{B} = \left\{ \vec{ b_1} , \ldots , \vec{ b_n}  \right\} $  is a basis for the vector space $ V$ of dimension $ n$. \\
   Let $ T: V \to \mathbb{R} ^{n}$ be defined as
   \[
   T \left( \vec{ v}  \right) = \left[ \vec{ v}  \right]_{ \mathcal{B}} 
   .\] 
   $ T$ is an isomorphism.
  }
  \pf{Proof:}{
   Let $ \vec{ v} , \vec{ w}  \in V$ $ \exists  \lambda_i , \alpha_i \qquad 1 \leq i \leq n$  such that
    \[
    \vec{ v} =  \sum\limits_{i=1}^{n} \alpha_i \vec{ b_i} \qquad \vec{ w} =  \sum\limits_{i=1}^{n} \lambda_i \vec{ b_i}
    .\] 
    \[
    \vec{ v} + \vec{ w} =  \sum\limits_{i=1}^{n} \left( \alpha_i + \lambda_i \right) \vec{ b_i}
    .\] 
    \[
    T \left( \vec{ v} +\vec{ w}  \right) = \left[ \vec{ v} + \vec{ w}  \right]_{ \mathcal{B}} = \begin{bmatrix}
    \lambda_1 + \alpha_1\\
    \lambda_2 + \alpha_2\\
    \vdots\\
    \lambda_n + \alpha_n\\
    \end{bmatrix}       _{ \mathcal{B}} =     \begin{bmatrix}
    \lambda_1 \\
    \lambda_2 \\
    \vdots\\
    \lambda_n \\
    \end{bmatrix}       _{ \mathcal{B}}  +    \begin{bmatrix}
     \alpha_1\\
     \alpha_2\\
    \vdots\\
     \alpha_n\\
    \end{bmatrix}       _{ \mathcal{B}}      = T \left( \vec{ v}  \right) + T \left( \vec{ w}  \right)
    .\] 
    Hence, $ T$ preserves vector addition.\\ 
    \\
    Suppose $ \beta \in \mathbb{R}$ 
    \[
    \beta \vec{ v} = \beta \sum\limits_{i=1}^{n} \lambda_i \vec{ b_i} = \sum\limits_{i=1}^{n} \beta \lambda_i \vec{ b_i}
    .\] 
    \[
    T \left(  \beta \vec{ v}  \right) = \begin{bmatrix}
    \beta\lambda_1\\
    \beta\lambda_2\\
    \vdots\\
    \beta\lambda_n\\
    \end{bmatrix} = \beta                \begin{bmatrix}
    \lambda_1\\
    \lambda_2\\
    \vdots\\
    \lambda_n\\
    \end{bmatrix}             = \beta T \left(  \vec{ v}  \right)
    .\] 
    Hence, $ T$ preserves scalar multiplication.\\
    \\
    Hence $ T$ is a linear transformation.\\
    \begin{align*}
      ker T &= \left\{ \vec{ v}  \in V \mid T \left(  \vec{ v}  \right) = \vec{ 0} _{ \mathcal{B}} \right\}\\
      &= \left\{ \sum\limits_{i=1}^{n} \lambda_i \vec{ b_i} \in V \mid  \begin{bmatrix}
      \lambda_1\\
      \lambda_2\\
     \vdots\\
      \lambda_n\\
      \end{bmatrix} _{ \mathcal{B}} = \begin{bmatrix}
      0\\
      0\\
      \vdots\\
      0\\
      \end{bmatrix} _{ \mathcal{B}} \right\}\\
      &= \left\{ \sum\limits_{i=1}^{n} \lambda_i \vec{ b_i} \in V \mid \lambda_i = 0 \forall i \right\}\\
      &= \left\{ \vec{ 0} _V \right\}\\
    .\end{align*}
    Hence $ T$ is an injective linear transformation. Since $ dim V = dim \mathbb{R} ^{n}$ by a previous lemma, $ T$ is an isomorphism.\\
    \\
    T is called the co-ordinate mapping from $ V$ to $ \mathbb{R} ^{n}$ determined by the basis $ \mathcal{B}$.\\
    \[
     \vec{ v}  \to \left[ \vec{ v}  \right] _{ \mathcal{B}}
    .\] 
  }
   \nt{
   \begin{enumerate}[label=(\roman*)]
     \item The entries in the co-ordinate vector depend on the choice of basis.\\
     \item                     The entries in the co-ordinate vector are the coefficients in the expansion of $ \vec{ v} $ relative to the basis.\\
     \item                                     By fixing a basis one can associate a vector in an abstract vector space $ V$, of dimension $ n$ with a vector in $ \mathbb{R} ^{n}$.\\
     \end{enumerate}
   }
   \ex{}{
    Let $ V = \mathcal{P} _2 \left[ x \right] = \left\{ a_0 + a_1 x +a_2 x^2  \mid a_i \in \mathbb{R}\right\} $.\\
    Let $ \mathcal{B} = \left\{ 1, x, x^2 \right\} $ and let $ \mathcal{C} =  \left\{ 1, 1+x , 1+x+x^2 \right\}$.\\
    $ \mathcal{B} $ and  $ \mathcal{C}$ form a basis for $ \mathcal{P}_2 \left[ x \right]$.\\
    Let $ \vec{ v} = 3 +2x +x^2 \in \mathcal{P} _2 \left[  x\right] $
    \begin{enumerate}[label=(\roman*)]
     \item Find $ \left[ \vec{ v}  \right]_{ \mathcal{B}} $
      \item   Find $ \left[ \vec{ v}  \right]_{ \mathcal{C}} $
      \end{enumerate}
    \textbf{Solution:}\\
    \textit{(i) }\\
    \[
    \vec{ v} = 3 + 2x + x^2 = 3 \cdot 1 + 2 \cdot x + 1 \cdot x^2
    .\] 
    \[
    \implies \left[ \vec{ v}  \right]_{ \mathcal{B}} = \begin{bmatrix}
    3\\
    2\\
    1\\
    \end{bmatrix}
    _{ \mathcal{B}}
    .\] 
    \\
    \textit{(ii) }\\
    \[
    \vec{ v} = 3 + 2x + x^2 = 3 \cdot 1 + 2 \cdot (1+x) - 1 \cdot (1+x+x^2)
    .\] 
    \[
    \implies \left[ \vec{ v}  \right]_{ \mathcal{C}} =  \begin{bmatrix}
    1\\
    1\\
    1\\
    \end{bmatrix}
    _{ \mathcal{C}}
    .\] 
   }

   \section{Expressing Vectors in $ \mathbb{R} ^2$ as co-ordinate vectors with respect to different bases}
   Let $ \mathcal{E} = \left\{ \vec{ e_1} , \vec{ e_2}  \right\} = \left\{ \begin{bmatrix}
   1\\
   0\\
   \end{bmatrix}
   , \begin{bmatrix}
   0\\
   1\\
   \end{bmatrix}
    \right\}$ be the standard basis for $ \mathbb{R} ^2$.\\
    Throughout what follows if there is no subscript on the co-ordinate vector, it is assumed that the co-ordinates are with respect to $ \mathcal{E}$ i.e.  $ \begin{bmatrix}
    x\\
    y\\
    \end{bmatrix}
    = \begin{bmatrix}
    x\\
    y\\
    \end{bmatrix}
    _{ \mathcal{E}}$ \\
    Let $ \mathcal{B} = \left\{ \begin{bmatrix}
    1\\
    0\\
    \end{bmatrix}
    , \begin{bmatrix}
    1\\
    2\\
    \end{bmatrix}
    \right\}$. $ \mathcal{B} $ forms a basis for $ \mathbb{R} ^2$.\\
    \textit{Find the co-ordinate vector  of $ \vec{ x} = \begin{bmatrix}
    1\\
    6\\
    \end{bmatrix}
   $ relative to the basis $ \mathcal{B}$.} i.e.  find $  \left[ \vec{ x}  \right] _{ \mathcal{B}}$.\\
  \[
  
    \begin{bmatrix}
    1\\
    6\\
    \end{bmatrix}
    = \lambda_1 \begin{bmatrix}
    1\\
    0\\
    \end{bmatrix}
    + \lambda_2 \begin{bmatrix}
    1\\
    2\\
    \end{bmatrix}
     \text{ where } \left[ \vec{ x}  \right]_{ \mathcal{B}} = \begin{bmatrix}
     \lambda_1\\
     \lambda_2\\
     \end{bmatrix}
  .\] 
   \begin{align*}
    &  \begin{bmatrix}
     1\\
     6\\
     \end{bmatrix}  =   \begin{bmatrix}
     1 & 1\\
     0 & 2\\
     \end{bmatrix} \begin{bmatrix}
     \lambda_1\\
     \lambda_2\\
     \end{bmatrix}
     _{ \mathcal{B}}\\
     &  \begin{bmatrix}
     1 & 1\\
     0 & 2\\
     \end{bmatrix} ^{-1} \begin{bmatrix}
     \lambda_1\\
     \lambda_2\\
     \end{bmatrix}
     _{ \mathcal{B}} \\
     &  \frac{1}{2} \begin{bmatrix}
     2 & -1\\
     0 & 1\\
     \end{bmatrix} \begin{bmatrix}
     1\\
     6\\
     \end{bmatrix}
     = \begin{bmatrix}
     \lambda_1\\
     \lambda_2\\
     \end{bmatrix}
     _{ \mathcal{B}}\\
     & \frac{1}{2} \begin{bmatrix}
     -4\\
     6\\
     \end{bmatrix}
     _{ \mathcal{B}}\\
     & \begin{bmatrix}
     -2\\
     3\\
     \end{bmatrix}
     _{ \mathcal{B}} = \begin{bmatrix}
     \lambda_1\\
     \lambda_2\\
     \end{bmatrix}
     _{ \mathcal{B}}\\
     & \implies \left[ \vec{ x}  \right]_{ \mathcal{B}} = \begin{bmatrix}
     \lambda_1\\
     \lambda_2\\
     \end{bmatrix}
     _{ \mathcal{B}} 
   .\end{align*}
    
   \begin{tikzpicture}[scale=0.9,
        axis/.style   = {thick, ->, >=Stealth},
        grid/.style   = {green!60, dashed, very thin},
        tick/.style   = {thick},
        violetpath/.style = {line width=4pt, violet!50, opacity=.55, rounded corners=1.5pt},
        violetline/.style = {thick, violet!70!black, >=Stealth},
        pinkvec/.style   = {thick, pink, >=Stealth}
      ]

  % ------------------------------------------------------------------
  % 1. Dotted grid
  \draw[grid] (-3.5,-0.5) grid (4.5,6.5);

  % ------------------------------------------------------------------
  % 2. Axes
  \draw[axis] (-3.5,0)  -- (4.5,0);     % x-axis
  \draw[axis] (0,-0.5)  -- (0,6.5);     % y-axis

  % ------------------------------------------------------------------
  % 3. Tick marks and labels (x)
  \foreach \x in {-3,-2,-1,1,2,3,4}{
      \draw[tick] (\x,0.14) -- (\x,-0.14) node[below] {\small $\x$};
  }

  % 4. Tick marks and labels (y)
  \foreach \y in {1,2,3,4,5,6}{
      \draw[tick] (0.14,\y) -- (-0.14,\y) node[left]  {\small $\y$};
  }

  % ------------------------------------------------------------------
  % 5. Small pink unit vector  (0,0) → (0,1)
  \draw[pinkvec] (0,0) -- (0,1);

  % ------------------------------------------------------------------
  % 6. L-shaped purple vector (0,0) → (1,0) → (1,6)
  %    Wide translucent “highlight” first …
  \draw[violetpath] (0,0) -- (1,0) -- (1,6);
  %    … then the narrow arrowed path on top for crisp outlines
  \draw[violetline,->] (0,0) -- (1,0);
  \draw[violetline,->] (1,0) -- (1,6);

  % (Optional) marker midway on the vertical part, as in the sketch:
  \fill[violet!50,opacity=.5] (1,2) circle (2.5pt);
\end{tikzpicture}
                   \\
   \\
                         \begin{tikzpicture}[scale=0.9,
        % ----- reusable styles ------------------------------------------
        axis/.style      = {very thick, ->, >=Stealth},
        tick/.style      = {very thick},
        diag/.style      = {blue!70, dashed, line width=.8pt},
        %  “high-lighter” underlay
        hlC/.style       = {line width=6pt, cyan!50, opacity=.35,
                            rounded corners=2pt},
        cyanvec/.style   = {very thick, cyan!70!black, ->, >=Stealth},
        pinkvec/.style   = {very thick, pink, ->, >=Stealth},
        purplevec/.style = {very thick, violet, ->, >=Stealth}
      ]

  % ----------------------------------------------------------------------
  % 0. Diagonal dashed family  y = x + k
  \foreach \k in {-3,-2,-1,0,1,2,3,4}{
      \draw[diag] (-3.5,{-3.5+\k}) -- (4.5,{4.5+\k});
  }

  % ----------------------------------------------------------------------
  % 1. Coordinate axes
  \draw[axis] (-3.5,0)  -- (4.5,0);   % x–axis
  \draw[axis] (0,-1.5)  -- (0,6.5);   % y–axis

  % 2. Tick marks and labels
  \foreach \x in {-3,-2,-1,1,2,3,4}{
      \draw[tick] (\x,0.14) -- (\x,-0.14) node[below] {\small $\x$};
  }
  \foreach \y in {-1,1,2,3,4,5,6}{
      \draw[tick] (0.14,\y) -- (-0.14,\y) node[left] {\small $\y$};
  }

  % ----------------------------------------------------------------------
  % 3. Turquoise vectors (from (-2,0))
  %    — translucent “high-lighter” first
  \draw[hlC] (-2,0) -- (0,0);
  \draw[hlC] (-2,0) -- (1,3);
  %    — crisp arrowed strokes on top
  \draw[cyanvec] (-2,0) -- (0,0);
  \draw[cyanvec] (-2,0) -- (1,3);

  % ----------------------------------------------------------------------
  % 4. Pink and purple basis vectors from the origin
  \draw[pinkvec]   (0,0) -- (1,0);   %  (1,0)
  \draw[purplevec] (0,0) -- (1,1);   %  (1,2)
\end{tikzpicture}
    \\
    \\


\end{document}                   
