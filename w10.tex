\documentclass{report}

%%%%%%%%%%%%%%%%%%%%%%%%%%%%%%%%%
% PACKAGE IMPORTS
%%%%%%%%%%%%%%%%%%%%%%%%%%%%%%%%%


\usepackage[tmargin=2cm,rmargin=1in,lmargin=1in,margin=0.85in,bmargin=2cm,footskip=.2in]{geometry}
\usepackage{amsmath,amsfonts,amsthm,amssymb,mathtools}
\usepackage[varbb]{newpxmath}
\usepackage{xfrac}
\usepackage[makeroom]{cancel}
\usepackage{bookmark}
\usepackage{enumitem}
\usepackage{hyperref,theoremref}
\hypersetup{
	pdftitle={Assignment},
	colorlinks=true, linkcolor=doc!90,
	bookmarksnumbered=true,
	bookmarksopen=true
}
\usepackage[most,many,breakable]{tcolorbox}
\usepackage{xcolor}
\usepackage{varwidth}
\usepackage{varwidth}
\usepackage{tocloft}
\usepackage{etoolbox}
\usepackage{derivative} %many derivativess partials
%\usepackage{authblk}
\usepackage{nameref}
\usepackage{multicol,array}
\usepackage{tikz-cd}
\usepackage[ruled,vlined,linesnumbered]{algorithm2e}
\usepackage{comment} % enables the use of multi-line comments (\ifx \fi) 
\usepackage{import}
\usepackage{xifthen}
\usepackage{pdfpages}
\usepackage{transparent}
\usepackage{verbatim}

\newcommand\mycommfont[1]{\footnotesize\ttfamily\textcolor{blue}{#1}}
\SetCommentSty{mycommfont}
\newcommand{\incfig}[1]{%
    \def\svgwidth{\columnwidth}
    \import{./figures/}{#1.pdf_tex}
}
\usepackage[tagged, highstructure]{accessibility}
\usepackage{tikzsymbols}
\renewcommand\qedsymbol{$\Laughey$}


%\usepackage{import}
%\usepackage{xifthen}
%\usepackage{pdfpages}
%\usepackage{transparent}


%%%%%%%%%%%%%%%%%%%%%%%%%%%%%%
% SELF MADE COLORS
%%%%%%%%%%%%%%%%%%%%%%%%%%%%%%



\definecolor{myg}{RGB}{56, 140, 70}
\definecolor{myb}{RGB}{45, 111, 177}
\definecolor{myr}{RGB}{199, 68, 64}
\definecolor{mytheorembg}{HTML}{F2F2F9}
\definecolor{mytheoremfr}{HTML}{00007B}
\definecolor{mylenmabg}{HTML}{FFFAF8}
\definecolor{mylenmafr}{HTML}{983b0f}
\definecolor{mypropbg}{HTML}{f2fbfc}
\definecolor{mypropfr}{HTML}{191971}
\definecolor{myexamplebg}{HTML}{F2FBF8}
\definecolor{myexamplefr}{HTML}{88D6D1}
\definecolor{myexampleti}{HTML}{2A7F7F}
\definecolor{mydefinitbg}{HTML}{E5E5FF}
\definecolor{mydefinitfr}{HTML}{3F3FA3}
\definecolor{notesgreen}{RGB}{0,162,0}
\definecolor{myp}{RGB}{197, 92, 212}
\definecolor{mygr}{HTML}{2C3338}
\definecolor{myred}{RGB}{127,0,0}
\definecolor{myyellow}{RGB}{169,121,69}
\definecolor{myexercisebg}{HTML}{F2FBF8}
\definecolor{myexercisefg}{HTML}{88D6D1}


%%%%%%%%%%%%%%%%%%%%%%%%%%%%
% TCOLORBOX SETUPS
%%%%%%%%%%%%%%%%%%%%%%%%%%%%

\setlength{\parindent}{1cm}
%================================
% THEOREM BOX
%================================

\tcbuselibrary{theorems,skins,hooks}
\newtcbtheorem[number within=section]{Theorem}{Theorem}
{%
	enhanced,
	breakable,
	colback = mytheorembg,
	frame hidden,
	boxrule = 0sp,
	borderline west = {2pt}{0pt}{mytheoremfr},
	sharp corners,
	detach title,
	before upper = \tcbtitle\par\smallskip,
	coltitle = mytheoremfr,
	fonttitle = \bfseries\sffamily,
	description font = \mdseries,
	separator sign none,
	segmentation style={solid, mytheoremfr},
}
{th}

\tcbuselibrary{theorems,skins,hooks}
\newtcbtheorem[number within=chapter]{theorem}{Theorem}
{%
	enhanced,
	breakable,
	colback = mytheorembg,
	frame hidden,
	boxrule = 0sp,
	borderline west = {2pt}{0pt}{mytheoremfr},
	sharp corners,
	detach title,
	before upper = \tcbtitle\par\smallskip,
	coltitle = mytheoremfr,
	fonttitle = \bfseries\sffamily,
	description font = \mdseries,
	separator sign none,
	segmentation style={solid, mytheoremfr},
}
{th}


\tcbuselibrary{theorems,skins,hooks}
\newtcolorbox{Theoremcon}
{%
	enhanced
	,breakable
	,colback = mytheorembg
	,frame hidden
	,boxrule = 0sp
	,borderline west = {2pt}{0pt}{mytheoremfr}
	,sharp corners
	,description font = \mdseries
	,separator sign none
}

%================================
% Corollery
%================================
\tcbuselibrary{theorems,skins,hooks}
\newtcbtheorem[number within=section]{Corollary}{Corollary}
{%
	enhanced
	,breakable
	,colback = myp!10
	,frame hidden
	,boxrule = 0sp
	,borderline west = {2pt}{0pt}{myp!85!black}
	,sharp corners
	,detach title
	,before upper = \tcbtitle\par\smallskip
	,coltitle = myp!85!black
	,fonttitle = \bfseries\sffamily
	,description font = \mdseries
	,separator sign none
	,segmentation style={solid, myp!85!black}
}
{th}
\tcbuselibrary{theorems,skins,hooks}
\newtcbtheorem[number within=chapter]{corollary}{Corollary}
{%
	enhanced
	,breakable
	,colback = myp!10
	,frame hidden
	,boxrule = 0sp
	,borderline west = {2pt}{0pt}{myp!85!black}
	,sharp corners
	,detach title
	,before upper = \tcbtitle\par\smallskip
	,coltitle = myp!85!black
	,fonttitle = \bfseries\sffamily
	,description font = \mdseries
	,separator sign none
	,segmentation style={solid, myp!85!black}
}
{th}


%================================
% LENMA
%================================

\tcbuselibrary{theorems,skins,hooks}
\newtcbtheorem[number within=section]{Lenma}{Lenma}
{%
	enhanced,
	breakable,
	colback = mylenmabg,
	frame hidden,
	boxrule = 0sp,
	borderline west = {2pt}{0pt}{mylenmafr},
	sharp corners,
	detach title,
	before upper = \tcbtitle\par\smallskip,
	coltitle = mylenmafr,
	fonttitle = \bfseries\sffamily,
	description font = \mdseries,
	separator sign none,
	segmentation style={solid, mylenmafr},
}
{th}

\tcbuselibrary{theorems,skins,hooks}
\newtcbtheorem[number within=chapter]{lenma}{Lenma}
{%
	enhanced,
	breakable,
	colback = mylenmabg,
	frame hidden,
	boxrule = 0sp,
	borderline west = {2pt}{0pt}{mylenmafr},
	sharp corners,
	detach title,
	before upper = \tcbtitle\par\smallskip,
	coltitle = mylenmafr,
	fonttitle = \bfseries\sffamily,
	description font = \mdseries,
	separator sign none,
	segmentation style={solid, mylenmafr},
}
{th}


%================================
% PROPOSITION
%================================

\tcbuselibrary{theorems,skins,hooks}
\newtcbtheorem[number within=section]{Prop}{Proposition}
{%
	enhanced,
	breakable,
	colback = mypropbg,
	frame hidden,
	boxrule = 0sp,
	borderline west = {2pt}{0pt}{mypropfr},
	sharp corners,
	detach title,
	before upper = \tcbtitle\par\smallskip,
	coltitle = mypropfr,
	fonttitle = \bfseries\sffamily,
	description font = \mdseries,
	separator sign none,
	segmentation style={solid, mypropfr},
}
{th}

\tcbuselibrary{theorems,skins,hooks}
\newtcbtheorem[number within=chapter]{prop}{Proposition}
{%
	enhanced,
	breakable,
	colback = mypropbg,
	frame hidden,
	boxrule = 0sp,
	borderline west = {2pt}{0pt}{mypropfr},
	sharp corners,
	detach title,
	before upper = \tcbtitle\par\smallskip,
	coltitle = mypropfr,
	fonttitle = \bfseries\sffamily,
	description font = \mdseries,
	separator sign none,
	segmentation style={solid, mypropfr},
}
{th}


%================================
% CLAIM
%================================

\tcbuselibrary{theorems,skins,hooks}
\newtcbtheorem[number within=section]{claim}{Claim}
{%
	enhanced
	,breakable
	,colback = myg!10
	,frame hidden
	,boxrule = 0sp
	,borderline west = {2pt}{0pt}{myg}
	,sharp corners
	,detach title
	,before upper = \tcbtitle\par\smallskip
	,coltitle = myg!85!black
	,fonttitle = \bfseries\sffamily
	,description font = \mdseries
	,separator sign none
	,segmentation style={solid, myg!85!black}
}
{th}



%================================
% Exercise
%================================

\tcbuselibrary{theorems,skins,hooks}
\newtcbtheorem[number within=section]{Exercise}{Exercise}
{%
	enhanced,
	breakable,
	colback = myexercisebg,
	frame hidden,
	boxrule = 0sp,
	borderline west = {2pt}{0pt}{myexercisefg},
	sharp corners,
	detach title,
	before upper = \tcbtitle\par\smallskip,
	coltitle = myexercisefg,
	fonttitle = \bfseries\sffamily,
	description font = \mdseries,
	separator sign none,
	segmentation style={solid, myexercisefg},
}
{th}

\tcbuselibrary{theorems,skins,hooks}
\newtcbtheorem[number within=chapter]{exercise}{Exercise}
{%
	enhanced,
	breakable,
	colback = myexercisebg,
	frame hidden,
	boxrule = 0sp,
	borderline west = {2pt}{0pt}{myexercisefg},
	sharp corners,
	detach title,
	before upper = \tcbtitle\par\smallskip,
	coltitle = myexercisefg,
	fonttitle = \bfseries\sffamily,
	description font = \mdseries,
	separator sign none,
	segmentation style={solid, myexercisefg},
}
{th}

%================================
% EXAMPLE BOX
%================================

\newtcbtheorem[number within=section]{Example}{Example}
{%
	colback = myexamplebg
	,breakable
	,colframe = myexamplefr
	,coltitle = myexampleti
	,boxrule = 1pt
	,sharp corners
	,detach title
	,before upper=\tcbtitle\par\smallskip
	,fonttitle = \bfseries
	,description font = \mdseries
	,separator sign none
	,description delimiters parenthesis
}
{ex}

\newtcbtheorem[number within=chapter]{example}{Example}
{%
	colback = myexamplebg
	,breakable
	,colframe = myexamplefr
	,coltitle = myexampleti
	,boxrule = 1pt
	,sharp corners
	,detach title
	,before upper=\tcbtitle\par\smallskip
	,fonttitle = \bfseries
	,description font = \mdseries
	,separator sign none
	,description delimiters parenthesis
}
{ex}

%================================
% DEFINITION BOX
%================================

\newtcbtheorem[number within=section]{Definition}{Definition}{enhanced,
	before skip=2mm,after skip=2mm, colback=red!5,colframe=red!80!black,boxrule=0.5mm,
	attach boxed title to top left={xshift=1cm,yshift*=1mm-\tcboxedtitleheight}, varwidth boxed title*=-3cm,
	boxed title style={frame code={
					\path[fill=tcbcolback]
					([yshift=-1mm,xshift=-1mm]frame.north west)
					arc[start angle=0,end angle=180,radius=1mm]
					([yshift=-1mm,xshift=1mm]frame.north east)
					arc[start angle=180,end angle=0,radius=1mm];
					\path[left color=tcbcolback!60!black,right color=tcbcolback!60!black,
						middle color=tcbcolback!80!black]
					([xshift=-2mm]frame.north west) -- ([xshift=2mm]frame.north east)
					[rounded corners=1mm]-- ([xshift=1mm,yshift=-1mm]frame.north east)
					-- (frame.south east) -- (frame.south west)
					-- ([xshift=-1mm,yshift=-1mm]frame.north west)
					[sharp corners]-- cycle;
				},interior engine=empty,
		},
	fonttitle=\bfseries,
	title={#2},#1}{def}
\newtcbtheorem[number within=chapter]{definition}{Definition}{enhanced,
	before skip=2mm,after skip=2mm, colback=red!5,colframe=red!80!black,boxrule=0.5mm,
	attach boxed title to top left={xshift=1cm,yshift*=1mm-\tcboxedtitleheight}, varwidth boxed title*=-3cm,
	boxed title style={frame code={
					\path[fill=tcbcolback]
					([yshift=-1mm,xshift=-1mm]frame.north west)
					arc[start angle=0,end angle=180,radius=1mm]
					([yshift=-1mm,xshift=1mm]frame.north east)
					arc[start angle=180,end angle=0,radius=1mm];
					\path[left color=tcbcolback!60!black,right color=tcbcolback!60!black,
						middle color=tcbcolback!80!black]
					([xshift=-2mm]frame.north west) -- ([xshift=2mm]frame.north east)
					[rounded corners=1mm]-- ([xshift=1mm,yshift=-1mm]frame.north east)
					-- (frame.south east) -- (frame.south west)
					-- ([xshift=-1mm,yshift=-1mm]frame.north west)
					[sharp corners]-- cycle;
				},interior engine=empty,
		},
	fonttitle=\bfseries,
	title={#2},#1}{def}



%================================
% Solution BOX
%================================

\makeatletter
\newtcbtheorem{question}{Question}{enhanced,
	breakable,
	colback=white,
	colframe=myb!80!black,
	attach boxed title to top left={yshift*=-\tcboxedtitleheight},
	fonttitle=\bfseries,
	title={#2},
	boxed title size=title,
	boxed title style={%
			sharp corners,
			rounded corners=northwest,
			colback=tcbcolframe,
			boxrule=0pt,
		},
	underlay boxed title={%
			\path[fill=tcbcolframe] (title.south west)--(title.south east)
			to[out=0, in=180] ([xshift=5mm]title.east)--
			(title.center-|frame.east)
			[rounded corners=\kvtcb@arc] |-
			(frame.north) -| cycle;
		},
	#1
}{def}
\makeatother

%================================
% SOLUTION BOX
%================================

\makeatletter
\newtcolorbox{solution}{enhanced,
	breakable,
	colback=white,
	colframe=myg!80!black,
	attach boxed title to top left={yshift*=-\tcboxedtitleheight},
	title=Solution,
	boxed title size=title,
	boxed title style={%
			sharp corners,
			rounded corners=northwest,
			colback=tcbcolframe,
			boxrule=0pt,
		},
	underlay boxed title={%
			\path[fill=tcbcolframe] (title.south west)--(title.south east)
			to[out=0, in=180] ([xshift=5mm]title.east)--
			(title.center-|frame.east)
			[rounded corners=\kvtcb@arc] |-
			(frame.north) -| cycle;
		},
}
\makeatother

%================================
% Question BOX
%================================

\makeatletter
\newtcbtheorem{qstion}{Question}{enhanced,
	breakable,
	colback=white,
	colframe=mygr,
	attach boxed title to top left={yshift*=-\tcboxedtitleheight},
	fonttitle=\bfseries,
	title={#2},
	boxed title size=title,
	boxed title style={%
			sharp corners,
			rounded corners=northwest,
			colback=tcbcolframe,
			boxrule=0pt,
		},
	underlay boxed title={%
			\path[fill=tcbcolframe] (title.south west)--(title.south east)
			to[out=0, in=180] ([xshift=5mm]title.east)--
			(title.center-|frame.east)
			[rounded corners=\kvtcb@arc] |-
			(frame.north) -| cycle;
		},
	#1
}{def}
\makeatother

\newtcbtheorem[number within=chapter]{wconc}{Wrong Concept}{
	breakable,
	enhanced,
	colback=white,
	colframe=myr,
	arc=0pt,
	outer arc=0pt,
	fonttitle=\bfseries\sffamily\large,
	colbacktitle=myr,
	attach boxed title to top left={},
	boxed title style={
			enhanced,
			skin=enhancedfirst jigsaw,
			arc=3pt,
			bottom=0pt,
			interior style={fill=myr}
		},
	#1
}{def}



%================================
% NOTE BOX
%================================

\usetikzlibrary{arrows,calc,shadows.blur}
\tcbuselibrary{skins}
\newtcolorbox{note}[1][]{%
	enhanced jigsaw,
	colback=gray!20!white,%
	colframe=gray!80!black,
	size=small,
	boxrule=1pt,
	title=\textbf{Note:-},
	halign title=flush center,
	coltitle=black,
	breakable,
	drop shadow=black!50!white,
	attach boxed title to top left={xshift=1cm,yshift=-\tcboxedtitleheight/2,yshifttext=-\tcboxedtitleheight/2},
	minipage boxed title=1.5cm,
	boxed title style={%
			colback=white,
			size=fbox,
			boxrule=1pt,
			boxsep=2pt,
			underlay={%
					\coordinate (dotA) at ($(interior.west) + (-0.5pt,0)$);
					\coordinate (dotB) at ($(interior.east) + (0.5pt,0)$);
					\begin{scope}
						\clip (interior.north west) rectangle ([xshift=3ex]interior.east);
						\filldraw [white, blur shadow={shadow opacity=60, shadow yshift=-.75ex}, rounded corners=2pt] (interior.north west) rectangle (interior.south east);
					\end{scope}
					\begin{scope}[gray!80!black]
						\fill (dotA) circle (2pt);
						\fill (dotB) circle (2pt);
					\end{scope}
				},
		},
	#1,
}

%%%%%%%%%%%%%%%%%%%%%%%%%%%%%%
% SELF MADE COMMANDS
%%%%%%%%%%%%%%%%%%%%%%%%%%%%%%


\newcommand{\thm}[2]{\begin{Theorem}{#1}{}#2\end{Theorem}}
\newcommand{\cor}[2]{\begin{Corollary}{#1}{}#2\end{Corollary}}
\newcommand{\mlenma}[2]{\begin{Lenma}{#1}{}#2\end{Lenma}}
\newcommand{\mprop}[2]{\begin{Prop}{#1}{}#2\end{Prop}}
\newcommand{\clm}[3]{\begin{claim}{#1}{#2}#3\end{claim}}
\newcommand{\wc}[2]{\begin{wconc}{#1}{}\setlength{\parindent}{1cm}#2\end{wconc}}
\newcommand{\thmcon}[1]{\begin{Theoremcon}{#1}\end{Theoremcon}}
\newcommand{\ex}[2]{\begin{Example}{#1}{}#2\end{Example}}
\newcommand{\dfn}[2]{\begin{Definition}[colbacktitle=red!75!black]{#1}{}#2\end{Definition}}
\newcommand{\dfnc}[2]{\begin{definition}[colbacktitle=red!75!black]{#1}{}#2\end{definition}}
\newcommand{\qs}[2]{\begin{question}{#1}{}#2\end{question}}
\newcommand{\pf}[2]{\begin{myproof}[#1]#2\end{myproof}}
\newcommand{\nt}[1]{\begin{note}#1\end{note}}

\newcommand*\circled[1]{\tikz[baseline=(char.base)]{
		\node[shape=circle,draw,inner sep=1pt] (char) {#1};}}
\newcommand\getcurrentref[1]{%
	\ifnumequal{\value{#1}}{0}
	{??}
	{\the\value{#1}}%
}
\newcommand{\getCurrentSectionNumber}{\getcurrentref{section}}
\newenvironment{myproof}[1][\proofname]{%
	\proof[\bfseries #1: ]%
}{\endproof}

\newcommand{\mclm}[2]{\begin{myclaim}[#1]#2\end{myclaim}}
\newenvironment{myclaim}[1][\claimname]{\proof[\bfseries #1: ]}{}

\newcounter{mylabelcounter}

\makeatletter
\newcommand{\setword}[2]{%
	\phantomsection
	#1\def\@currentlabel{\unexpanded{#1}}\label{#2}%
}
\makeatother




\tikzset{
	symbol/.style={
			draw=none,
			every to/.append style={
					edge node={node [sloped, allow upside down, auto=false]{$#1$}}}
		}
}


% deliminators
\DeclarePairedDelimiter{\abs}{\lvert}{\rvert}
\DeclarePairedDelimiter{\norm}{\lVert}{\rVert}

\DeclarePairedDelimiter{\ceil}{\lceil}{\rceil}
\DeclarePairedDelimiter{\floor}{\lfloor}{\rfloor}
\DeclarePairedDelimiter{\round}{\lfloor}{\rceil}

\newsavebox\diffdbox
\newcommand{\slantedromand}{{\mathpalette\makesl{d}}}
\newcommand{\makesl}[2]{%
\begingroup
\sbox{\diffdbox}{$\mathsurround=0pt#1\mathrm{#2}$}%
\pdfsave
\pdfsetmatrix{1 0 0.2 1}%
\rlap{\usebox{\diffdbox}}%
\pdfrestore
\hskip\wd\diffdbox
\endgroup
}
\newcommand{\dd}[1][]{\ensuremath{\mathop{}\!\ifstrempty{#1}{%
\slantedromand\@ifnextchar^{\hspace{0.2ex}}{\hspace{0.1ex}}}%
{\slantedromand\hspace{0.2ex}^{#1}}}}
\ProvideDocumentCommand\dv{o m g}{%
  \ensuremath{%
    \IfValueTF{#3}{%
      \IfNoValueTF{#1}{%
        \frac{\dd #2}{\dd #3}%
      }{%
        \frac{\dd^{#1} #2}{\dd #3^{#1}}%
      }%
    }{%
      \IfNoValueTF{#1}{%
        \frac{\dd}{\dd #2}%
      }{%
        \frac{\dd^{#1}}{\dd #2^{#1}}%
      }%
    }%
  }%
}
\providecommand*{\pdv}[3][]{\frac{\partial^{#1}#2}{\partial#3^{#1}}}
%  - others
\DeclareMathOperator{\Lap}{\mathcal{L}}
\DeclareMathOperator{\Var}{Var} % varience
\DeclareMathOperator{\Cov}{Cov} % covarience
\DeclareMathOperator{\E}{E} % expected

% Since the amsthm package isn't loaded

% I prefer the slanted \leq
\let\oldleq\leq % save them in case they're every wanted
\let\oldgeq\geq
\renewcommand{\leq}{\leqslant}
\renewcommand{\geq}{\geqslant}

% % redefine matrix env to allow for alignment, use r as default
% \renewcommand*\env@matrix[1][r]{\hskip -\arraycolsep
%     \let\@ifnextchar\new@ifnextchar
%     \array{*\c@MaxMatrixCols #1}}


%\usepackage{framed}
%\usepackage{titletoc}
%\usepackage{etoolbox}
%\usepackage{lmodern}


%\patchcmd{\tableofcontents}{\contentsname}{\sffamily\contentsname}{}{}

%\renewenvironment{leftbar}
%{\def\FrameCommand{\hspace{6em}%
%		{\color{myyellow}\vrule width 2pt depth 6pt}\hspace{1em}}%
%	\MakeFramed{\parshape 1 0cm \dimexpr\textwidth-6em\relax\FrameRestore}\vskip2pt%
%}
%{\endMakeFramed}

%\titlecontents{chapter}
%[0em]{\vspace*{2\baselineskip}}
%{\parbox{4.5em}{%
%		\hfill\Huge\sffamily\bfseries\color{myred}\thecontentspage}%
%	\vspace*{-2.3\baselineskip}\leftbar\textsc{\small\chaptername~\thecontentslabel}\\\sffamily}
%{}{\endleftbar}
%\titlecontents{section}
%[8.4em]
%{\sffamily\contentslabel{3em}}{}{}
%{\hspace{0.5em}\nobreak\itshape\color{myred}\contentspage}
%\titlecontents{subsection}
%[8.4em]
%{\sffamily\contentslabel{3em}}{}{}  
%{\hspace{0.5em}\nobreak\itshape\color{myred}\contentspage}



%%%%%%%%%%%%%%%%%%%%%%%%%%%%%%%%%%%%%%%%%%%
% TABLE OF CONTENTS
%%%%%%%%%%%%%%%%%%%%%%%%%%%%%%%%%%%%%%%%%%%

\usepackage{tikz}
\definecolor{doc}{RGB}{0,60,110}
\usepackage{titletoc}
\contentsmargin{0cm}
\titlecontents{chapter}[3.7pc]
{\addvspace{30pt}%
	\begin{tikzpicture}[remember picture, overlay]%
		\draw[fill=doc!60,draw=doc!60] (-7,-.1) rectangle (-0.9,.5);%
		\pgftext[left,x=-3.5cm,y=0.2cm]{\color{white}\Large\sc\bfseries Chapter\ \thecontentslabel};%
	\end{tikzpicture}\color{doc!60}\large\sc\bfseries}%
{}
{}
{\;\titlerule\;\large\sc\bfseries Page \thecontentspage
	\begin{tikzpicture}[remember picture, overlay]
		\draw[fill=doc!60,draw=doc!60] (2pt,0) rectangle (4,0.1pt);
	\end{tikzpicture}}%
\titlecontents{section}[3.7pc]
{\addvspace{2pt}}
{\contentslabel[\thecontentslabel]{2pc}}
{}
{\hfill\small \thecontentspage}
[]
\titlecontents*{subsection}[3.7pc]
{\addvspace{-1pt}\small}
{}
{}
{\ --- \small\thecontentspage}
[ \textbullet\ ][]

\makeatletter
\renewcommand{\tableofcontents}{%
	\chapter*{%
	  \vspace*{-20\p@}%
	  \begin{tikzpicture}[remember picture, overlay]%
		  \pgftext[right,x=15cm,y=0.2cm]{\color{doc!60}\Huge\sc\bfseries \contentsname};%
		  \draw[fill=doc!60,draw=doc!60] (13,-.75) rectangle (20,1);%
		  \clip (13,-.75) rectangle (20,1);
		  \pgftext[right,x=15cm,y=0.2cm]{\color{white}\Huge\sc\bfseries \contentsname};%
	  \end{tikzpicture}}%
	\@starttoc{toc}}
\makeatother


%From M275 "Topology" at SJSU
\newcommand{\id}{\mathrm{id}}
\newcommand{\taking}[1]{\xrightarrow{#1}}
\newcommand{\inv}{^{-1}}

%From M170 "Introduction to Graph Theory" at SJSU
\DeclareMathOperator{\diam}{diam}
\DeclareMathOperator{\ord}{ord}
\newcommand{\defeq}{\overset{\mathrm{def}}{=}}

%From the USAMO .tex files
\newcommand{\ts}{\textsuperscript}
\newcommand{\dg}{^\circ}
\newcommand{\ii}{\item}

% % From Math 55 and Math 145 at Harvard
% \newenvironment{subproof}[1][Proof]{%
% \begin{proof}[#1] \renewcommand{\qedsymbol}{$\blacksquare$}}%
% {\end{proof}}

\newcommand{\liff}{\leftrightarrow}
\newcommand{\lthen}{\rightarrow}
\newcommand{\opname}{\operatorname}
\newcommand{\surjto}{\twoheadrightarrow}
\newcommand{\injto}{\hookrightarrow}
\newcommand{\On}{\mathrm{On}} % ordinals
\DeclareMathOperator{\img}{im} % Image
\DeclareMathOperator{\Img}{Im} % Image
\DeclareMathOperator{\coker}{coker} % Cokernel
\DeclareMathOperator{\Coker}{Coker} % Cokernel
\DeclareMathOperator{\Ker}{Ker} % Kernel
\DeclareMathOperator{\rank}{rank}
\DeclareMathOperator{\Spec}{Spec} % spectrum
\DeclareMathOperator{\Tr}{Tr} % trace
\DeclareMathOperator{\pr}{pr} % projection
\DeclareMathOperator{\ext}{ext} % extension
\DeclareMathOperator{\pred}{pred} % predecessor
\DeclareMathOperator{\dom}{dom} % domain
\DeclareMathOperator{\ran}{ran} % range
\DeclareMathOperator{\Hom}{Hom} % homomorphism
\DeclareMathOperator{\Mor}{Mor} % morphisms
\DeclareMathOperator{\End}{End} % endomorphism

\newcommand{\eps}{\epsilon}
\newcommand{\veps}{\varepsilon}
\newcommand{\ol}{\overline}
\newcommand{\ul}{\underline}
\newcommand{\wt}{\widetilde}
\newcommand{\wh}{\widehat}
\newcommand{\vocab}[1]{\textbf{\color{blue} #1}}
\providecommand{\half}{\frac{1}{2}}
\newcommand{\dang}{\measuredangle} %% Directed angle
\newcommand{\ray}[1]{\overrightarrow{#1}}
\newcommand{\seg}[1]{\overline{#1}}
\newcommand{\arc}[1]{\wideparen{#1}}
\DeclareMathOperator{\cis}{cis}
\DeclareMathOperator*{\lcm}{lcm}
\DeclareMathOperator*{\argmin}{arg min}
\DeclareMathOperator*{\argmax}{arg max}
\newcommand{\cycsum}{\sum_{\mathrm{cyc}}}
\newcommand{\symsum}{\sum_{\mathrm{sym}}}
\newcommand{\cycprod}{\prod_{\mathrm{cyc}}}
\newcommand{\symprod}{\prod_{\mathrm{sym}}}
\newcommand{\Qed}{\begin{flushright}\qed\end{flushright}}
\newcommand{\parinn}{\setlength{\parindent}{1cm}}
\newcommand{\parinf}{\setlength{\parindent}{0cm}}
% \newcommand{\norm}{\|\cdot\|}
\newcommand{\inorm}{\norm_{\infty}}
\newcommand{\opensets}{\{V_{\alpha}\}_{\alpha\in I}}
\newcommand{\oset}{V_{\alpha}}
\newcommand{\opset}[1]{V_{\alpha_{#1}}}
\newcommand{\lub}{\text{lub}}
\newcommand{\del}[2]{\frac{\partial #1}{\partial #2}}
\newcommand{\Del}[3]{\frac{\partial^{#1} #2}{\partial^{#1} #3}}
\newcommand{\deld}[2]{\dfrac{\partial #1}{\partial #2}}
\newcommand{\Deld}[3]{\dfrac{\partial^{#1} #2}{\partial^{#1} #3}}
\newcommand{\lm}{\lambda}
\newcommand{\uin}{\mathbin{\rotatebox[origin=c]{90}{$\in$}}}
\newcommand{\usubset}{\mathbin{\rotatebox[origin=c]{90}{$\subset$}}}
\newcommand{\lt}{\left}
\newcommand{\rt}{\right}
\newcommand{\bs}[1]{\boldsymbol{#1}}
\newcommand{\exs}{\exists}
\newcommand{\st}{\strut}
\newcommand{\dps}[1]{\displaystyle{#1}}

\newcommand{\sol}{\setlength{\parindent}{0cm}\textbf{\textit{Solution:}}\setlength{\parindent}{1cm} }
\newcommand{\solve}[1]{\setlength{\parindent}{0cm}\textbf{\textit{Solution: }}\setlength{\parindent}{1cm}#1 \Qed}

%--------------------------------------------------
% LIE ALGEBRAS
%--------------------------------------------------
\newcommand*{\kb}{\mathfrak{b}}  % Borel subalgebra
\newcommand*{\kg}{\mathfrak{g}}  % Lie algebra
\newcommand*{\kh}{\mathfrak{h}}  % Cartan subalgebra
\newcommand*{\kn}{\mathfrak{n}}  % Nilradical
\newcommand*{\ku}{\mathfrak{u}}  % Unipotent algebra
\newcommand*{\kz}{\mathfrak{z}}  % Center of algebra

%--------------------------------------------------
% HOMOLOGICAL ALGEBRA
%--------------------------------------------------
\DeclareMathOperator{\Ext}{Ext} % Ext functor
\DeclareMathOperator{\Tor}{Tor} % Tor functor

%--------------------------------------------------
% MATRIX & GROUP NOTATION
%--------------------------------------------------
\DeclareMathOperator{\GL}{GL} % General Linear Group
\DeclareMathOperator{\SL}{SL} % Special Linear Group
\newcommand*{\gl}{\operatorname{\mathfrak{gl}}} % General linear Lie algebra
\newcommand*{\sl}{\operatorname{\mathfrak{sl}}} % Special linear Lie algebra

%--------------------------------------------------
% NUMBER SETS
%--------------------------------------------------
\newcommand*{\RR}{\mathbb{R}}
\newcommand*{\NN}{\mathbb{N}}
\newcommand*{\ZZ}{\mathbb{Z}}
\newcommand*{\QQ}{\mathbb{Q}}
\newcommand*{\CC}{\mathbb{C}}
\newcommand*{\PP}{\mathbb{P}}
\newcommand*{\HH}{\mathbb{H}}
\newcommand*{\FF}{\mathbb{F}}
\newcommand*{\EE}{\mathbb{E}} % Expected Value

%--------------------------------------------------
% MATH SCRIPT, FRAKTUR, AND BOLD SYMBOLS
%--------------------------------------------------
\newcommand*{\mcA}{\mathcal{A}}
\newcommand*{\mcB}{\mathcal{B}}
\newcommand*{\mcC}{\mathcal{C}}
\newcommand*{\mcD}{\mathcal{D}}
\newcommand*{\mcE}{\mathcal{E}}
\newcommand*{\mcF}{\mathcal{F}}
\newcommand*{\mcG}{\mathcal{G}}
\newcommand*{\mcH}{\mathcal{H}}

\newcommand*{\mfA}{\mathfrak{A}}  \newcommand*{\mfB}{\mathfrak{B}}
\newcommand*{\mfC}{\mathfrak{C}}  \newcommand*{\mfD}{\mathfrak{D}}
\newcommand*{\mfE}{\mathfrak{E}}  \newcommand*{\mfF}{\mathfrak{F}}
\newcommand*{\mfG}{\mathfrak{G}}  \newcommand*{\mfH}{\mathfrak{H}}

\usepackage{bm} % Ensure bold math works correctly
\newcommand*{\bmA}{\bm{A}}
\newcommand*{\bmB}{\bm{B}}
\newcommand*{\bmC}{\bm{C}}
\newcommand*{\bmD}{\bm{D}}
\newcommand*{\bmE}{\bm{E}}
\newcommand*{\bmF}{\bm{F}}
\newcommand*{\bmG}{\bm{G}}
\newcommand*{\bmH}{\bm{H}}

%--------------------------------------------------
% FUNCTIONAL ANALYSIS & ALGEBRA
%--------------------------------------------------
\DeclareMathOperator{\Aut}{Aut} % Automorphism group
\DeclareMathOperator{\Inn}{Inn} % Inner automorphisms
\DeclareMathOperator{\Syl}{Syl} % Sylow subgroups
\DeclareMathOperator{\Gal}{Gal} % Galois group
\DeclareMathOperator{\sign}{sign} % Sign function


%\usepackage[tagged, highstructure]{accessibility}
\usepackage{tocloft}
\usepackage{arydshln}
\usetikzlibrary{arrows.meta, decorations.pathreplacing}




\begin{document}
\title{Linear Algebra I}
\author{Lecture Notes Provided by Dr.~Miriam Logan.}
\date{}
\maketitle
\tableofcontents
\newpage  
\section{Linear Trasnformations Determined by Matrices}
 Suppose $ A \in M _{ m \times  n}\left(  \mathbb{R}\right) $                                                 then the mapping $ T: \mathbb{R} ^{n} \to \mathbb{R} ^{m}$ defined by 
 \[
 T \left( \vec{ x}  \right) = A \vec{ x} \forall  \vec{ x} \in \mathbb{R} ^{n}
 .\]                    is a linear trnasformation.\\
 \textit{Why?} \\
 Let $ \vec{ x} , \vec{ y}  \in \mathbb{R}^{n} $, $ \lambda \in \mathbb{R}$
 \[
 T \left( \vec{ x} + \vec{ y}  \right) = A \left( \vec{ x} + \vec{ y}  \right) = A \vec{ x} + A \vec{ y} = T \left( \vec{ x}  \right) + T \left( \vec{ y}  \right)
 .\] 
 $ \implies T$ preserves addition.\\
 \[
 T \left( \lambda \vec{ x}  \right) = A \left( \lambda \vec{ x}  \right) = \lambda A \vec{ x} = \lambda T \left( \vec{ x}  \right)
 .\] 
 $ \implies T$ preserves scalar multiplication.\\
 \\
 In fact every linear transformation from $ \mathbb{R} ^{n}$ to $ \mathbb{R} ^{m}$ is defined by left multiplication by an $ m \times n$ matrix.\\
 We will show this shortly.\\
 \\
 \\
 \thm{}
 {
	 Suppose $ V$, $ W$ are vector spaces. Evey linear transformation $ T: V \to W$ is uniquely determined by the values $ T \left(  \vec{ v_1}  \right) , T \left( \vec{ v_2}  \right) , \ldots , T \left( \vec{ v_n}  \right) $ where $ \left\{ \vec{ v_1} ,\vec{ v_2} ,\ldots , \vec{ v_n}  \right\}$ is a basis for $ V$.
 }
  \textit{How do we prove this?} \\
  We take an arbitrary vector in the space and see can we determine its image under $ T$ just by knowing the images of the basis vectors.\\
  \\
  \pf{Proof:}{
  Suppose $ \vec{ v} \in V $. Since  $ \left\{ \vec{ v_1} ,\vec{ v_2} ,\ldots , \vec{ v_n}  \right\}$  is a basis for $ V$, there exist scalars $ c_1 , c_2 , \ldots , c_n \in \mathbb{R}$ such that $ \vec{ v} = c_1 \vec{ v_1} + c_2 \vec{ v_2} + \ldots +  c_n \vec{ v_n}  $ \\
  Hence,     
        \begin{align*}
        	T \left( \vec{ v}  \right) & = T \left( c_1 \vec{ v_1} + c_2 \vec{ v_2} + \ldots +  c_n \vec{ v_n}  \right)\\
		& = T \left( c_1 \vec{ v_1}  \right) + T \left( c_2 \vec{ v_2}  \right) + \ldots +  T \left( c_n \vec{ v_n}  \right)\\
		& = c_1 T \left( \vec{ v_1}  \right) + c_2 T \left( \vec{ v_2}  \right) + \ldots +  c_n T \left( \vec{ v_n}  \right)\\
        .\end{align*}
   Since $ T$ preserves addition and scalar multiplication.\\
   Hence, the image under $ T$ of any $ \vec{ v} \in V$ is uniquely determined by the images of the basis vectors.\\
  }
   \ex{}{
   \textit{Find the image of the basis vectors $ \vec{ e_1} = \begin{bmatrix}
   1\\
   0\\
   \end{bmatrix}
   $ and $ \vec{ e_2} =  \begin{bmatrix}
   0\\
   1\\
   \end{bmatrix}
$ under the linear transformation $ T: \mathbb{R} ^{2} \to \mathbb{R} ^{3}$ given by}
\[
T \begin{bmatrix}
x\\
y\\
\end{bmatrix}
= \begin{bmatrix}
1 & -3 \\
3 & 5 \\
-1 & 7 \\
\end{bmatrix} \begin{bmatrix}
x\\
y\\
\end{bmatrix}
.\] 
\textbf{Solution:}\\
\[
T \begin{bmatrix}
1\\
0\\
\end{bmatrix}
\begin{bmatrix}
1 & -3 \\
3 & 5 \\
-1 & 7 \\
\end{bmatrix} \begin{bmatrix}
1\\
0\\
\end{bmatrix}
= \begin{bmatrix}
1\\
3\\
-1\\
\end{bmatrix}
.\] 
\[
T \begin{bmatrix}
0\\
1\\
\end{bmatrix}
= \begin{bmatrix}
1 & -3 \\
3 & 5 \\
-1 & 7 \\
\end{bmatrix} \begin{bmatrix}
0\\
1\\
\end{bmatrix}
= \begin{bmatrix}
-3\\
5\\
7\\
\end{bmatrix}
.\] 
   }
     \nt{
	     In the previous example, the image of the standard basis vectors $ \left\{ \vec{ e_1} , \vec{ e_2}  \right\} $  form the columns of the matrix $ A$ .\\
	     \\
	     This is true in general, if $ A \in M _{ m \times  n} \left(  \mathbb{R} \right) $ and $ T: \mathbb{R} ^{n} \to \mathbb{R} ^{ m}$ is defined by 
	     \[
	     T \left( \vec{ x}  \right) = A \vec{ x} 
	     .\] 
	     then $ T \left( \vec{ e_j}  \right) = A \vec{ e_j} = $  $j^{\text{th}}$ column of $ A$                                        i.e.  the columns of $ A$ are the images of the basis vectors .
     }
      \thm{}
      {
	      Every linear Transformation $ T: \mathbb{R} ^{n} \to \mathbb{R} ^{m}$ is defined by left multiplication by an $ m \times n$ matrix $ A$. If $ \left\{ \vec{ b_1} , \vec{ b_2} , \ldots , \vec{ b_n}  \right\} $ is a basis for $ \mathbb{R} ^{n}$ then the vectors $ \left\{ T \left( \vec{ b_1}  \right) , T \left( \vec{ b_2}  \right) , \ldots , T \left( \vec{ b_n}  \right)  \right\}$ are columns of $ A$.
      }
      \pf{Proof:}{
      Suppose $ \vec{ v} \in \mathbb{R} ^{ n}$ . $ \exists c_1,c_2,\ldots c_n \in \mathbb{R}$ such that $ \vec{ v} = \sum\limits_{i=1}^{n} c_i \vec{ b_i} 
      $
      \begin{align*}
      	T \left( \vec{ v}  \right) & = \sum\limits_{i=1}^{n} c_i T \left( \vec{ b_i}  \right) \\
	= c_1 T \left( \vec{ b_1}  \right) + c_2 T \left( \vec{ b_2}  \right) + \ldots + c_n T \left( \vec{ b_n}  \right) 
      .\end{align*}
      where the last line is a linear combination of vectors in $ \mathbb{R} ^{ m}$ which can be expressed as the following matrix product.
               \[
=
\begin{bmatrix}
\big| & \big| &        & \big| \\
T(\hat b_{1}) & T(\hat b_{2}) & \cdots & T(\hat b_{n}) \\
\big| & \big| &        & \big|
\end{bmatrix}
\begin{bmatrix}
c_{1} \\ c_{2} \\ \vdots \\ c_{n}
\end{bmatrix}
\]





      i.e. $ T \left( \vec{ v}  \right) = A \vec{ v} $ where $ A$ is the  $m \times n$ matric whose $j^{\text{th}}$ column is $ T \left( \vec{ b_j}  \right)$ i.e. the columns of $ A$ are the images of the basis vectors under $ T$.\\
      }
      
      \nt{
       The matrix of the linear transformation depends on the choice of basis for $ \mathbb{R} ^{n}$. And so, the basis choice  should be stated along with the matrix.\\
       If $ \mathcal{B} = \left\{ \vec{ b_1} , \vec{ b_2} ,\ldots, \vec{ b_n}  \right\}$ then we use $ \left[ T \right] _{ \mathcal{B}}$ to denote the matrix of $ T$ with respect to the basis $ \mathcal{B}$.\\
      }
      For the moment we will choose $ \mathcal{E} = \left\{ \vec{ e_1} ,\vec{ e_2} ,\ldots , \vec{ e_n}  \right\}$ (standard basis) as the basis for $ \mathbb{R} ^{n}$ unless otherwise stated.\\
     \nt{
     A linear transformation $ T: \mathbb{R} ^{1} \to \mathbb{R} ^{1}$  is given by $  T \left(  \vec{ x}  \right) = \alpha \vec{ x}  \forall  \alpha \in \mathbb{R} $. We can visualise this by thinking of $ T$ mapping the $ x$ axis to the line through the origin with slope $ \alpha$. Note that we need $ \mathbb{R} ^2 $ to visualise this. For linear transformations from $ \mathbb{R} ^{n}$ to $ \mathbb{R} ^{m}$ with $ n >1 $ we need to think differently in order to visualise them. But, we can also view $ T : \mathbb{R} ^{1} \to \mathbb{R} ^{1}$. $ T \left( \vec{ x}  \right)  = \alpha \vec{ x} $    just using a one dimensional line.\\
     Suppose $ \alpha=2 $, $ T \left( \vec{ x}  \right) = 2 \vec{ x} $
       \begin{tikzpicture}[>=Stealth,x=1cm,y=1cm] % x-scale = 1 cm per unit
  %% --- 1.  horizontal axis ----------------------------------
  \draw[very thick,->] (-6.7,0) -- (6.7,0) node[right] {$s$};

  %% --- 2.  tick-marks & labels ------------------------------
  \foreach \x in {-6,...,6}
    {\draw (\x,0) -- (\x,0.2) node[below=4pt] {\x};}

  %% --- 3.  pink “jump” arrows -------------------------------
  % (large one first so the shorter ones sit on top nicely)
  \draw[magenta,thick,->] (-6,0.2) .. controls (-5,1) and (-3,1) .. (-2,0.2);

  % left-hand cluster
  \draw[magenta,thick,->] (-6,0.2) .. controls (-5.8,0.4) and (-5.2,0.4) .. (-5,0.2);
  \draw[magenta,thick,->] (-4,0.2) .. controls (-3.5,0.6) and (-3.2,0.6) .. (-3,0.2);
  \draw[magenta,thick,->] (-3,0.2) .. controls (-2.8,0.5) and (-2.2,0.5) .. (-2,0.2);
  \draw[magenta,thick,->] (-2,0.2) .. controls (-1.8,0.4) and (-1.2,0.4) .. (-1,0.2);

  % middle & right-hand cluster
  \draw[magenta,thick,->] (0,0.2) .. controls (0.2,0.5) and (0.8,0.5) .. (1,0.2);
  \draw[magenta,thick,->] (1,0.2) .. controls (1.2,0.5) and (1.8,0.5) .. (2,0.2);
  \draw[magenta,thick,->] (2,0.2) .. controls (2.5,0.9) and (3.5,0.9) .. (4,0.2);
  \draw[magenta,thick,->] (3,0.2) .. controls (3.25,0.6) and (3.75,0.6) .. (4,0.2);
  \draw[magenta,thick,->] (4,0.2) .. controls (4.5,1) and (5.5,1) .. (6,0.2);
\end{tikzpicture}
       \\
       XXX
          \\
     $ T$  streches the $ x$ axis by a factor of $ 2$.\\
     }
      
       \ex{}{
	       Let $ \mathcal{E} = \left\{ \vec{ e_1} ,\vec{ e_2}  \right\} \subseteq \mathbb{R} ^2$ and suppose that the matrix $ A_i$ is the matrix of the linear transformation $ T_i: \mathbb{R} ^{2} \to \mathbb{R} ^{2}$ with respect to the basis $ \mathcal{E}$. Describe the effect that the linear transformation $ T_i$ has $ \mathbb{R} ^2$ 
	
                \begin{enumerate}[label=(\roman*)]
                \item 
		\raggedcolumns
		\begin{multicols}{2}
		\begin{align*}
			A_1 & = \begin{bmatrix}
			1 & 0\\
			0 & -1\\
			\end{bmatrix}\\
			&  T_1 \begin{bmatrix}
			1\\
			0\\
			\end{bmatrix}
			= \begin{bmatrix}
			1\\
			0\\
			\end{bmatrix}
			\\
			T_1 \begin{bmatrix}
			0\\
			1\\
			\end{bmatrix}
			& = \begin{bmatrix}
			0\\
			-1\\
			\end{bmatrix}
		.\end{align*}
		
		\break
		     \begin{align*}
		     	T_1 : \mathbb{R} ^2 & \to \mathbb{R} ^2\\
			& T_1 \begin{bmatrix}
			x\\
			y\\
			\end{bmatrix}
			= A_1 \begin{bmatrix}
			x\\
			y\\
			\end{bmatrix}
			\\
			T_1 \begin{bmatrix}
			x\\
			y\\
			\end{bmatrix}
			= \begin{bmatrix}
			1 & 0\\
			0 & -1\\
			\end{bmatrix} \begin{bmatrix}
			x\\
			y\\
			\end{bmatrix}
			= \begin{bmatrix}
			x\\
			-y\\
			\end{bmatrix}
		     .\end{align*}
		\end{multicols}
		
		\begin{tikzpicture}[line cap=round,line join=round,>=Stealth,
                    every node/.style={font=\small}]
  %% ------------------------------------------------- axes
  \draw[very thick,->] (-4,0) -- (4.8,0);        % x–axis
  \draw[very thick,->] (0,-3) -- (0,3);          % y–axis

  %% ------------------------------------------------- green arrow on +x
  \draw[very thick,teal!70!black,-{Stealth[length=4pt]}] (0,0) -- (3.5,0);

  %% ------------------------------------------------- pink “rotation” arrow
  % straight up-down segment
  \draw[pink!80!magenta,very thick] (0,0) -- (0,2.3);
  % curved part with arrow-tip at top
  \draw[pink!80!magenta,very thick,-{Stealth[length=4pt]}]
        (0,2.3) .. controls (-1.3,1.0) and (-1.3,-1.0) .. (0,-2.3);

  %% ------------------------------------------------- LEFT dotted vertical (blue)
  \begin{scope}[blue]
    \draw[dotted,thick] (-2,2) -- (-2,-2);
    \fill (-2, 2) circle (2pt) (-2,-2) circle (2pt);

    \node[anchor=west] at (-1.8, 2) {$\displaystyle\begin{bmatrix}x\\[-2pt]-y\end{bmatrix}$};
    \node[anchor=west] at (-1.8,-2) {$\displaystyle\begin{bmatrix}x\\y\end{bmatrix}$};
  \end{scope}

  %% ------------------------------------------------- RIGHT dotted vertical (violet)
  \begin{scope}[violet]
    \draw[dotted,thick] (2.8, 2) -- (2.8,-2);
    \fill (2.8, 2) circle (2pt) (2.8,-2) circle (2pt);

    \node[anchor=west] at (3.0, 2) {$\displaystyle\begin{bmatrix}x\\y\end{bmatrix}$};
    \node[anchor=west] at (3.0,-2) {$\displaystyle\begin{bmatrix}x\\[-2pt]-y\end{bmatrix}$};
  \end{scope}
\end{tikzpicture}
\\
XXX Green arrow and pink arrow INCLUDE XXX

		$ T_1$ fixes the x-component and changes the sing of the y-component.\\
		$ T_1$ reflects points in the $ x$-axis.\\
                \item 
			\raggedcolumns
			\begin{multicols}{2}
			\begin{align*}
				A_2 & = \begin{bmatrix}
				1 & 0\\
				0 & 0\\
				\end{bmatrix}\\
				&  T_2 \begin{bmatrix}
				1\\
				0\\
				\end{bmatrix}
				= \begin{bmatrix}
				1\\
				0\\
				\end{bmatrix}
				\\
				& T_2 \begin{bmatrix}
				0\\
				1\\
				\end{bmatrix}
				= \begin{bmatrix}
				0\\
				0\\
				\end{bmatrix}
			.\end{align*}
			\break
			\begin{align*}
				    T_2 : \mathbb{R} ^2 \to \mathbb{R} ^2\\
				    T_2 = \begin{bmatrix}
				    1 & 0\\
				    0 & 0\\
				    \end{bmatrix} \begin{bmatrix}
				    x\\
				    y\\
				    \end{bmatrix}
				    = \begin{bmatrix}
				    x\\
				    0\\
				    \end{bmatrix}
			.\end{align*}
			\end{multicols}
			\begin{tikzpicture}[line cap=round,line join=round,>=Stealth,
                    every node/.style={inner sep=0pt, font=\small}]

  %--- colours ---------------------------------------------------------------
  \colorlet{axisblue}{blue!80!black}
  \colorlet{greenarrow}{teal!70!black}
  \colorlet{pinkpt}{magenta!70!red}
  \colorlet{violetpt}{violet}

  %--- 1. coordinate axes ----------------------------------------------------
  \draw[very thick,axisblue,->] (-4,0) -- (4.5,0);   % x–axis
  \draw[very thick,axisblue,->] (0,-4.5) -- (0,4.5); % y–axis

  %--- 2. highlighted green segment on +x ------------------------------------
  \draw[very thick,greenarrow,-{Stealth[length=5pt]}] (0,0) -- (3,0);

  %--- 3. magenta dotted column on the y–axis --------------------------------
  \begin{scope}[pinkpt]
    \draw[dotted,thick] (0,-4) -- (0,4);
    \foreach \y in {-4,-2,0,2,4}
      \fill (0,\y) circle (3pt);
  \end{scope}

  %--- 4. violet dotted column to the right ----------------------------------
  \begin{scope}[violetpt]
    \draw[dotted,thick] (3.5,-4) -- (3.5,4);
    \foreach \y in {-4,-2,0,2,4}
      \fill (3.5,\y) circle (3pt);
  \end{scope}

\end{tikzpicture}
\\
 XXX Make green and pink arrow more visible XXX
 \\
			      $ T_2$ fixes the x axis and projects the y-axis orthogonally onto the origin\\
			      $ T_2$ orthogonally projects $ \mathbb{R} ^2$ onto the $ x$-axis.\\
                \item 
			\[
			A_3 = \begin{bmatrix}
			k & 0\\
			0 & 1\\
			\end{bmatrix}  , \qquad k >0
			.\] 
			\[
			T_3 \begin{bmatrix}
			1\\
			0\\
			\end{bmatrix}
			= \begin{bmatrix}
			k\\
			0\\
			\end{bmatrix}
			, \qquad T_3 \begin{bmatrix}
			x\\
			0\\
			\end{bmatrix}       = x \left(  T_3 \begin{bmatrix}
			1\\
			0\\
			\end{bmatrix}
			 \right) 
			    = x \begin{bmatrix}
			    k\\
			    0\\
			    \end{bmatrix}
			    = k \begin{bmatrix}
			    x\\
			    0\\
			    \end{bmatrix}
			.\] 
			\[
			T_3 \begin{bmatrix}
			0\\
			1\\
			\end{bmatrix}
			= \begin{bmatrix}
			0\\
			1\\
			\end{bmatrix}
			\qquad \implies            T_3 \text{ scales vectors along the x-axis by a factor of } k 
			.\] 
			\[
			T_3 \begin{bmatrix}
			x\\
			y\\
			\end{bmatrix}
			= \begin{bmatrix}
			k & 0\\
			0 & 1\\
			\end{bmatrix} \begin{bmatrix}
			x\\
			y\\
			\end{bmatrix}
			= \begin{bmatrix}
			kx\\
			y\\
			\end{bmatrix}
			.\] 
            XXX Include SIDE BY SIDE XXX
		
			If $ k=1$ $ A_3$ is the identity matrix and $ T_3$ fixes the plane 
			\[
			\begin{bmatrix}
			1 & 0\\
			0 & 1\\
			\end{bmatrix} \begin{bmatrix}
			x\\
			y\\
			\end{bmatrix}
			= \begin{bmatrix}
			x\\
			y\\
			\end{bmatrix}
			
			.\] 
                \end{enumerate}
       }
    \ex{}{
    \textit{Find the matrix of the linear transformation with respect to the standard basis vectors, that reflects $ \mathbb{R} ^2$ in the line $ y=x$ .}\\
    \[
	    \mathcal{E} = \left\{ \begin{bmatrix}
	    1\\
	    0\\
	    \end{bmatrix}
	    , \begin{bmatrix}
	    0\\
	    1\\
	    \end{bmatrix}
	     \right\}  \qquad T: \mathbb{R} ^2 \to \mathbb{R} ^2
    .\] 
    The matrix of the linear transformation is 
    \[
A \;=\;
\bigl[\, T(\mathbf e_{1}) \;\; T(\mathbf e_{2}) \bigr]
\;=\;
\begin{bmatrix}
 0 & 1 \\
 1 & 0
\end{bmatrix}.
\]
    \begin{tikzpicture}[line cap=round,line join=round,>=Stealth,
                    every node/.style={font=\small},
                    x=1cm,y=1cm]    % 1 cm = 1 unit

% ------------------------------------------------------------------
% colours (tweak to taste)
\colorlet{axisblue}{blue!80!black}
\colorlet{greenv}{teal!80!black}
\colorlet{purplev}{violet}

% ------------------------------------------------------------------
% coordinate axes
\draw[very thick,axisblue,->] (-4,0) -- (4.5,0);   % x–axis
\draw[very thick,axisblue,->] (0,-3.5) -- (0,4);   % y–axis

% ------------------------------------------------------------------
% y = x  (dashed red) ---------------------------------------------
\draw[red,dashed,thick] (-4,-4) -- (4,4);
\node[red,anchor=south west] at (3.8,3.8) {$y=x$};

% ------------------------------------------------------------------
% green original vector  (0,1) -------------------------------------
\draw[greenv,very thick,-{Stealth[length=4pt]}] (0,0) -- (0,2);
\node[greenv,anchor=east] at (-0.1,2)
     {$\displaystyle\begin{bmatrix}0\\[-2pt]1\end{bmatrix}$};

% ------------------------------------------------------------------
% purple image vector  T(0,1) = (1,0)  -----------------------------
\fill[purplev] (2,0) circle (3pt);   % little dot at (2,0)
\node[anchor=south, purplev] at (2,0.15)
     {$\displaystyle
       T\!\begin{bmatrix}0\\[-2pt]1\end{bmatrix}
       =\begin{bmatrix}1\\[-2pt]0\end{bmatrix}$};

\end{tikzpicture}



    \textit{Check!} 
    \[
    T \begin{bmatrix}
    x\\
    y\\
    \end{bmatrix}
    = \begin{bmatrix}
    0 & 1\\
    1 & 0\\
    \end{bmatrix} \begin{bmatrix}
    x\\
    y\\
    \end{bmatrix}
    = \begin{bmatrix}
    y\\
    x\\
    \end{bmatrix}
    .\] 
      \\
      \\
    \begin{tikzpicture}[line cap=round,line join=round,>=Stealth,
                    every node/.style={font=\small},
                    x=1cm,y=1cm]          % «1 unit = 1 cm» makes tweaking easy
% ------------------------------------------------------------------
% adjustable “numbers’’ for x and y  (just pick y>x)
\def\xval{2}   % ← x–coordinate
\def\yval{4}   % ← y–coordinate

% helpful aliases --------------------------------------------------
\coordinate (O) at (0,0);
\coordinate (A) at (\xval,\yval);    %  [x, y]
\coordinate (B) at (\xval,\xval);    %  [x, x]
\coordinate (C) at (\yval,\xval);    %  [y, x]
\coordinate (D) at (\yval,\yval);    %  [y, y]

% ------------------------------------------------------------------
% axes -------------------------------------------------------------
\draw[very thick,blue!80!black,->] (-0.5*\yval-0.5,0) -- (1.3*\yval,0);
\draw[very thick,blue!80!black,->] (0,-0.5*\yval-0.5) -- (0,1.3*\yval);

% axis tick-labels “x’’ and “y’’ -----------------------------------
\foreach \p/\label in {\xval/x,\yval/y}{
  \draw (\p,0) -- (\p,-0.15);
  \node[below] at (\p,-0.15) {\(\label\)};
  \draw (0,\p) -- (-0.15,\p);
  \node[left]  at (-0.15,\p) {\(\label\)};
}

% ------------------------------------------------------------------
% dashed red line  y = x -------------------------------------------
\draw[red!70,dashed,thick] (-0.5*\yval-0.5,-0.5*\yval-0.5)
                           -- (1.3*\yval,1.3*\yval)
                           node[above,right,red!70] {$y=x$};

% ------------------------------------------------------------------
% braces that measure  y - x ---------------------------------------
\draw[decorate,decoration={brace,amplitude=4pt,mirror},teal!70!black]
      (0,\xval) -- (0,\yval)
      node[midway,left=4pt,teal!70!black] {$y-x$};

\draw[decorate,decoration={brace,amplitude=4pt},teal!70!black]
      (\xval,0) -- (\yval,0)
      node[midway,below=4pt,teal!70!black] {$y-x$};

% ------------------------------------------------------------------
% the green triangle  A-C-B ----------------------------------------
\draw[very thick,teal!70!black] (A) -- (C) -- (B) -- cycle;

% inner “right-angle’’ marker (optional aesthetic detail)
\path let \p1=(C),\p2=(B),\p3=(A) in
      coordinate (M) at ($(\p1)!(\p2)!(\p3)$);
\draw[teal!70!black] ($(M)+(0.15,0)$) -- (M) -- ($(M)+(0,0.15)$);

% ------------------------------------------------------------------
% magenta dashed altitude  A-B -------------------------------------
\draw[magenta!70,dashed,thick] (A) -- (B)
      node[midway,right=4pt,magenta!70] {$y-x$};

% ------------------------------------------------------------------
% the four labelled dots -------------------------------------------
\foreach \pt/\pos/\txt in {
   A/above right/{\(\bigl[\begin{smallmatrix}x\\ y\end{smallmatrix}\bigr]\)},
   B/below left/{\(\bigl[\begin{smallmatrix}x\\ x\end{smallmatrix}\bigr]\)},
   C/right/{\(\bigl[\begin{smallmatrix}y\\ x\end{smallmatrix}\bigr]\)},
   D/above/{\(\bigl[\begin{smallmatrix}y\\ y\end{smallmatrix}\bigr]\)}}
   {\fill (\pt) circle (2.5pt);
    \node[\pos] at (\pt) {\txt};}

\end{tikzpicture}
       \\
       XXX FIX DIGRAM XXX
       \\
    }
    
    \nt{
    A vector lying on the line $ y=x$ should be fixed by reflection in the line - which it is, since $ \begin{bmatrix}
    0 & 1\\
    1 & 0\\
    \end{bmatrix}\begin{bmatrix}
    x\\
    x\\
    \end{bmatrix}
    = \begin{bmatrix}
    x\\
    x\\
    \end{bmatrix}
    $
    }
  \ex{}{
	  Let $ \mathcal{E} = \left\{ \vec{ e_1} , \vec{ e_2}  \right\}= \left\{ \begin{bmatrix}
	  1\\
	  0\\
	  \end{bmatrix}
	  , \begin{bmatrix}
	  0\\
	  1\\
	  \end{bmatrix}
	   \right\}$ \\
	   Find the matrix of the transformation of $ \mathbb{R} ^2$ that rotates the plane by an angle $ \theta$ in the counter-clockwise direction.\\
        \begin{tikzpicture}[line cap=round,line join=round,>=Stealth,
                    every node/.style={font=\small},
                    x=1cm,y=1cm]        % 1 cm = 1 unit

% ----- you can change θ here once and everything updates  ------------------
\def\ang{35} % degrees, positive = anticlockwise

% ---------------------------------------------------------------------------
% axes (dark-blue)
\draw[very thick,blue!80!black,->] (-5,0) -- (5,0);
\draw[very thick,blue!80!black,->] (0,-3) -- (0,5);

% ---------------------------------------------------------------------------
% basis vectors before rotation
\draw[very thick,teal!70!black,-{Stealth[length=4pt]}] (0,0) -- (4,0)
      node[below right=1pt] {\(\bigl[\!\begin{smallmatrix}1\\0\end{smallmatrix}\!\bigr]\)};
\draw[very thick,cyan!70!black,-{Stealth[length=4pt]}] (0,0) -- (0,4)
      node[above left] {\(\bigl[\!\begin{smallmatrix}0\\1\end{smallmatrix}\!\bigr]\)};

% ---------------------------------------------------------------------------
% images under the rotation  T
\coordinate (Te1) at ({4*cos(\ang)},{4*sin(\ang)});
\coordinate (Te2) at ({-4*sin(\ang)},{ 4*cos(\ang)});

\draw[very thick,magenta!80,-{Stealth[length=4pt]}] (0,0) -- (Te1)
      node[above=1pt, magenta!80] %
      {$T\!\bigl[\!\begin{smallmatrix}1\\0\end{smallmatrix}\!\bigr]
        =\bigl[\!\begin{smallmatrix}\cos\theta\\\sin\theta\end{smallmatrix}\!\bigr]$};

\draw[very thick,violet,-{Stealth[length=4pt]}] (0,0) -- (Te2)
      node[left,violet] %
      {$T\!\bigl[\!\begin{smallmatrix}0\\1\end{smallmatrix}\!\bigr]
        =\bigl[\!\begin{smallmatrix}-\sin\theta\\\cos\theta\end{smallmatrix}\!\bigr]$};

% ---------------------------------------------------------------------------
% angle arcs  (two copies just for the visual)
\pic[draw=black!60,->,"$\theta$",angle radius=10pt,angle eccentricity=1.25]
     {angle={(4,0)}--(0,0)--(Te1)};

\pic[draw=black!60,->,"$\theta$",angle radius=12pt,angle eccentricity=1.25]
     {angle={(Te1)}--(0,0)--(0,4)};

\end{tikzpicture}
  \\
  XXX INCLUDE ANLGLE OF THETA XXX
  \\




	   \[
	   \implies \text{ Matrix of rotation is } \begin{bmatrix}
	   \cos \theta  & - \sin \theta  \\
	   \sin \theta  & \cos \theta\\
	   \end{bmatrix}
	   .\] 
  }
   \ex{}{
   Find a linear transformation $ T: \mathbb{R} ^2 \to \mathbb{R} ^2 $  such that  $ T \begin{bmatrix}
   1\\
   1\\
   \end{bmatrix}
   = \begin{bmatrix}
   4\\
   -2\\
   \end{bmatrix}
   $ and $ T \begin{bmatrix}
   -2\\
   -3\\
   \end{bmatrix}
   = \begin{bmatrix}
   1\\
   -1\\
   \end{bmatrix}
   $ i.e.  given $ \begin{bmatrix}
   x\\
   y\\
   \end{bmatrix}
    \in \mathbb{R} ^2$ we need to find a formula for $ T \begin{bmatrix}
    x\\
    y\\
    \end{bmatrix}
    $ .\\
    \\
    \textbf{Solution:}\\
          \[
		  \left\{ \begin{bmatrix}
		  1\\
		  1\\
		  \end{bmatrix}
		  , \begin{bmatrix}
		  -2\\
		  -3\\
		  \end{bmatrix}
		   \right\}  \text{ forms a basis for } \mathbb{R} ^2
          .\] 
	  $ \implies$ given any vector  $ \begin{bmatrix}
	  x\\
	  y\\
	  \end{bmatrix}
	  \in \mathbb{R} ^2$ there exists scalars $ c_1, c_2 \in \mathbb{R} $  such that
	  \[
	  \begin{bmatrix}
	  x\\
	  y\\
	  \end{bmatrix}
	  = c_1 \begin{bmatrix}
	  1\\
	  1\\
	  \end{bmatrix}
	  + c_2 \begin{bmatrix}
	  -2\\
	  -3\\
	  \end{bmatrix}
	  .\] 
	  and so, $ T \begin{bmatrix}
	  x\\
	  y\\
	  \end{bmatrix}
	  = c_1 T \begin{bmatrix}
	  1\\
	  1\\
	  \end{bmatrix}
	  + c_2 T \begin{bmatrix}
	  -2\\
	  -3\\
	  \end{bmatrix}
	  $ by the linearity of $ T$ .\\
	  \[
	  \text{ i.e. } T \begin{bmatrix}
	  x\\
	  y\\
	  \end{bmatrix}
	  = c_1 \begin{bmatrix}
	  4\\
	  -2\\
	  \end{bmatrix}
	  + c_2 \begin{bmatrix}
	  1\\
	  -1\\
	  \end{bmatrix}
	  .\] 
	  \textit{goal:} to express $ c_1$ and $ c_2$ in terms of $ x$ and $ y$ .\\ 
	  \begin{align*}
		  & \begin{bmatrix}
	  	1 & -2\\
	  	1 & -3\\
	  	\end{bmatrix} ^{-1} \begin{bmatrix}
	  	x\\
	  	y\\
	  	\end{bmatrix}
	  	= \begin{bmatrix}
	  	c_1\\
	  	c_2\\
	  	\end{bmatrix}       \\
		& \frac{1}{-1} \begin{bmatrix}
		-3 & 2\\
		-1 & 1\\
		\end{bmatrix} \begin{bmatrix}
		x\\
		y\\
		\end{bmatrix}
		= \begin{bmatrix}
		c_1\\
		c_2\\
		\end{bmatrix}          \\
		& \begin{bmatrix}
		3 & -2\\
		 1& -1 \\
		\end{bmatrix}   \begin{bmatrix}
		x\\
		y\\
		\end{bmatrix}
		= \begin{bmatrix}
		c_1\\
		c_2\\
		\end{bmatrix}      \\
		& \implies c_1 = 32 x - 21 y, \qquad c_2 = -x + y
	  .\end{align*}
	   \[
	   \text{ Hence, } T \begin{bmatrix}
	   x\\
	   y\\
	   \end{bmatrix}
	   = \left( 3x -2y \right) \begin{bmatrix}
	   4\\
	   -2\\
	   \end{bmatrix}
	   + \left( x-y \right) \begin{bmatrix}
	   1\\
	   -1\\
	   \end{bmatrix}
	   .\] 
	   \[
	   T \begin{bmatrix}
	   x\\
	   y\\
	   \end{bmatrix}
	   = \begin{bmatrix}
	   12x-8y+x-y\\
	   -6x +4y -x +y\\
	   \end{bmatrix}
	   = \begin{bmatrix}
	   13x -9y\\
	   -7x +5y\\
	   \end{bmatrix}
	   .\] 
	   \[
	   \text{ i.e. } T \begin{bmatrix}
	   x\\
	   y\\
	   \end{bmatrix}
	   = \begin{bmatrix}
	   13 &  -9\\
	   -7 & 5\\
	   \end{bmatrix} \begin{bmatrix}
	   x\\
	   y\\
	   \end{bmatrix}
	   .\] 
   }
   \ex{}{
	   \textit{Find the linear transformation $ T: \mathbb{R} ^2 \to \mathbb{R} ^2$ such that}
	   \[
	   T  \begin{bmatrix}
	   1\\
	   1\\
	   1\\
	   \end{bmatrix}
	   = \begin{bmatrix}
	   3\\
	   4\\
	   \end{bmatrix}
	   \qquad T \begin{bmatrix}
	   1\\
	   2\\
	   1\\
	   \end{bmatrix}
	   = \begin{bmatrix}
	   4\\
	   6\\
	   \end{bmatrix}
	    \qquad  T \begin{bmatrix}
	    1\\
	    2\\
	    2\\
	    \end{bmatrix}
	    = \begin{bmatrix}
	    3\\
	    7\\
	    \end{bmatrix}
	   .\] 
	   i.e.  describe $ T \begin{bmatrix}
	   x\\
	   y\\
	   z\\
	   \end{bmatrix}
	   $ where $ \begin{bmatrix}
	   x\\
	   y\\
	   z\\
	   \end{bmatrix}
	   \in \mathbb{R} ^3$ \\
	   \[
	   T \begin{bmatrix}
	   x\\
	   y\\
	   z\\
	   \end{bmatrix}
	   = \begin{bmatrix}
	   3x +y -z \\
	   x+2y +z\\
	   \end{bmatrix}
	   .\] 
   }
   \section{Kernal and Image }
   Suppose $ T: V \to W$ is a linear transformation between vector spaces $ V$ and $ W$ .\\
   \dfn{Kernal of a Transformation :}{
   The \underline{kernal} of $ T$ is defined to be the set of all vectors $ \vec{ v}  \in V$ such that $ T \left( \vec{ v}  \right) = \vec{ 0} _W$ . It is denoted by $ \ker T$ .\\
   \[
   \text{ i.e. } \ker T = \left\{ \vec{ v}  \in V \mid T \left( \vec{ v}  \right) = \vec{ 0} _W \right\}
   .\] 
   }
   \dfn{Image of a Transformation :}{
	   The \underline{image} of $ T$  is the set of all vectors $ \vec{ w}  \in W$ such that there exists a vector $ \vec{ v}  \in V$ with $ T \left( \vec{ v}  \right) = \vec{ w} $ . It is denoted by $ \text{Im } T$ .\\
	   \[
	   \text{i.e. } \text{Im } T = \left\{ T \left( \vec{ v}  \right) \mid \vec{ v}  \in V \right\} 
	   .\] 
   }
     \ex{}{
     Conside the linear transformation \\
     $ D : \mathcal{P}_k \left[ x \right] \to \mathcal{P} _k \left[ x \right]$ defined by
     $ D \left( p \left( x \right)  \right) = p' \left( x \right) $ \\
     Find (i) the kernal of $ D$ , (ii) the image of $ D$ .\\
     \begin{enumerate}[label=(\roman*)]
     \item 
	     \begin{align*}
	     	ker D & = \left\{ p \left( x \right)  \in \mathcal{P}_k \left[ x \right] \mid D \left( p \left( x \right)  \right) = 0 \right\}  \\
			     	& = \left\{ p \left( x \right)  \in \mathcal{P}_k \left[ x \right] \mid p' \left( x \right) = 0 \right\}  \\
							     	& = \left\{ a_0 + a_1 x + \ldots + a_k x^k \in \mathcal{P}_k \left[ x \right] \mid a_1 + 2 a_2 x + \ldots + k a_k x ^{k-1}=0 \right\}  \\
								& = \left\{  \sum\limits_{i=0}^{k} a_i x^i \mid a_1 = a_2 = \ldots = a_k = 0 \right\}  \\
								& = \left\{ a_0 \in \mathbb{R}  \right\}  \\
								& = \text{ set of all constant polynomials } \\
								Im D & = \left\{ p \left( x \right)  \in \mathcal{P}_k \left[ x \right] \mid p \left( x \right) = D \left( q \left( x \right)  \right) \text{ for some } q \left( x \right)  \in \mathcal{P}_k \left[ x \right]  \right\}  \\
	     .\end{align*}
	     Let $ q \left( x \right)  = \sum\limits_{i=0}^{k} a_i x^i$ \\
	     Then $ D \left( q \left( x \right)  \right)  = \sum\limits_{i=1}^{k} i a_i x^{i-1} = a_1 + 2 a_2 x + \ldots + k a_k x ^{k-1}$ .\\
	     \\
$ \implies Im D = \left\{ p\left( x \right) \in \mathcal{P}_k \left[ x \right] \mid \text{ degree of} p \left( x \right)  \text{ is at most } k-1\right\}$\\
$ Im D = \mathcal{P} _{k-1}\left[ x \right]$
   
    
    \nt{
   Suppose $ A \in M _{m \times n} $ and let $ T: \mathbb{R} ^n \to \mathbb{R} ^m$ be defined as $ T \left( \vec{ x}  \right) = A\vec{x}  \forall  \vec{ x}  \in \mathbb{R} ^{n}$.
\\
\textit{Recall: $ A$ is the $m \times n$ matrix whos $i^{\text{th}}$ column is} $ T \left(  \vec{ e_i}  \right) $ \\
\begin{align*}
	\text{ ker } T & = \left\{ \vec{ x}  \in \mathbb{R} ^n \mid T \left( \vec{ x}  \right) = \vec{ 0}\right\}\\
	& = \left\{ \vec{ x}  \in \mathbb{R} ^n \mid A\vec{x}  = \vec{ 0}\right\}\\
	& = \left\{ \mathcal{N} \left( A \right) = \text{ null space of } A \right\} \\
.\end{align*}
   }
                     
	     
     \item 
	     \begin{align*}
	     	Im T & = \left\{ \vec{ y}  \in \mathbb{R} ^{ m} \mid \vec{ y} = T \left( \vec{ x}  \right) \text{ for some } \vec{ x}  \in \mathbb{R} ^n \right\}  \\
			     	& = \left\{ \vec{ y}  \in \mathbb{R} ^{ m} \mid \vec{ y} = A\vec{x}  \text{ for some } \vec{ x}  \in \mathbb{R} ^n \right\}  \\
							     	& = \left\{ \vec{ y}  \in \mathbb{R} ^{ m} \mid  \vec{ y} \text{ lies in the span of the columns of } A \right\}  \\
								& = \mathcal{C} \left( A \right) \text{ (the column space of } A \text{) }\\
	     .\end{align*}
     \end{enumerate}
     
     
     }
     
   
       


\end{document}               
