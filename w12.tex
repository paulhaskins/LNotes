\documentclass{report}

%%%%%%%%%%%%%%%%%%%%%%%%%%%%%%%%%
% PACKAGE IMPORTS
%%%%%%%%%%%%%%%%%%%%%%%%%%%%%%%%%


\usepackage[tmargin=2cm,rmargin=1in,lmargin=1in,margin=0.85in,bmargin=2cm,footskip=.2in]{geometry}
\usepackage{amsmath,amsfonts,amsthm,amssymb,mathtools}
\usepackage[varbb]{newpxmath}
\usepackage{xfrac}
\usepackage[makeroom]{cancel}
\usepackage{bookmark}
\usepackage{enumitem}
\usepackage{hyperref,theoremref}
\hypersetup{
	pdftitle={Assignment},
	colorlinks=true, linkcolor=doc!90,
	bookmarksnumbered=true,
	bookmarksopen=true
}
\usepackage[most,many,breakable]{tcolorbox}
\usepackage{xcolor}
\usepackage{varwidth}
\usepackage{varwidth}
\usepackage{tocloft}
\usepackage{etoolbox}
\usepackage{derivative} %many derivativess partials
%\usepackage{authblk}
\usepackage{nameref}
\usepackage{multicol,array}
\usepackage{tikz-cd}
\usepackage[ruled,vlined,linesnumbered]{algorithm2e}
\usepackage{comment} % enables the use of multi-line comments (\ifx \fi) 
\usepackage{import}
\usepackage{xifthen}
\usepackage{pdfpages}
\usepackage{transparent}
\usepackage{verbatim}

\newcommand\mycommfont[1]{\footnotesize\ttfamily\textcolor{blue}{#1}}
\SetCommentSty{mycommfont}
\newcommand{\incfig}[1]{%
    \def\svgwidth{\columnwidth}
    \import{./figures/}{#1.pdf_tex}
}
\usepackage[tagged, highstructure]{accessibility}
\usepackage{tikzsymbols}
\renewcommand\qedsymbol{$\Laughey$}


%\usepackage{import}
%\usepackage{xifthen}
%\usepackage{pdfpages}
%\usepackage{transparent}


%%%%%%%%%%%%%%%%%%%%%%%%%%%%%%
% SELF MADE COLORS
%%%%%%%%%%%%%%%%%%%%%%%%%%%%%%



\definecolor{myg}{RGB}{56, 140, 70}
\definecolor{myb}{RGB}{45, 111, 177}
\definecolor{myr}{RGB}{199, 68, 64}
\definecolor{mytheorembg}{HTML}{F2F2F9}
\definecolor{mytheoremfr}{HTML}{00007B}
\definecolor{mylenmabg}{HTML}{FFFAF8}
\definecolor{mylenmafr}{HTML}{983b0f}
\definecolor{mypropbg}{HTML}{f2fbfc}
\definecolor{mypropfr}{HTML}{191971}
\definecolor{myexamplebg}{HTML}{F2FBF8}
\definecolor{myexamplefr}{HTML}{88D6D1}
\definecolor{myexampleti}{HTML}{2A7F7F}
\definecolor{mydefinitbg}{HTML}{E5E5FF}
\definecolor{mydefinitfr}{HTML}{3F3FA3}
\definecolor{notesgreen}{RGB}{0,162,0}
\definecolor{myp}{RGB}{197, 92, 212}
\definecolor{mygr}{HTML}{2C3338}
\definecolor{myred}{RGB}{127,0,0}
\definecolor{myyellow}{RGB}{169,121,69}
\definecolor{myexercisebg}{HTML}{F2FBF8}
\definecolor{myexercisefg}{HTML}{88D6D1}


%%%%%%%%%%%%%%%%%%%%%%%%%%%%
% TCOLORBOX SETUPS
%%%%%%%%%%%%%%%%%%%%%%%%%%%%

\setlength{\parindent}{1cm}
%================================
% THEOREM BOX
%================================

\tcbuselibrary{theorems,skins,hooks}
\newtcbtheorem[number within=section]{Theorem}{Theorem}
{%
	enhanced,
	breakable,
	colback = mytheorembg,
	frame hidden,
	boxrule = 0sp,
	borderline west = {2pt}{0pt}{mytheoremfr},
	sharp corners,
	detach title,
	before upper = \tcbtitle\par\smallskip,
	coltitle = mytheoremfr,
	fonttitle = \bfseries\sffamily,
	description font = \mdseries,
	separator sign none,
	segmentation style={solid, mytheoremfr},
}
{th}

\tcbuselibrary{theorems,skins,hooks}
\newtcbtheorem[number within=chapter]{theorem}{Theorem}
{%
	enhanced,
	breakable,
	colback = mytheorembg,
	frame hidden,
	boxrule = 0sp,
	borderline west = {2pt}{0pt}{mytheoremfr},
	sharp corners,
	detach title,
	before upper = \tcbtitle\par\smallskip,
	coltitle = mytheoremfr,
	fonttitle = \bfseries\sffamily,
	description font = \mdseries,
	separator sign none,
	segmentation style={solid, mytheoremfr},
}
{th}


\tcbuselibrary{theorems,skins,hooks}
\newtcolorbox{Theoremcon}
{%
	enhanced
	,breakable
	,colback = mytheorembg
	,frame hidden
	,boxrule = 0sp
	,borderline west = {2pt}{0pt}{mytheoremfr}
	,sharp corners
	,description font = \mdseries
	,separator sign none
}

%================================
% Corollery
%================================
\tcbuselibrary{theorems,skins,hooks}
\newtcbtheorem[number within=section]{Corollary}{Corollary}
{%
	enhanced
	,breakable
	,colback = myp!10
	,frame hidden
	,boxrule = 0sp
	,borderline west = {2pt}{0pt}{myp!85!black}
	,sharp corners
	,detach title
	,before upper = \tcbtitle\par\smallskip
	,coltitle = myp!85!black
	,fonttitle = \bfseries\sffamily
	,description font = \mdseries
	,separator sign none
	,segmentation style={solid, myp!85!black}
}
{th}
\tcbuselibrary{theorems,skins,hooks}
\newtcbtheorem[number within=chapter]{corollary}{Corollary}
{%
	enhanced
	,breakable
	,colback = myp!10
	,frame hidden
	,boxrule = 0sp
	,borderline west = {2pt}{0pt}{myp!85!black}
	,sharp corners
	,detach title
	,before upper = \tcbtitle\par\smallskip
	,coltitle = myp!85!black
	,fonttitle = \bfseries\sffamily
	,description font = \mdseries
	,separator sign none
	,segmentation style={solid, myp!85!black}
}
{th}


%================================
% LENMA
%================================

\tcbuselibrary{theorems,skins,hooks}
\newtcbtheorem[number within=section]{Lenma}{Lenma}
{%
	enhanced,
	breakable,
	colback = mylenmabg,
	frame hidden,
	boxrule = 0sp,
	borderline west = {2pt}{0pt}{mylenmafr},
	sharp corners,
	detach title,
	before upper = \tcbtitle\par\smallskip,
	coltitle = mylenmafr,
	fonttitle = \bfseries\sffamily,
	description font = \mdseries,
	separator sign none,
	segmentation style={solid, mylenmafr},
}
{th}

\tcbuselibrary{theorems,skins,hooks}
\newtcbtheorem[number within=chapter]{lenma}{Lenma}
{%
	enhanced,
	breakable,
	colback = mylenmabg,
	frame hidden,
	boxrule = 0sp,
	borderline west = {2pt}{0pt}{mylenmafr},
	sharp corners,
	detach title,
	before upper = \tcbtitle\par\smallskip,
	coltitle = mylenmafr,
	fonttitle = \bfseries\sffamily,
	description font = \mdseries,
	separator sign none,
	segmentation style={solid, mylenmafr},
}
{th}


%================================
% PROPOSITION
%================================

\tcbuselibrary{theorems,skins,hooks}
\newtcbtheorem[number within=section]{Prop}{Proposition}
{%
	enhanced,
	breakable,
	colback = mypropbg,
	frame hidden,
	boxrule = 0sp,
	borderline west = {2pt}{0pt}{mypropfr},
	sharp corners,
	detach title,
	before upper = \tcbtitle\par\smallskip,
	coltitle = mypropfr,
	fonttitle = \bfseries\sffamily,
	description font = \mdseries,
	separator sign none,
	segmentation style={solid, mypropfr},
}
{th}

\tcbuselibrary{theorems,skins,hooks}
\newtcbtheorem[number within=chapter]{prop}{Proposition}
{%
	enhanced,
	breakable,
	colback = mypropbg,
	frame hidden,
	boxrule = 0sp,
	borderline west = {2pt}{0pt}{mypropfr},
	sharp corners,
	detach title,
	before upper = \tcbtitle\par\smallskip,
	coltitle = mypropfr,
	fonttitle = \bfseries\sffamily,
	description font = \mdseries,
	separator sign none,
	segmentation style={solid, mypropfr},
}
{th}


%================================
% CLAIM
%================================

\tcbuselibrary{theorems,skins,hooks}
\newtcbtheorem[number within=section]{claim}{Claim}
{%
	enhanced
	,breakable
	,colback = myg!10
	,frame hidden
	,boxrule = 0sp
	,borderline west = {2pt}{0pt}{myg}
	,sharp corners
	,detach title
	,before upper = \tcbtitle\par\smallskip
	,coltitle = myg!85!black
	,fonttitle = \bfseries\sffamily
	,description font = \mdseries
	,separator sign none
	,segmentation style={solid, myg!85!black}
}
{th}



%================================
% Exercise
%================================

\tcbuselibrary{theorems,skins,hooks}
\newtcbtheorem[number within=section]{Exercise}{Exercise}
{%
	enhanced,
	breakable,
	colback = myexercisebg,
	frame hidden,
	boxrule = 0sp,
	borderline west = {2pt}{0pt}{myexercisefg},
	sharp corners,
	detach title,
	before upper = \tcbtitle\par\smallskip,
	coltitle = myexercisefg,
	fonttitle = \bfseries\sffamily,
	description font = \mdseries,
	separator sign none,
	segmentation style={solid, myexercisefg},
}
{th}

\tcbuselibrary{theorems,skins,hooks}
\newtcbtheorem[number within=chapter]{exercise}{Exercise}
{%
	enhanced,
	breakable,
	colback = myexercisebg,
	frame hidden,
	boxrule = 0sp,
	borderline west = {2pt}{0pt}{myexercisefg},
	sharp corners,
	detach title,
	before upper = \tcbtitle\par\smallskip,
	coltitle = myexercisefg,
	fonttitle = \bfseries\sffamily,
	description font = \mdseries,
	separator sign none,
	segmentation style={solid, myexercisefg},
}
{th}

%================================
% EXAMPLE BOX
%================================

\newtcbtheorem[number within=section]{Example}{Example}
{%
	colback = myexamplebg
	,breakable
	,colframe = myexamplefr
	,coltitle = myexampleti
	,boxrule = 1pt
	,sharp corners
	,detach title
	,before upper=\tcbtitle\par\smallskip
	,fonttitle = \bfseries
	,description font = \mdseries
	,separator sign none
	,description delimiters parenthesis
}
{ex}

\newtcbtheorem[number within=chapter]{example}{Example}
{%
	colback = myexamplebg
	,breakable
	,colframe = myexamplefr
	,coltitle = myexampleti
	,boxrule = 1pt
	,sharp corners
	,detach title
	,before upper=\tcbtitle\par\smallskip
	,fonttitle = \bfseries
	,description font = \mdseries
	,separator sign none
	,description delimiters parenthesis
}
{ex}

%================================
% DEFINITION BOX
%================================

\newtcbtheorem[number within=section]{Definition}{Definition}{enhanced,
	before skip=2mm,after skip=2mm, colback=red!5,colframe=red!80!black,boxrule=0.5mm,
	attach boxed title to top left={xshift=1cm,yshift*=1mm-\tcboxedtitleheight}, varwidth boxed title*=-3cm,
	boxed title style={frame code={
					\path[fill=tcbcolback]
					([yshift=-1mm,xshift=-1mm]frame.north west)
					arc[start angle=0,end angle=180,radius=1mm]
					([yshift=-1mm,xshift=1mm]frame.north east)
					arc[start angle=180,end angle=0,radius=1mm];
					\path[left color=tcbcolback!60!black,right color=tcbcolback!60!black,
						middle color=tcbcolback!80!black]
					([xshift=-2mm]frame.north west) -- ([xshift=2mm]frame.north east)
					[rounded corners=1mm]-- ([xshift=1mm,yshift=-1mm]frame.north east)
					-- (frame.south east) -- (frame.south west)
					-- ([xshift=-1mm,yshift=-1mm]frame.north west)
					[sharp corners]-- cycle;
				},interior engine=empty,
		},
	fonttitle=\bfseries,
	title={#2},#1}{def}
\newtcbtheorem[number within=chapter]{definition}{Definition}{enhanced,
	before skip=2mm,after skip=2mm, colback=red!5,colframe=red!80!black,boxrule=0.5mm,
	attach boxed title to top left={xshift=1cm,yshift*=1mm-\tcboxedtitleheight}, varwidth boxed title*=-3cm,
	boxed title style={frame code={
					\path[fill=tcbcolback]
					([yshift=-1mm,xshift=-1mm]frame.north west)
					arc[start angle=0,end angle=180,radius=1mm]
					([yshift=-1mm,xshift=1mm]frame.north east)
					arc[start angle=180,end angle=0,radius=1mm];
					\path[left color=tcbcolback!60!black,right color=tcbcolback!60!black,
						middle color=tcbcolback!80!black]
					([xshift=-2mm]frame.north west) -- ([xshift=2mm]frame.north east)
					[rounded corners=1mm]-- ([xshift=1mm,yshift=-1mm]frame.north east)
					-- (frame.south east) -- (frame.south west)
					-- ([xshift=-1mm,yshift=-1mm]frame.north west)
					[sharp corners]-- cycle;
				},interior engine=empty,
		},
	fonttitle=\bfseries,
	title={#2},#1}{def}



%================================
% Solution BOX
%================================

\makeatletter
\newtcbtheorem{question}{Question}{enhanced,
	breakable,
	colback=white,
	colframe=myb!80!black,
	attach boxed title to top left={yshift*=-\tcboxedtitleheight},
	fonttitle=\bfseries,
	title={#2},
	boxed title size=title,
	boxed title style={%
			sharp corners,
			rounded corners=northwest,
			colback=tcbcolframe,
			boxrule=0pt,
		},
	underlay boxed title={%
			\path[fill=tcbcolframe] (title.south west)--(title.south east)
			to[out=0, in=180] ([xshift=5mm]title.east)--
			(title.center-|frame.east)
			[rounded corners=\kvtcb@arc] |-
			(frame.north) -| cycle;
		},
	#1
}{def}
\makeatother

%================================
% SOLUTION BOX
%================================

\makeatletter
\newtcolorbox{solution}{enhanced,
	breakable,
	colback=white,
	colframe=myg!80!black,
	attach boxed title to top left={yshift*=-\tcboxedtitleheight},
	title=Solution,
	boxed title size=title,
	boxed title style={%
			sharp corners,
			rounded corners=northwest,
			colback=tcbcolframe,
			boxrule=0pt,
		},
	underlay boxed title={%
			\path[fill=tcbcolframe] (title.south west)--(title.south east)
			to[out=0, in=180] ([xshift=5mm]title.east)--
			(title.center-|frame.east)
			[rounded corners=\kvtcb@arc] |-
			(frame.north) -| cycle;
		},
}
\makeatother

%================================
% Question BOX
%================================

\makeatletter
\newtcbtheorem{qstion}{Question}{enhanced,
	breakable,
	colback=white,
	colframe=mygr,
	attach boxed title to top left={yshift*=-\tcboxedtitleheight},
	fonttitle=\bfseries,
	title={#2},
	boxed title size=title,
	boxed title style={%
			sharp corners,
			rounded corners=northwest,
			colback=tcbcolframe,
			boxrule=0pt,
		},
	underlay boxed title={%
			\path[fill=tcbcolframe] (title.south west)--(title.south east)
			to[out=0, in=180] ([xshift=5mm]title.east)--
			(title.center-|frame.east)
			[rounded corners=\kvtcb@arc] |-
			(frame.north) -| cycle;
		},
	#1
}{def}
\makeatother

\newtcbtheorem[number within=chapter]{wconc}{Wrong Concept}{
	breakable,
	enhanced,
	colback=white,
	colframe=myr,
	arc=0pt,
	outer arc=0pt,
	fonttitle=\bfseries\sffamily\large,
	colbacktitle=myr,
	attach boxed title to top left={},
	boxed title style={
			enhanced,
			skin=enhancedfirst jigsaw,
			arc=3pt,
			bottom=0pt,
			interior style={fill=myr}
		},
	#1
}{def}



%================================
% NOTE BOX
%================================

\usetikzlibrary{arrows,calc,shadows.blur}
\tcbuselibrary{skins}
\newtcolorbox{note}[1][]{%
	enhanced jigsaw,
	colback=gray!20!white,%
	colframe=gray!80!black,
	size=small,
	boxrule=1pt,
	title=\textbf{Note:-},
	halign title=flush center,
	coltitle=black,
	breakable,
	drop shadow=black!50!white,
	attach boxed title to top left={xshift=1cm,yshift=-\tcboxedtitleheight/2,yshifttext=-\tcboxedtitleheight/2},
	minipage boxed title=1.5cm,
	boxed title style={%
			colback=white,
			size=fbox,
			boxrule=1pt,
			boxsep=2pt,
			underlay={%
					\coordinate (dotA) at ($(interior.west) + (-0.5pt,0)$);
					\coordinate (dotB) at ($(interior.east) + (0.5pt,0)$);
					\begin{scope}
						\clip (interior.north west) rectangle ([xshift=3ex]interior.east);
						\filldraw [white, blur shadow={shadow opacity=60, shadow yshift=-.75ex}, rounded corners=2pt] (interior.north west) rectangle (interior.south east);
					\end{scope}
					\begin{scope}[gray!80!black]
						\fill (dotA) circle (2pt);
						\fill (dotB) circle (2pt);
					\end{scope}
				},
		},
	#1,
}

%%%%%%%%%%%%%%%%%%%%%%%%%%%%%%
% SELF MADE COMMANDS
%%%%%%%%%%%%%%%%%%%%%%%%%%%%%%


\newcommand{\thm}[2]{\begin{Theorem}{#1}{}#2\end{Theorem}}
\newcommand{\cor}[2]{\begin{Corollary}{#1}{}#2\end{Corollary}}
\newcommand{\mlenma}[2]{\begin{Lenma}{#1}{}#2\end{Lenma}}
\newcommand{\mprop}[2]{\begin{Prop}{#1}{}#2\end{Prop}}
\newcommand{\clm}[3]{\begin{claim}{#1}{#2}#3\end{claim}}
\newcommand{\wc}[2]{\begin{wconc}{#1}{}\setlength{\parindent}{1cm}#2\end{wconc}}
\newcommand{\thmcon}[1]{\begin{Theoremcon}{#1}\end{Theoremcon}}
\newcommand{\ex}[2]{\begin{Example}{#1}{}#2\end{Example}}
\newcommand{\dfn}[2]{\begin{Definition}[colbacktitle=red!75!black]{#1}{}#2\end{Definition}}
\newcommand{\dfnc}[2]{\begin{definition}[colbacktitle=red!75!black]{#1}{}#2\end{definition}}
\newcommand{\qs}[2]{\begin{question}{#1}{}#2\end{question}}
\newcommand{\pf}[2]{\begin{myproof}[#1]#2\end{myproof}}
\newcommand{\nt}[1]{\begin{note}#1\end{note}}

\newcommand*\circled[1]{\tikz[baseline=(char.base)]{
		\node[shape=circle,draw,inner sep=1pt] (char) {#1};}}
\newcommand\getcurrentref[1]{%
	\ifnumequal{\value{#1}}{0}
	{??}
	{\the\value{#1}}%
}
\newcommand{\getCurrentSectionNumber}{\getcurrentref{section}}
\newenvironment{myproof}[1][\proofname]{%
	\proof[\bfseries #1: ]%
}{\endproof}

\newcommand{\mclm}[2]{\begin{myclaim}[#1]#2\end{myclaim}}
\newenvironment{myclaim}[1][\claimname]{\proof[\bfseries #1: ]}{}

\newcounter{mylabelcounter}

\makeatletter
\newcommand{\setword}[2]{%
	\phantomsection
	#1\def\@currentlabel{\unexpanded{#1}}\label{#2}%
}
\makeatother




\tikzset{
	symbol/.style={
			draw=none,
			every to/.append style={
					edge node={node [sloped, allow upside down, auto=false]{$#1$}}}
		}
}


% deliminators
\DeclarePairedDelimiter{\abs}{\lvert}{\rvert}
\DeclarePairedDelimiter{\norm}{\lVert}{\rVert}

\DeclarePairedDelimiter{\ceil}{\lceil}{\rceil}
\DeclarePairedDelimiter{\floor}{\lfloor}{\rfloor}
\DeclarePairedDelimiter{\round}{\lfloor}{\rceil}

\newsavebox\diffdbox
\newcommand{\slantedromand}{{\mathpalette\makesl{d}}}
\newcommand{\makesl}[2]{%
\begingroup
\sbox{\diffdbox}{$\mathsurround=0pt#1\mathrm{#2}$}%
\pdfsave
\pdfsetmatrix{1 0 0.2 1}%
\rlap{\usebox{\diffdbox}}%
\pdfrestore
\hskip\wd\diffdbox
\endgroup
}
\newcommand{\dd}[1][]{\ensuremath{\mathop{}\!\ifstrempty{#1}{%
\slantedromand\@ifnextchar^{\hspace{0.2ex}}{\hspace{0.1ex}}}%
{\slantedromand\hspace{0.2ex}^{#1}}}}
\ProvideDocumentCommand\dv{o m g}{%
  \ensuremath{%
    \IfValueTF{#3}{%
      \IfNoValueTF{#1}{%
        \frac{\dd #2}{\dd #3}%
      }{%
        \frac{\dd^{#1} #2}{\dd #3^{#1}}%
      }%
    }{%
      \IfNoValueTF{#1}{%
        \frac{\dd}{\dd #2}%
      }{%
        \frac{\dd^{#1}}{\dd #2^{#1}}%
      }%
    }%
  }%
}
\providecommand*{\pdv}[3][]{\frac{\partial^{#1}#2}{\partial#3^{#1}}}
%  - others
\DeclareMathOperator{\Lap}{\mathcal{L}}
\DeclareMathOperator{\Var}{Var} % varience
\DeclareMathOperator{\Cov}{Cov} % covarience
\DeclareMathOperator{\E}{E} % expected

% Since the amsthm package isn't loaded

% I prefer the slanted \leq
\let\oldleq\leq % save them in case they're every wanted
\let\oldgeq\geq
\renewcommand{\leq}{\leqslant}
\renewcommand{\geq}{\geqslant}

% % redefine matrix env to allow for alignment, use r as default
% \renewcommand*\env@matrix[1][r]{\hskip -\arraycolsep
%     \let\@ifnextchar\new@ifnextchar
%     \array{*\c@MaxMatrixCols #1}}


%\usepackage{framed}
%\usepackage{titletoc}
%\usepackage{etoolbox}
%\usepackage{lmodern}


%\patchcmd{\tableofcontents}{\contentsname}{\sffamily\contentsname}{}{}

%\renewenvironment{leftbar}
%{\def\FrameCommand{\hspace{6em}%
%		{\color{myyellow}\vrule width 2pt depth 6pt}\hspace{1em}}%
%	\MakeFramed{\parshape 1 0cm \dimexpr\textwidth-6em\relax\FrameRestore}\vskip2pt%
%}
%{\endMakeFramed}

%\titlecontents{chapter}
%[0em]{\vspace*{2\baselineskip}}
%{\parbox{4.5em}{%
%		\hfill\Huge\sffamily\bfseries\color{myred}\thecontentspage}%
%	\vspace*{-2.3\baselineskip}\leftbar\textsc{\small\chaptername~\thecontentslabel}\\\sffamily}
%{}{\endleftbar}
%\titlecontents{section}
%[8.4em]
%{\sffamily\contentslabel{3em}}{}{}
%{\hspace{0.5em}\nobreak\itshape\color{myred}\contentspage}
%\titlecontents{subsection}
%[8.4em]
%{\sffamily\contentslabel{3em}}{}{}  
%{\hspace{0.5em}\nobreak\itshape\color{myred}\contentspage}



%%%%%%%%%%%%%%%%%%%%%%%%%%%%%%%%%%%%%%%%%%%
% TABLE OF CONTENTS
%%%%%%%%%%%%%%%%%%%%%%%%%%%%%%%%%%%%%%%%%%%

\usepackage{tikz}
\definecolor{doc}{RGB}{0,60,110}
\usepackage{titletoc}
\contentsmargin{0cm}
\titlecontents{chapter}[3.7pc]
{\addvspace{30pt}%
	\begin{tikzpicture}[remember picture, overlay]%
		\draw[fill=doc!60,draw=doc!60] (-7,-.1) rectangle (-0.9,.5);%
		\pgftext[left,x=-3.5cm,y=0.2cm]{\color{white}\Large\sc\bfseries Chapter\ \thecontentslabel};%
	\end{tikzpicture}\color{doc!60}\large\sc\bfseries}%
{}
{}
{\;\titlerule\;\large\sc\bfseries Page \thecontentspage
	\begin{tikzpicture}[remember picture, overlay]
		\draw[fill=doc!60,draw=doc!60] (2pt,0) rectangle (4,0.1pt);
	\end{tikzpicture}}%
\titlecontents{section}[3.7pc]
{\addvspace{2pt}}
{\contentslabel[\thecontentslabel]{2pc}}
{}
{\hfill\small \thecontentspage}
[]
\titlecontents*{subsection}[3.7pc]
{\addvspace{-1pt}\small}
{}
{}
{\ --- \small\thecontentspage}
[ \textbullet\ ][]

\makeatletter
\renewcommand{\tableofcontents}{%
	\chapter*{%
	  \vspace*{-20\p@}%
	  \begin{tikzpicture}[remember picture, overlay]%
		  \pgftext[right,x=15cm,y=0.2cm]{\color{doc!60}\Huge\sc\bfseries \contentsname};%
		  \draw[fill=doc!60,draw=doc!60] (13,-.75) rectangle (20,1);%
		  \clip (13,-.75) rectangle (20,1);
		  \pgftext[right,x=15cm,y=0.2cm]{\color{white}\Huge\sc\bfseries \contentsname};%
	  \end{tikzpicture}}%
	\@starttoc{toc}}
\makeatother


%From M275 "Topology" at SJSU
\newcommand{\id}{\mathrm{id}}
\newcommand{\taking}[1]{\xrightarrow{#1}}
\newcommand{\inv}{^{-1}}

%From M170 "Introduction to Graph Theory" at SJSU
\DeclareMathOperator{\diam}{diam}
\DeclareMathOperator{\ord}{ord}
\newcommand{\defeq}{\overset{\mathrm{def}}{=}}

%From the USAMO .tex files
\newcommand{\ts}{\textsuperscript}
\newcommand{\dg}{^\circ}
\newcommand{\ii}{\item}

% % From Math 55 and Math 145 at Harvard
% \newenvironment{subproof}[1][Proof]{%
% \begin{proof}[#1] \renewcommand{\qedsymbol}{$\blacksquare$}}%
% {\end{proof}}

\newcommand{\liff}{\leftrightarrow}
\newcommand{\lthen}{\rightarrow}
\newcommand{\opname}{\operatorname}
\newcommand{\surjto}{\twoheadrightarrow}
\newcommand{\injto}{\hookrightarrow}
\newcommand{\On}{\mathrm{On}} % ordinals
\DeclareMathOperator{\img}{im} % Image
\DeclareMathOperator{\Img}{Im} % Image
\DeclareMathOperator{\coker}{coker} % Cokernel
\DeclareMathOperator{\Coker}{Coker} % Cokernel
\DeclareMathOperator{\Ker}{Ker} % Kernel
\DeclareMathOperator{\rank}{rank}
\DeclareMathOperator{\Spec}{Spec} % spectrum
\DeclareMathOperator{\Tr}{Tr} % trace
\DeclareMathOperator{\pr}{pr} % projection
\DeclareMathOperator{\ext}{ext} % extension
\DeclareMathOperator{\pred}{pred} % predecessor
\DeclareMathOperator{\dom}{dom} % domain
\DeclareMathOperator{\ran}{ran} % range
\DeclareMathOperator{\Hom}{Hom} % homomorphism
\DeclareMathOperator{\Mor}{Mor} % morphisms
\DeclareMathOperator{\End}{End} % endomorphism

\newcommand{\eps}{\epsilon}
\newcommand{\veps}{\varepsilon}
\newcommand{\ol}{\overline}
\newcommand{\ul}{\underline}
\newcommand{\wt}{\widetilde}
\newcommand{\wh}{\widehat}
\newcommand{\vocab}[1]{\textbf{\color{blue} #1}}
\providecommand{\half}{\frac{1}{2}}
\newcommand{\dang}{\measuredangle} %% Directed angle
\newcommand{\ray}[1]{\overrightarrow{#1}}
\newcommand{\seg}[1]{\overline{#1}}
\newcommand{\arc}[1]{\wideparen{#1}}
\DeclareMathOperator{\cis}{cis}
\DeclareMathOperator*{\lcm}{lcm}
\DeclareMathOperator*{\argmin}{arg min}
\DeclareMathOperator*{\argmax}{arg max}
\newcommand{\cycsum}{\sum_{\mathrm{cyc}}}
\newcommand{\symsum}{\sum_{\mathrm{sym}}}
\newcommand{\cycprod}{\prod_{\mathrm{cyc}}}
\newcommand{\symprod}{\prod_{\mathrm{sym}}}
\newcommand{\Qed}{\begin{flushright}\qed\end{flushright}}
\newcommand{\parinn}{\setlength{\parindent}{1cm}}
\newcommand{\parinf}{\setlength{\parindent}{0cm}}
% \newcommand{\norm}{\|\cdot\|}
\newcommand{\inorm}{\norm_{\infty}}
\newcommand{\opensets}{\{V_{\alpha}\}_{\alpha\in I}}
\newcommand{\oset}{V_{\alpha}}
\newcommand{\opset}[1]{V_{\alpha_{#1}}}
\newcommand{\lub}{\text{lub}}
\newcommand{\del}[2]{\frac{\partial #1}{\partial #2}}
\newcommand{\Del}[3]{\frac{\partial^{#1} #2}{\partial^{#1} #3}}
\newcommand{\deld}[2]{\dfrac{\partial #1}{\partial #2}}
\newcommand{\Deld}[3]{\dfrac{\partial^{#1} #2}{\partial^{#1} #3}}
\newcommand{\lm}{\lambda}
\newcommand{\uin}{\mathbin{\rotatebox[origin=c]{90}{$\in$}}}
\newcommand{\usubset}{\mathbin{\rotatebox[origin=c]{90}{$\subset$}}}
\newcommand{\lt}{\left}
\newcommand{\rt}{\right}
\newcommand{\bs}[1]{\boldsymbol{#1}}
\newcommand{\exs}{\exists}
\newcommand{\st}{\strut}
\newcommand{\dps}[1]{\displaystyle{#1}}

\newcommand{\sol}{\setlength{\parindent}{0cm}\textbf{\textit{Solution:}}\setlength{\parindent}{1cm} }
\newcommand{\solve}[1]{\setlength{\parindent}{0cm}\textbf{\textit{Solution: }}\setlength{\parindent}{1cm}#1 \Qed}

%--------------------------------------------------
% LIE ALGEBRAS
%--------------------------------------------------
\newcommand*{\kb}{\mathfrak{b}}  % Borel subalgebra
\newcommand*{\kg}{\mathfrak{g}}  % Lie algebra
\newcommand*{\kh}{\mathfrak{h}}  % Cartan subalgebra
\newcommand*{\kn}{\mathfrak{n}}  % Nilradical
\newcommand*{\ku}{\mathfrak{u}}  % Unipotent algebra
\newcommand*{\kz}{\mathfrak{z}}  % Center of algebra

%--------------------------------------------------
% HOMOLOGICAL ALGEBRA
%--------------------------------------------------
\DeclareMathOperator{\Ext}{Ext} % Ext functor
\DeclareMathOperator{\Tor}{Tor} % Tor functor

%--------------------------------------------------
% MATRIX & GROUP NOTATION
%--------------------------------------------------
\DeclareMathOperator{\GL}{GL} % General Linear Group
\DeclareMathOperator{\SL}{SL} % Special Linear Group
\newcommand*{\gl}{\operatorname{\mathfrak{gl}}} % General linear Lie algebra
\newcommand*{\sl}{\operatorname{\mathfrak{sl}}} % Special linear Lie algebra

%--------------------------------------------------
% NUMBER SETS
%--------------------------------------------------
\newcommand*{\RR}{\mathbb{R}}
\newcommand*{\NN}{\mathbb{N}}
\newcommand*{\ZZ}{\mathbb{Z}}
\newcommand*{\QQ}{\mathbb{Q}}
\newcommand*{\CC}{\mathbb{C}}
\newcommand*{\PP}{\mathbb{P}}
\newcommand*{\HH}{\mathbb{H}}
\newcommand*{\FF}{\mathbb{F}}
\newcommand*{\EE}{\mathbb{E}} % Expected Value

%--------------------------------------------------
% MATH SCRIPT, FRAKTUR, AND BOLD SYMBOLS
%--------------------------------------------------
\newcommand*{\mcA}{\mathcal{A}}
\newcommand*{\mcB}{\mathcal{B}}
\newcommand*{\mcC}{\mathcal{C}}
\newcommand*{\mcD}{\mathcal{D}}
\newcommand*{\mcE}{\mathcal{E}}
\newcommand*{\mcF}{\mathcal{F}}
\newcommand*{\mcG}{\mathcal{G}}
\newcommand*{\mcH}{\mathcal{H}}

\newcommand*{\mfA}{\mathfrak{A}}  \newcommand*{\mfB}{\mathfrak{B}}
\newcommand*{\mfC}{\mathfrak{C}}  \newcommand*{\mfD}{\mathfrak{D}}
\newcommand*{\mfE}{\mathfrak{E}}  \newcommand*{\mfF}{\mathfrak{F}}
\newcommand*{\mfG}{\mathfrak{G}}  \newcommand*{\mfH}{\mathfrak{H}}

\usepackage{bm} % Ensure bold math works correctly
\newcommand*{\bmA}{\bm{A}}
\newcommand*{\bmB}{\bm{B}}
\newcommand*{\bmC}{\bm{C}}
\newcommand*{\bmD}{\bm{D}}
\newcommand*{\bmE}{\bm{E}}
\newcommand*{\bmF}{\bm{F}}
\newcommand*{\bmG}{\bm{G}}
\newcommand*{\bmH}{\bm{H}}

%--------------------------------------------------
% FUNCTIONAL ANALYSIS & ALGEBRA
%--------------------------------------------------
\DeclareMathOperator{\Aut}{Aut} % Automorphism group
\DeclareMathOperator{\Inn}{Inn} % Inner automorphisms
\DeclareMathOperator{\Syl}{Syl} % Sylow subgroups
\DeclareMathOperator{\Gal}{Gal} % Galois group
\DeclareMathOperator{\sign}{sign} % Sign function


%\usepackage[tagged, highstructure]{accessibility}
\usepackage{tocloft}
\usepackage{arydshln}
\usetikzlibrary{arrows.meta, decorations.pathreplacing}
\usepackage{tikz-cd}


\loadspellchecklist[en][wordlist.txt]
\setupspellchecking[state=start]
\definecolor{outer}{RGB}{0,150,95}   % greenish (vector-space level)
\definecolor{inner}{RGB}{90,50,160}  % violet / navy (coordinate level)
\tikzset{
  space/.style ={text=outerclr},                    % for V, W, R^n, R^m
  coord/.style ={text=innerclr},                    % for v, Tv, columns
  oarr /.style ={outerclr, -{Latex[length=3.5mm]}}, % outer arrows
  iarr /.style ={innerclr, -{Latex[length=3mm]}},   % inner arrows
  every node/.style={font=\small}
}


\begin{document}
\title{Linear Algebra I}
\author{Lecture Notes Provided by Dr.~Miriam Logan.}
\date{}
\maketitle
\tableofcontents
\newpage  
  
In general, suppose $ \mathcal{B} = \left\{ \vec{ b_1},\vec{ b_2}  \right\} $        forms a basis for $ \mathbb{R} ^2 $, for every vector $ \vec{ x} \in \mathbb{R} ^2  \exists  \lambda_1 , \lambda_2 \in \mathbb{R}$ such that
\[
\vec{ x} = \lambda_1 \vec{ b_1} + \lambda_2 \vec{ b_2}
.\] 
In this case, $ \left[ \vec{ x}  \right] _{  \mathcal{B}}= \begin{bmatrix}
\lambda_1\\
\lambda_2\\
\end{bmatrix}
_{  \mathcal{B}} $\\
To find $ \lambda_1$, $ \lambda_{2}$ we use the fact that the co-ordinate mapping $ \vec{ x} \to \left[ \vec{ x}  \right] _{ \mathcal{B}}$ determined by any base $ \mathcal{B}$ is a bijective linear transformation. \\
Hence,
\begin{align*}
 \left[ \vec{ x}  \right] _{  \mathcal{B}} = \left[ \lambda_1 \vec{ b_1} + \lambda_2 \vec{ b_2}  \right] _{  \mathcal{B}}\\
 &= \lambda_1 \left[ \vec{ b_1}  \right] _{  \mathcal{B}} + \lambda_2 \left[ \vec{ b_2}  \right] _{  \mathcal{B}}\\
 &= \left[ \left[ \vec{ b_1}  \right] _{ \mathcal{E}} \left[ \vec{ b_2}  \right] _{ \mathcal{E}} \right] \begin{bmatrix}
 \lambda_1\\
 \lambda_2\\
 \end{bmatrix}
 _{ \mathcal{B}}\\
\text{ i.e. } \left[ \vec{ x}  \right] _{  \mathcal{B}} = \left[  \left[ \vec{ b_1}  \right] _{  \mathcal{E}} \left[ \vec{ b_2}  \right] _{  \mathcal{E}} \right] \left[ \vec{ x}  \right] _{ \mathcal{B}}\\
\text{ and so, }  \left[ \left[ \vec{ b_1}  \right] _{ \mathcal{E}} \left[ \vec{ b_2}  \right] _{ \mathcal{E}}\right] ^{-1} \left[ \vec{ x}  \right] _{ \mathcal{E}} = \left[ \vec{ x}  \right] _{  \mathcal{B}}\\
.\end{align*}
  Note: the inverse exists because the co-ordinate mapping is a bijective linear transformation.\\
  \nt{
  We use $ P _{ \mathcal{B} \to \mathcal{E}}$ to denote the matrix $  \left[ \left[ \vec{ b_1}  \right] _{ \mathcal{E}} \left[ \vec{ b_2}  \right] _{ \mathcal{E}}\right] $  since it "takes in " co-ordinates in the basis $ \mathcal{B}$ and returns co-ordinates in the basis $ \mathcal{E}$.\\
  Note that $ \left( P _{ \mathcal{B} \to \mathcal{E}} \right) ^{-1}$ does the opposite, it takes in co-ordinates in the basis $ \mathcal{E}$ and returns co-ordinates in the basis $ \mathcal{B}$.\\
  Hence, conclude that
  \[
  \left( P _{ \mathcal{B} \to \mathcal{E}} ^{ -1}\right)             = P _{ \mathcal{E} \to \mathcal{B}}
  .\] 
  Note that the columns of $ P _{ \mathcal{B} \to \mathcal{E}}$ are the basis vectors of $ \mathcal{B}$ written with respect to the basis $ \mathcal{E}$.\\
  }
  \thm{}
  {
   Let $ \mathcal{B} = \left\{ \vec{ b_1} ,\vec{ b_2} ,\ldots , \vec{ b_n}  \right\}$ and $ \mathcal{C} = \left\{ \vec{ c_1} , \vec{ c_2} , \ldots , \vec{ c_n}  \right\} $ be bases of the vector space $ V$. Thene there exists a unqie $n \times n$  matrix $ P _{ \mathcal{B} \to \mathcal{C}}$ such that for every vector $ \vec{ x} \in V$,
   \[
    \left[ \vec{ x}  \right] _{ c\alpha} = P _{ \mathcal{B} \to \mathcal{C}} \left[ \vec{ x}  \right] _{ \mathcal{B}}
   .\] 
   where the columns of $ P_{ \mathcal{B} \to \mathcal{C}}$  are the basis vectors of $ \mathcal{B}$ written with respect to the basis $ \mathcal{C}$.\\
   \[
   \text{ i.e. } P _{ \mathcal{B} \to \mathcal{C}} = \left[ \left[ \vec{ b_1}  \right] _{ \mathcal{C}} \left[ \vec{ b_2}  \right] _{ \mathcal{C}} \ldots  \left[ \vec{ b_n}  \right] _{ \mathcal{C}}  \right]
   .\] 
  }
  \dfn{Change of Co-ordinate Matrix :}{
     The matrix $ P _{ \mathcal{B} \to \mathcal{C}}$ is called the change of co-ordinate matrix from the basis $ \mathcal{B}$ to the basis $ \mathcal{C}$.\\
  }
  \nt{
  The columns of $ P _{ \mathcal{B} \to \mathcal{C}}$ form a basis  for $ V$, hence, $ P _{ \mathcal{B} \to \mathcal{C}}$ is invertible and $ \left( P _{ \mathcal{B} \to c\alpha} \right) ^{-1} = P _{ \mathcal{C} \to \mathcal{B}}$ is the change of co-ordinate matrix from the basis $ \mathcal{C}$ to the basis $ \mathcal{B}$.\\
  }
  \ex{}{
   Let $ \mathcal{B} = \left\{ \begin{bmatrix}
   1\\
   0\\
   \end{bmatrix}
   , \begin{bmatrix}
   1\\
   2\\
   \end{bmatrix}
   \right\}$ and $ \mathcal{C} = \left\{ \begin{bmatrix}
   1\\
   4\\
   \end{bmatrix}
   , \begin{bmatrix}
   -2\\
   -10\\
   \end{bmatrix}
   \right\}$ 
   \begin{enumerate}[label=(\roman*)]
     \item Find  $ P _{ \mathcal{B} \to \mathcal{C}}$
     \item   Find $ P _{ \mathcal{C} \to \mathcal{B}}$
     \item   Suppose $ \left[ \vec{ v}  \right] _{ \mathcal{C}} = \begin{bmatrix}
     -7\\
     0\\
     \end{bmatrix}
     _{ \mathcal{C}}$, find $ \left[ \vec{ v}  \right] _{ \mathcal{B}}$.
     \end{enumerate}
     \textbf{Solution:}\\
     The columns of $ P _{ \mathcal{B} \to \mathcal{C}}$  are $ \left[ \vec{ b_1} \right] _{ \mathcal{C}} $ and $ \left[ \vec{ b_2} \right] _{ \mathcal{C}}$.
     \begin{align*}
      \begin{bmatrix}
      1\\
      0\\
      \end{bmatrix}
      = \lambda_1 \begin{bmatrix}
      1\\
      4\\
      \end{bmatrix}
      + \lambda_2 \begin{bmatrix}
      -2\\
      -10\\
     \end{bmatrix} \qquad  \text{where } \left[ \vec{ b_1}  \right]  _{ \mathcal{C}} = \begin{bmatrix}
     \lambda_1\\
     \lambda_2\\
     \end{bmatrix}
     _{ \mathcal{C}}\\
     & \begin{bmatrix}
     1\\
     0\\
     \end{bmatrix}
     = \begin{bmatrix}
     1 & -2\\
     4 & -10\\
     \end{bmatrix} \begin{bmatrix}
     \lambda_1\\
     \lambda_2\\
     \end{bmatrix}
     _{ \mathcal{C}} \\
     &  \begin{bmatrix}
     1 & -2\\
     4 & -10\\
     \end{bmatrix}^{-1} \begin{bmatrix}
     1\\
     0\\
     \end{bmatrix}
     = \begin{bmatrix}
     \lambda_1\\
     \lambda_2\\
     \end{bmatrix} _{ \mathcal{C}}\\
     &  - \frac{ 1  }{ 2 } \begin{bmatrix}
     -10 & 2\\
     -4 & 1\\
     \end{bmatrix} \begin{bmatrix}
     1\\
     0\\
     \end{bmatrix}         = \begin{bmatrix}
     \lambda_1\\
     \lambda_2\\
     \end{bmatrix}
     _{ \mathcal{C}}\\
     &  - \frac{ 1  }{ 2 }  \begin{bmatrix}
     -10\\
     -4\\
     \end{bmatrix}
     _{ \mathcal{C}} = \begin{bmatrix}
     \lambda_1\\
     \lambda_2\\
     \end{bmatrix}
     _{ \mathcal{C}}\\
     &  \begin{bmatrix}
     5\\
     2\\
     \end{bmatrix}
     _{ \mathcal{C}} = \begin{bmatrix}
     \lambda_1\\
     \lambda_2\\
     \end{bmatrix} _{ \mathcal{C}}  = \left[ \vec{ b_1}  \right] _{ \mathcal{C}}\\
     .\end{align*}
    Similarly, 
    \[
     \left[ \vec{ b_2}  \right] _{ \mathcal{C}} = - \frac{ 1  }{ 2 } \begin{bmatrix}
     -10 & 2\\
     -4 & 1\\
     \end{bmatrix} \begin{bmatrix}
     1\\
     2\\
     \end{bmatrix}
     = - \frac{ 1  }{ 2 }  \begin{bmatrix}
     -6\\
     -2\\
     \end{bmatrix}
      _{ \mathcal{C}}
    .\] 
    \[
     \left[ \vec{ b_2}  \right] _{ \mathcal{C}} = \begin{bmatrix}
     3\\
     1\\
     \end{bmatrix}        _{ \mathcal{C}}
    .\] 
    \[
    \text{ Hence, } P _{ \mathcal{B} \to \mathcal{C}} =  \begin{bmatrix}
    5 & 3\\
    2 & 1\\
    \end{bmatrix}
    .\] 
    \\
    \\
    \textit{(ii)}\\
    \begin{align*}
     \left[ \vec{ v}  \right] _{ \mathcal{C}} = \begin{bmatrix}
     -7\\
     9\\
     \end{bmatrix}
     _{ \mathcal{C}}\\
     \left[ \vec{ v}  \right] _{ \mathcal{B}} \left[ \vec{ v}  \right] _{ \mathcal{C}} = \begin{bmatrix}
     -1 & 3\\
     2 & -5\\
     \end{bmatrix}  \begin{bmatrix}
     -7\\
     9\\
     \end{bmatrix}
     _ { \mathcal{C}}\\
     \implies \left[ \vec{ v}  \right] _{ \mathcal{B}} = \begin{bmatrix}
     34\\
     -59\\
     \end{bmatrix}
     _{ \mathcal{B}}\\
    .\end{align*}
  }
  
  \ex{}{


   \begin{enumerate}[label=(\roman*)]
     \item                    Let $ V = \mathcal{P}_2 \left[ x \right] $. Find the change of co-ordinate matrix from the basis $ \mathcal{B} = \left\{ 1 -2x +x^2, 3-5x+4x^2 , 2x+3x^2 \right\}$ to the standard basis $ \mathcal{C} = \left\{ 1, x, x^2 \right\}$.\\
   i.e.  find $ P _{ \mathcal{B} \to \mathcal{C}}$ \\
           
     \item 
      Find the $ \mathcal{B}$ co-ordinates of the vector $ -1 +2x$ i.e.  find $ \left[ -1 +2x  \right] _{ \mathcal{B}}$.
     \end{enumerate}
     
    \\
    \textbf{Solution:}\\
    $ P _{ \mathcal{B} \to \mathcal{C}} = \left[
  \begin{array}{c|c|c}
    \bigl[\vec b_{1}\bigr]_{e} &
    \bigl[\vec b_{2}\bigr]_{e} &
    \bigl[\vec b_{3}\bigr]_{e}
  \end{array}
\right] $
     
    \[
    P _{ \mathcal{B} \to \mathcal{C}} = \begin{bmatrix}
    1 & 3 & 0\\
    -2 & -5 & 2\\
    1 & 4 & 3\\
    \end{bmatrix}
    .\] 
\\
\textit{(ii)}\\
\[
 \left[ -1 +2x \right] _{ \mathcal{B}} = \begin{bmatrix}
 \alpha_1\\
 \alpha_2\\
 \alpha_3\\
 \end{bmatrix}
  _{ \mathcal{B}} 
.\] 
where $ -1+2x = \alpha_1 \vec{ b_1}+ \alpha_2 \vec{ b_2} + \alpha_3 \vec{ b_3}  $ and we can solve for $ \alpha_1, \alpha_2 , \alpha_3$ or, alternatively, 
\[
 \left[ -1 +2x \right] _{ \mathcal{B}} = P _{ \mathcal{B} \to \mathcal{C}} ^{-1} \left[ -1 +2x  \right] _{ \mathcal{C}}
.\] 
\[
 \left[ -1 + 2x \right] _{ \mathcal{B}} = \left(  P _{ \mathcal{B} \to \mathcal{C}} \right) ^{-1}  \begin{bmatrix}
 -1\\
 2\\
 0\\
 \end{bmatrix}
  _{ \mathcal{C}} = \begin{bmatrix}
  -23 & -9 & 6\\
  8 & 3 & -2\\
  -3 & -1 & 1\\
  \end{bmatrix} \begin{bmatrix}
  -1\\
  2\\
  0\\
  \end{bmatrix}        _{ \mathcal{C}}
.\] 
\[
 \left[ -1 +2x \right] _{ \mathcal{B}} = \begin{bmatrix}
 5\\
 -2\\
 1\\
 \end{bmatrix}
 _{ \mathcal{B}}
.\] 
Check!\\
\[
5 \left( 1 -2x +x^2 \right) -2 \left( 3-5x+4x^2 \right) +1 \left( 2x+3x^2 \right) = -1 +2x
.\] 

  }
  
  \section{Linear Transformations and Co-ordinate Vectors}
 Suppose $ T: V \to W$ is a linear transformation between vector spaces $ V$ and $ W$.\\
  Let $ \mathcal{B} = \left\{ \vec{ v_1} , \vec{ v_2} , \ldots , \vec{ v_n}  \right\}$ be a basis for $ V$ and $ \mathcal{C} = \left\{ \vec{ w_1} , \vec{ w_2} , \ldots , \vec{ w_m}  \right\}$ be a basis for $ W$.\\
  We know there exists a bijective linear transformation from $ V \to \mathbb{R} ^{n}$ determined by the co-ordinate mapping $ \vec{ v} \to \left[ \vec{ v}  \right] _{ \mathcal{B}}$, equally the co-ordinate mapping  from $ W \to \mathbb{R} ^{m}$ given by $ \vec{ W} \to \left[ \vec{ W}  \right] _{ \mathcal{C}}$ is an isomorphism.\\
   Suppose $ \vec{ v} = \sum\limits_{i=1}^{n} b_i \vec{ v_i}$ and     $ T \left( \vec{ v}  \right) = \sum\limits_{i=1}^{m} c_i \vec{ w_i}$,\\
   \\
                            \\
 \[\begin{tikzcd}
	V && T && W \\
	& {\vec{v}} && {T\left(\vec{v} \right)} \\
	{\left[ \quad \right]_{\mathcal{B}}} &&&& {\left[ \quad \right]_{\mathcal{C}}} \\
	& \begin{array}{c} \begin{bmatrix} b_1\\ b_2 \\ \vdots \\ b_n\end{bmatrix}_{\mathcal{B}} \end{array} && \begin{array}{c} \begin{bmatrix} c_1\\ c_2 \\ \vdots \\ c_n\end{bmatrix}_{\mathcal{C}} \end{array} \\
	{\mathbb{R}^n} && {\left[T \right]_{\mathcal{B}, \mathcal{C}}} && {\mathbb{R}^m}
	\arrow[from=1-1, to=1-5]
	\arrow[from=1-1, to=5-1]
	\arrow[from=1-5, to=5-5]
	\arrow[from=2-2, to=2-4]
	\arrow[from=2-2, to=4-2]
	\arrow[from=2-4, to=4-4]
	\arrow[from=4-2, to=4-4]
	\arrow[from=5-1, to=5-5]
\end{tikzcd}\]


\\



     \\
     \\
 These mappings give rise to a linear transformation:
 \[
   \left[ T \right] _{ \mathcal{B}  \mathcal{C} } : \mathbb{R} ^{n} \to \mathbb{R} ^{m} \text{ defined by}
 .\] 
 \[
   \left[ T \right] _{ \mathcal{B}  \mathcal{C} } = \left[  \right] _{ \mathcal{C}} {\circ}  T \circ \left[  \right] _{ \mathcal{B}} ^{-1} 
 .\] 
 \[
 \text{ Note:} \left[ T \right] _{ \mathcal{B}  \mathcal{C} } \left(  \begin{bmatrix}
 b_1\\
 b_2\\
 \vdots\\
 b_n\\
 \end{bmatrix}
   \right)     = \begin{bmatrix}
   c_1\\
   c_2\\
   \vdots \\
   c_m\\
   \end{bmatrix}  _{ \mathcal{C}}
 .\] 
 \nt{
 The composition of linear trnansformations is a linear transformation (we will show this later).\\
}
Since $  \left[ T \right] _{ \mathcal{B} \mathcal{C}}$ is a composition of linear transformations it is a linear transformation.\\
    \thm{}
    {
      $ \left[ T \right] _{ \mathcal{B} \mathcal{C}}$ is a linear transformation from $ \mathbb{R} ^{n}$ to $ \mathbb{R} ^{m}$, hence there exists an $n \times n$  matrix which we will denote by $ \left[ T \right]_{ \mathcal{B} \mathcal{C}}$ such that 
      \[
        \left[ T \left( \vec{ v}  \right)  \right] _{ \mathcal{C}}= \left[ T \right] _{ \mathcal{B} \mathcal{C}} \left[ \vec{ v}  \right] _{ \mathcal{B}} \forall  \vec{ v} \in V
      .\] 
    }
      \pf{Proof:}{
        Let $ \vec{ v} \in V$  $ \left[ \vec{ v}  \right] _{ \mathcal{B}} = \begin{bmatrix}
        b_1\\
        b_2\\
        \vdots\\
        b_n\\
        \end{bmatrix}          
         $
         $ \vec{ v} = \sum\limits_{i=1}^{n} b_i \vec{ v_i} 
         $   $ T \left(  \vec{ v}  \right) = \sum\limits_{i=1}^{n} b_i T \left( \vec{ v_i}  \right) 
         $
         \[
          \left[ T \left( \vec{ v}  \right)  \right] _{ \mathcal{C}} = \sum\limits_{i=1}^{n} b_i \left[ T \left( \vec{ v_i}  \right)  \right] _{ \mathcal{C}} =  A  = \begin{bmatrix}
        b_1\\
        b_2\\
        \vdots\\
        b_n\\
      \end{bmatrix}         = A \left[ \vec{ v}  \right] _{ \mathcal{B}}                          
         .\] 
         whrer $ A$ is the $m \times n$  matrix whose $i^{\text{th}}$ column is $ \left[ T \left( \vec{ v_i}  \right)  \right] _{ \mathcal{C}}$.\\
         Hence $ \left[ T \right]_{ \mathcal{B} , \mathcal{C}} =  \left[
\begin{array}{ccc}
\big| &        & \big|\\
\left[T(\vec v_1)\right]_{e} & \cdots & \left[T(\vec v_n)\right]_{e}\\
\big| &        & \big|
\end{array}
\right]$
      }
      
    \dfn{ Matrix of a Transformation :}{
    The matrix $ \left[ T \right]_{ \mathcal{B} , \mathcal{C}}$ whose $i^{\text{th}}$  column is the image of the $i^{\text{th}}$  basis vector in $ \mathcal{B}$ written with respect to $ \mathcal{C}$, is called the matrix of linear transformation.
    }

    In conclusion: Every linear transformation from one finite dimensional vector space to another can be represented by a matrix.\\
    \\
    \nt{
      We can use the matrix $ \left[ T \right] _{ \mathcal{B} \mathcal{C}}$ to determine the rank and nullity of $ T$, since
      \[
      null T = dim \left( \ker T \right) = dim \left( \ker \left[ T \right] _{ \mathcal{B} \mathcal{C}} \right) 
      .\] 
      \[
        rk T = dim \left( \text{Im} T \right) = dim \left( \mathcal{C} \left( \left[ T \right] _{ \mathcal{B} \mathcal{C}} \right)  \right) 
      .\] 
    }
  \ex{}{
    Let $ D: \mathcal{P}_2 \left[ x \right] \to \mathcal{P}_1 \left[ x \right]$  be given as $ D \left( p \left( x \right)  \right) = p' \left( x \right) $  \\
    Let $ \mathcal{B} = \left\{ 1 ,x, x^2 \right\}$ and $ \mathcal{C} = \left\{ 1, x \right\}$ be a basis for $ \mathcal{P}_2 \left[ x \right]$ and $ \mathcal{P}_1 \left[ x \right]$ respectively.\\
    \textit{Find the matrix of the transformation  relative to these bases.}\\
    \\
    \textbf{Solution:}\\
    \[
    \left[ D \right] _{ \mathcal{B} , \mathcal{C}} = \left[ \left[ D \left( 1 \right)  \right] _{ \mathcal{C}} \left[ D \left( x \right)  \right] _{ \mathcal{C}} \left[ D \left( x^2 \right)  \right] _{ \mathcal{C}}  \right] 
    .\] 
    \[
    D \left( 1 \right) =0 \qquad  D \left( x \right) = 1 \qquad D \left( x^2 \right) = 2x
    .\] 
    \[
    \left[ D \left( 1  \right) \right] _{ \mathcal{C}} = \begin{bmatrix}
    0\\
    0\\
    \end{bmatrix}
    _{ \mathcal{C}} \qquad \left[ D \left( x  \right) \right] _{ \mathcal{C}} = \begin{bmatrix}
    1\\
    0\\
    \end{bmatrix}
    _{ \mathcal{C}} \qquad \left[ D \left( x^2  \right) \right] _{ \mathcal{C}} = \begin{bmatrix}
    0\\
    2\\
    \end{bmatrix}
    _{ \mathcal{C}}
    .\] 
    \[
    \implies \left[ D \right] _{ \mathcal{B} , \mathcal{C}}  = \begin{bmatrix}
    0 & 1 & 0\\
    0 & 0 & 2\\
    \end{bmatrix}
    .\] 
  }
   
    \[\begin{tikzcd}
     \mathcal{P}_2\left[ x \right]  && D && \mathcal{P}_1\left[ x \right]  \\
	& {a+bx+cx^2} && {b+2cx} \\
	{ \quad \qquad \left[ \quad \right]_{\mathcal{B}}} &&&& { \qquad \qquad \left[ \quad \right]_{\mathcal{C}}} \\
	& \begin{array}{c} \begin{bmatrix} a\\ b\\ c\end{bmatrix}_{\mathcal{B}} \end{array} && \begin{array}{c} \begin{bmatrix} b\\ 2c\end{bmatrix}_{\mathcal{C}} \end{array} \\
	{\mathbb{R}^3} && {\left[D \right]_{\mathcal{B}, \mathcal{C}}} && {\mathbb{R}^2}
	\arrow[from=1-1, to=1-5]
	\arrow[from=1-1, to=5-1]
	\arrow[from=1-5, to=5-5]
	\arrow[from=2-2, to=2-4]
	\arrow[from=2-2, to=4-2]
	\arrow[from=2-4, to=4-4]
	\arrow[from=4-2, to=4-4]
	\arrow[from=5-1, to=5-5]
\end{tikzcd}\] \\
\\





 \\
 \\

Indeed,
\[
[D]_{\mathcal{B},\mathcal{C}}
  \begin{bmatrix}a\\ b\\ c\end{bmatrix}_{\mathcal{B}}
  \;=\;
  \begin{bmatrix}
    0 & 1 & 0\\
    0 & 0 & 2
  \end{bmatrix}
  \begin{bmatrix}a\\ b\\ c\end{bmatrix}_{\mathcal{B}}
  \;=\;
  \begin{bmatrix}b\\ 2c\end{bmatrix}_{\mathcal{C}}.
\]

  \\
  \nt{
    The matrix of the transformation $ \left[ D \right] _{ \mathcal{B} \mathcal{C}}$ was relatively straightforward. It could have been more with a different choice of bases.\\
    This is true in general, the matrix of a linear transformation can be quite simple if the bases are suitably chosen as we'll see in the next example.\\
    }
  
    \ex{}{
         Suppose $ T: \mathbb{R} ^2 \to \mathbb{R} ^2$ is a linear transformation that reflects $ \mathbb{R} ^2$ in the line thorough the origin that passes through $ \begin{bmatrix}
         2\\
         1\\
         \end{bmatrix}
         $.\\
         \textit{ Find the matrix of the transformation relative to the standard basis $ \mathcal{E} = \left\{  \begin{bmatrix}
         1\\
         0\\
         \end{bmatrix}
         , \begin{bmatrix}
         0\\
         1\\
         \end{bmatrix}
          \right\} $  }\\
          i.e.  Find $ \left[ T \right] _{ \mathcal{E} \mathcal{E}}$\\
          \\
          \textbf{Solution:}\\
          We know that $ T \begin{bmatrix}
          2\\
          1\\
          \end{bmatrix}
          = \begin{bmatrix}
          2\\
          1\\
          \end{bmatrix}
          $ and $  T \left(  \lambda \begin{bmatrix}
          2\\
          1\\
          \end{bmatrix}
           \right) = \lambda \begin{bmatrix}
           2\\
           1\\
           \end{bmatrix}
            \forall  \lambda \in \mathbb{R}$ and $ T \left(  \vec{ v}  \right) \in - \vec{ v} \forall  \vec{ v} \perp $ to $ \begin{bmatrix}
            2\\
            1\\
            \end{bmatrix}
            $ ie $ T \begin{bmatrix}
            -1\\
            2\\
            \end{bmatrix}
            = \begin{bmatrix}
            1\\
            -2\\
            \end{bmatrix}
            $          \\
            \\
          \begin{tikzpicture}[>=stealth,scale=0.9,
        every node/.style={font=\small}]
% ------------------------------------------------------------------
% Colours
\colorlet{axes}{blue!70!black}
\colorlet{vec}{violet}
\colorlet{exam}{magenta!70}

% ------------------------------------------------------------------
% Coordinate axes --------------------------------------------------
\draw[very thick,->,axes] (-3.6,0) -- (3.6,0);   % x–axis
\draw[very thick,->,axes] (0,-2.8) -- (0,3.4);    % y–axis

% Tick marks & labels
\foreach \x in {-3,-2,-1,1,2,3}{
  \draw[axes] (\x,0.12)--(\x,-0.12) node[below] {\(\x\)};
}
\foreach \y in {-2,-1,1,2,3}{
  \draw[axes] (0.12,\y)--(-0.12,\y) node[left] {\(\y\)};
}

% ------------------------------------------------------------------
% Violet reference direction (dashed guideline + arrow) ------------
\draw[dashed,thick,vec] (-3,-1.5) -- (3,1.5);
\draw[thick,->,vec] (0,0) -- (2,1);

% ------------------------------------------------------------------
% Example vector and its image (magenta) ---------------------------
\path (-1,2) node[exam,fill=exam,circle,inner sep=2.5pt]{} coordinate(A)
      ( 0,0) coordinate(O)
      ( 1,-2) node[exam,fill=exam,circle,inner sep=2.5pt]{} coordinate(B);

\draw[exam,very thick] (A) -- (O);
\draw[exam,very thick,->] (O) -- (B);

% ------------------------------------------------------------------
% Labelling the example --------------------------------------------
\node[exam,left]  at (A) {$\bigl[\!\begin{smallmatrix}-1\\2\end{smallmatrix}\!\bigr]$};

\node[exam,right] at ($(B)+(0.25,0.15)$)
      {$T\!\Bigl(\!\begin{bmatrix}-1\\[2pt]2\end{bmatrix}\!\Bigr)
        =\begin{bmatrix}1\\[-2pt]-2\end{bmatrix}$};
\end{tikzpicture}

 \\

            But, in terms of describing $ T$ in general we need to know the image under $ T$ of the basis vectors of $ \mathbb{R} ^2$ \\
            \\
            Let $ \mathcal{B} = \left\{  \begin{bmatrix}
            2\\
            1\\
            \end{bmatrix}
            , \begin{bmatrix}
            -1\\
           2\\
            \end{bmatrix}  \right\} $  $ \mathcal{B} $ is a basis for $ \mathbb{R} ^2$.\\
            \[
            T \left(  \vec{ b_1}  \right) = \vec{ b_1}  \qquad  T \left(  \vec{ b_2}  \right) = - \vec{ b_2}
            .\] 
            Relative to $ \mathcal{B}$ the matrix of the transformation is
            \[
            \left[ T \right] _{ \mathcal{B} , \mathcal{B}} = \left[ \left[ T \left(  \vec{ b_1}  \right)  \right] _{ \mathcal{B}} \left[ T \left(  \vec{ b_2}  \right)  \right] _{ \mathcal{B}}  \right] = \begin{bmatrix}
            1 & 0\\
            0 & -1\\
            \end{bmatrix}
            .\] 
    
            \\
            \\
            \begin{tikzpicture}[>=Latex, every node/.style={font=\small}]
% ------------------------------------------------------------------
% 1. outer corner nodes ------------------------------------------------
\node (TL) at (0, 2) {$\mathbb{R}^{2}$};
\node (TR) at (7, 2) {$\mathbb{R}^{2}$};
\node (BL) at (0,-2) {$\mathbb{R}^{2}$};
\node (BR) at (7,-2) {$\mathbb{R}^{2}$};

% 2. outer arrows ------------------------------------------------------
\draw[->,thick] (TL) -- node[above] {$\bigl[T\bigr]_{\mathcal{B},\mathcal{B}}$} (TR);
\draw[->,thick] (BL) -- node[below] {$\bigl[T\bigr]_{\varepsilon,\varepsilon}$} (BR);
\draw[->,thick] (TL) -- node[left=6pt] {$P_{\mathcal{B}\to\varepsilon}$} (BL);
\draw[->,thick] (TR) -- node[right=6pt] {$P_{\mathcal{B}\to\varepsilon}$} (BR);

% 3. explicit matrix next to the left arrow ---------------------------
\node[anchor=east] at ($(TL)!0.5!(BL)+(-1.2,0)$)
      {$P_{\mathcal{B}\to\varepsilon}
        =\begin{bmatrix} 2 & -1 \\[2pt] 1 & 2 \end{bmatrix}$};

% ------------------------------------------------------------------
% 4. inner nodes (coordinate vectors) ---------------------------------
\node (xB)  at (2.5, 0.9) {$[\tilde{x}]_{\mathcal{B}}$};
\node (TxB) at (5.0, 0.9) {$[T(\tilde{x})]_{\mathcal{B}}$};
\node (xE)  at (2.5,-0.9) {$[\tilde{x}]_{\varepsilon}$};
\node (TxE) at (5.0,-0.9) {$[T(\tilde{x})]_{\varepsilon}$};

% 5. inner arrows ------------------------------------------------------
\draw[->,thick] (xB)  -- (TxB);      % horizontal inner arrow
\draw[->,thick] (xB)  -- (xE);       % left inner vertical arrow
\draw[->,thick] (TxB) -- (TxE);      % right inner vertical arrow
\end{tikzpicture}
\\
\\
            \[
            P _{ \mathcal{B} \to \mathcal{E}} = \left[ \left[ \vec{ b_1}  \right] _{ \mathcal{E}} \left[ \vec{ b_2}  \right] _{ \mathcal{E}}  \right] = \begin{bmatrix}
            2 & -1\\
            1 & 2\\
            \end{bmatrix}
            .\] 
            \[
\bigl[\,T\,\bigr]_{\varepsilon,\varepsilon}\,[\vec{x}]_{\varepsilon}
\;=\;
\underbrace{%
    P_{\mathcal B\to\varepsilon} \circ\;
  \underbrace{%
    \bigl[\,T\,\bigr]_{\mathcal B,\mathcal B} \circ\;
    \underbrace{%
      P_{\varepsilon\to\mathcal B}\, \left(   [\vec{x}]_{\varepsilon} \right)
    }_{\displaystyle[\vec{x}]_{\mathcal B}}
  }_{\displaystyle[T(\vec{x})]_{\mathcal B}}
}_{\displaystyle T(\vec{x})}
\]
\begin{align*}
  \implies \left[ T \right] _{ \mathcal{E} \mathcal{E}} = \begin{bmatrix}
  2 & -1\\
  1 & 2\\
  \end{bmatrix} \begin{bmatrix}
  1 & 0\\
  0 & -1\\
  \end{bmatrix} \begin{bmatrix}
  2 & 1\\
  -1 & 2\\
  \end{bmatrix} \left(  \frac{1}{5} \right)          \\
  & =  \frac{1}{5} \begin{bmatrix}
  2 & 1\\
  1 & -2\\
  \end{bmatrix} \begin{bmatrix}
  2 & 1\\
  -1 & 2\\
  \end{bmatrix}\\
  \left[ T \right]_{ \mathcal{E} \mathcal{E}} = \left( \frac{1}{5} \right) \begin{bmatrix}
  3 & 4\\
  4 & -3\\
  \end{bmatrix}
.\end{align*}
Check!  We know $ \left[ T \right] _{ \mathcal{E} \mathcal{E}} \begin{bmatrix}
2\\
1\\
\end{bmatrix}
= \begin{bmatrix}
2\\
1\\
\end{bmatrix}
$ and $ \left[ T \right] _{ \mathcal{E} \mathcal{E}} \begin{bmatrix}
-1\\
2\\
\end{bmatrix}
= \begin{bmatrix}
1\\
-2\\
\end{bmatrix}
$ 
\[
\left( \frac{1}{5} \right) \begin{bmatrix}
3 & 4\\
4 & -3\\
\end{bmatrix}\begin{bmatrix}
2\\
1\\
\end{bmatrix}
= \frac{1}{5} \begin{bmatrix}
10\\
5\\
\end{bmatrix}
= \begin{bmatrix}
2\\
1\\
\end{bmatrix}
.\] 
 \[
 \frac{1}{5} \begin{bmatrix}
 3 & 4\\
 4 & -3\\
 \end{bmatrix} \begin{bmatrix}
 1\\
 -2\\
 \end{bmatrix}
  = \frac{1}{5} \begin{bmatrix}
  -5\\
  10\\
  \end{bmatrix}
  = \begin{bmatrix}
  -1\\
  2\\
  \end{bmatrix}
 .\] 
  }
 \nt{
   For any linear transformation $ T: \mathbb{R} ^2 \to \mathbb{R} ^2 $ that reflects $ \mathbb{R} ^2$ in a line through the origin, there exists a basis $ \mathcal{B}$ of $ \mathbb{R} ^2$ such that $ \left[ T \right] _{ \mathcal{B} \mathcal{B}}$  is $ \begin{bmatrix}
   1 & 0\\
   0 & -1\\
 \end{bmatrix}$ where $  \mathcal{B} = \left\{  \vec{ v_1} , \vec{ v_2}  \right\} $  and $ T \left( \vec{ v_1}  \right) = \vec{ v_1} $, $ T \left( \vec{ v_2}  \right) = - \vec{ v_2} $ and $ T $ is a reflection of $ \mathbb{R} ^2$ in the line through $ \vec{ v_1} $, i.e.  all reflections of $ \mathbb{R} ^2$ in a line through the origin have the same very simple matrix representation if the basis for $ \mathbb{R}$ is the once chose above.\\
 }
  \nt{
  In general, one important goal of linear algebra is to find simple expressions for various expresions for various structures. weve already seen this:\\
  For example, 
  \begin{enumerate}[label=(\roman*)]
    \item $ n$- dimensional vector spaces described as the span of $ n$  basis vectors.\\
    \item  rank + nullity of a linear transformation put some structure on the domain space
    \item  Subspaces within vector spaces.
      \item Change of basis resulting in a reflection having a very simple matrix representation.
    \end{enumerate}
  }
   \ex{}{
   Let $ \phi : M_{2 \times  2} \to M_{2 \times  2}$   bed defined as 
   \[
   \phi \left( X \right) = \begin{bmatrix}
   1 & 1\\
   1 & 0\\
   \end{bmatrix} \left( X \right) \qquad  \forall  X \in M_{2 \times 2}
   .\] 
   \textit{
     \begin{enumerate}[label=(\roman*)]
       \item Show that $ \phi$ is a linear transformation.\\
       \item Find the matrix of the transformation with respect to the standard basis for $ M_{2 \times 2}$, 
         \[
           \mathcal{E} \;=\;
\left\{
\underbrace{\begin{pmatrix} 1 & 0 \\ 0 & 0 \end{pmatrix}}_{\vec{e}_{1}},
\;
\underbrace{\begin{pmatrix} 0 & 1 \\ 0 & 0 \end{pmatrix}}_{\vec{e}_{2}},
\;
\underbrace{\begin{pmatrix} 0 & 0 \\ 1 & 0 \end{pmatrix}}_{\vec{e}_{3}},
\;
\underbrace{\begin{pmatrix} 0 & 0 \\ 0 & 1 \end{pmatrix}}_{\vec{e}_{4}}
\right\}
         .\] 
       \item 
   \end{enumerate} }
       \\
       \\
        \textbf{Solution:}\\
        \textit{(i)}\\
        Let $ X , Y \in M_{2 \times 2}$ , $ \lambda \in \mathbb{R}$
        \begin{align*}
          \phi  \left(  X + Y \right) = \begin{bmatrix}
          1 & 1\\
          1 & 0\\
          \end{bmatrix} \left(  X + Y \right)\\
          &= \begin{bmatrix}
          1 & 1\\
          1 & 0\\
          \end{bmatrix} \left(  X \right) + \begin{bmatrix}
          1 & 1\\
          1 & 0\\
          \end{bmatrix} \left(  Y \right)\\
          &= \phi \left(  X \right) + \phi \left(  Y \right)\\
          \phi \left(  \lambda X \right) &= \begin{bmatrix}
          1 & 1\\
          1 & 0\\
          \end{bmatrix} \left( \lambda X \right) \\
          &= \lambda \begin{bmatrix}
          1 & 1\\
          1 & 0\\
          \end{bmatrix} \left(  X \right)\\
          &= \lambda \phi \left(  X \right)
        .\end{align*}
        Hence $ \phi$ is a linear transformation.\\
        \\
        \textit{(ii)}\\
        We want to find $ \left[ \phi  \right]_{ \mathcal{E} \mathcal{E}}$
        \[
          \left[ \phi  \right]_{ \mathcal{E} \mathcal{E}} = \left[ \left[ \phi \left( \vec{ e_1}  \right)  \right]_{ \mathcal{E}} \left[ \phi \left( \vec{ e_2}  \right)  \right] _{ \mathcal{E}} \left[ \phi \left( \vec{ e_3}  \right)  \right] _{ \mathcal{E}} \left[ \phi \left( \vec{ e_4}  \right)  \right] _{ \mathcal{E}}  \right]
        .\] 
        \[
        \phi \left( \vec{ e_1}  \right) = \begin{bmatrix}
        1 & 1\\
        1 & 0\\
        \end{bmatrix} \begin{bmatrix}
        1 & 0\\
        0 & 0\\
        \end{bmatrix} = \begin{bmatrix}
        1 & 0\\
        1 & 0\\
        \end{bmatrix} = \vec{ e_1} + \vec{ e_3} 
        .\] 
        \[
        \implies \left[ \phi \left( \vec{ e_1}  \right)  \right] _{ \mathcal{E}} = \begin{bmatrix}
        1\\
        0\\
        1\\
        0\\
        \end{bmatrix}
        .\] 
        \[
        \phi \left( \vec{ e_2}  \right) = \begin{bmatrix}
        1 & 1\\
        1 & 0\\
        \end{bmatrix} \begin{bmatrix}
        0 & 1\\
        0 & 0\\
        \end{bmatrix} = \vec{ e_2} + \vec{ e_4}
        .\] 
        \[
        \implies \left[ \phi \left( \vec{ e_2}  \right)  \right] _{ \mathcal{E}} = \begin{bmatrix}
        0\\
        1\\
        0\\
        1\\
        \end{bmatrix}
          _{ \mathcal{E}}
        .\] 
        \[
        \phi \left( \vec{ e_3}  \right) = \begin{bmatrix}
        1 & 1\\
        1 & 0\\
        \end{bmatrix} \begin{bmatrix}
        0 & 0\\
        1 & 0\\
        \end{bmatrix} = \begin{bmatrix}
        1 & 0\\
        0 & 0\\
        \end{bmatrix} = \vec{ e_1}
        .\] 
        \[
        \implies \left[ \phi \left( \vec{ e_3}  \right)  \right] _{ \mathcal{E}} = \begin{bmatrix}
        1\\
        0\\
        0\\
        0\\
        \end{bmatrix}
        _{ \mathcal{E}} 
        .\]
           \[
           \phi  \left(  \vec{ e_4}  \right) = \begin{bmatrix}
           1 & 1\\
           1 & 0\\
           \end{bmatrix} \begin{bmatrix}
           0 & 0\\
           0 & 1\\
           \end{bmatrix} = \begin{bmatrix}
           0 & 1\\
           0 & 0\\
           \end{bmatrix} = \vec{ e_2} 
           .\] 
           \[
           \implies \left[ \phi \left( \vec{ e_4}  \right)  \right] _{ \mathcal{E}} = \begin{bmatrix}
           0\\
           1\\
           0\\
           0\\
           \end{bmatrix}
            _{ \mathcal{E}}
           .\] 
           \[
            \left[ \phi  \right]_{ \mathcal{E} \mathcal{E}} = \begin{bmatrix}
            1 & 0 & 1 & 0\\
            0 & 1 & 0 & 1\\
            1 & 0 & 0 & 0\\
            0 & 1 & 0 & 0\\
            \end{bmatrix}
           .\] 

     \begin{tikzpicture}[>=Latex,
                    outer/.style={font=\small,draw=none,text=violet!80!black},
                    inner/.style={font=\small,draw=none,text=blue!70!black},
                    oarr /.style={-Latex,line width=.9pt,violet!80!black},
                    iarr/.style={-Latex,line width=.8pt,blue!70!black}]

%-------------------------------------------------------------------
% 1.  outer corner nodes (vector spaces)
%-------------------------------------------------------------------
\node[outer] (TL) at (0, 2) {$\mathcal{M}_{2\times 2}$};
\node[outer] (TR) at (8, 2) {$\mathcal{M}_{2\times 2}$};
\node[outer] (BL) at (0,-2) {$\mathbb{R}^{4}$};
\node[outer] (BR) at (8,-2) {$\mathbb{R}^{4}$};

% outer arrows
\draw[oarr] (TL) -- node[above] {$\phi$} (TR);
\draw[oarr] (TL) -- node[left]  {$[\;]_{\mathcal{E}}$} (BL);
\draw[oarr] (TR) -- node[right] {$[\;]_{\mathcal{E}}$} (BR);
\draw[oarr] (BL) -- node[below] {$\bigl[\phi\bigr]_{\mathcal{E},\mathcal{E}}$} (BR);

%-------------------------------------------------------------------
% 2.  inner nodes  (actual matrices / coordinate columns)
%-------------------------------------------------------------------
% matrices
\node[inner] (MatL) at (2.3, 0.9)
      {$\displaystyle
        \begin{bmatrix}
          a & b\\
          c & d
        \end{bmatrix}$};

\node[inner] (MatR) at (5.7, 0.9)
      {$\displaystyle
        \begin{bmatrix}
          a+c & b+d\\
          a   &  b
        \end{bmatrix}$};

% coordinate column vectors
\node[inner] (VecL) at (2.3,-0.9)
      {$\displaystyle
        \begin{bmatrix}
          a\\ b\\ c\\ d
        \end{bmatrix}_{\! \mathcal{E}}$};

\node[inner] (VecR) at (5.7,-0.9)
      {$\displaystyle
        \begin{bmatrix}
          a+c\\ b+d\\ a\\ b
        \end{bmatrix}_{\! \mathcal{E}}$};

% inner arrows
\draw[iarr] (MatL) -- (MatR);
\draw[iarr] (MatL) -- (VecL);
\draw[iarr] (MatR) -- (VecR);
\draw[iarr] (VecL) -- (VecR);

\end{tikzpicture}
\\
\\

    \[
      \text{ since } \left[ \phi  \right]  _{ \mathcal{E} \mathcal{E}} \begin{bmatrix}
      a\\
      b\\
      c\\
      d\\
      \end{bmatrix}
       _{ \mathcal{E}} = \begin{bmatrix}
       1 & 0 & 1 & 0\\
       0 & 1 & 0 & 1\\
       1 & 0 & 0 & 0\\
       0 & 1 & 0 & 0\\
       \end{bmatrix} \begin{bmatrix}
       a\\
       b\\
       c\\
       d\\
       \end{bmatrix}
        _{ \mathcal{E}} = \begin{bmatrix}
        a+c\\
        b+d\\
        a\\
        b\\
        \end{bmatrix}_{ \mathcal{E}}
    .\] 
   }
   
   \mlem{ }{
   Let $ U$, $ V$, $ W$ be vector spaces and let $ \alpha : U \to V $ and $ \phi : V \to W $ be linear transformations.\\
   Let $ \mathcal{E} $ be a basis of $ U$. $ \mathcal{F}$ be a basis of $ V$ and $ \mathcal{G}$ be a basis of $ W$.\\
   \begin{enumerate}[label=(\roman*)]
     \item $ \phi \cdot  \alpha$ is a linnear transformation and,
     \item     $ \left[ \phi  \circ \alpha \right]_{ \mathcal{E} \mathcal{G}} = \left[ \phi  \right]_{ \mathcal{F} \mathcal{G}} \left[ \alpha \right] _{ \mathcal{E} \mathcal{F}}$
     \end{enumerate}
   }
   \pf{Proof:}{
             \begin{enumerate}[label=(\roman*)]
               \item  Let $ \vec{ u_1} , \vec{ u_2}  \in U$ by linearity of $ \alpha$ we have
                 \[
                 \left( \phi \circ \alpha  \right) \left( \vec{ u_1} + \vec{ u_2}  \right) = \phi \left( \alpha \left( \vec{ u_1} +\vec{ u_2}  \right)  \right)  
                 .\] 
                 by linearty of $ \phi $
                 \[
                 = \phi  \left( \alpha \left( \vec{ u_1}  \right) + \alpha \left( \vec{ u_2}  \right)  \right) 
                 .\] 
                 Let $ \vec{ u} \in U$ , $ \lambda \in \mathbb{R}$
                 \[
                 \left( \phi \circ \alpha \right) \left( \lambda \vec{ u}  \right) = \phi \left( \alpha \left( \lambda \vec{ u}  \right)  \right)  = \phi \left( \lambda \alpha \left( \vec{ u}  \right)  \right) = \lambda \phi \left( \alpha \left( \vec{ u}  \right)  \right)
                 .\] 
                 \[
                 \implies \phi  \circ \alpha  \text{ is a linear transformation.}
                 .\] 
               \item 
                 Let $ \vec{ u} \in U$
                 \[
                   \left[ \left( \phi \circ \alpha \right)  \left( \vec{ u}  \right) \right] _{ \mathcal{G}} = \left[ \phi  \left( \alpha \left( \vec{ u}  \right)  \right)  \right] _{ \mathcal{G}} = \left[ \phi  \right] _{ \mathcal{F} \mathcal{G}} \left[ \alpha \left( \vec{ u}  \right)  \right] _{ \mathcal{F}}  = \left[ \phi  \right]_{ \mathcal{F} \mathcal{G}} \left[ \alpha \right]  _{ \mathcal{E} \mathcal{F}} \left[ \vec{ u}  \right] _{ \mathcal{E} }
                 .\] 
                 \[
                 \text{ But, } \left[ \left( \phi \circ \alpha \right)  \left( \vec{ u}  \right)  \right] _{ \mathcal{G}} = \left[ \phi  \circ \alpha  \right] _{ \mathcal{E} \mathcal{G}} \left[ \vec{ u}  \right] _{ \mathcal{G}}
                 .\] 
                 \[
                 \implies \left[ \phi  \circ \alpha  \right] _{ \mathcal{E} \mathcal{G}} = \left[ \phi  \right]_{ \mathcal{F} \mathcal{G}} \left[ \alpha  \right] _{ \mathcal{E} \mathcal{F}}
                 .\] 
               \end{enumerate}
   }
   \ex{}{
     Let $ T: \mathcal{P} _2 \left[ x \right] \to \mathcal{P} _3 \left[ x \right]$, $ T \left( p \left( x \right)  \right) = x \cdot  p \left( x \right) $ \\
     Let $ \mathcal{E} = \left\{  1, x , x^2\right\}$ is a basis for $ \mathcal{P} _2 \left[  x\right] $, $ \mathcal{F} = \left\{ 1, x ,x^2 , x^3 \right\}$ a basis for $ \mathcal{P} _3 \left[ x \right]$\\
     \[
       \left[ T \right]  _{ \mathcal{E} , \mathcal{F}} = \left[ \left[ T \left( 1 \right)  \right] _{ \mathcal{F}} \left[ T \left( x \right)  \right] _{ \mathcal{F}} \left[ T \left( x^2 \right)  \right] _{ \mathcal{F}}  \right]
     .\] 
     \[
       \left[ T \right] _{ \mathcal{E} , \mathcal{F}} = \left[ \left[ x \right] _{ \mathcal{F}} \left[ x^2 \right] _{ \mathcal{F}} \left[ x^3 \right] _{ \mathcal{F}}  \right]
     .\] 
     \[
       \left[ T \right] _{ \mathcal{E} \mathcal{F}}  = 4\begin{bmatrix}
       0 & 0 & 0 \\
       1 & 0 & 0 \\
       0 & 1 & 0 \\
       0 & 0 & 1 \\
       \end{bmatrix}
     .\] 
   }
  \ex{}{
    Let $ D: \mathcal{P} _3 \left[ x \right] \to \mathcal{P} \left[ x \right]$, $ D \left( p \left( x \right)  \right) = p' \left( x \right) $.\\
    We've seen a similar example before.\\
    Taking $ \mathcal{E} = \left\{ 1, x , x^2 \right\} $ to be a basis for $ \mathcal{P} _2 \left[ x \right]$  and $ \mathcal{F} = \left\{ 1, x ,x^2 , x^3 \right\}$  as a basis for $ \mathcal{P} _3 \left[ x \right]$\\
    \[
      \left[ D \right]  _{ \mathcal{F} \mathcal{E}} = \left[ \left[ D \left( 1 \right)  \right] _{ \mathcal{E}} \left[ D \left( x \right)  \right] _{ \mathcal{E}} \left[ D \left( x^2 \right)  \right] _{ \mathcal{E}}  \right]  = \begin{bmatrix}
      0 & 1 & 0 & 0\\
      0 & 0 & 2 & 0\\
      0 & 0 & 0 & 3\\
      \end{bmatrix}
    .\] 
    \textbf{Note:} $ \mathcal{E} = \left\{ 1,x,x^2 \right\}$, $ D \circ T : \mathcal{P}_2 \left[ x \right] \to \mathcal{P}_2 \left[ x \right]$, $ \left( D \circ T \right) = x p' \left( x \right)  $
    \[
      \left[ D \circ T \right]  _{ \mathcal{E} , \mathcal{E}} =  \left[ \left[ \left( D \circ T \right) \left( 1 \right)   \right] _{ \mathcal{E}} \left[  \left( D \circ T \right) \left( x \right)   \right] _{ \mathcal{E}}  \left[ \left( D \circ T \right) \left( x^2 \right)   \right]_{ \mathcal{E}}\right] = \begin{bmatrix}
      1 & 0 & 0\\
      0 & 2 & 0\\
      0 & 0 & 3\\
    \end{bmatrix}                = \left[ D \right] _{ \mathcal{F} , \mathcal{E}} \cdot \left[ T \right] _{ \mathcal{E} \mathcal{F}} 
    .\] 

 \[
\begin{tikzpicture}[>=Latex,
                    every node/.style={font=\small},
                    x=1cm,y=1cm]

\node (P2L) at (0, 1.8) {$\mathcal P_{2}[x]$};
\node (P3)  at (4, 1.8) {$\mathcal P_{3}[x]$};
\node (P2R) at (8, 1.8) {$\mathcal P_{2}[x]$};

\node (R3L) at (0, 0)   {$\mathbb R^{3}$};
\node (R4)  at (4, 0)   {$\mathbb R^{4}$};
\node (R3R) at (8, 0)   {$\mathbb R^{3}$};

\draw[->,thick] (P2L) -- node[above] {$T$}               (P3);
\draw[->,thick] (P3)  -- node[above] {$D$}               (P2R);

\draw[->,thick] (P2L) -- node[left]  {$[\;]_{\mathcal{E}}$}      (R3L);
\draw[->,thick] (P3)  -- node[right] {$[\;]_{\mathcal F}$}(R4);
\draw[->,thick] (P2R) -- node[right] {$[\;]_{\mathcal{E}}$}      (R3R);

\draw[->,thick] (R3L) -- node[below] {$[T]_{\mathcal{E},\mathcal F}$} (R4);
\draw[->,thick] (R4)  -- node[below] {$[D]_{\mathcal F,\mathcal{E}}$} (R3R);

\draw[->,thick, bend left=18]  (P2L) to node[above] {$D\!\circ\!T$} (P2R);
\draw[->,thick, bend right=18] (R3L) to node[below] {$[D]_{\mathcal F,\mathcal{E}}\!\circ\![T]_{\mathcal{E},\mathcal F}$}
                               (R3R);

\node at (6.3,-1.05) {$  =\;\bigl[(D\!\circ\!T)\bigr]_{\mathcal{E},\mathcal{E}}$};

\end{tikzpicture}
\] 
  }
   
  


\end{document} 
