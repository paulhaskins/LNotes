\documentclass{report}

%%%%%%%%%%%%%%%%%%%%%%%%%%%%%%%%%
% PACKAGE IMPORTS
%%%%%%%%%%%%%%%%%%%%%%%%%%%%%%%%%


\usepackage[tmargin=2cm,rmargin=1in,lmargin=1in,margin=0.85in,bmargin=2cm,footskip=.2in]{geometry}
\usepackage{amsmath,amsfonts,amsthm,amssymb,mathtools}
\usepackage[varbb]{newpxmath}
\usepackage{xfrac}
\usepackage[makeroom]{cancel}
\usepackage{bookmark}
\usepackage{enumitem}
\usepackage{hyperref,theoremref}
\hypersetup{
	pdftitle={Assignment},
	colorlinks=true, linkcolor=doc!90,
	bookmarksnumbered=true,
	bookmarksopen=true
}
\usepackage[most,many,breakable]{tcolorbox}
\usepackage{xcolor}
\usepackage{varwidth}
\usepackage{varwidth}
\usepackage{tocloft}
\usepackage{etoolbox}
\usepackage{derivative} %many derivativess partials
%\usepackage{authblk}
\usepackage{nameref}
\usepackage{multicol,array}
\usepackage{tikz-cd}
\usepackage[ruled,vlined,linesnumbered]{algorithm2e}
\usepackage{comment} % enables the use of multi-line comments (\ifx \fi) 
\usepackage{import}
\usepackage{xifthen}
\usepackage{pdfpages}
\usepackage{transparent}
\usepackage{verbatim}

\newcommand\mycommfont[1]{\footnotesize\ttfamily\textcolor{blue}{#1}}
\SetCommentSty{mycommfont}
\newcommand{\incfig}[1]{%
    \def\svgwidth{\columnwidth}
    \import{./figures/}{#1.pdf_tex}
}
\usepackage[tagged, highstructure]{accessibility}
\usepackage{tikzsymbols}
\renewcommand\qedsymbol{$\Laughey$}


%\usepackage{import}
%\usepackage{xifthen}
%\usepackage{pdfpages}
%\usepackage{transparent}


%%%%%%%%%%%%%%%%%%%%%%%%%%%%%%
% SELF MADE COLORS
%%%%%%%%%%%%%%%%%%%%%%%%%%%%%%



\definecolor{myg}{RGB}{56, 140, 70}
\definecolor{myb}{RGB}{45, 111, 177}
\definecolor{myr}{RGB}{199, 68, 64}
\definecolor{mytheorembg}{HTML}{F2F2F9}
\definecolor{mytheoremfr}{HTML}{00007B}
\definecolor{mylenmabg}{HTML}{FFFAF8}
\definecolor{mylenmafr}{HTML}{983b0f}
\definecolor{mypropbg}{HTML}{f2fbfc}
\definecolor{mypropfr}{HTML}{191971}
\definecolor{myexamplebg}{HTML}{F2FBF8}
\definecolor{myexamplefr}{HTML}{88D6D1}
\definecolor{myexampleti}{HTML}{2A7F7F}
\definecolor{mydefinitbg}{HTML}{E5E5FF}
\definecolor{mydefinitfr}{HTML}{3F3FA3}
\definecolor{notesgreen}{RGB}{0,162,0}
\definecolor{myp}{RGB}{197, 92, 212}
\definecolor{mygr}{HTML}{2C3338}
\definecolor{myred}{RGB}{127,0,0}
\definecolor{myyellow}{RGB}{169,121,69}
\definecolor{myexercisebg}{HTML}{F2FBF8}
\definecolor{myexercisefg}{HTML}{88D6D1}


%%%%%%%%%%%%%%%%%%%%%%%%%%%%
% TCOLORBOX SETUPS
%%%%%%%%%%%%%%%%%%%%%%%%%%%%

\setlength{\parindent}{1cm}
%================================
% THEOREM BOX
%================================

\tcbuselibrary{theorems,skins,hooks}
\newtcbtheorem[number within=section]{Theorem}{Theorem}
{%
	enhanced,
	breakable,
	colback = mytheorembg,
	frame hidden,
	boxrule = 0sp,
	borderline west = {2pt}{0pt}{mytheoremfr},
	sharp corners,
	detach title,
	before upper = \tcbtitle\par\smallskip,
	coltitle = mytheoremfr,
	fonttitle = \bfseries\sffamily,
	description font = \mdseries,
	separator sign none,
	segmentation style={solid, mytheoremfr},
}
{th}

\tcbuselibrary{theorems,skins,hooks}
\newtcbtheorem[number within=chapter]{theorem}{Theorem}
{%
	enhanced,
	breakable,
	colback = mytheorembg,
	frame hidden,
	boxrule = 0sp,
	borderline west = {2pt}{0pt}{mytheoremfr},
	sharp corners,
	detach title,
	before upper = \tcbtitle\par\smallskip,
	coltitle = mytheoremfr,
	fonttitle = \bfseries\sffamily,
	description font = \mdseries,
	separator sign none,
	segmentation style={solid, mytheoremfr},
}
{th}


\tcbuselibrary{theorems,skins,hooks}
\newtcolorbox{Theoremcon}
{%
	enhanced
	,breakable
	,colback = mytheorembg
	,frame hidden
	,boxrule = 0sp
	,borderline west = {2pt}{0pt}{mytheoremfr}
	,sharp corners
	,description font = \mdseries
	,separator sign none
}

%================================
% Corollery
%================================
\tcbuselibrary{theorems,skins,hooks}
\newtcbtheorem[number within=section]{Corollary}{Corollary}
{%
	enhanced
	,breakable
	,colback = myp!10
	,frame hidden
	,boxrule = 0sp
	,borderline west = {2pt}{0pt}{myp!85!black}
	,sharp corners
	,detach title
	,before upper = \tcbtitle\par\smallskip
	,coltitle = myp!85!black
	,fonttitle = \bfseries\sffamily
	,description font = \mdseries
	,separator sign none
	,segmentation style={solid, myp!85!black}
}
{th}
\tcbuselibrary{theorems,skins,hooks}
\newtcbtheorem[number within=chapter]{corollary}{Corollary}
{%
	enhanced
	,breakable
	,colback = myp!10
	,frame hidden
	,boxrule = 0sp
	,borderline west = {2pt}{0pt}{myp!85!black}
	,sharp corners
	,detach title
	,before upper = \tcbtitle\par\smallskip
	,coltitle = myp!85!black
	,fonttitle = \bfseries\sffamily
	,description font = \mdseries
	,separator sign none
	,segmentation style={solid, myp!85!black}
}
{th}


%================================
% LENMA
%================================

\tcbuselibrary{theorems,skins,hooks}
\newtcbtheorem[number within=section]{Lenma}{Lenma}
{%
	enhanced,
	breakable,
	colback = mylenmabg,
	frame hidden,
	boxrule = 0sp,
	borderline west = {2pt}{0pt}{mylenmafr},
	sharp corners,
	detach title,
	before upper = \tcbtitle\par\smallskip,
	coltitle = mylenmafr,
	fonttitle = \bfseries\sffamily,
	description font = \mdseries,
	separator sign none,
	segmentation style={solid, mylenmafr},
}
{th}

\tcbuselibrary{theorems,skins,hooks}
\newtcbtheorem[number within=chapter]{lenma}{Lenma}
{%
	enhanced,
	breakable,
	colback = mylenmabg,
	frame hidden,
	boxrule = 0sp,
	borderline west = {2pt}{0pt}{mylenmafr},
	sharp corners,
	detach title,
	before upper = \tcbtitle\par\smallskip,
	coltitle = mylenmafr,
	fonttitle = \bfseries\sffamily,
	description font = \mdseries,
	separator sign none,
	segmentation style={solid, mylenmafr},
}
{th}


%================================
% PROPOSITION
%================================

\tcbuselibrary{theorems,skins,hooks}
\newtcbtheorem[number within=section]{Prop}{Proposition}
{%
	enhanced,
	breakable,
	colback = mypropbg,
	frame hidden,
	boxrule = 0sp,
	borderline west = {2pt}{0pt}{mypropfr},
	sharp corners,
	detach title,
	before upper = \tcbtitle\par\smallskip,
	coltitle = mypropfr,
	fonttitle = \bfseries\sffamily,
	description font = \mdseries,
	separator sign none,
	segmentation style={solid, mypropfr},
}
{th}

\tcbuselibrary{theorems,skins,hooks}
\newtcbtheorem[number within=chapter]{prop}{Proposition}
{%
	enhanced,
	breakable,
	colback = mypropbg,
	frame hidden,
	boxrule = 0sp,
	borderline west = {2pt}{0pt}{mypropfr},
	sharp corners,
	detach title,
	before upper = \tcbtitle\par\smallskip,
	coltitle = mypropfr,
	fonttitle = \bfseries\sffamily,
	description font = \mdseries,
	separator sign none,
	segmentation style={solid, mypropfr},
}
{th}


%================================
% CLAIM
%================================

\tcbuselibrary{theorems,skins,hooks}
\newtcbtheorem[number within=section]{claim}{Claim}
{%
	enhanced
	,breakable
	,colback = myg!10
	,frame hidden
	,boxrule = 0sp
	,borderline west = {2pt}{0pt}{myg}
	,sharp corners
	,detach title
	,before upper = \tcbtitle\par\smallskip
	,coltitle = myg!85!black
	,fonttitle = \bfseries\sffamily
	,description font = \mdseries
	,separator sign none
	,segmentation style={solid, myg!85!black}
}
{th}



%================================
% Exercise
%================================

\tcbuselibrary{theorems,skins,hooks}
\newtcbtheorem[number within=section]{Exercise}{Exercise}
{%
	enhanced,
	breakable,
	colback = myexercisebg,
	frame hidden,
	boxrule = 0sp,
	borderline west = {2pt}{0pt}{myexercisefg},
	sharp corners,
	detach title,
	before upper = \tcbtitle\par\smallskip,
	coltitle = myexercisefg,
	fonttitle = \bfseries\sffamily,
	description font = \mdseries,
	separator sign none,
	segmentation style={solid, myexercisefg},
}
{th}

\tcbuselibrary{theorems,skins,hooks}
\newtcbtheorem[number within=chapter]{exercise}{Exercise}
{%
	enhanced,
	breakable,
	colback = myexercisebg,
	frame hidden,
	boxrule = 0sp,
	borderline west = {2pt}{0pt}{myexercisefg},
	sharp corners,
	detach title,
	before upper = \tcbtitle\par\smallskip,
	coltitle = myexercisefg,
	fonttitle = \bfseries\sffamily,
	description font = \mdseries,
	separator sign none,
	segmentation style={solid, myexercisefg},
}
{th}

%================================
% EXAMPLE BOX
%================================

\newtcbtheorem[number within=section]{Example}{Example}
{%
	colback = myexamplebg
	,breakable
	,colframe = myexamplefr
	,coltitle = myexampleti
	,boxrule = 1pt
	,sharp corners
	,detach title
	,before upper=\tcbtitle\par\smallskip
	,fonttitle = \bfseries
	,description font = \mdseries
	,separator sign none
	,description delimiters parenthesis
}
{ex}

\newtcbtheorem[number within=chapter]{example}{Example}
{%
	colback = myexamplebg
	,breakable
	,colframe = myexamplefr
	,coltitle = myexampleti
	,boxrule = 1pt
	,sharp corners
	,detach title
	,before upper=\tcbtitle\par\smallskip
	,fonttitle = \bfseries
	,description font = \mdseries
	,separator sign none
	,description delimiters parenthesis
}
{ex}

%================================
% DEFINITION BOX
%================================

\newtcbtheorem[number within=section]{Definition}{Definition}{enhanced,
	before skip=2mm,after skip=2mm, colback=red!5,colframe=red!80!black,boxrule=0.5mm,
	attach boxed title to top left={xshift=1cm,yshift*=1mm-\tcboxedtitleheight}, varwidth boxed title*=-3cm,
	boxed title style={frame code={
					\path[fill=tcbcolback]
					([yshift=-1mm,xshift=-1mm]frame.north west)
					arc[start angle=0,end angle=180,radius=1mm]
					([yshift=-1mm,xshift=1mm]frame.north east)
					arc[start angle=180,end angle=0,radius=1mm];
					\path[left color=tcbcolback!60!black,right color=tcbcolback!60!black,
						middle color=tcbcolback!80!black]
					([xshift=-2mm]frame.north west) -- ([xshift=2mm]frame.north east)
					[rounded corners=1mm]-- ([xshift=1mm,yshift=-1mm]frame.north east)
					-- (frame.south east) -- (frame.south west)
					-- ([xshift=-1mm,yshift=-1mm]frame.north west)
					[sharp corners]-- cycle;
				},interior engine=empty,
		},
	fonttitle=\bfseries,
	title={#2},#1}{def}
\newtcbtheorem[number within=chapter]{definition}{Definition}{enhanced,
	before skip=2mm,after skip=2mm, colback=red!5,colframe=red!80!black,boxrule=0.5mm,
	attach boxed title to top left={xshift=1cm,yshift*=1mm-\tcboxedtitleheight}, varwidth boxed title*=-3cm,
	boxed title style={frame code={
					\path[fill=tcbcolback]
					([yshift=-1mm,xshift=-1mm]frame.north west)
					arc[start angle=0,end angle=180,radius=1mm]
					([yshift=-1mm,xshift=1mm]frame.north east)
					arc[start angle=180,end angle=0,radius=1mm];
					\path[left color=tcbcolback!60!black,right color=tcbcolback!60!black,
						middle color=tcbcolback!80!black]
					([xshift=-2mm]frame.north west) -- ([xshift=2mm]frame.north east)
					[rounded corners=1mm]-- ([xshift=1mm,yshift=-1mm]frame.north east)
					-- (frame.south east) -- (frame.south west)
					-- ([xshift=-1mm,yshift=-1mm]frame.north west)
					[sharp corners]-- cycle;
				},interior engine=empty,
		},
	fonttitle=\bfseries,
	title={#2},#1}{def}



%================================
% Solution BOX
%================================

\makeatletter
\newtcbtheorem{question}{Question}{enhanced,
	breakable,
	colback=white,
	colframe=myb!80!black,
	attach boxed title to top left={yshift*=-\tcboxedtitleheight},
	fonttitle=\bfseries,
	title={#2},
	boxed title size=title,
	boxed title style={%
			sharp corners,
			rounded corners=northwest,
			colback=tcbcolframe,
			boxrule=0pt,
		},
	underlay boxed title={%
			\path[fill=tcbcolframe] (title.south west)--(title.south east)
			to[out=0, in=180] ([xshift=5mm]title.east)--
			(title.center-|frame.east)
			[rounded corners=\kvtcb@arc] |-
			(frame.north) -| cycle;
		},
	#1
}{def}
\makeatother

%================================
% SOLUTION BOX
%================================

\makeatletter
\newtcolorbox{solution}{enhanced,
	breakable,
	colback=white,
	colframe=myg!80!black,
	attach boxed title to top left={yshift*=-\tcboxedtitleheight},
	title=Solution,
	boxed title size=title,
	boxed title style={%
			sharp corners,
			rounded corners=northwest,
			colback=tcbcolframe,
			boxrule=0pt,
		},
	underlay boxed title={%
			\path[fill=tcbcolframe] (title.south west)--(title.south east)
			to[out=0, in=180] ([xshift=5mm]title.east)--
			(title.center-|frame.east)
			[rounded corners=\kvtcb@arc] |-
			(frame.north) -| cycle;
		},
}
\makeatother

%================================
% Question BOX
%================================

\makeatletter
\newtcbtheorem{qstion}{Question}{enhanced,
	breakable,
	colback=white,
	colframe=mygr,
	attach boxed title to top left={yshift*=-\tcboxedtitleheight},
	fonttitle=\bfseries,
	title={#2},
	boxed title size=title,
	boxed title style={%
			sharp corners,
			rounded corners=northwest,
			colback=tcbcolframe,
			boxrule=0pt,
		},
	underlay boxed title={%
			\path[fill=tcbcolframe] (title.south west)--(title.south east)
			to[out=0, in=180] ([xshift=5mm]title.east)--
			(title.center-|frame.east)
			[rounded corners=\kvtcb@arc] |-
			(frame.north) -| cycle;
		},
	#1
}{def}
\makeatother

\newtcbtheorem[number within=chapter]{wconc}{Wrong Concept}{
	breakable,
	enhanced,
	colback=white,
	colframe=myr,
	arc=0pt,
	outer arc=0pt,
	fonttitle=\bfseries\sffamily\large,
	colbacktitle=myr,
	attach boxed title to top left={},
	boxed title style={
			enhanced,
			skin=enhancedfirst jigsaw,
			arc=3pt,
			bottom=0pt,
			interior style={fill=myr}
		},
	#1
}{def}



%================================
% NOTE BOX
%================================

\usetikzlibrary{arrows,calc,shadows.blur}
\tcbuselibrary{skins}
\newtcolorbox{note}[1][]{%
	enhanced jigsaw,
	colback=gray!20!white,%
	colframe=gray!80!black,
	size=small,
	boxrule=1pt,
	title=\textbf{Note:-},
	halign title=flush center,
	coltitle=black,
	breakable,
	drop shadow=black!50!white,
	attach boxed title to top left={xshift=1cm,yshift=-\tcboxedtitleheight/2,yshifttext=-\tcboxedtitleheight/2},
	minipage boxed title=1.5cm,
	boxed title style={%
			colback=white,
			size=fbox,
			boxrule=1pt,
			boxsep=2pt,
			underlay={%
					\coordinate (dotA) at ($(interior.west) + (-0.5pt,0)$);
					\coordinate (dotB) at ($(interior.east) + (0.5pt,0)$);
					\begin{scope}
						\clip (interior.north west) rectangle ([xshift=3ex]interior.east);
						\filldraw [white, blur shadow={shadow opacity=60, shadow yshift=-.75ex}, rounded corners=2pt] (interior.north west) rectangle (interior.south east);
					\end{scope}
					\begin{scope}[gray!80!black]
						\fill (dotA) circle (2pt);
						\fill (dotB) circle (2pt);
					\end{scope}
				},
		},
	#1,
}

%%%%%%%%%%%%%%%%%%%%%%%%%%%%%%
% SELF MADE COMMANDS
%%%%%%%%%%%%%%%%%%%%%%%%%%%%%%


\newcommand{\thm}[2]{\begin{Theorem}{#1}{}#2\end{Theorem}}
\newcommand{\cor}[2]{\begin{Corollary}{#1}{}#2\end{Corollary}}
\newcommand{\mlenma}[2]{\begin{Lenma}{#1}{}#2\end{Lenma}}
\newcommand{\mprop}[2]{\begin{Prop}{#1}{}#2\end{Prop}}
\newcommand{\clm}[3]{\begin{claim}{#1}{#2}#3\end{claim}}
\newcommand{\wc}[2]{\begin{wconc}{#1}{}\setlength{\parindent}{1cm}#2\end{wconc}}
\newcommand{\thmcon}[1]{\begin{Theoremcon}{#1}\end{Theoremcon}}
\newcommand{\ex}[2]{\begin{Example}{#1}{}#2\end{Example}}
\newcommand{\dfn}[2]{\begin{Definition}[colbacktitle=red!75!black]{#1}{}#2\end{Definition}}
\newcommand{\dfnc}[2]{\begin{definition}[colbacktitle=red!75!black]{#1}{}#2\end{definition}}
\newcommand{\qs}[2]{\begin{question}{#1}{}#2\end{question}}
\newcommand{\pf}[2]{\begin{myproof}[#1]#2\end{myproof}}
\newcommand{\nt}[1]{\begin{note}#1\end{note}}

\newcommand*\circled[1]{\tikz[baseline=(char.base)]{
		\node[shape=circle,draw,inner sep=1pt] (char) {#1};}}
\newcommand\getcurrentref[1]{%
	\ifnumequal{\value{#1}}{0}
	{??}
	{\the\value{#1}}%
}
\newcommand{\getCurrentSectionNumber}{\getcurrentref{section}}
\newenvironment{myproof}[1][\proofname]{%
	\proof[\bfseries #1: ]%
}{\endproof}

\newcommand{\mclm}[2]{\begin{myclaim}[#1]#2\end{myclaim}}
\newenvironment{myclaim}[1][\claimname]{\proof[\bfseries #1: ]}{}

\newcounter{mylabelcounter}

\makeatletter
\newcommand{\setword}[2]{%
	\phantomsection
	#1\def\@currentlabel{\unexpanded{#1}}\label{#2}%
}
\makeatother




\tikzset{
	symbol/.style={
			draw=none,
			every to/.append style={
					edge node={node [sloped, allow upside down, auto=false]{$#1$}}}
		}
}


% deliminators
\DeclarePairedDelimiter{\abs}{\lvert}{\rvert}
\DeclarePairedDelimiter{\norm}{\lVert}{\rVert}

\DeclarePairedDelimiter{\ceil}{\lceil}{\rceil}
\DeclarePairedDelimiter{\floor}{\lfloor}{\rfloor}
\DeclarePairedDelimiter{\round}{\lfloor}{\rceil}

\newsavebox\diffdbox
\newcommand{\slantedromand}{{\mathpalette\makesl{d}}}
\newcommand{\makesl}[2]{%
\begingroup
\sbox{\diffdbox}{$\mathsurround=0pt#1\mathrm{#2}$}%
\pdfsave
\pdfsetmatrix{1 0 0.2 1}%
\rlap{\usebox{\diffdbox}}%
\pdfrestore
\hskip\wd\diffdbox
\endgroup
}
\newcommand{\dd}[1][]{\ensuremath{\mathop{}\!\ifstrempty{#1}{%
\slantedromand\@ifnextchar^{\hspace{0.2ex}}{\hspace{0.1ex}}}%
{\slantedromand\hspace{0.2ex}^{#1}}}}
\ProvideDocumentCommand\dv{o m g}{%
  \ensuremath{%
    \IfValueTF{#3}{%
      \IfNoValueTF{#1}{%
        \frac{\dd #2}{\dd #3}%
      }{%
        \frac{\dd^{#1} #2}{\dd #3^{#1}}%
      }%
    }{%
      \IfNoValueTF{#1}{%
        \frac{\dd}{\dd #2}%
      }{%
        \frac{\dd^{#1}}{\dd #2^{#1}}%
      }%
    }%
  }%
}
\providecommand*{\pdv}[3][]{\frac{\partial^{#1}#2}{\partial#3^{#1}}}
%  - others
\DeclareMathOperator{\Lap}{\mathcal{L}}
\DeclareMathOperator{\Var}{Var} % varience
\DeclareMathOperator{\Cov}{Cov} % covarience
\DeclareMathOperator{\E}{E} % expected

% Since the amsthm package isn't loaded

% I prefer the slanted \leq
\let\oldleq\leq % save them in case they're every wanted
\let\oldgeq\geq
\renewcommand{\leq}{\leqslant}
\renewcommand{\geq}{\geqslant}

% % redefine matrix env to allow for alignment, use r as default
% \renewcommand*\env@matrix[1][r]{\hskip -\arraycolsep
%     \let\@ifnextchar\new@ifnextchar
%     \array{*\c@MaxMatrixCols #1}}


%\usepackage{framed}
%\usepackage{titletoc}
%\usepackage{etoolbox}
%\usepackage{lmodern}


%\patchcmd{\tableofcontents}{\contentsname}{\sffamily\contentsname}{}{}

%\renewenvironment{leftbar}
%{\def\FrameCommand{\hspace{6em}%
%		{\color{myyellow}\vrule width 2pt depth 6pt}\hspace{1em}}%
%	\MakeFramed{\parshape 1 0cm \dimexpr\textwidth-6em\relax\FrameRestore}\vskip2pt%
%}
%{\endMakeFramed}

%\titlecontents{chapter}
%[0em]{\vspace*{2\baselineskip}}
%{\parbox{4.5em}{%
%		\hfill\Huge\sffamily\bfseries\color{myred}\thecontentspage}%
%	\vspace*{-2.3\baselineskip}\leftbar\textsc{\small\chaptername~\thecontentslabel}\\\sffamily}
%{}{\endleftbar}
%\titlecontents{section}
%[8.4em]
%{\sffamily\contentslabel{3em}}{}{}
%{\hspace{0.5em}\nobreak\itshape\color{myred}\contentspage}
%\titlecontents{subsection}
%[8.4em]
%{\sffamily\contentslabel{3em}}{}{}  
%{\hspace{0.5em}\nobreak\itshape\color{myred}\contentspage}



%%%%%%%%%%%%%%%%%%%%%%%%%%%%%%%%%%%%%%%%%%%
% TABLE OF CONTENTS
%%%%%%%%%%%%%%%%%%%%%%%%%%%%%%%%%%%%%%%%%%%

\usepackage{tikz}
\definecolor{doc}{RGB}{0,60,110}
\usepackage{titletoc}
\contentsmargin{0cm}
\titlecontents{chapter}[3.7pc]
{\addvspace{30pt}%
	\begin{tikzpicture}[remember picture, overlay]%
		\draw[fill=doc!60,draw=doc!60] (-7,-.1) rectangle (-0.9,.5);%
		\pgftext[left,x=-3.5cm,y=0.2cm]{\color{white}\Large\sc\bfseries Chapter\ \thecontentslabel};%
	\end{tikzpicture}\color{doc!60}\large\sc\bfseries}%
{}
{}
{\;\titlerule\;\large\sc\bfseries Page \thecontentspage
	\begin{tikzpicture}[remember picture, overlay]
		\draw[fill=doc!60,draw=doc!60] (2pt,0) rectangle (4,0.1pt);
	\end{tikzpicture}}%
\titlecontents{section}[3.7pc]
{\addvspace{2pt}}
{\contentslabel[\thecontentslabel]{2pc}}
{}
{\hfill\small \thecontentspage}
[]
\titlecontents*{subsection}[3.7pc]
{\addvspace{-1pt}\small}
{}
{}
{\ --- \small\thecontentspage}
[ \textbullet\ ][]

\makeatletter
\renewcommand{\tableofcontents}{%
	\chapter*{%
	  \vspace*{-20\p@}%
	  \begin{tikzpicture}[remember picture, overlay]%
		  \pgftext[right,x=15cm,y=0.2cm]{\color{doc!60}\Huge\sc\bfseries \contentsname};%
		  \draw[fill=doc!60,draw=doc!60] (13,-.75) rectangle (20,1);%
		  \clip (13,-.75) rectangle (20,1);
		  \pgftext[right,x=15cm,y=0.2cm]{\color{white}\Huge\sc\bfseries \contentsname};%
	  \end{tikzpicture}}%
	\@starttoc{toc}}
\makeatother


%From M275 "Topology" at SJSU
\newcommand{\id}{\mathrm{id}}
\newcommand{\taking}[1]{\xrightarrow{#1}}
\newcommand{\inv}{^{-1}}

%From M170 "Introduction to Graph Theory" at SJSU
\DeclareMathOperator{\diam}{diam}
\DeclareMathOperator{\ord}{ord}
\newcommand{\defeq}{\overset{\mathrm{def}}{=}}

%From the USAMO .tex files
\newcommand{\ts}{\textsuperscript}
\newcommand{\dg}{^\circ}
\newcommand{\ii}{\item}

% % From Math 55 and Math 145 at Harvard
% \newenvironment{subproof}[1][Proof]{%
% \begin{proof}[#1] \renewcommand{\qedsymbol}{$\blacksquare$}}%
% {\end{proof}}

\newcommand{\liff}{\leftrightarrow}
\newcommand{\lthen}{\rightarrow}
\newcommand{\opname}{\operatorname}
\newcommand{\surjto}{\twoheadrightarrow}
\newcommand{\injto}{\hookrightarrow}
\newcommand{\On}{\mathrm{On}} % ordinals
\DeclareMathOperator{\img}{im} % Image
\DeclareMathOperator{\Img}{Im} % Image
\DeclareMathOperator{\coker}{coker} % Cokernel
\DeclareMathOperator{\Coker}{Coker} % Cokernel
\DeclareMathOperator{\Ker}{Ker} % Kernel
\DeclareMathOperator{\rank}{rank}
\DeclareMathOperator{\Spec}{Spec} % spectrum
\DeclareMathOperator{\Tr}{Tr} % trace
\DeclareMathOperator{\pr}{pr} % projection
\DeclareMathOperator{\ext}{ext} % extension
\DeclareMathOperator{\pred}{pred} % predecessor
\DeclareMathOperator{\dom}{dom} % domain
\DeclareMathOperator{\ran}{ran} % range
\DeclareMathOperator{\Hom}{Hom} % homomorphism
\DeclareMathOperator{\Mor}{Mor} % morphisms
\DeclareMathOperator{\End}{End} % endomorphism

\newcommand{\eps}{\epsilon}
\newcommand{\veps}{\varepsilon}
\newcommand{\ol}{\overline}
\newcommand{\ul}{\underline}
\newcommand{\wt}{\widetilde}
\newcommand{\wh}{\widehat}
\newcommand{\vocab}[1]{\textbf{\color{blue} #1}}
\providecommand{\half}{\frac{1}{2}}
\newcommand{\dang}{\measuredangle} %% Directed angle
\newcommand{\ray}[1]{\overrightarrow{#1}}
\newcommand{\seg}[1]{\overline{#1}}
\newcommand{\arc}[1]{\wideparen{#1}}
\DeclareMathOperator{\cis}{cis}
\DeclareMathOperator*{\lcm}{lcm}
\DeclareMathOperator*{\argmin}{arg min}
\DeclareMathOperator*{\argmax}{arg max}
\newcommand{\cycsum}{\sum_{\mathrm{cyc}}}
\newcommand{\symsum}{\sum_{\mathrm{sym}}}
\newcommand{\cycprod}{\prod_{\mathrm{cyc}}}
\newcommand{\symprod}{\prod_{\mathrm{sym}}}
\newcommand{\Qed}{\begin{flushright}\qed\end{flushright}}
\newcommand{\parinn}{\setlength{\parindent}{1cm}}
\newcommand{\parinf}{\setlength{\parindent}{0cm}}
% \newcommand{\norm}{\|\cdot\|}
\newcommand{\inorm}{\norm_{\infty}}
\newcommand{\opensets}{\{V_{\alpha}\}_{\alpha\in I}}
\newcommand{\oset}{V_{\alpha}}
\newcommand{\opset}[1]{V_{\alpha_{#1}}}
\newcommand{\lub}{\text{lub}}
\newcommand{\del}[2]{\frac{\partial #1}{\partial #2}}
\newcommand{\Del}[3]{\frac{\partial^{#1} #2}{\partial^{#1} #3}}
\newcommand{\deld}[2]{\dfrac{\partial #1}{\partial #2}}
\newcommand{\Deld}[3]{\dfrac{\partial^{#1} #2}{\partial^{#1} #3}}
\newcommand{\lm}{\lambda}
\newcommand{\uin}{\mathbin{\rotatebox[origin=c]{90}{$\in$}}}
\newcommand{\usubset}{\mathbin{\rotatebox[origin=c]{90}{$\subset$}}}
\newcommand{\lt}{\left}
\newcommand{\rt}{\right}
\newcommand{\bs}[1]{\boldsymbol{#1}}
\newcommand{\exs}{\exists}
\newcommand{\st}{\strut}
\newcommand{\dps}[1]{\displaystyle{#1}}

\newcommand{\sol}{\setlength{\parindent}{0cm}\textbf{\textit{Solution:}}\setlength{\parindent}{1cm} }
\newcommand{\solve}[1]{\setlength{\parindent}{0cm}\textbf{\textit{Solution: }}\setlength{\parindent}{1cm}#1 \Qed}

%--------------------------------------------------
% LIE ALGEBRAS
%--------------------------------------------------
\newcommand*{\kb}{\mathfrak{b}}  % Borel subalgebra
\newcommand*{\kg}{\mathfrak{g}}  % Lie algebra
\newcommand*{\kh}{\mathfrak{h}}  % Cartan subalgebra
\newcommand*{\kn}{\mathfrak{n}}  % Nilradical
\newcommand*{\ku}{\mathfrak{u}}  % Unipotent algebra
\newcommand*{\kz}{\mathfrak{z}}  % Center of algebra

%--------------------------------------------------
% HOMOLOGICAL ALGEBRA
%--------------------------------------------------
\DeclareMathOperator{\Ext}{Ext} % Ext functor
\DeclareMathOperator{\Tor}{Tor} % Tor functor

%--------------------------------------------------
% MATRIX & GROUP NOTATION
%--------------------------------------------------
\DeclareMathOperator{\GL}{GL} % General Linear Group
\DeclareMathOperator{\SL}{SL} % Special Linear Group
\newcommand*{\gl}{\operatorname{\mathfrak{gl}}} % General linear Lie algebra
\newcommand*{\sl}{\operatorname{\mathfrak{sl}}} % Special linear Lie algebra

%--------------------------------------------------
% NUMBER SETS
%--------------------------------------------------
\newcommand*{\RR}{\mathbb{R}}
\newcommand*{\NN}{\mathbb{N}}
\newcommand*{\ZZ}{\mathbb{Z}}
\newcommand*{\QQ}{\mathbb{Q}}
\newcommand*{\CC}{\mathbb{C}}
\newcommand*{\PP}{\mathbb{P}}
\newcommand*{\HH}{\mathbb{H}}
\newcommand*{\FF}{\mathbb{F}}
\newcommand*{\EE}{\mathbb{E}} % Expected Value

%--------------------------------------------------
% MATH SCRIPT, FRAKTUR, AND BOLD SYMBOLS
%--------------------------------------------------
\newcommand*{\mcA}{\mathcal{A}}
\newcommand*{\mcB}{\mathcal{B}}
\newcommand*{\mcC}{\mathcal{C}}
\newcommand*{\mcD}{\mathcal{D}}
\newcommand*{\mcE}{\mathcal{E}}
\newcommand*{\mcF}{\mathcal{F}}
\newcommand*{\mcG}{\mathcal{G}}
\newcommand*{\mcH}{\mathcal{H}}

\newcommand*{\mfA}{\mathfrak{A}}  \newcommand*{\mfB}{\mathfrak{B}}
\newcommand*{\mfC}{\mathfrak{C}}  \newcommand*{\mfD}{\mathfrak{D}}
\newcommand*{\mfE}{\mathfrak{E}}  \newcommand*{\mfF}{\mathfrak{F}}
\newcommand*{\mfG}{\mathfrak{G}}  \newcommand*{\mfH}{\mathfrak{H}}

\usepackage{bm} % Ensure bold math works correctly
\newcommand*{\bmA}{\bm{A}}
\newcommand*{\bmB}{\bm{B}}
\newcommand*{\bmC}{\bm{C}}
\newcommand*{\bmD}{\bm{D}}
\newcommand*{\bmE}{\bm{E}}
\newcommand*{\bmF}{\bm{F}}
\newcommand*{\bmG}{\bm{G}}
\newcommand*{\bmH}{\bm{H}}

%--------------------------------------------------
% FUNCTIONAL ANALYSIS & ALGEBRA
%--------------------------------------------------
\DeclareMathOperator{\Aut}{Aut} % Automorphism group
\DeclareMathOperator{\Inn}{Inn} % Inner automorphisms
\DeclareMathOperator{\Syl}{Syl} % Sylow subgroups
\DeclareMathOperator{\Gal}{Gal} % Galois group
\DeclareMathOperator{\sign}{sign} % Sign function


%\usepackage[tagged, highstructure]{accessibility}
\usepackage{tocloft}
\usepackage{arydshln}




\begin{document}
\title{Linear Algebra I}
\author{Lecture Notes Provided by Dr.~Miriam Logan.}
\date{}
\maketitle
\tableofcontents
\newpage  
 For the following definitions let $ V$ be a vector space and $ \left\{ \vec{ v_1} , \vec{ v_2} ,\ldots , \vec{  v_k}  \right\} \subseteq V $.\\
 \dfn{Span :}{
	 span $\left\{ \vec{ v_1} , \vec{ v_2} ,\ldots , \vec{  v_k}  \right\} = \left\{ \sum\limits_{i=1}^{k} c_i \vec{ v_i} \mid c_i \in \mathbb{R}
	  \right\} $ is the set of all linear combinations of the vectors in the set.
 }
 \dfn{Linear Independence :}{
  The set $\left\{ \vec{ v_1} , \vec{ v_2} ,\ldots , \vec{  v_k}  \right\}$ is linearly independent  if $ \sum\limits_{i=1}^{k} c_i \vec{ v_i} = \vec{ 0} $ implies that $ c_i = 0$ for all $ i = 1,2,\ldots,k$.
 }

 \dfn{ Basis :}{
 A basis for $ V$ is a linearly independent set of vectors that spans $ V$. All bases of $ V$ have the same number of vectors, which is called the dimension of $ V$.
 }
   \dfn{Subspace :}{
   Suppose $ V$ is a vector space and $ U \subseteq V$. If $ U$ is a vector space then $ U$ is called a subspace of $ V$. 
   }
   
\section{Bases for Vector Spaces}
	
\ex{}{
The standard basis of $ \mathbb{R}^n$ is the set of vectors $ \left\{ \vec{ e_1} , \vec{ e_2} ,\ldots , \vec{  e_n}  \right\}$ where
\[
\vec{ e_1} = \begin{bmatrix}
1\\
0\\
\vdots\\
0\\
\end{bmatrix} \quad \vec{ e_2} =  \begin{bmatrix}
0\\
1\\
\vdots\\
0\\
\end{bmatrix} 
\ldots \vec{ e_j} = } 
.\] 
}

\ex{}{
\[
	V = \mathcal{P}_k \left[ x \right]  = \left\{ a_0 + a_1 x + a_2 x^2 +\ldots + a_k x^{k} \mid  a_i \in \mathbb{R}\right\} 
.\] 
$ \left\{ 1,x,x^2,\ldots, x^{k} \right\}$  clearly spans $ \mathcal{P}_k \left[ x \right] $ \\
\textit{Is the set linearly independent?}\\
Suppose $ c_0 + c_1 x + c_2 x^2 +\ldots + c_k \in \mathbb{R}$ such that 
\[
c_0 + c_1 x + c_2 x^2 +\ldots + c_k x^{k} = 0 + 0x + \ldots + 0x^{k}
.\] 
Compare coefficients we get
\[
c_0 = 0, c_1 = 0, c_2 = 0, \ldots, c_k = 0
.\] 
i.e. $ c_i =0 \forall i$ \\
Hence the set is linearly independent and so        $ \left\{ 1,x,x^2,\ldots, x^{k} \right\}$        forms a basis for $ \mathcal{P}_k \left[ x \right] $.
}
 \nt{
 The dimension of $ \mathcal{P}_k \left[ x \right] $ is $ k+1$ since the basis has $ k+1$ vectors.\\
 }
 \ex{}{
	 Which of the following sets forms a basis for $ \mathcal{P} _2 \left[ x \right] $?\\
	 \\
	 \begin{enumerate} [label=(\alph*)]
	 \item $ \left\{ 1,1+x,1+x+x^2 \right\} $\\
	 \item $ \left\{ x+x^2,1,2+2x+3x^2 \right\} $\\
	 \item $ \left\{ x,5x-x^2,2 \right\} $\\
	 \item $ \left\{ 1+x^2,x-3x^2,1+x-3x^2 \right\} $\\
	 \end{enumerate}
	 \textbf{Note:} dim $ \left( \mathcal{P}_2 \left[ x \right]  \right) =3$ so \textbf{(b)} or \textbf{(c)}  don't form a basis for $ \mathcal{P}_2 \left[  x\right] $.\\ 
	 \\
	 \textbf{(a)}  \textit{Linearly Independent?}\\
	 Suppose $ c_0 , c_1 , c_2 \in \mathbb{R}$ such that
	 \[
	 c_0\left( 1 \right) +c_1 \left( 1+x \right) + c_2 \left( 1+x+x^2 \right) = 0+ -x + 0x^2
	 .\] 
	 i.e. $ c_0+c_1+c_2 =0$,  $ c_1+c_2 =0$, $ c_2 = 0$\\
	 Using back substitution we get $ c_2 = 0$, $ c_1 = 0$, $ c_0 = 0$\\
	 $ \implies$ the set is linearly independent.\\
	 \textbf{Note:}  We could have also have considered the augmented matrix associated with the system  and observed that since there is a pivto in every column $ c_0 = c_1 = c_2 = 0$ 
	 
	 \[
	 \left[
	 \begin{array}{ccc;{2pt/2pt}c}  
	 1 & 1 & 1 & 0\\
	 0 & 1 & 1 & 0\\
	 0 & 0 & 1 & 0\\
	 \end{array}
	 \right]
	 .\] 
	 Does $  \left\{ 1, 1+x , 1+x+x^2 \right\}$  span $ \mathcal{P}_2 \left[ x \right] $?\\\\
	 Let $ a +bx +dx^2$ be an arbitrary vector in $ \mathcal{P}_2 \left[ x \right] $. Does there exist $ c_0,c_1,c_2 \in \mathbb{R}$ such that
	 \[
	 c_0\left( 1 \right) c_1 \left( 1+x \right) + c_2 \left( 1+x+x^2 \right) = a + bx + dx^2
	 .\] 
	 Comparing coefficients we get
	 \begin{lalign*}
	 	c_0 + c_1 + c_2 &= a\\
			 	c_1 + c_2 &= b\\
							 	c_2 &= d
	 .\end{lalign*}
	 i.e. does there exist a solution to the system:
	 \[
	 \left[
	 \begin{array}{ccc;{2pt/2pt}c}  
	 1 & 1 & 1 & a\\
	 0 & 1 & 1 & b\\
	 0 & 0 & 1& d\\
	 \end{array}
	 \right]
	 .\]         Yes, since there is a pivot in every row.\\
Hence the set forms a basis for $ \mathcal{P}_2 \left[ x \right] $ \\
\\
\textbf{(d)} \\
\[
	\left\{ 1+x^2, x -3x^2, 1+x-3x^2 \right\} 
.\] 
Linearly independent?\\
Suppose $ c_0, c_1, c_2 \in \mathbb{R}$ such that
\[
c_0 \left( 1+x^2 \right) + c_1 \left( x -3x^2 \right) + c_2 \left( 1+x-3x^2 \right) = 0 + 0x + 0x^2
.\] 
i.e. $ c_0+c_2 =0$, $ c_1+ c_2 =0$, $ c_0 -3c_1 -3c_2 =0$



   
\[
\text{Augmented matrix:}\;
\left[
  \begin{array}{ccc:c}
    1 & 0 & 1 & 0\\
    0 & 1 & 1 & 0\\
    1 & -3 & -3 & 0
  \end{array}
\right]
\;\xrightarrow{\;r_{3}-r_{1}\;}
\left[
  \begin{array}{ccc:c}
    1 & 0 & 1 & 0\\
    0 & 1 & 1 & 0\\
    0 & -3 & -4 & 0
  \end{array}
\right].
\]

 
 \[
\xrightarrow[\,r_{3}+3r_{2}\,]{}
\left[
  \begin{array}{ccc:c}
    1 & 0 & 1 & 0\\
    0 & 1 & 1 & 0\\
    0 & 0 & -1 & 0
  \end{array}
\right].
\]   
Pivot in every column $ \implies$ only solution is $ c_0 = c_1 = c_2 = 0$\\
$ \implies$ The set is linearly independent.\\
Does the set $ \left\{ 1+x^2, x-3x^2, 1+x-3x^2 \right\} $  span  $ \mathcal{P}_2 \left[ x \right] $?\\
i.e. given $ a +bx +dx^2 \in \mathcal{P}_2 \left[ x \right]$ does there exist $ c_0,c_1,c_2 \in \mathbb{R}$ such that
\[
c_0 \left( 1+x^2 \right) + c_1 \left( x -3x^2 \right) + c_2 \left( 1+x-3x^2 \right) = a + bx + dx^2
.\] 
i.e for fixed $ a,b,d \in \mathbb{R}$ does there exist a solution to the system
\raggedcolumns
\begin{multicols}{2}
\begin{align*}
	c_0+c_2 =a\\
	c_1+c_2 = b\\
	c_0 -3c_1 -3c_2 = d\\
.\end{align*}

\break
 \[
 \left[
 \begin{array}{ccc;{2pt/2pt}c}  
 1 & 0 & 1 & a\\
 0 & 1 & 1 & b\\
 1 & -3 & -3 & d\\
 \end{array}
 \right]
 .\] 

\end{multicols}
     We know that the echelon form of the augmented matrix has a pivot in every row, hence there is a solution to the system for all $ a,b,d \in \mathbb{R}$.\\
Hence the set $ \left\{ 1+x^2, x-3x^2, 1+x-3x^2 \right\} $ spans $ \mathcal{P}_2 \left[ x \right] $ and thus forms a basis for $ \mathcal{P}_2 \left[ x \right] $.\\
 }   
 \ex{}{
 \[
	 \text{ Let} U = \text{ span } \left\{ 1+x-2x^2+3x^3, 2+3x^3, 2+3x -4x^2 +6x^3, -7-4x+15x^2-18x^3, 5-4x-11x^2+12x^3 \right\} 
 .\] 
 \textit{Find a basis for $ U$ }\\
  The set of vectors clearly spans $ U$, we just need to check if the set is linearly independent, and if not we need to reduce the set to a linearly independent set.\\
  Suppose $ c_1,c_2,c_3,c_4 \in \mathbb{R} $ such that 
  \[
  c_1 \left( 1+x-2x^2+3x^3 \right) + c_2 \left( 2+3x^3 \right) + c_3 \left( 2+3x -4x^2 +6x^3 \right) + c_4 \left( -7-4x+15x^2-18x^3 \right) = 0
  .\] 
  Comparing coefficients we get the following system of equations:
  \begin{align*}
  	c_1 + 2c_2 -7c_3 +5 c_4 =0\\
	  	c_1 + 3c_3 -4 c_3-4c_4 =0\\
		  	-2c_1 -4c_2 + 15c_3 -11c_4 =0\\
			  	3c_1 + 3c_2 + 6c_3 + 18c_4 =0
  .\end{align*}
  \[
  \left[
  \begin{array}{cccc;{2pt/2pt}c}  
  1 & 2 & -7 & 5 & 0\\
  1 & 3 & -4 & -4 & 0\\
  -2 & -4 & 15 & -11 & 0\\
  3 & 6 & -18 & 12 & 0\\
  \end{array}
\right] \xrightarrow[r_1+ r_3\\ r_4-3r_1]{r_2-r_1} \left[
\begin{array}{cccc;{2pt/2pt}c}  
1 & 2 & -7 & 5 & 0\\
0 & 1 & 3 & -9 & 0\\
0 & 0 & 1 & -1 & 0\\
0 & 0 & 3 & -3 & 0\\
\end{array}
\right]     
  .\] 
  $ c_4$ is a free variable, this implies there are non-trivial solutions to the system.\\
  $ \implies$ The four polynomials are not linearly independent. But note that the first three polynomials are linearly independent, hence 
  \[
  \left\{ 1+x-2x^2+3x^3, 2+3x^3, 2+3x -4x^2 +6x^3 , -7-4x+15x^2- 18x^3 \right\}
  .\]  forms a basis for $ U$.\\
 }
   \ex{}{
Let $ U_1 = \left\{ a +bx^2 \mid  a ,b \in \mathbb{R} \right\} $.\\
\textit{Is $ U_1$ a subspace of $ \mathcal{P}_2 \left[ x \right] $?}\\
\underline{zero vector:} \\
$ 0 + 0 x^2 \in U_1$. Yes $ U_1$ contains the zero vector.\\
\\
\underline{closed under addition:}\\
Let $ a+bx^2$, $ c+ dx^2 \in U_1$ 
\[
a+bx^2 + c+ dx^2 = (a+c) + (b+d)x^2 \in U_1
.\] 
$ \implies U_1$ is closed under addition.\\
\underline{closed under scalar multiplication:}\\
Let $ a+bx^2 \in U_1$ and $ \lambda \in \mathbb{R}$, then
\[
\lambda \left( a+bx^2 \right) = \lambda a + \lambda b x^2 \in U_1
.\] 
$ \implies U_1$ is closed under scalar multiplication.\\\\
Hence $ U_1$ is a subspace of $ \mathcal{P}_2 \left[ x \right] $.\\
   }
 \ex{}{
    Let $ U_2 = \left\{ a +bx^2   \mid  a ,b \in \mathbb{R} , a \cdot b = 0 \right\} $.\\
    \textit{Is $ U_2$ a subspace of $ \mathcal{P}_2 \left[ x \right] $?}\\
    \underline{zero vector:} \\
    $ 0 + 0x^2 \in U_2$ since $ \left( 0 \right) \left( 0 \right) =0$. Yes $ U_2$ contains the zero vector.\\
    \underline{closed under addition:}\\
    Let $ a+bx^2$, $ c+ dx^2 \in U_2$ such that $ ab = 0$ and $ cd = 0$.\\
    \begin{align*}
    	a+bx^2 + c+ dx^2 &= (a+c) + (b+d)x^2\\
	    	&= \left( a+c \right) \left( b + d  \right) = ab + ad + bc + cd \\
			    	&= 0 + ad + bc + 0 = ad + bc\\
    .\end{align*}
     Which is not necessarily zero.\\
     Counter example: $ 5x^2$, $ 7 \in U_2$
     but $ 7+ 5x^2  \notin U_2 $ \\
     Hence $ U_2$ is not closed under addition.\\
     $ \implies U_2$ is not a subspace of $ \mathcal{P}_2 \left[ x \right] $.\\
 }
  \ex{}{
  \textit{Determine if the following sets are bases for $ \mathcal{P}_2 \left[ x \right] $:}\\
  \[
	  A = \left\{ 1, 1-x, 2-4x +x^2, 6-18x +9x^2 =x^3 \right\} 
  .\] 
  Note  that $ \dim \left( \mathcal{P}_3 \left[ x \right] \right) = 4$  $ \implies$ if $ A$ is linearly independent then it is a basis for $ \mathcal{P}_3 \left[ x \right] $.\\
  Suppose $ c_1,c_2,c_3,c_4 \in \mathbb{R}$ such that
  \[
  c_1 \left( 1 \right) + c_2 \left( 1-x \right) + c_3 \left( 2-4x +x^2 \right) + c_4 \left( 6-18x +9x^2 =x^3 \right) = 0
  .\] 
  \[
   \text{ i.e. } c_1 + c_2 + 2c_3 + 6c_4 + x \left( -c_2 -4c_3 -18 c_4 \right) + x^2 \left( c_3 + 9 c_4 \right) + x^3 \left(  - c_4 \right) = 0 + 0x + 0x^2 + 0x^3
  .\] 
  Equating coefficients we get the following system of equations:
  \begin{align*}
  	   c_1 + c_2 + 2c_3 + 6c_4 &= 0\\
   -c_2 -4c_3 -18 c_4 &= 0\\
   c_3 + 9 c_4 &= 0\\
   -c_4 &= 0
  .\end{align*}
  to determine if the set is linearly independent we need to determine if the following system has only free variables:
  \[
  \left[
  \begin{array}{cccc;{2pt/2pt}c}  
  1 & 1 & 2 & 6 & 0\\
  0 & -1 & -4 & -18 & 0\\
  0 & 0 & 1 & 9 & 0\\
  0 & 0 & 0 & -1 & 0\\
  \end{array}
  \right]
  .\]               Pivot in every column $ \implies$ $ A$ is linearly independent and hence $ A$ is a basis for $ \mathcal{P}_3 \left[ x \right] $.\\
  }


\ex{}{
Which of the following sets is a basis for $ M_{2\times  2}$ 
\begin{enumerate} [label=(\alph*)]
\item 
	\[
		\left\{ \begin{bmatrix}
		5 & 0\\
		0 & 0\\
		\end{bmatrix} , \begin{bmatrix}
		-2 & 7\\
		0 & 0\\
		\end{bmatrix}, \begin{bmatrix}
		6 & 3\\
		-1 & 0\\
		\end{bmatrix} \right\}
	.\] 
\item 
	\[
		\left\{ \begin{bmatrix}
		1 & 1\\
		1 & 1\\
		\end{bmatrix}, \begin{bmatrix}
		0 & 1\\
		1 & 1\\
		\end{bmatrix}, \begin{bmatrix}
		0 & 0 \\
		1 & 1\\
		\end{bmatrix}, \begin{bmatrix}
		0 & 0\\
		0 & 1\\
		\end{bmatrix} \right\} 
	.\] 
\item 
	\[
		\left\{ \begin{bmatrix}
		1 & 1\\
		1 & 1\\
		\end{bmatrix}, \begin{bmatrix}
		2 & -3\\
		0 & 4\\
		\end{bmatrix}\right\}
	.\] 
	\item 
		\[
			\left\{ \begin{bmatrix}
			2 & 0\\
			0 & -2\\
			\end{bmatrix} , \begin{bmatrix}
			0 & 3\\
			3 & 0\\
			\end{bmatrix}, \begin{bmatrix}
			4 & 0\\
			0 & 0\\
			\end{bmatrix}, \begin{bmatrix}
			2 & 9\\
			9 & -10\\
			\end{bmatrix}\right\} 
		.\] 
\end{enumerate}
  \textbf{Solution:}\\
  Note that $ \dim \left( M_{2\times 2} \right) = 4$ \\
  $ \implies$ all bases for $ M_{2\times 2}$ have 4 vectors.\\
  This eliminates \textbf{(a)}, and \textbf{(c)}.\\
  \textbf{(b)}: since dim $ \left( M_{2\times 2} \right) = 4$ all we need to show is that the set is linearly independent.\\
  Suppose $ c_1,c_2,c_3,c_4 \in \mathbb{R}$ such that
  \[
  c_1 \begin{bmatrix}
  1 & 1\\
  1 & 1\\
  \end{bmatrix} + c_2 \begin{bmatrix}
  0 & 1\\
  1 & 1\\
  \end{bmatrix} + c_3 \begin{bmatrix}
  0 & 0\\
  1 & 1\\
  \end{bmatrix} + c_4 \begin{bmatrix}
  0 & 0\\
  0 & 1\\
  \end{bmatrix} = \begin{bmatrix}
  0 & 0\\
  0 & 0\\
  \end{bmatrix}
  .\] 
   \[
   \text{ i.e. } \begin{bmatrix}
   c_1 & c_1+c_2\\
   c_1+c_2 +c_3 & c_1 +c_2 +c_3 +c_4\\
   \end{bmatrix}= \begin{bmatrix}
   0 & 0\\
   0 & 0\\
   \end{bmatrix}
   .\] 
   equating entries:
   \begin{align*}
   	    c_1 &= 0\\
    c_1 + c_2 &= 0\\
    c_1 + c_2 + c_3 &= 0\\
    c_1 +c_2 +c_3 +c_4 &= 0
   .\end{align*}
   By observation (or looking at the augmented matrix) we get that the only solution is $ c_1 = c_2 = c_3 = c_4 = 0$ i.e. the set is linearly independent.\\
   $ \implies$ \textbf{(b)} is a basis for $ M_{2\times 2}$.\\
  \textbf{(d)}:  Similarly we check that the set is linearly independent $ c_1$, $ c_2$, $ c_3$, $ c_4 \in \mathbb{R}$ such that
  \[
  c_1 \begin{bmatrix}
  2 & 0\\
  0 & -2\\
  \end{bmatrix} + c_2 \begin{bmatrix}
  0 & 3\\
  3 & 0\\
  \end{bmatrix}+ c_3 \begin{bmatrix}
  4 & 0\\
  0 & 0\\
  \end{bmatrix} + c_4 \begin{bmatrix}
  4 & 0\\
  0 & 0\\
  \end{bmatrix} = \begin{bmatrix}
  0 & 0\\
  0 & 0\\
  \end{bmatrix}
  .\] 
  After similar calculations we get:\\
  \[
  \left[
  \begin{array}{cccc;{2pt/2pt}c}  
  2 & 0 & 4 & 2 & 0\\
  0 & 3 & 0 & 9 & 0\\
  0 & 0 & 0 & 0 & 0\\
  0 & 0 & 0 & -8 & 0\\
  \end{array}
  \right]
  .\] 
  Only 3 pivots, one free variable \\
  $ \implies$ the set is linearly dependent and does not form a basis for $ M_{2\times 2}$.\\
}
\dfn{Trace of a Matrix :}{
Suppose $ A$ is an $n \times n$  matrix with $ (i,j) $ entry $ a_{ij}$. The trace of $ A$ is defined as
\[
a_{11} + a_{22} + \ldots + a_{nn} = \sum\limits_{i=1}^{n} a_{ii}
.\] 
}

 \ex{}{
	 Let $ U = \left\{ A \in M_{n \times  n } \mid \text{trace } A =0\right\} $.\\
	 \textit{Is $ U$ a subspace of $ M_{n \times n}$?}\\ 
	 Examples of $ 2\times 2$ matrices with trace $ 0$ are
	 \[
	 \begin{bmatrix}
	 3 & 0\\
	 7 & -3\\
	 \end{bmatrix} \qquad \begin{bmatrix}
	 a & b\\
	 c & -a\\
	 \end{bmatrix} \qquad a,b,c \in \mathbb{R}
	 .\] 
	 \underline{zero vector:}\\
	 \[
	 \begin{bmatrix}
	 0 & 0 & 0 & \dots  & 0 \\
	 0 & 0 & 0 & \dots  & 0 \\
	 \vdots & \vdots & \vdots & \ddots & \vdots \\
	 0 & 0 & 0 & \dots  & 0\end{bmatrix} \in U \text{ since }  \left( 0 \right) \left( n \right) =0
	 .\] 
	 \underline{closed under addition:}\\
	 Let $ A$, $ B \in U$ \\
	 Suppose $ (i,j) $ entry of $ A$ is $ a_{ij}$ and the $(i,j)$ entry of $ B$ is $ b_{ij}$.\\
	\[
	\sum\limits_{i=1}^{n} a_{ii}=0 \qquad \sum\limits_{i=1}^{n} b_{ii} = 0
	.\] 
	The $ (i,j) $ entry of $ A+B$ is $ a_{ij} + b_{ij}$, hence
	\[
	\text{ trace } \left( A+B \right) = \sum\limits_{i=1}^{n} \left( a_{ii} + b_{ii} \right) = \sum\limits_{i=1}^{n} a_{ii} + \sum\limits_{i=1}^{n} b_{ii} = 0+0=0
	.\] 
	$ \implies U$ is closed under addition.\\
	\underline{ closed under scalar multiplication:}\\
	 Let $ A \in U$ and  $ \implies \text{ trace } \left( A \right) =0 = \sum\limits_{i=1}^{n} a_{ii}$.\\
	 Let $ \lambda \in \mathbb{R}$, then the $(i,j)$ entry of $ \lambda A$ is $ \lambda a_{ij}$.\\
	 \[
	 \implies \text{ trace } \left( \lambda A \right) = \sum\limits_{i=1}^{n} \lambda a_{ii} = \lambda \sum\limits_{i=1}^{n} a_{ii} = \lambda \cdot 0 = 0
	 .\] 
	 $ \implies U$ is closed under scalar multiplication.\\
	 Hence $ U$ is a subspace of $ M_{n \times n}$.\\
 }


 \section{Linear Transformations}
 	
 \dfn{Linear Transformation :}{
 A linear transformation $ T$ from a vector space $ V$ to a vector space $ W$, $ T: V \to W$ is a map that assigns to each vector $ \vec{ v} \in V$ a unique vector $ T \left( \vec{ x} \right) \in W$ such that $ T$ preserves vector addition and scalar multiplication, i.e. 
 \begin{enumerate}[label=(\roman*)]
 \item $ T \left( \vec{ v_1} + \vec{ v_2} \right) = T \left( \vec{ v_1} \right) + T \left( \vec{ v_2} \right)$ for all $ \vec{ v_1},\vec{ v_2} \in V$.
 \item      $ T \left( \lambda \vec{ v}  \right) = \lambda T \left( \vec{ v}  \right)  \forall  \lambda \in \mathbb{R}, \vec{ v}  \in V$\\
	 \begin{tikzpicture}[>=stealth, every node/.style={inner sep=1pt}, scale=1.1]

  %--- the two vector-space “blobs” --------------------------
  \draw (-2,0) ellipse (1.6 and 3);          % V
  \node at (-2,  3.3) {$V$};

  \draw ( 4,0) ellipse (1.6 and 3);          % W
  \node at ( 4,  3.3) {$W$};

  %--- the label for the transformation ---------------------
  \draw[->] (-0.5, 2.9)  
            .. controls ( 1, 3.4) and ( 2.5, 3.4) .. 
            ( 3.5, 2.9)
            node[midway, above] {$T$};

  %--- points in V ------------------------------------------
  \node (v1)  at (-2,  2) {$\vec v_{1}$};
  \node (v2)  at (-2,  1) {$\vec v_{2}$};
  \node (v12) at (-2,  0) {$\vec v_{1}+\vec v_{2}$};
  \node (v)   at (-2, -1) {$\vec v$};
  \node (lv)  at (-2, -2) {$\lambda\vec v$};

  %--- images in W ------------------------------------------
  \node (Tv1) at ( 4,  2) {$T(\vec v_{1})$};
  \node (Tv2) at ( 4,  1) {$T(\vec v_{2})$};
  \node (Tsum)at ( 5.5,  0) {$T(\vec v_{1})+T(\vec v_{2})\;\neq\;T(\vec v_{1}+\vec v_{2})$};
  \node (Tv)  at ( 4, -1) {$T(\vec v)$};
  \node (lTv) at ( 4, -2) {$\lambda\,T(\vec v)=T(\lambda\vec v)$};

  %--- arrows between corresponding points ------------------
  \foreach \from/\to in {v1/Tv1, v2/Tv2, v12/Tsum, v/Tv, lv/lTv}
      \draw[->] (\from) -- (\to);

\end{tikzpicture}

     
 \end{enumerate}
}

   \mlem{}{
   Suppose $ T: V \to W$ is a linear transformation. Let $ \vec{ 0}_v $ and $ \vec{ 0}_w$ be the zero vectors in $ V$ and $ W$ respectively. Then
   \begin{enumerate}[label=(\roman*)]
   \item $ T \left( \vec{ 0}_v \right) = \vec{ 0}_w$
   \item      $ T \left( - \vec{ v}  \right) = - T \left( \vec{ v}  \right)  \forall  \vec{ v}  \in V$
   \end{enumerate}
  }
    \pf{Proof:}{
    (i)\\
          We note that $ 0 _{  \vec{ v} } = \vec{ 0 _v} \forall \vec{ v}  \in V $ also $ 0 _{  \vec{ w} } = \vec{ 0 _w} \forall \vec{ w}  \in W $.\\
	  $ T \left( \vec{ 0}_{  v } \right) = T \left( 0_{ \vec{ v} } \right) $  for some $ \vec{ v}  \in V$\\
	  $ = 0 T \left( \vec{ v}  \right) $ by the property of linear transformations\\
	  $ = 0 \vec{ 0}_w$ since $ T \left( \vec{ v}  \right) \in W$\\
	  \\
	  (ii)\\
	  Let $ \vec{ v} \in V$ 
	  
   	  $ T \left( - \vec{ v}  \right) = \left( -1 \right) T \left( \vec{ v}  \right) $ by the property of linear transformations\\
	  $ = - T \left( \vec{ v}  \right) $ since $ T \left( \vec{ v}  \right) \in W$\\
 }
   \ex{}{
	   Suppose $ T : \mathcal{P}_k \left[ x \right]  \to \mathcal{P} _{k-1}\left[x \right] $ is defined  as $ T\left( p \left( x \right)  \right) = p ^{1}\left( x\right) $. Show that $ T$ is a linear transformation.\\
	   \underline{Preserves vector addition:}\\
	   Suppose $ p\left( x \right) $, $ q \left( x \right)  \in \mathcal{P}_k \left[ x \right] $.
	   \begin{align*}
	   		    T \left( p \left( x \right) + q \left( x \right)  \right) &=  \frac{d}{dx} \left(  p \left( x \right) + q \left( x \right)  \right) \\
			    	   		    &= \frac{d}{dx} \left( p \left( x \right)  \right) + \frac{d}{dx} \left( q \left( x \right)  \right)\\
						    &= p' \left( x \right) + q' \left( x \right)\\
						    			    	   		    &= T \left( p \left( x \right)  \right) + T \left( q \left( x \right)  \right)
	   .\end{align*}
	   $ \implies T$  preserves vector addition.\\
	   \underline{Preserves scalar multiplication:}\\
	   Suppose $ p\left[ x \right]  \in \mathcal{P}_k \left[ x \right]$, $ \lambda \in \mathbb{R}$ \\
	   \begin{align*}
	   	        T \left( \lambda p\left( x \right)  \right) = \frac{d}{dx} \left( \lambda p \left( x \right)  \right) = \lambda  \frac{ d   }{ dx }\left( p \left( x \right)  \right) \\
			&= \lambda T \left( p \left( x \right)  \right) 
	   .\end{align*}
	   $ \implies T $ preserves scalar multiplication.\\
	   Hence $ T$ is a linear transformation.
   }  
   \ex{}{
   Show that the map 
\[
	T: \mathcal{P}_2 \left[ x \right] \to \mathcal{P}_3   \left[ x \right]
.\] 
\[
T \left( p \left( x \right)  \right) = x \left( p \left( x \right)  \right) 
.\]                  is a linear transformation
\begin{enumerate}[label=(\roman*)]
\item Let $ p\left( x \right) $, $ q \left( x \right)  \in \mathcal{P}_2 \left[ x \right]$
 \[
 T \left( p \left( x \right) + q \left( x \right)  \right) = x \left( p \left( x \right) + q \left( x \right)  \right) = x \left( p \left( x \right)  \right) + x \left( q \left( x \right)  \right) 
 .\]        
 \[
 = T \left(  p \left( x \right)  \right) + T \left( q \left( x \right)  \right) 
 .\] 
\item                     Let $ p \left( x \right)  \in \mathcal{P}_2 \left[  x\right] $ and $ \lambda \in \mathbb{R}$
 \begin{align*}
 	T \left( \lambda p\left( x \right)  \right) = x \left( \lambda p \left( x \right)  \right) = \lambda \left( x p\left( x \right)  \right) \\
	\lambda T \left( p \left( x \right)  \right) 
 .\end{align*}  
 $ \implies T$ is a linear transformation.
\end{enumerate}
   }
   
   \ex{}{
	   Let $ \alpha : \mathcal{P} \left[ x \right] \to \mathcal{P} \left[ x \right] $ be defined as $ \alpha \left(  p \left( x \right)  \right) = 3 p \left( x \right) p ' \left( x \right)$
    \textit{Is $ \alpha$ a linear transformation?}\\
    Let $ p \left( x \right) , q \left( x \right)  \in \mathcal{P} \left[ x \right]$
    \[
    \alpha \left( p \left( x \right) + q \left( x \right)  \right) = 3 \left( p \left( x \right) + q \left( x \right)  \right) \left( p \left( x \right) + q \left( x \right)  \right) '
    .\] 
    and $ 3 \left(  p\left( x \right) + q \left( x \right)  \right) \left( p \left( x \right) + q \left( x \right)  \right) '$ is not, in general equal to $ 3 p \left( x \right) p' \left( x \right) + 3 q \left( x \right) q' \left( x \right)  $
    $ \implies \alpha$ is not a linear transformation\\
    \\
    Counterexample:\\
    Consider $ p\left( x \right) = x+5$, $ q \left( x \right) = x^2 -x +2$
    \begin{align*}
    	\alpha \left( p\left( x \right) + q\left( x \right)  \right) = \alpha \left(  x^2 +7 \right) = 3 \left( x^2+7 \right) \left( 2x \right) \\
	&= \alpha \left(  p \left( x \right)  \right) = 3 \left( x+5 \right) \\
	&=  \alpha \left(  q \left(  x \right)  \right) = 3 \left( x^2 - x + 2 \right)                \left( 2x-1 \right) 
    .\end{align*}
    and, $ 3 \left( x^2+7 \right) \left( 2x \right)  \neq 3 \left( x+5 \right) + 3 \left( x^2-x +2 \right)  \left( 2x-1 \right) $
   }
   \ex{}{
   Suppose $ A$ is the following $ 3 \times 2$ matrix 
   \[
   A = \begin{bmatrix}
 1& -3 \\\
3 & 5 \\\
-1& 7 \\\
\end{bmatrix}        
   .\]               We define a mapping $ T: \mathbb{R} ^2 \to \mathbb{R} ^2 $ by 
   \[
   T \left(  \vec{ x}  \right) = A \vec{ x} \forall  \vec{ x} \in \mathbb{R} ^2 
   .\] 
   $ T$ is a linear transformation - \textit{why?}\\
   \\
   Let $ \vec{ u} $, $ \vec{ v} \in \mathbb{R} ^2$, $ \lambda \in \mathbb{R}$ \\
   \\
   \begin{align*}
   	T \left(  \vec{ u} + \vec{ v}  \right) = A \left( \vec{ u} + \vec{ v}  \right) = A \vec{ u} + A \vec{ v} \\
	&= T \left( \vec{ u}  \right) + T \left( \vec{ v}  \right) 
   .\end{align*}
   $ \implies T $ preserves addition \\
   \\
   \[
   T \left(  \lambda \vec{ u }  \right) = A \left(  \lambda \vec{ u }  \right) = \lambda \left( A \vec{ u}  \right) = \lambda T \left(  \vec{ u}  \right)  
   .\] 
   $ \implies T$ preserves scalar multiplication\\
   Hence $ T$ is a linear transformation.
   }
 \nt{
 There was nothing special about the matrix $ A$ in the previous example that influenced the fact that $ A$ determined a linear transformation from $ \mathbb{R} ^2 \to \mathbb{R} ^3$. All $ 3 \times 2$ matrices determine linear transformations from $ \mathbb{R} ^2 \to \mathbb{R} ^3$ defined by $ T: \mathbb{R} ^2 \to \mathbb{R} ^3$, $ T \left( \vec{ x}  \right) = A \vec{ x} $ where $ A$ is any $ 3 \times 2$ matrix.\\
 \\
 In general, if $ A \in M _{m \times  n} \left(  \mathbb{R} \right) $  then the mapping $ T: \mathbb{R} ^{n} \to \mathbb{R} ^{ m}$ defined by $ T \left( \vec{ x}  \right) = A \vec{ x} $, $ \forall  \vec{ x} \in \mathbb{R} ^{ n}$ is a linear transformation.\\
 }
   
     
 












\end{document}                                          
