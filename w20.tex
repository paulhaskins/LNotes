\documentclass{report}

%%%%%%%%%%%%%%%%%%%%%%%%%%%%%%%%%
% PACKAGE IMPORTS
%%%%%%%%%%%%%%%%%%%%%%%%%%%%%%%%%


\usepackage[tmargin=2cm,rmargin=1in,lmargin=1in,margin=0.85in,bmargin=2cm,footskip=.2in]{geometry}
\usepackage{amsmath,amsfonts,amsthm,amssymb,mathtools}
\usepackage[varbb]{newpxmath}
\usepackage{xfrac}
\usepackage[makeroom]{cancel}
\usepackage{bookmark}
\usepackage{enumitem}
\usepackage{hyperref,theoremref}
\hypersetup{
	pdftitle={Assignment},
	colorlinks=true, linkcolor=doc!90,
	bookmarksnumbered=true,
	bookmarksopen=true
}
\usepackage[most,many,breakable]{tcolorbox}
\usepackage{xcolor}
\usepackage{varwidth}
\usepackage{varwidth}
\usepackage{tocloft}
\usepackage{etoolbox}
\usepackage{derivative} %many derivativess partials
%\usepackage{authblk}
\usepackage{nameref}
\usepackage{multicol,array}
\usepackage{tikz-cd}
\usepackage[ruled,vlined,linesnumbered]{algorithm2e}
\usepackage{comment} % enables the use of multi-line comments (\ifx \fi) 
\usepackage{import}
\usepackage{xifthen}
\usepackage{pdfpages}
\usepackage{transparent}
\usepackage{verbatim}

\newcommand\mycommfont[1]{\footnotesize\ttfamily\textcolor{blue}{#1}}
\SetCommentSty{mycommfont}
\newcommand{\incfig}[1]{%
    \def\svgwidth{\columnwidth}
    \import{./figures/}{#1.pdf_tex}
}
\usepackage[tagged, highstructure]{accessibility}
\usepackage{tikzsymbols}
\renewcommand\qedsymbol{$\Laughey$}


%\usepackage{import}
%\usepackage{xifthen}
%\usepackage{pdfpages}
%\usepackage{transparent}


%%%%%%%%%%%%%%%%%%%%%%%%%%%%%%
% SELF MADE COLORS
%%%%%%%%%%%%%%%%%%%%%%%%%%%%%%



\definecolor{myg}{RGB}{56, 140, 70}
\definecolor{myb}{RGB}{45, 111, 177}
\definecolor{myr}{RGB}{199, 68, 64}
\definecolor{mytheorembg}{HTML}{F2F2F9}
\definecolor{mytheoremfr}{HTML}{00007B}
\definecolor{mylenmabg}{HTML}{FFFAF8}
\definecolor{mylenmafr}{HTML}{983b0f}
\definecolor{mypropbg}{HTML}{f2fbfc}
\definecolor{mypropfr}{HTML}{191971}
\definecolor{myexamplebg}{HTML}{F2FBF8}
\definecolor{myexamplefr}{HTML}{88D6D1}
\definecolor{myexampleti}{HTML}{2A7F7F}
\definecolor{mydefinitbg}{HTML}{E5E5FF}
\definecolor{mydefinitfr}{HTML}{3F3FA3}
\definecolor{notesgreen}{RGB}{0,162,0}
\definecolor{myp}{RGB}{197, 92, 212}
\definecolor{mygr}{HTML}{2C3338}
\definecolor{myred}{RGB}{127,0,0}
\definecolor{myyellow}{RGB}{169,121,69}
\definecolor{myexercisebg}{HTML}{F2FBF8}
\definecolor{myexercisefg}{HTML}{88D6D1}


%%%%%%%%%%%%%%%%%%%%%%%%%%%%
% TCOLORBOX SETUPS
%%%%%%%%%%%%%%%%%%%%%%%%%%%%

\setlength{\parindent}{1cm}
%================================
% THEOREM BOX
%================================

\tcbuselibrary{theorems,skins,hooks}
\newtcbtheorem[number within=section]{Theorem}{Theorem}
{%
	enhanced,
	breakable,
	colback = mytheorembg,
	frame hidden,
	boxrule = 0sp,
	borderline west = {2pt}{0pt}{mytheoremfr},
	sharp corners,
	detach title,
	before upper = \tcbtitle\par\smallskip,
	coltitle = mytheoremfr,
	fonttitle = \bfseries\sffamily,
	description font = \mdseries,
	separator sign none,
	segmentation style={solid, mytheoremfr},
}
{th}

\tcbuselibrary{theorems,skins,hooks}
\newtcbtheorem[number within=chapter]{theorem}{Theorem}
{%
	enhanced,
	breakable,
	colback = mytheorembg,
	frame hidden,
	boxrule = 0sp,
	borderline west = {2pt}{0pt}{mytheoremfr},
	sharp corners,
	detach title,
	before upper = \tcbtitle\par\smallskip,
	coltitle = mytheoremfr,
	fonttitle = \bfseries\sffamily,
	description font = \mdseries,
	separator sign none,
	segmentation style={solid, mytheoremfr},
}
{th}


\tcbuselibrary{theorems,skins,hooks}
\newtcolorbox{Theoremcon}
{%
	enhanced
	,breakable
	,colback = mytheorembg
	,frame hidden
	,boxrule = 0sp
	,borderline west = {2pt}{0pt}{mytheoremfr}
	,sharp corners
	,description font = \mdseries
	,separator sign none
}

%================================
% Corollery
%================================
\tcbuselibrary{theorems,skins,hooks}
\newtcbtheorem[number within=section]{Corollary}{Corollary}
{%
	enhanced
	,breakable
	,colback = myp!10
	,frame hidden
	,boxrule = 0sp
	,borderline west = {2pt}{0pt}{myp!85!black}
	,sharp corners
	,detach title
	,before upper = \tcbtitle\par\smallskip
	,coltitle = myp!85!black
	,fonttitle = \bfseries\sffamily
	,description font = \mdseries
	,separator sign none
	,segmentation style={solid, myp!85!black}
}
{th}
\tcbuselibrary{theorems,skins,hooks}
\newtcbtheorem[number within=chapter]{corollary}{Corollary}
{%
	enhanced
	,breakable
	,colback = myp!10
	,frame hidden
	,boxrule = 0sp
	,borderline west = {2pt}{0pt}{myp!85!black}
	,sharp corners
	,detach title
	,before upper = \tcbtitle\par\smallskip
	,coltitle = myp!85!black
	,fonttitle = \bfseries\sffamily
	,description font = \mdseries
	,separator sign none
	,segmentation style={solid, myp!85!black}
}
{th}


%================================
% LENMA
%================================

\tcbuselibrary{theorems,skins,hooks}
\newtcbtheorem[number within=section]{Lenma}{Lenma}
{%
	enhanced,
	breakable,
	colback = mylenmabg,
	frame hidden,
	boxrule = 0sp,
	borderline west = {2pt}{0pt}{mylenmafr},
	sharp corners,
	detach title,
	before upper = \tcbtitle\par\smallskip,
	coltitle = mylenmafr,
	fonttitle = \bfseries\sffamily,
	description font = \mdseries,
	separator sign none,
	segmentation style={solid, mylenmafr},
}
{th}

\tcbuselibrary{theorems,skins,hooks}
\newtcbtheorem[number within=chapter]{lenma}{Lenma}
{%
	enhanced,
	breakable,
	colback = mylenmabg,
	frame hidden,
	boxrule = 0sp,
	borderline west = {2pt}{0pt}{mylenmafr},
	sharp corners,
	detach title,
	before upper = \tcbtitle\par\smallskip,
	coltitle = mylenmafr,
	fonttitle = \bfseries\sffamily,
	description font = \mdseries,
	separator sign none,
	segmentation style={solid, mylenmafr},
}
{th}


%================================
% PROPOSITION
%================================

\tcbuselibrary{theorems,skins,hooks}
\newtcbtheorem[number within=section]{Prop}{Proposition}
{%
	enhanced,
	breakable,
	colback = mypropbg,
	frame hidden,
	boxrule = 0sp,
	borderline west = {2pt}{0pt}{mypropfr},
	sharp corners,
	detach title,
	before upper = \tcbtitle\par\smallskip,
	coltitle = mypropfr,
	fonttitle = \bfseries\sffamily,
	description font = \mdseries,
	separator sign none,
	segmentation style={solid, mypropfr},
}
{th}

\tcbuselibrary{theorems,skins,hooks}
\newtcbtheorem[number within=chapter]{prop}{Proposition}
{%
	enhanced,
	breakable,
	colback = mypropbg,
	frame hidden,
	boxrule = 0sp,
	borderline west = {2pt}{0pt}{mypropfr},
	sharp corners,
	detach title,
	before upper = \tcbtitle\par\smallskip,
	coltitle = mypropfr,
	fonttitle = \bfseries\sffamily,
	description font = \mdseries,
	separator sign none,
	segmentation style={solid, mypropfr},
}
{th}


%================================
% CLAIM
%================================

\tcbuselibrary{theorems,skins,hooks}
\newtcbtheorem[number within=section]{claim}{Claim}
{%
	enhanced
	,breakable
	,colback = myg!10
	,frame hidden
	,boxrule = 0sp
	,borderline west = {2pt}{0pt}{myg}
	,sharp corners
	,detach title
	,before upper = \tcbtitle\par\smallskip
	,coltitle = myg!85!black
	,fonttitle = \bfseries\sffamily
	,description font = \mdseries
	,separator sign none
	,segmentation style={solid, myg!85!black}
}
{th}



%================================
% Exercise
%================================

\tcbuselibrary{theorems,skins,hooks}
\newtcbtheorem[number within=section]{Exercise}{Exercise}
{%
	enhanced,
	breakable,
	colback = myexercisebg,
	frame hidden,
	boxrule = 0sp,
	borderline west = {2pt}{0pt}{myexercisefg},
	sharp corners,
	detach title,
	before upper = \tcbtitle\par\smallskip,
	coltitle = myexercisefg,
	fonttitle = \bfseries\sffamily,
	description font = \mdseries,
	separator sign none,
	segmentation style={solid, myexercisefg},
}
{th}

\tcbuselibrary{theorems,skins,hooks}
\newtcbtheorem[number within=chapter]{exercise}{Exercise}
{%
	enhanced,
	breakable,
	colback = myexercisebg,
	frame hidden,
	boxrule = 0sp,
	borderline west = {2pt}{0pt}{myexercisefg},
	sharp corners,
	detach title,
	before upper = \tcbtitle\par\smallskip,
	coltitle = myexercisefg,
	fonttitle = \bfseries\sffamily,
	description font = \mdseries,
	separator sign none,
	segmentation style={solid, myexercisefg},
}
{th}

%================================
% EXAMPLE BOX
%================================

\newtcbtheorem[number within=section]{Example}{Example}
{%
	colback = myexamplebg
	,breakable
	,colframe = myexamplefr
	,coltitle = myexampleti
	,boxrule = 1pt
	,sharp corners
	,detach title
	,before upper=\tcbtitle\par\smallskip
	,fonttitle = \bfseries
	,description font = \mdseries
	,separator sign none
	,description delimiters parenthesis
}
{ex}

\newtcbtheorem[number within=chapter]{example}{Example}
{%
	colback = myexamplebg
	,breakable
	,colframe = myexamplefr
	,coltitle = myexampleti
	,boxrule = 1pt
	,sharp corners
	,detach title
	,before upper=\tcbtitle\par\smallskip
	,fonttitle = \bfseries
	,description font = \mdseries
	,separator sign none
	,description delimiters parenthesis
}
{ex}

%================================
% DEFINITION BOX
%================================

\newtcbtheorem[number within=section]{Definition}{Definition}{enhanced,
	before skip=2mm,after skip=2mm, colback=red!5,colframe=red!80!black,boxrule=0.5mm,
	attach boxed title to top left={xshift=1cm,yshift*=1mm-\tcboxedtitleheight}, varwidth boxed title*=-3cm,
	boxed title style={frame code={
					\path[fill=tcbcolback]
					([yshift=-1mm,xshift=-1mm]frame.north west)
					arc[start angle=0,end angle=180,radius=1mm]
					([yshift=-1mm,xshift=1mm]frame.north east)
					arc[start angle=180,end angle=0,radius=1mm];
					\path[left color=tcbcolback!60!black,right color=tcbcolback!60!black,
						middle color=tcbcolback!80!black]
					([xshift=-2mm]frame.north west) -- ([xshift=2mm]frame.north east)
					[rounded corners=1mm]-- ([xshift=1mm,yshift=-1mm]frame.north east)
					-- (frame.south east) -- (frame.south west)
					-- ([xshift=-1mm,yshift=-1mm]frame.north west)
					[sharp corners]-- cycle;
				},interior engine=empty,
		},
	fonttitle=\bfseries,
	title={#2},#1}{def}
\newtcbtheorem[number within=chapter]{definition}{Definition}{enhanced,
	before skip=2mm,after skip=2mm, colback=red!5,colframe=red!80!black,boxrule=0.5mm,
	attach boxed title to top left={xshift=1cm,yshift*=1mm-\tcboxedtitleheight}, varwidth boxed title*=-3cm,
	boxed title style={frame code={
					\path[fill=tcbcolback]
					([yshift=-1mm,xshift=-1mm]frame.north west)
					arc[start angle=0,end angle=180,radius=1mm]
					([yshift=-1mm,xshift=1mm]frame.north east)
					arc[start angle=180,end angle=0,radius=1mm];
					\path[left color=tcbcolback!60!black,right color=tcbcolback!60!black,
						middle color=tcbcolback!80!black]
					([xshift=-2mm]frame.north west) -- ([xshift=2mm]frame.north east)
					[rounded corners=1mm]-- ([xshift=1mm,yshift=-1mm]frame.north east)
					-- (frame.south east) -- (frame.south west)
					-- ([xshift=-1mm,yshift=-1mm]frame.north west)
					[sharp corners]-- cycle;
				},interior engine=empty,
		},
	fonttitle=\bfseries,
	title={#2},#1}{def}



%================================
% Solution BOX
%================================

\makeatletter
\newtcbtheorem{question}{Question}{enhanced,
	breakable,
	colback=white,
	colframe=myb!80!black,
	attach boxed title to top left={yshift*=-\tcboxedtitleheight},
	fonttitle=\bfseries,
	title={#2},
	boxed title size=title,
	boxed title style={%
			sharp corners,
			rounded corners=northwest,
			colback=tcbcolframe,
			boxrule=0pt,
		},
	underlay boxed title={%
			\path[fill=tcbcolframe] (title.south west)--(title.south east)
			to[out=0, in=180] ([xshift=5mm]title.east)--
			(title.center-|frame.east)
			[rounded corners=\kvtcb@arc] |-
			(frame.north) -| cycle;
		},
	#1
}{def}
\makeatother

%================================
% SOLUTION BOX
%================================

\makeatletter
\newtcolorbox{solution}{enhanced,
	breakable,
	colback=white,
	colframe=myg!80!black,
	attach boxed title to top left={yshift*=-\tcboxedtitleheight},
	title=Solution,
	boxed title size=title,
	boxed title style={%
			sharp corners,
			rounded corners=northwest,
			colback=tcbcolframe,
			boxrule=0pt,
		},
	underlay boxed title={%
			\path[fill=tcbcolframe] (title.south west)--(title.south east)
			to[out=0, in=180] ([xshift=5mm]title.east)--
			(title.center-|frame.east)
			[rounded corners=\kvtcb@arc] |-
			(frame.north) -| cycle;
		},
}
\makeatother

%================================
% Question BOX
%================================

\makeatletter
\newtcbtheorem{qstion}{Question}{enhanced,
	breakable,
	colback=white,
	colframe=mygr,
	attach boxed title to top left={yshift*=-\tcboxedtitleheight},
	fonttitle=\bfseries,
	title={#2},
	boxed title size=title,
	boxed title style={%
			sharp corners,
			rounded corners=northwest,
			colback=tcbcolframe,
			boxrule=0pt,
		},
	underlay boxed title={%
			\path[fill=tcbcolframe] (title.south west)--(title.south east)
			to[out=0, in=180] ([xshift=5mm]title.east)--
			(title.center-|frame.east)
			[rounded corners=\kvtcb@arc] |-
			(frame.north) -| cycle;
		},
	#1
}{def}
\makeatother

\newtcbtheorem[number within=chapter]{wconc}{Wrong Concept}{
	breakable,
	enhanced,
	colback=white,
	colframe=myr,
	arc=0pt,
	outer arc=0pt,
	fonttitle=\bfseries\sffamily\large,
	colbacktitle=myr,
	attach boxed title to top left={},
	boxed title style={
			enhanced,
			skin=enhancedfirst jigsaw,
			arc=3pt,
			bottom=0pt,
			interior style={fill=myr}
		},
	#1
}{def}



%================================
% NOTE BOX
%================================

\usetikzlibrary{arrows,calc,shadows.blur}
\tcbuselibrary{skins}
\newtcolorbox{note}[1][]{%
	enhanced jigsaw,
	colback=gray!20!white,%
	colframe=gray!80!black,
	size=small,
	boxrule=1pt,
	title=\textbf{Note:-},
	halign title=flush center,
	coltitle=black,
	breakable,
	drop shadow=black!50!white,
	attach boxed title to top left={xshift=1cm,yshift=-\tcboxedtitleheight/2,yshifttext=-\tcboxedtitleheight/2},
	minipage boxed title=1.5cm,
	boxed title style={%
			colback=white,
			size=fbox,
			boxrule=1pt,
			boxsep=2pt,
			underlay={%
					\coordinate (dotA) at ($(interior.west) + (-0.5pt,0)$);
					\coordinate (dotB) at ($(interior.east) + (0.5pt,0)$);
					\begin{scope}
						\clip (interior.north west) rectangle ([xshift=3ex]interior.east);
						\filldraw [white, blur shadow={shadow opacity=60, shadow yshift=-.75ex}, rounded corners=2pt] (interior.north west) rectangle (interior.south east);
					\end{scope}
					\begin{scope}[gray!80!black]
						\fill (dotA) circle (2pt);
						\fill (dotB) circle (2pt);
					\end{scope}
				},
		},
	#1,
}

%%%%%%%%%%%%%%%%%%%%%%%%%%%%%%
% SELF MADE COMMANDS
%%%%%%%%%%%%%%%%%%%%%%%%%%%%%%


\newcommand{\thm}[2]{\begin{Theorem}{#1}{}#2\end{Theorem}}
\newcommand{\cor}[2]{\begin{Corollary}{#1}{}#2\end{Corollary}}
\newcommand{\mlenma}[2]{\begin{Lenma}{#1}{}#2\end{Lenma}}
\newcommand{\mprop}[2]{\begin{Prop}{#1}{}#2\end{Prop}}
\newcommand{\clm}[3]{\begin{claim}{#1}{#2}#3\end{claim}}
\newcommand{\wc}[2]{\begin{wconc}{#1}{}\setlength{\parindent}{1cm}#2\end{wconc}}
\newcommand{\thmcon}[1]{\begin{Theoremcon}{#1}\end{Theoremcon}}
\newcommand{\ex}[2]{\begin{Example}{#1}{}#2\end{Example}}
\newcommand{\dfn}[2]{\begin{Definition}[colbacktitle=red!75!black]{#1}{}#2\end{Definition}}
\newcommand{\dfnc}[2]{\begin{definition}[colbacktitle=red!75!black]{#1}{}#2\end{definition}}
\newcommand{\qs}[2]{\begin{question}{#1}{}#2\end{question}}
\newcommand{\pf}[2]{\begin{myproof}[#1]#2\end{myproof}}
\newcommand{\nt}[1]{\begin{note}#1\end{note}}

\newcommand*\circled[1]{\tikz[baseline=(char.base)]{
		\node[shape=circle,draw,inner sep=1pt] (char) {#1};}}
\newcommand\getcurrentref[1]{%
	\ifnumequal{\value{#1}}{0}
	{??}
	{\the\value{#1}}%
}
\newcommand{\getCurrentSectionNumber}{\getcurrentref{section}}
\newenvironment{myproof}[1][\proofname]{%
	\proof[\bfseries #1: ]%
}{\endproof}

\newcommand{\mclm}[2]{\begin{myclaim}[#1]#2\end{myclaim}}
\newenvironment{myclaim}[1][\claimname]{\proof[\bfseries #1: ]}{}

\newcounter{mylabelcounter}

\makeatletter
\newcommand{\setword}[2]{%
	\phantomsection
	#1\def\@currentlabel{\unexpanded{#1}}\label{#2}%
}
\makeatother




\tikzset{
	symbol/.style={
			draw=none,
			every to/.append style={
					edge node={node [sloped, allow upside down, auto=false]{$#1$}}}
		}
}


% deliminators
\DeclarePairedDelimiter{\abs}{\lvert}{\rvert}
\DeclarePairedDelimiter{\norm}{\lVert}{\rVert}

\DeclarePairedDelimiter{\ceil}{\lceil}{\rceil}
\DeclarePairedDelimiter{\floor}{\lfloor}{\rfloor}
\DeclarePairedDelimiter{\round}{\lfloor}{\rceil}

\newsavebox\diffdbox
\newcommand{\slantedromand}{{\mathpalette\makesl{d}}}
\newcommand{\makesl}[2]{%
\begingroup
\sbox{\diffdbox}{$\mathsurround=0pt#1\mathrm{#2}$}%
\pdfsave
\pdfsetmatrix{1 0 0.2 1}%
\rlap{\usebox{\diffdbox}}%
\pdfrestore
\hskip\wd\diffdbox
\endgroup
}
\newcommand{\dd}[1][]{\ensuremath{\mathop{}\!\ifstrempty{#1}{%
\slantedromand\@ifnextchar^{\hspace{0.2ex}}{\hspace{0.1ex}}}%
{\slantedromand\hspace{0.2ex}^{#1}}}}
\ProvideDocumentCommand\dv{o m g}{%
  \ensuremath{%
    \IfValueTF{#3}{%
      \IfNoValueTF{#1}{%
        \frac{\dd #2}{\dd #3}%
      }{%
        \frac{\dd^{#1} #2}{\dd #3^{#1}}%
      }%
    }{%
      \IfNoValueTF{#1}{%
        \frac{\dd}{\dd #2}%
      }{%
        \frac{\dd^{#1}}{\dd #2^{#1}}%
      }%
    }%
  }%
}
\providecommand*{\pdv}[3][]{\frac{\partial^{#1}#2}{\partial#3^{#1}}}
%  - others
\DeclareMathOperator{\Lap}{\mathcal{L}}
\DeclareMathOperator{\Var}{Var} % varience
\DeclareMathOperator{\Cov}{Cov} % covarience
\DeclareMathOperator{\E}{E} % expected

% Since the amsthm package isn't loaded

% I prefer the slanted \leq
\let\oldleq\leq % save them in case they're every wanted
\let\oldgeq\geq
\renewcommand{\leq}{\leqslant}
\renewcommand{\geq}{\geqslant}

% % redefine matrix env to allow for alignment, use r as default
% \renewcommand*\env@matrix[1][r]{\hskip -\arraycolsep
%     \let\@ifnextchar\new@ifnextchar
%     \array{*\c@MaxMatrixCols #1}}


%\usepackage{framed}
%\usepackage{titletoc}
%\usepackage{etoolbox}
%\usepackage{lmodern}


%\patchcmd{\tableofcontents}{\contentsname}{\sffamily\contentsname}{}{}

%\renewenvironment{leftbar}
%{\def\FrameCommand{\hspace{6em}%
%		{\color{myyellow}\vrule width 2pt depth 6pt}\hspace{1em}}%
%	\MakeFramed{\parshape 1 0cm \dimexpr\textwidth-6em\relax\FrameRestore}\vskip2pt%
%}
%{\endMakeFramed}

%\titlecontents{chapter}
%[0em]{\vspace*{2\baselineskip}}
%{\parbox{4.5em}{%
%		\hfill\Huge\sffamily\bfseries\color{myred}\thecontentspage}%
%	\vspace*{-2.3\baselineskip}\leftbar\textsc{\small\chaptername~\thecontentslabel}\\\sffamily}
%{}{\endleftbar}
%\titlecontents{section}
%[8.4em]
%{\sffamily\contentslabel{3em}}{}{}
%{\hspace{0.5em}\nobreak\itshape\color{myred}\contentspage}
%\titlecontents{subsection}
%[8.4em]
%{\sffamily\contentslabel{3em}}{}{}  
%{\hspace{0.5em}\nobreak\itshape\color{myred}\contentspage}



%%%%%%%%%%%%%%%%%%%%%%%%%%%%%%%%%%%%%%%%%%%
% TABLE OF CONTENTS
%%%%%%%%%%%%%%%%%%%%%%%%%%%%%%%%%%%%%%%%%%%

\usepackage{tikz}
\definecolor{doc}{RGB}{0,60,110}
\usepackage{titletoc}
\contentsmargin{0cm}
\titlecontents{chapter}[3.7pc]
{\addvspace{30pt}%
	\begin{tikzpicture}[remember picture, overlay]%
		\draw[fill=doc!60,draw=doc!60] (-7,-.1) rectangle (-0.9,.5);%
		\pgftext[left,x=-3.5cm,y=0.2cm]{\color{white}\Large\sc\bfseries Chapter\ \thecontentslabel};%
	\end{tikzpicture}\color{doc!60}\large\sc\bfseries}%
{}
{}
{\;\titlerule\;\large\sc\bfseries Page \thecontentspage
	\begin{tikzpicture}[remember picture, overlay]
		\draw[fill=doc!60,draw=doc!60] (2pt,0) rectangle (4,0.1pt);
	\end{tikzpicture}}%
\titlecontents{section}[3.7pc]
{\addvspace{2pt}}
{\contentslabel[\thecontentslabel]{2pc}}
{}
{\hfill\small \thecontentspage}
[]
\titlecontents*{subsection}[3.7pc]
{\addvspace{-1pt}\small}
{}
{}
{\ --- \small\thecontentspage}
[ \textbullet\ ][]

\makeatletter
\renewcommand{\tableofcontents}{%
	\chapter*{%
	  \vspace*{-20\p@}%
	  \begin{tikzpicture}[remember picture, overlay]%
		  \pgftext[right,x=15cm,y=0.2cm]{\color{doc!60}\Huge\sc\bfseries \contentsname};%
		  \draw[fill=doc!60,draw=doc!60] (13,-.75) rectangle (20,1);%
		  \clip (13,-.75) rectangle (20,1);
		  \pgftext[right,x=15cm,y=0.2cm]{\color{white}\Huge\sc\bfseries \contentsname};%
	  \end{tikzpicture}}%
	\@starttoc{toc}}
\makeatother


%From M275 "Topology" at SJSU
\newcommand{\id}{\mathrm{id}}
\newcommand{\taking}[1]{\xrightarrow{#1}}
\newcommand{\inv}{^{-1}}

%From M170 "Introduction to Graph Theory" at SJSU
\DeclareMathOperator{\diam}{diam}
\DeclareMathOperator{\ord}{ord}
\newcommand{\defeq}{\overset{\mathrm{def}}{=}}

%From the USAMO .tex files
\newcommand{\ts}{\textsuperscript}
\newcommand{\dg}{^\circ}
\newcommand{\ii}{\item}

% % From Math 55 and Math 145 at Harvard
% \newenvironment{subproof}[1][Proof]{%
% \begin{proof}[#1] \renewcommand{\qedsymbol}{$\blacksquare$}}%
% {\end{proof}}

\newcommand{\liff}{\leftrightarrow}
\newcommand{\lthen}{\rightarrow}
\newcommand{\opname}{\operatorname}
\newcommand{\surjto}{\twoheadrightarrow}
\newcommand{\injto}{\hookrightarrow}
\newcommand{\On}{\mathrm{On}} % ordinals
\DeclareMathOperator{\img}{im} % Image
\DeclareMathOperator{\Img}{Im} % Image
\DeclareMathOperator{\coker}{coker} % Cokernel
\DeclareMathOperator{\Coker}{Coker} % Cokernel
\DeclareMathOperator{\Ker}{Ker} % Kernel
\DeclareMathOperator{\rank}{rank}
\DeclareMathOperator{\Spec}{Spec} % spectrum
\DeclareMathOperator{\Tr}{Tr} % trace
\DeclareMathOperator{\pr}{pr} % projection
\DeclareMathOperator{\ext}{ext} % extension
\DeclareMathOperator{\pred}{pred} % predecessor
\DeclareMathOperator{\dom}{dom} % domain
\DeclareMathOperator{\ran}{ran} % range
\DeclareMathOperator{\Hom}{Hom} % homomorphism
\DeclareMathOperator{\Mor}{Mor} % morphisms
\DeclareMathOperator{\End}{End} % endomorphism

\newcommand{\eps}{\epsilon}
\newcommand{\veps}{\varepsilon}
\newcommand{\ol}{\overline}
\newcommand{\ul}{\underline}
\newcommand{\wt}{\widetilde}
\newcommand{\wh}{\widehat}
\newcommand{\vocab}[1]{\textbf{\color{blue} #1}}
\providecommand{\half}{\frac{1}{2}}
\newcommand{\dang}{\measuredangle} %% Directed angle
\newcommand{\ray}[1]{\overrightarrow{#1}}
\newcommand{\seg}[1]{\overline{#1}}
\newcommand{\arc}[1]{\wideparen{#1}}
\DeclareMathOperator{\cis}{cis}
\DeclareMathOperator*{\lcm}{lcm}
\DeclareMathOperator*{\argmin}{arg min}
\DeclareMathOperator*{\argmax}{arg max}
\newcommand{\cycsum}{\sum_{\mathrm{cyc}}}
\newcommand{\symsum}{\sum_{\mathrm{sym}}}
\newcommand{\cycprod}{\prod_{\mathrm{cyc}}}
\newcommand{\symprod}{\prod_{\mathrm{sym}}}
\newcommand{\Qed}{\begin{flushright}\qed\end{flushright}}
\newcommand{\parinn}{\setlength{\parindent}{1cm}}
\newcommand{\parinf}{\setlength{\parindent}{0cm}}
% \newcommand{\norm}{\|\cdot\|}
\newcommand{\inorm}{\norm_{\infty}}
\newcommand{\opensets}{\{V_{\alpha}\}_{\alpha\in I}}
\newcommand{\oset}{V_{\alpha}}
\newcommand{\opset}[1]{V_{\alpha_{#1}}}
\newcommand{\lub}{\text{lub}}
\newcommand{\del}[2]{\frac{\partial #1}{\partial #2}}
\newcommand{\Del}[3]{\frac{\partial^{#1} #2}{\partial^{#1} #3}}
\newcommand{\deld}[2]{\dfrac{\partial #1}{\partial #2}}
\newcommand{\Deld}[3]{\dfrac{\partial^{#1} #2}{\partial^{#1} #3}}
\newcommand{\lm}{\lambda}
\newcommand{\uin}{\mathbin{\rotatebox[origin=c]{90}{$\in$}}}
\newcommand{\usubset}{\mathbin{\rotatebox[origin=c]{90}{$\subset$}}}
\newcommand{\lt}{\left}
\newcommand{\rt}{\right}
\newcommand{\bs}[1]{\boldsymbol{#1}}
\newcommand{\exs}{\exists}
\newcommand{\st}{\strut}
\newcommand{\dps}[1]{\displaystyle{#1}}

\newcommand{\sol}{\setlength{\parindent}{0cm}\textbf{\textit{Solution:}}\setlength{\parindent}{1cm} }
\newcommand{\solve}[1]{\setlength{\parindent}{0cm}\textbf{\textit{Solution: }}\setlength{\parindent}{1cm}#1 \Qed}

%--------------------------------------------------
% LIE ALGEBRAS
%--------------------------------------------------
\newcommand*{\kb}{\mathfrak{b}}  % Borel subalgebra
\newcommand*{\kg}{\mathfrak{g}}  % Lie algebra
\newcommand*{\kh}{\mathfrak{h}}  % Cartan subalgebra
\newcommand*{\kn}{\mathfrak{n}}  % Nilradical
\newcommand*{\ku}{\mathfrak{u}}  % Unipotent algebra
\newcommand*{\kz}{\mathfrak{z}}  % Center of algebra

%--------------------------------------------------
% HOMOLOGICAL ALGEBRA
%--------------------------------------------------
\DeclareMathOperator{\Ext}{Ext} % Ext functor
\DeclareMathOperator{\Tor}{Tor} % Tor functor

%--------------------------------------------------
% MATRIX & GROUP NOTATION
%--------------------------------------------------
\DeclareMathOperator{\GL}{GL} % General Linear Group
\DeclareMathOperator{\SL}{SL} % Special Linear Group
\newcommand*{\gl}{\operatorname{\mathfrak{gl}}} % General linear Lie algebra
\newcommand*{\sl}{\operatorname{\mathfrak{sl}}} % Special linear Lie algebra

%--------------------------------------------------
% NUMBER SETS
%--------------------------------------------------
\newcommand*{\RR}{\mathbb{R}}
\newcommand*{\NN}{\mathbb{N}}
\newcommand*{\ZZ}{\mathbb{Z}}
\newcommand*{\QQ}{\mathbb{Q}}
\newcommand*{\CC}{\mathbb{C}}
\newcommand*{\PP}{\mathbb{P}}
\newcommand*{\HH}{\mathbb{H}}
\newcommand*{\FF}{\mathbb{F}}
\newcommand*{\EE}{\mathbb{E}} % Expected Value

%--------------------------------------------------
% MATH SCRIPT, FRAKTUR, AND BOLD SYMBOLS
%--------------------------------------------------
\newcommand*{\mcA}{\mathcal{A}}
\newcommand*{\mcB}{\mathcal{B}}
\newcommand*{\mcC}{\mathcal{C}}
\newcommand*{\mcD}{\mathcal{D}}
\newcommand*{\mcE}{\mathcal{E}}
\newcommand*{\mcF}{\mathcal{F}}
\newcommand*{\mcG}{\mathcal{G}}
\newcommand*{\mcH}{\mathcal{H}}

\newcommand*{\mfA}{\mathfrak{A}}  \newcommand*{\mfB}{\mathfrak{B}}
\newcommand*{\mfC}{\mathfrak{C}}  \newcommand*{\mfD}{\mathfrak{D}}
\newcommand*{\mfE}{\mathfrak{E}}  \newcommand*{\mfF}{\mathfrak{F}}
\newcommand*{\mfG}{\mathfrak{G}}  \newcommand*{\mfH}{\mathfrak{H}}

\usepackage{bm} % Ensure bold math works correctly
\newcommand*{\bmA}{\bm{A}}
\newcommand*{\bmB}{\bm{B}}
\newcommand*{\bmC}{\bm{C}}
\newcommand*{\bmD}{\bm{D}}
\newcommand*{\bmE}{\bm{E}}
\newcommand*{\bmF}{\bm{F}}
\newcommand*{\bmG}{\bm{G}}
\newcommand*{\bmH}{\bm{H}}

%--------------------------------------------------
% FUNCTIONAL ANALYSIS & ALGEBRA
%--------------------------------------------------
\DeclareMathOperator{\Aut}{Aut} % Automorphism group
\DeclareMathOperator{\Inn}{Inn} % Inner automorphisms
\DeclareMathOperator{\Syl}{Syl} % Sylow subgroups
\DeclareMathOperator{\Gal}{Gal} % Galois group
\DeclareMathOperator{\sign}{sign} % Sign function


%\usepackage[tagged, highstructure]{accessibility}
\usepackage{tocloft}
\usepackage{arydshln}
\usetikzlibrary{arrows.meta, decorations.pathreplacing}
\usepackage{tikz-cd}
\usepackage{polynom}
\usepackage{pifont}
\newcommand{\pistar}{{\zf\symbol{"4A}}}
% a tiny helper for a stretched phantom (for the underbrace)
\newcommand\mc[1]{\multicolumn{1}{c}{#1}}
\usepackage{pgfplots}
\pgfplotsset{compat=1.17}



\begin{document}
\title{Linear Algebra I}
\author{Lecture Notes Provided by Dr.~Miriam Logan.}
\date{}
\maketitle
\tableofcontents
\newpage  
   
A $  2 \times  2$ orthogonal matrix is either a rotation or of $ \mathbb{R} ^2$ about the origin or a reflection of $ \mathbb{R} ^2$ about a line through the origin.  \\
Let $ A = \begin{bmatrix}
a & b\\
c & d\\
\end{bmatrix}$ be an orthogonal matrix. \\
We know that $ a ^2 + c^2 =1$, $ b^2 + d^2 = 1$, and $ ab + cd = 0$. \\
$ \implies a^2 = 1- c^2$, $ \qquad  d^2 = 1 - b^2$\\
and $ ab = -cd $ \\
 \begin{align*}
   \implies a^2 b^2 &= c^2 d^2 \\
   \implies \left(1 - c^2\right)b^2 &= c^2 \left(1 - b^2\right)\\
   \implies b^2 - c^2 b^2 &= c^2 - c^2 b^2\\
   \implies b^2 &= c^2\\
   \implies b &= \pm c
 .\end{align*}
 Similarly, we can show that $ a = \pm d$. \\
 Case $ 1$ : $ a = d$ , $ b = c$
 \begin{align*}
  ab+cd =0\\
  \implies ac+ ca = 0\\
  2ac = 0\\
  \implies a = 0 \text{ or } c = 0\\
  \text{ and hence } d = 0 \text{ or } b = 0\\
 .\end{align*}
 \[
 a=d=0  \qquad  \begin{bmatrix}
 0 & b\\
 c & 0\\
 \end{bmatrix} \qquad  b^2=1 \qquad c^2=1
 .\] 
 \[
 \underbrace{ \begin{bmatrix}
 0 & 1\\
 1 & 0\\
\end{bmatrix} }_{ \text{reflection in } y = x} \qquad \underbrace{ \begin{bmatrix}
0 & -1\\
 1&0 \\
\end{bmatrix} }_{ \text{rotation by } \frac{\pi}{2} \text{ anticlockwise} } \qquad  \underbrace{ \begin{bmatrix}
0 & 1\\
-1 & 0\\
\end{bmatrix} }_{ \text{rotation by } \frac{3\pi}{2} \text{ anticlockwise}} \qquad \underbrace{ \begin{bmatrix}
0 & -1\\
-1 & 0\\
\end{bmatrix} }_{ \text{reflection in } y = -x} 
 .\] 
 Case $ 2$ : $ a = -d$, $ b = -c$
 same as case $ 1$ \\
 \\
 Case $ 3$ : $ b=c$ and $ a = -d$\\
 \[
 \begin{bmatrix}
 a & c\\
 c & -a\\
 \end{bmatrix} \qquad  a^2 + c^2 = 1 \qquad  \implies \begin{bmatrix}
 a\\
 c\\
 \end{bmatrix}
  \text{ is a vector on the unit circle for some } \theta \in [0, 2\pi)
 .\] 
 \[
 \implies \begin{bmatrix}
 a\\
 c\\
 \end{bmatrix}
 = \begin{bmatrix}
 \cos \theta \\
 \sin \theta \\
 \end{bmatrix}
 .\]
 \[
 \implies A = \begin{bmatrix}
 \cos \theta  & \sin \theta \\
 \sin \theta  & - \cos \theta \\
 \end{bmatrix}
 .\] 
 to show that this matrix represents a reflection in $ \mathbb{R} ^2$ we just need to show that it is similar to $ \begin{bmatrix}
 1 & 0\\
 0 & -1\\
 \end{bmatrix}$ i.e. that it has $ 1$ and minus $ 1$ as eigenvalues. \\
 \begin{align*}
  \chi _A \left( \lambda \right) = \left( \cos \theta  -\lambda \right) \left( - \cos \theta - \lambda  \right) - \sin ^2 \theta &= 0\\
  - \cos ^2 \theta + \lambda ^2 - \sin ^2 \theta &= 0\\
  \lambda ^2 &= 1\\
  \lambda &= \pm 1
  \implies A \text{ is similar to } \begin{bmatrix}
  1 & 0\\
  0 & -1\\
  \end{bmatrix}
 .\end{align*}
 In fact, $ A$ is a reflection of $ \mathbb{R} ^2$ in the line through the origin that makes an angle of $  \frac{\theta}{2}$ with the $ x$-axis. \\
 \\
 Case $ 4$ : $ b = -c$ and $ a = d$\\
 \[
 A = \begin{bmatrix}
 a & -c\\
 c & a\\
 \end{bmatrix} 
 .\] 
 \[
 \text{ As before, } \qquad  A = \begin{bmatrix}
  \cos \theta  & - \sin \theta \\
 \sin \theta  & \cos \theta \\
 \end{bmatrix} \text{ for some } \theta \in [0, 2\pi)
 .\] 
 this is a rotation of $ \mathbb{R} ^2$ about the origin by an angle of $ \theta$ anticlockwise. \\
 \\
 \thm{Spectral Theorem}
 {
   Suppose $ A$ is a real symmetric $n \times n$  matrix. There is an orthogonal matrix $ B$ such that $ B ^{ T}A B = B ^{-1}A B$ is diagonal. The entries along the diagonal are the eigenvalues and the columns of $ B$ are the eigenvectors of $ A$. \\
 }
 \ex{}{
  Let $ \mathcal{P} _1 \left[ x \right]$ be the space of all polynomials of degree at most $ 1$. \\
  Let 
  \[
  \langle f,g  \rangle = \int_{-1}^{1} 3x f \left( x \right) g \left( x \right) dx \qquad  \forall  f, g \in \mathcal{P} _1 \left[ x \right]
  .\] 
  Find the matrix $ A$ of this bilinear form with respect to the standard basis $ \mathcal{F} =\left\{ 1, x \right\}$ and find an orthogonal matrix $ B$ such that $ B ^{ T} A B$ is diagonal. \\
  \textbf{Solution:} \\
  \[
  \left( A \right) _{ ij} = \langle \vec{ b_i} , \vec{ b_j}   \rangle \text{ where } \vec{ b_1} = 1, \vec{ b_2} = x
  .\] 
  \begin{align*}
   \langle \vec{ b_i} , \vec{ b_j}   \rangle &= \langle x ^{i-1}, x ^{j-1}  \rangle \\
   &= \int_{-1}^{1} 3x \left( x ^{i-1} \right) \left( x ^{j-1} \right) dx\\
   &= 3 \int_{-1}^{1} x ^{i+j-1} dx\\
   &= 3 \left[ \frac{x ^{i+j}}{i+j} \right]_{-1}^{1}\\
   &= 3 \left( \frac{1}{i+j} - \frac{(-1)^{i+j}}{i+j} \right)\\
   a_{ 1 1} = 3 \left( \frac{1}{2} - \frac{1}{2} \right)  = 0\qquad  a _{ 12 } = 3 \left( \frac{1}{3} - \frac{(-1)^{3}}{3} \right) = 2\\
   a _{ 21 } = 3 \left( \frac{1}{3} - \frac{(-1)^{3}}{3} \right) = 2 \qquad  a _{ 22 } = 3 \left( \frac{1}{4} - \frac{1}{4} \right) = 0\\
  .\end{align*}
  \[
  \implies A = \begin{bmatrix}
  0 & 2\\
  2 & 0\\
 \end{bmatrix} \qquad  A \text{ is symmetric}
  .\] 
  To find $ B$ we look for the eigenvalues.
  \begin{align*}
   \chi _A \left( \lambda \right) &= \lambda^2 -4 =0\\
   \lambda &= \pm 2
  .\end{align*}
  \[
  \lambda=2 \to \vec{ v_1} = \begin{bmatrix}
  1\\
  1\\
  \end{bmatrix}
   \qquad  \lambda = -2 \to \vec{ v_2} =  \begin{bmatrix}
   1\\
   -1\\
   \end{bmatrix}
  .\] 
  Eigenvalues are distinct thus eigenvectors are orthogonal. \\
  \[
  \text{ Let } B = \begin{bmatrix}
  \frac{ 1  }{ \sqrt{2}  } & \frac{1}{ \sqrt{2} } \\
    \frac{1}{ \sqrt{2} }& - \frac{1}{ \sqrt{2} }\\
  \end{bmatrix} \qquad B ^{ T} = B ^{-1}
  .\] 
  \[
  B ^{ T} A B = B^{-1} A B = \begin{bmatrix}
  2 & 0\\
  0 & -2\\
  \end{bmatrix}
  .\] 
 }
  \ex{}{
   Suppose $ \left\{ \vec{ v_1} ,\ldots , \vec{ v_n}  \right\}  $ forms an orthonormal basis of $ \mathbb{R} ^n$ and consider the $n \times n$  matrix $ A= I_n - 2 \vec{ v_1} \left( \vec{ v_1}  \right) ^{T} $ . Show that $ A$ is 
   \begin{enumerate}[label=(\roman*)]
     \item symmetric
     \item orthogonal 
     \item    diagonalizable
     \end{enumerate}
   \textbf{Solution:} \\
   \begin{enumerate}[label=(\roman*).]  
     \item 
      \begin{align*}
       A^{T} &= \left( I_n - 2 \vec{ v_1} \left( \vec{ v_1}  \right) ^{T} \right) ^{T}\\
       &= I_n ^{T} - 2 \left( \vec{ v_1} \left( \vec{ v_1}  \right) ^{T} \right) ^{T}\\
       &= I_n - 2 \left( \vec{ v_1}  \right)  \vec{ v_1} ^{T} \qquad  \left( BC \right) ^{T} = C ^{T} B ^{T}\\
       &= A\\
       \implies A \text{ is symmetric.}
      .\end{align*}
     \item 
      \underline{Orthogonal:} \\
      by (i)\\
      \begin{align*}
       A^{T}A = A A &= \left( I_n - 2 \vec{ v_1} \left( \vec{ v_1}  \right) ^{T} \right) \left( I_n - 2 \vec{ v_1} \left( \vec{ v_1}  \right) ^{T} \right)\\
       &= I_n - 4 \left( \vec{ v_1}  \right) \left( \vec{ v_1}  \right) ^{T} + 4 \left( \vec{ v_1}  \right) \left( \vec{ v_1}  \right) ^{T} \left( \vec{ v_1}  \right) \left( \vec{ v_1}  \right) ^{T}\\
       &= I_n - 4 \left( \vec{ v_1}  \right) \left( \vec{ v_1}  \right) ^{T} + 4 \left( \vec{ v_1}  \right) \left( \vec{ v_1}  \right) ^{T}\\
       &= I_n\\ 
       &\implies A^{T}A = I_n \text{ i.e. } A \text{ is orthogonal.}
      .\end{align*}
     \item
      $ A$ being diagonalizable follows from the spectral theorem since $ A$ is orthogonal and symmetric. \\
   \end{enumerate}
   
   It can also be shown that each $ \vec{ v_i} $ is an eigenvector of $ A$  and hence, there exists a basis of $ \mathbb{R} ^{n}$ consisting of eigenvectors of $ A$. \\
   \begin{align*}
    A \vec{ v_1} &= \left( I_n - 2 \vec{ v_1} \left( \vec{ v_1}  \right) ^{T} \right) \vec{ v_1}\\
    &= \vec{ v_1} - 2 \vec{ v_1} \left( \vec{ v_1}  \right) ^{T} \vec{ v_1}\\
    &= \vec{ v_1} - 2 \vec{ v_1} = - \vec{ v_1}\\
    \implies \vec{ v_1} \text{ is an eigenvector of } A \text{ with eigenvalue } -1.\\
    A \vec{ v_k} &= \left( I_n - 2 \vec{ v_1} \left( \vec{ v_1}  \right) ^{T} \right) \vec{ v_k} \qquad  k \neq 1\\
    &= \left( I_n -2 \vec{ v_1} \left( \vec{ v_1}  \right) ^{T} \right) \left( \vec{ v_k} \right)\\
    &= \vec{ v_k} - 2 \vec{ v_1} \left( \underbrace{ \left( \vec{ v_1}  \right) ^{T} \left( \vec{ v_k}  \right)  }_{ =0 } \right)\\
    \implies A \vec{ v_k} = \vec{ v_k} 
    \text{ i.e. } \vec{ v_k} \text{ is an eigenvector of } A \text{ with eigenvalue } 1 \text{ for } k \neq 1.\\
   .\end{align*}
   $ \implies \left\{ \vec{ v_1} , \ldots , \vec{ v_k}  \right\} $  is an orthonormal basis consisting of eigenvectors of $ A$ and if $ B = \begin{bmatrix}
   \vline & \vline & \vline\\
   \vec{ v_1}  & \ldots & \vec{ v_k} \\
   \vline & \vline & \vline\\
   \end{bmatrix}$ then 
   \[
   B ^{ T} A B = \begin{bmatrix}
       -1 & 0 & 0 & \dots  & 0 \\
       0 & 1 & 0 & \dots  & 0 \\
       \vdots & \vdots & \vdots & \ddots & \vdots \\
       0 & 0 & 0 & \dots  & 1\end{bmatrix}
   .\] 
     
     
  }
  \section{Quadratic Forms}
 Consider a quadratic form $ q$ in the variables $ x_1, \ldots , x_n$,
 
 \[
 q \left( x_1,  \ldots , x_n \right) = \sum\limits_{i \le j}^{}  \alpha_{ij} x_i x_j 
 .\] 
 We can view this quadratic form as a particular instance of a symmetric bilinear form evaluated at  $ \langle  \vec{ x} , \vec{ x}   \rangle $ as follows:\\

 Let $ A$ be a symmetric $n \times n$  matrix, $ \left( A \right) _{ (i,j) } =a_{ij}$ and Let 
 \[
 \langle ,  \rangle : \mathbb{R} ^{n} \times  \mathbb{R} ^{n} \to \mathbb{R} 
 .\] 
 be defined as
 \begin{align*}
  \langle \vec{ x} , \vec{ y}   \rangle &= \left( \vec{ x}  \right) ^{T} A \vec{ y} \\
  \langle \vec{ x} ,\vec{ x}   \rangle =& \left( \vec{ x}  \right) ^{T} A \vec{ x} \\
  &= \sum\limits_{1 \leq i,j \leq n}^{} a_{ij} x_i x_j\\
 .\end{align*}
 Note: in this expansion both $ a_{ij} x_i x_j$ and $ a_{ji} x_j x_i$ appear, and since $ a_{ij} = a_{ji}$, and $ x_i x_j= x_j x_i \implies a_{ij} x_i x_j + a_{ji}x_j x_i = 2a_{ij}x_i x_j $ if $ i \neq j$
 \[
 \implies \langle \vec{ x} ,\vec{ x}   \rangle = q \left( x_1, \ldots , x_n \right) = \sum\limits_{i \le j}^{}  \alpha_{ij} x_i x_j 
 .\] 
 $ \iff \alpha_{ij}= 2 a_{ij}$ and $ \alpha _{ ij} = \alpha _{ i i}$ for $ i \neq j$.\\
   
  
 \dfn{Quadratic Form :}{
   A quadratic form in $ n$ variables is a function that has the following form:
   \[
   q \left( x_1, \ldots , x_n \right) = \sum\limits_{i \le j}^{}  \alpha_{ij} x_i x_j
   .\] 
   It can be written as 
   \[
   q \left( \vec{ x}  \right) = \left( \vec{ x}  \right) ^{T} A \vec{ x}
   .\] 
   where $ A$ is a symmetric $n \times n$  matrix with $ \alpha_{ ii} = A_{ii}$ and $ \alpha_{ij}= \frac{1}{2} A_{ij}$
 }
 \ex{}{
 $ q \left( \vec{ x}  \right) = x_1 x_2 \qquad  \vec{ x} \in \mathbb{R} ^2$
 \begin{enumerate}[label=(\roman*)]
   \item Find $ A$, the matrix of the quadratic form
   \item 
     Find a matrix $ B$ such that $ B ^{ T} A B$ is diagonal.
   \end{enumerate}
          \[
          q \left( x \right) = 0 x_1 x_1 + 1 x_1 x_2 + 0 x_2 x_2
          .\] 
          \[
          \implies q \left(  \vec{ x}  \right) = \left( \vec{ x}  \right) ^{T} A \vec{ x}  \qquad  \text{ where } A = \begin{bmatrix}
          0 & \frac{1}{2}\\
          \frac{1}{2} & 0\\
          \end{bmatrix}
          .\] 
          $ A$ is a symmetric matrix $ \implies \exists $ an orthogonal matrix $ B$ such that $ B ^{ T} A B$ is diagonal. \\
          Eigenvalues:
          \[
          \text{ det }  \left( A - \lambda I \right) = \lambda ^2 - \frac{1}{4}= 0 \qquad  \lambda = \pm \frac{1}{2}
          .\] 
          Eigenvectors:
          \[
          \lambda = \frac{1}{2} \to \vec{ v_1} = \begin{bmatrix}
          1\\
          1\\
          \end{bmatrix}
           \qquad  \lambda = -\frac{1}{2} \to \vec{ v_2} =  \begin{bmatrix}
           -1\\
           1\\
           \end{bmatrix}
           
          .\] 
          unit vectors:
          \[
          \vec{ w_1} = \frac{1}{\sqrt{2}} \begin{bmatrix}
          1\\
          1\\
          \end{bmatrix}
          \qquad  \vec{ w_2} = \frac{1}{\sqrt{2}} \begin{bmatrix}
          -1\\
          1\\
          \end{bmatrix}
          
          .\] 
   
          Let $ \mathcal{F} = \left\{ \vec{ w_1} ,\vec{ w_2}  \right\} $ \\
          \[
          B = \begin{bmatrix}
          \frac{1}{ \sqrt{2} } & - \frac{1}{ \sqrt{2} }\\
          \frac{1}{ \sqrt{2} } & \frac{1}{ \sqrt{2} }\\
          \end{bmatrix}             \qquad  B \text{ is an orthogonal matrix.}
          .\]
          \[
          B ^{ T} A B = B ^{-1} A B = \begin{bmatrix}
          \frac{1}{2} & 0\\
          0 & - \frac{1}{2}\\
          \end{bmatrix}
          .\] 
          Note $ B = P _{ \mathcal{F} \to \mathcal{E}}$ is a change of basis matrix from $ \mathcal{F} $ to $ \mathcal{E}$.\\
          Hence
          \[
           \left[ \vec{ x}  \right] _{ \mathcal{E}} = B \left[ \vec{ x}  \right] _{ \mathcal{F}} 
          .\] 
          and so 
          \begin{align*}
           \left( \left( \left[ \vec{ x}  \right] _{ \mathcal{E}}\right) ^{T} A \left( \left[ \vec{ x}  \right] _{ \mathcal{E}}\right)  \right) = \left( B \left[ \vec{ x}  \right] _{ \mathcal{F}}\right) ^{T} A \left( B \left[ \vec{ x}  \right] _{ \mathcal{F}}\right)\\
           &= \left( \left[ \vec{ x}  \right] _{ \mathcal{F}} ^{T} B ^{T} A B \left[ \vec{ x}  \right] _{ \mathcal{F}} \right)\\
          .\end{align*}
          $ \implies$ with respect to the basis $ \mathcal{F}$, the marix of $ q$ is diagonal,
          \[
          B ^{T} A B = \begin{bmatrix}
          \frac{1}{2} & 0\\
          0 & - \frac{1}{2}\\
          \end{bmatrix}
          .\]
          \[
           \text{ If } \left[ \vec{ x}  \right] _{ \mathcal{E}} = \begin{bmatrix}
           x_1\\
           x_2\\
           \end{bmatrix}
           _{ \mathcal{E}} \qquad  \text{ and } \left[ \vec{ x}  \right] _{ \mathcal{F}} = \begin{bmatrix}
           f_1\\
           f_2\\
           \end{bmatrix}
           _{ \mathcal{F}}
          .\] 
          \[
           \implies q \left(  \left[ \vec{ x}  \right] _{ \mathcal{F}} \right) = \frac{1}{2} f_1 ^2 - \frac{1}{2} f_2 ^2 
          .\] 
          while as,
          \[
           q \left( \left[ \vec{ x}  \right] _{ \mathcal{E}}\right) = x_1 x_2 
          .\] 
          \\
          Recall: If $ \mathcal{F} = \left\{  \vec{ v_1} , \ldots , \vec{ v_n}  \right\}$ and $  \mathcal{E} = \left\{  \vec{ w_1} , \ldots , \vec{ w_n} \right\} $ are different bases for $ \mathbb{R} ^{n}$ and suppose $ A \in M _{ n \times n}\left( \mathbb{R} \right) $  is the matrix of a bilinear form $ \langle ,  \rangle : \mathbb{R} ^{n} \times \mathbb{R} ^{n} \to \mathbb{R} $ with repect to $ \mathcal{E}$. And suppose $ B = P _{ \mathcal{F}\to \mathcal{E} }$ ie $ B = \begin{bmatrix}
          \vline & \vline & \vline\\
          \vec{ v_1}  & \ldots & \vec{ v_n} \\
          \vline & \vline & \vline\\
          \end{bmatrix}$
          then the matric of $ \langle ,  \rangle $ with respect to $ \mathcal{F}$ is $ B ^{T}A B $.\\
          \textit{ i.e. $ A$ and $ B^{T}A B$ represent the same bilinear form with respect to different bases.}\\
 } 
 
 \dfn{ :}{
 Two $n \times n$  matrices $ A, B \in M _{ n \times  n\left(  \mathbb{R} \right) }$    are said to be congruent if there exists an invertible matrix $ C$ such that \[
 B = C ^{T} A C
 .\] 
 Note:
 \[
 \left( C ^{T} \right) ^{-1} B C ^{-1} =A \\
 .\] 
 \[
 \left( C ^{-1} \right) ^{ T} B C ^{-1} =A 
 .\] 
 Congruent matrices represent the same bilinear form with respect to different bases just as similar matrices represent the same linear operator with respect to different bases. \\
 }
 \nt{
 Given a quadratic form 
 \[
 q \left( \vec{ x}  \right) = \left( \vec{ x}  \right) ^{T} A \vec{ x}
 .\] 
 Since $ A$ is symmetric, by the spectral theorem, there exists an orthogonal matrix $ B$ such that $ B ^{T} A B$ is diagonal (where the diagonal entries are eigenvalues). \\
 \begin{align*}
  q \left( \left[  \vec{ x}  \right]_{ \mathcal{E}} \right) ^{T} A \left( \left[ \vec{ x}  \right] \right) \\
  &= \left( \left[  \vec{ x}  \right]_{ \mathcal{F}} \right) ^{T} B ^{T} A B \left( \left[  \vec{ x}  \right]_{ \mathcal{F}} \right)\\
  &= \sum\limits_{i=1}^{n} \lambda_i \left( y_i \right) ^2
 .\end{align*}
 where $ \left[ \vec{ x}  \right]_{ \mathcal{F}}= \begin{bmatrix}
 y_1\\
 y_2\\
 \vdots \\
 y_n\\
 \end{bmatrix}_{ \mathcal{F}}
  $                         and $ \lambda_i$ are the eigenvalues of $ A$. \\
 }
  \ex{}{
  \[
  q \left( x_1 ,x_2 ,x_3 \right) = \left( x_1 \right) ^2 - \left( x_2 \right) ^2  + \left( x_3 \right) ^2 - 2 x_1 x_2 - 2 x_1 x_3 - 2 x_2 x_3
  .\] 
  \begin{enumerate}[label=(\roman*)]
    \item Find $ A$ the matrix of the quadratic form.
    \item Find a basis $ \mathcal{F} $ of $ \mathbb{R} ^3$ with respect to which the matrix of $ q$ is a diagonal matrix and find the diagonal matrix.
    \end{enumerate}
  \textbf{Solution:} \\
  \begin{enumerate}[label=(\roman*)]
    \item 
     \[
     A = \begin{bmatrix}
     1 & -1 & -1\\
     -1 & -1 & -1\\
     -1 & -1 & 1\\
     \end{bmatrix}
     .\] 
     \[
     \text{ Eigenvalues} \qquad  \lambda =1 \qquad  \lambda = 2 \qquad  \lambda= -2
     .\] 
     \[
     \text{ Eigenvectors: } \qquad   \begin{bmatrix}
     1\\
     -1\\
     1\\
     \end{bmatrix}
      \qquad  \begin{bmatrix}
      -1\\
      0\\
      1\\
      \end{bmatrix}
       \qquad  \begin{bmatrix}
       1\\
       2\\
       1\\
       \end{bmatrix}
     .\] 
     \[
     \text{Let } \mathcal{F} = \left\{  \frac{1}{ \sqrt{3} } \begin{bmatrix}
     1\\
     -1\\
     1\\
     \end{bmatrix}
      , \frac{1}{ \sqrt{2} } \begin{bmatrix}
      -1\\
      0\\
      1\\
      \end{bmatrix}
       , \frac{1}{ \sqrt{6} } \begin{bmatrix}
       1\\
       2\\
       1\\
       \end{bmatrix}
     \right\}
     .\] 
     $ \mathcal{F}$ is an orthonormal basis of $ \mathbb{R} ^3$. \\
     \[
     \text{ Let } B = \begin{bmatrix}
     \frac{1}{ \sqrt{3} } & - \frac{1}{ \sqrt{2} } & \frac{1}{ \sqrt{6} } \\
      - \frac{1}{ \sqrt{3} }& 0 & \frac{2}{ \sqrt{6} }\\
      \frac{1}{ \sqrt{3} }& \frac{1}{ \sqrt{2} }  & \frac{1}{ \sqrt{6} }\\
     \end{bmatrix}  \qquad  \text{ is an orthogonal matrix.}
     .\] 
     \[
     \text{ and } B ^{ T} A B = \begin{bmatrix}
     1 & 0 & 0\\
     0 & 2 & 0\\
     0 & 0 & -2\\
     \end{bmatrix}
     .\] 
     with the change of variables $ \left[ \vec{ x}  \right] _{ \mathcal{E}} B \left[ \vec{ x}  \right] _{ \mathcal{F}}, q$ can be expressed as 
     \begin{align*}
      q \left( \left[ \vec{ x}  \right] _{ \mathcal{E}}\right) = \left[ \vec{ x}  \right] _{ \mathcal{E}} ^{T} A \left[ \vec{ x}  \right] _{ \mathcal{E}}\\
      &= \left[ \vec{ x}  \right]  _{ \mathcal{F}} ^{T} B ^{T} A B \left[ \vec{ x}  \right] _{ \mathcal{F}}\\
      &= y_1^2+2 y_2^2 - 2 y_3^2\\
      \text{ if } \left[ \vec{ x}  \right] _{ \mathcal{F}} &= \begin{bmatrix}
      y_1\\
      y_2\\
      y_3\\
      \end{bmatrix}
     .\end{align*}
    \end{enumerate}

   }
   \dfn{Postive Definite Matrices :}{
             A $n \times n$  matrix $ A$ is said to be positive definite if 
             \[
              \langle \vec{ x} ,\vec{ x}   \rangle = \left( \vec{ x}  \right) ^{T} A \vec{ x} \ge 0 \qquad  \forall  \vec{ x}  \in \mathbb{R} ^{n} \text{ with equality} \iff \vec{ x} = \vec{ 0} 
             .\] 
   }

\thm{(Test for Postive Definiteness)}
   {
      Suppose $ A$ is a real symmetric matrix, then $ A$ is positive definite ifand only if:
      \begin{enumerate}[label=(\arabic*).]  
        \item Definition: $ \left( \vec{ x}  \right) ^{T}A \vec{ x} > 0 \qquad  \forall  \vec{ x }  \neq \vec{ 0} $
        \item    Eigenvalues: $ \lambda >0 \qquad  \forall $ eigenvalues $ \lambda $ of $ A$
        \item Sylvester's Criterion: the determinants of all the $ k \times  k$ upper left submatrices are all positive 


         \begin{tikzpicture}[every node/.style={anchor=base}]
  % ---------- MATRIX ----------
  \matrix (A) [matrix of math nodes,
               left delimiter=(, right delimiter=)] {
    a_{11} & a_{12} & a_{13} & \cdots & \cdots & a_{1n} \\
    a_{21} & a_{22} & a_{23} & \cdots & \cdots & a_{2n} \\
    a_{31} & a_{32} & a_{33} & \cdots & \cdots & a_{3n} \\
    \vdots & \vdots & \vdots & \ddots &        & \vdots \\
    a_{n1} & a_{n2} & a_{n3} & \cdots & \cdots & a_{nn} \\
  };

  % ---------- DASHED BLUE RECTANGLES ----------
  \draw[dashed,blue] (A-1-1.north west) rectangle (A-1-1.south east);  % 1×1 block
  \draw[dashed,blue] (A-1-1.north west) rectangle (A-2-2.south east);  % 2×2 block
  \draw[dashed,blue] (A-1-1.north west) rectangle (A-3-3.south east);  % 3×3 block
\end{tikzpicture}
      \end{enumerate}
               
   }
   
    \ex{}{
    Let $  A = \begin{bmatrix}
    3 & 1\\
    1 & 4\\
    \end{bmatrix}$. Is $ A$ positive definite? \\
    Sylvester: $ 3 >0$
    \[
    \text{ det }  A    = 12-1 = 11 >0 \qquad  \text{ Yes!}
    .\]
    or equivalently,\\
    \begin{align*}
     \text{ det } \left( A - \lambda I \right) &= \left( 3 - \lambda \right) \left( 4 -\lambda \right) -1\\
     &= \lambda ^2 - 7 \lambda + 11\\
     \lambda &= \frac{7 \pm \sqrt{49 - 44}}{2} = \frac{7 \pm \sqrt{5}}{2}\\
     \lambda _1 &= \frac{7 + \sqrt{5}}{2} >0\\
     \lambda _2 &= \frac{7 - \sqrt{5}}{2} >0\\
     \implies A \text{ is positive definite.}
    .\end{align*}
    }
    \ex{}{
    Let $ A = \begin{bmatrix}
    3 & 1 & 0\\
    1 & 2 & 5\\
    0 & 5 & x\\
    \end{bmatrix}$ \\
    \textit{Find the valies of $ x$ for which $ A$ is positive definite.} \\
    \textbf{Solution:} \\
    \[
     \text{ det } \left[ 3 \right]  =3 >0 
    .\] 
    \[
    \text{ det }  \begin{bmatrix}
    3 & 1\\
    1 & 2\\
    \end{bmatrix} = 5 >0
    .\] 
    \[
    \text{ det } A = 3 \left( 2x -25 \right) -1 \left( x \right) >0\\
    .\] 
    \begin{align*}
     6x -75 -x &>0\\
     5x -75 &>0\\
     x &>15\\
    .\end{align*}
    $ \implies A$ is positive definite $ \iff x >15$
    }
    \ex{}{
    Find the values of $ x$ for which 
    \[
    A = \begin{bmatrix}
    2 & x & 0\\
    x & x+4 & x-4\\
    0 & x-4 & x+1\\
    \end{bmatrix} \text{ is positive definite.}
    .\] 
     \[
     \text{ det } \left[ 2 \right] = 2 >0
     .\] 
     \begin{align*}
      \text{ det } \begin{bmatrix}
      2 & x\\
      x & x+4\\
     \end{bmatrix} = 2x + 8 - x^2 &>0\\
     - \left( x^2 -2x -8 \right) &>0\\
     - \left( x -4 \right) \left( x +2 \right) &>0\\
     -2 < x &<4
     .\end{align*}

  \\
  \\

  \begin{tikzpicture}
  \begin{axis}[
      xmin=-9, xmax=7,
      ymin=-400, ymax=400,
      axis lines=middle,      % if this key is missing on old TeX, use the two lines below instead
      % axis x line=middle,
      % axis y line=middle,
      xlabel={$x$}, ylabel={$f(x)$},
      grid=both,
      samples=300,
      domain=-9:7,
      thick,
  ]
    % ----- the cubic  f(x) = -(x-1)(x+6)(x-4) ---------------------------
    \addplot[blue] {-(x-1)*(x+6)*(x-4)};

    % ----- mark its three real zeros -----------------------------------
    \addplot[only marks, mark=*] coordinates {(-6,0) (1,0) (4,0)};
  \end{axis}
\end{tikzpicture}





  \\
     \begin{align*}
      \text{ det } A &= 2 \left[ \left( x+4 \right) \left( x+1 \right) - \left( x-4 \right) ^2  \right] - x \left( x \right) \left( x+1 \right) \\
      &= 2 \left[ x^2 +5x +4 - \left( x^2 -8x +16 \right) \right] - x^3 -x^2\\
      &= 2 \left[ 13x -12 \right] - x^3 -x^2\\
      &= - x^3 -x^2 + 26x -24\\
      &= - \left( x^3 + x^2 - 26x + 24 \right)\\
      & -1 \left( x-1 \right) \left( x^2 + 2x -24 \right)\\
      &= -1 \left( x-1 \right) \left( x+6 \right) \left( x-4 \right)>0\\
       x < -6 \qquad \text{ or } \qquad  1 < x < 4
     .\end{align*}

    }

    \dfn{Positive Definite Quadratic Form :}{
       A quadratic form $ q$ is said to be positive definite if $ q \left( \vec{ x}  \right) \ge 0 \qquad  \forall  \vec{ x} $ \\ with equality if and only if $ \vec{ x} = \vec{ 0}$.\\
    }

    \ex{}{
    Find the values of a for which the quadratic form
    \[
    q \left( \vec{ x}  \right) = a x_1 ^2 + x_2^2 + 5 x_3 ^2 + 2 x_1 x_2 + 2 x_1 x_3 + 2 x_2 x_3 \text{ is positive definite.}
    .\] 
    \[
    q \left( \vec{ x}  \right) = \left( \vec{ x}  \right) ^{T} A \vec{ x} = \left( \vec{ x}  \right) ^{T} \begin{bmatrix}
    a & 1 & 1\\
    1 & 1 & a\\
    1 & a & 5\\
    \end{bmatrix}             \vec{ x} 
    .\]  
    $ q$ is positive definite if and only if the matrix $ A$ is positive definite.\\
    \begin{enumerate}[label=(\arabic*).]  
     \item $ \text{ det } \left[ a \right]  =a >0 $

      \item 
       \[
       \text{ det }  \begin{bmatrix}
       a & 1\\
       1 & 1\\
       \end{bmatrix} = a -1 >0  \text{ i.e. } a >1
       .\] 
      \item 
       \begin{align*}
        \text{ det } \begin{bmatrix}
        a & 1 & 1\\
        1 & 1 & a\\
        1 & a & 5\\
       \end{bmatrix} &= a \left( 5 - a^2 \right) - 1 \left( 1 - a \right) + 1 \left( a - 1 \right)\\
        &= 5a - a^3 -5 + a + a -1\\
        &= -a^3 + 7a -6 >0\\
        &= - \left( a-1 \right) \left( a^2 + 2a -6 \right) >0\\
        &= - \left( a-1 \right) \left( a+3 \right) \left( a-2 \right) >0\\
        a<-3 \qquad  \text{ or } \qquad  1 < a < 2
       .\end{align*}
              \\
         \begin{tikzpicture}
  \begin{axis}[
      xmin=-5, xmax=5,
      ymin=-20, ymax=20,
      axis lines=middle,
      xlabel={$a$}, ylabel={$f(a)$},
      grid=both,            % optional: light background grid
      samples=200,          % number of evaluation points
      domain=-5:5,
      thick,
  ]
    % ----- the cubic f(a) = -(a-1)(a+3)(a-2) ---------------------------
    \addplot[blue] {-(x-1)*(x+3)*(x-2)};
    
    % ----- mark its three real zeros -----------------------------------
    \addplot[only marks, mark=*] coordinates {(-3,0) (1,0) (2,0)};
  \end{axis}
\end{tikzpicture}

       \\
       \\
       Putting everything together, we have $ A$ is positive definite if and only if $ 1 < a < 2$.
    \end{enumerate}
    
    }
    
    
 
   \thm{(Test for Postive Definiteness)}
   {
      Suppose $ A$ is a real symmetric matrix, then $ A$ is positive definite ifand only if:
      \begin{enumerate}[label=(\arabic*).]  
        \item Definition: $ \left( \vec{ x}  \right) ^{T}A \vec{ x} > 0 \qquad  \forall  \vec{ x }  \neq \vec{ 0} $
        \item    Eigenvalues: $ \lambda >0 \qquad  \forall $ eigenvalues $ \lambda $ of $ A$
        \item Sylvester's Criterion: the determinants of all the $ k \times  k$ upper left submatrices are all positive 
        \end{enumerate}
    }
      
      \pf{Proof:}{
       (1) $ \iff$ (2)\\
       $ A$ is symmetric $ \iff $   $ \exists $ an orthogonal matrix $ B$ such that $ B ^{T} A B = \begin{bmatrix}
           \lambda_1 & 0 & 0 & \dots  & 0 \\
           0 & \lambda_2 & 0 & \dots  & 0 \\
           \vdots & \vdots & \vdots & \ddots & \vdots \\
           0 & 0 & 0 & \dots  & \lambda_n\end{bmatrix} $

           with $ \lambda_i$  being the eigenvalues of $ A$.\\
           Let $ \mathcal{F}$ be the basis of $ \mathbb{R} ^{n}$ given by the vectors from the columns of $ B$, and suppose 
           \[
            \left[ \vec{ x}  \right] _{ \mathcal{F}} = \begin{bmatrix}
            y_1\\
            \vdots\\
            y_n\\
            \end{bmatrix}
           .\] 
           Then 
           \[
            \left( \left[ \vec{ x}  \right] _{ \mathcal{E}}\right)   ^{T} A \left( \left[ \vec{ x}  \right] _{ \mathcal{E}}\right) = \left(  \left[ \vec{ x}  \right] _{ \mathcal{F}} \right) ^{T} B ^{T} A B \left( B \left[ \vec{ x}  \right] _{ \mathcal{F}} \right)   = \sum\limits_{i=1}^{n} \lambda_i \left( y_i \right) ^2>0
           .\] 

          \[
           \iff \lambda_i \forall  i
          .\] 
          (1) $ \iff $  (3)
          \[
          \left( \vec{ x}  \right) ^{T}A \vec{ x} >0 \alpha \vec{ x}  \neq \vec{ 0} 
          .\] 
          \[
           \iff \sum\limits_{1 \leq i,j \leq n}^{}  a_{ij} x_i x_j >0 \qquad  \forall  \vec{ x}  \neq \vec{ 0}
          .\] 
          If we choose $ \vec{ x} $ such that 
          \[
           x_{k+1} = x_{k+2} = \ldots = x_n = 0 
          .\] 
          then
          \[
          \sum\limits_{1 \leq i , j \leq k}^{} a_{ (i,j) } x_i x_j >0 \qquad  \forall  \vec{ x}  \neq \vec{ 0}
          .\] 
          $ \iff$  the $ k \times k $ upper left submatrix is positive definite.\\
          $  \iff $ Its eigenvalues are positive.\\
          $ \iff  \text{ det } A_k >0$ 
      }
      
\end{document}
