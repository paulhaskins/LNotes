\documentclass{report}

%%%%%%%%%%%%%%%%%%%%%%%%%%%%%%%%%
% PACKAGE IMPORTS
%%%%%%%%%%%%%%%%%%%%%%%%%%%%%%%%%


\usepackage[tmargin=2cm,rmargin=1in,lmargin=1in,margin=0.85in,bmargin=2cm,footskip=.2in]{geometry}
\usepackage{amsmath,amsfonts,amsthm,amssymb,mathtools}
\usepackage[varbb]{newpxmath}
\usepackage{xfrac}
\usepackage[makeroom]{cancel}
\usepackage{bookmark}
\usepackage{enumitem}
\usepackage{hyperref,theoremref}
\hypersetup{
	pdftitle={Assignment},
	colorlinks=true, linkcolor=doc!90,
	bookmarksnumbered=true,
	bookmarksopen=true
}
\usepackage[most,many,breakable]{tcolorbox}
\usepackage{xcolor}
\usepackage{varwidth}
\usepackage{varwidth}
\usepackage{tocloft}
\usepackage{etoolbox}
\usepackage{derivative} %many derivativess partials
%\usepackage{authblk}
\usepackage{nameref}
\usepackage{multicol,array}
\usepackage{tikz-cd}
\usepackage[ruled,vlined,linesnumbered]{algorithm2e}
\usepackage{comment} % enables the use of multi-line comments (\ifx \fi) 
\usepackage{import}
\usepackage{xifthen}
\usepackage{pdfpages}
\usepackage{transparent}
\usepackage{verbatim}

\newcommand\mycommfont[1]{\footnotesize\ttfamily\textcolor{blue}{#1}}
\SetCommentSty{mycommfont}
\newcommand{\incfig}[1]{%
    \def\svgwidth{\columnwidth}
    \import{./figures/}{#1.pdf_tex}
}
\usepackage[tagged, highstructure]{accessibility}
\usepackage{tikzsymbols}
\renewcommand\qedsymbol{$\Laughey$}


%\usepackage{import}
%\usepackage{xifthen}
%\usepackage{pdfpages}
%\usepackage{transparent}


%%%%%%%%%%%%%%%%%%%%%%%%%%%%%%
% SELF MADE COLORS
%%%%%%%%%%%%%%%%%%%%%%%%%%%%%%



\definecolor{myg}{RGB}{56, 140, 70}
\definecolor{myb}{RGB}{45, 111, 177}
\definecolor{myr}{RGB}{199, 68, 64}
\definecolor{mytheorembg}{HTML}{F2F2F9}
\definecolor{mytheoremfr}{HTML}{00007B}
\definecolor{mylenmabg}{HTML}{FFFAF8}
\definecolor{mylenmafr}{HTML}{983b0f}
\definecolor{mypropbg}{HTML}{f2fbfc}
\definecolor{mypropfr}{HTML}{191971}
\definecolor{myexamplebg}{HTML}{F2FBF8}
\definecolor{myexamplefr}{HTML}{88D6D1}
\definecolor{myexampleti}{HTML}{2A7F7F}
\definecolor{mydefinitbg}{HTML}{E5E5FF}
\definecolor{mydefinitfr}{HTML}{3F3FA3}
\definecolor{notesgreen}{RGB}{0,162,0}
\definecolor{myp}{RGB}{197, 92, 212}
\definecolor{mygr}{HTML}{2C3338}
\definecolor{myred}{RGB}{127,0,0}
\definecolor{myyellow}{RGB}{169,121,69}
\definecolor{myexercisebg}{HTML}{F2FBF8}
\definecolor{myexercisefg}{HTML}{88D6D1}


%%%%%%%%%%%%%%%%%%%%%%%%%%%%
% TCOLORBOX SETUPS
%%%%%%%%%%%%%%%%%%%%%%%%%%%%

\setlength{\parindent}{1cm}
%================================
% THEOREM BOX
%================================

\tcbuselibrary{theorems,skins,hooks}
\newtcbtheorem[number within=section]{Theorem}{Theorem}
{%
	enhanced,
	breakable,
	colback = mytheorembg,
	frame hidden,
	boxrule = 0sp,
	borderline west = {2pt}{0pt}{mytheoremfr},
	sharp corners,
	detach title,
	before upper = \tcbtitle\par\smallskip,
	coltitle = mytheoremfr,
	fonttitle = \bfseries\sffamily,
	description font = \mdseries,
	separator sign none,
	segmentation style={solid, mytheoremfr},
}
{th}

\tcbuselibrary{theorems,skins,hooks}
\newtcbtheorem[number within=chapter]{theorem}{Theorem}
{%
	enhanced,
	breakable,
	colback = mytheorembg,
	frame hidden,
	boxrule = 0sp,
	borderline west = {2pt}{0pt}{mytheoremfr},
	sharp corners,
	detach title,
	before upper = \tcbtitle\par\smallskip,
	coltitle = mytheoremfr,
	fonttitle = \bfseries\sffamily,
	description font = \mdseries,
	separator sign none,
	segmentation style={solid, mytheoremfr},
}
{th}


\tcbuselibrary{theorems,skins,hooks}
\newtcolorbox{Theoremcon}
{%
	enhanced
	,breakable
	,colback = mytheorembg
	,frame hidden
	,boxrule = 0sp
	,borderline west = {2pt}{0pt}{mytheoremfr}
	,sharp corners
	,description font = \mdseries
	,separator sign none
}

%================================
% Corollery
%================================
\tcbuselibrary{theorems,skins,hooks}
\newtcbtheorem[number within=section]{Corollary}{Corollary}
{%
	enhanced
	,breakable
	,colback = myp!10
	,frame hidden
	,boxrule = 0sp
	,borderline west = {2pt}{0pt}{myp!85!black}
	,sharp corners
	,detach title
	,before upper = \tcbtitle\par\smallskip
	,coltitle = myp!85!black
	,fonttitle = \bfseries\sffamily
	,description font = \mdseries
	,separator sign none
	,segmentation style={solid, myp!85!black}
}
{th}
\tcbuselibrary{theorems,skins,hooks}
\newtcbtheorem[number within=chapter]{corollary}{Corollary}
{%
	enhanced
	,breakable
	,colback = myp!10
	,frame hidden
	,boxrule = 0sp
	,borderline west = {2pt}{0pt}{myp!85!black}
	,sharp corners
	,detach title
	,before upper = \tcbtitle\par\smallskip
	,coltitle = myp!85!black
	,fonttitle = \bfseries\sffamily
	,description font = \mdseries
	,separator sign none
	,segmentation style={solid, myp!85!black}
}
{th}


%================================
% LENMA
%================================

\tcbuselibrary{theorems,skins,hooks}
\newtcbtheorem[number within=section]{Lenma}{Lenma}
{%
	enhanced,
	breakable,
	colback = mylenmabg,
	frame hidden,
	boxrule = 0sp,
	borderline west = {2pt}{0pt}{mylenmafr},
	sharp corners,
	detach title,
	before upper = \tcbtitle\par\smallskip,
	coltitle = mylenmafr,
	fonttitle = \bfseries\sffamily,
	description font = \mdseries,
	separator sign none,
	segmentation style={solid, mylenmafr},
}
{th}

\tcbuselibrary{theorems,skins,hooks}
\newtcbtheorem[number within=chapter]{lenma}{Lenma}
{%
	enhanced,
	breakable,
	colback = mylenmabg,
	frame hidden,
	boxrule = 0sp,
	borderline west = {2pt}{0pt}{mylenmafr},
	sharp corners,
	detach title,
	before upper = \tcbtitle\par\smallskip,
	coltitle = mylenmafr,
	fonttitle = \bfseries\sffamily,
	description font = \mdseries,
	separator sign none,
	segmentation style={solid, mylenmafr},
}
{th}


%================================
% PROPOSITION
%================================

\tcbuselibrary{theorems,skins,hooks}
\newtcbtheorem[number within=section]{Prop}{Proposition}
{%
	enhanced,
	breakable,
	colback = mypropbg,
	frame hidden,
	boxrule = 0sp,
	borderline west = {2pt}{0pt}{mypropfr},
	sharp corners,
	detach title,
	before upper = \tcbtitle\par\smallskip,
	coltitle = mypropfr,
	fonttitle = \bfseries\sffamily,
	description font = \mdseries,
	separator sign none,
	segmentation style={solid, mypropfr},
}
{th}

\tcbuselibrary{theorems,skins,hooks}
\newtcbtheorem[number within=chapter]{prop}{Proposition}
{%
	enhanced,
	breakable,
	colback = mypropbg,
	frame hidden,
	boxrule = 0sp,
	borderline west = {2pt}{0pt}{mypropfr},
	sharp corners,
	detach title,
	before upper = \tcbtitle\par\smallskip,
	coltitle = mypropfr,
	fonttitle = \bfseries\sffamily,
	description font = \mdseries,
	separator sign none,
	segmentation style={solid, mypropfr},
}
{th}


%================================
% CLAIM
%================================

\tcbuselibrary{theorems,skins,hooks}
\newtcbtheorem[number within=section]{claim}{Claim}
{%
	enhanced
	,breakable
	,colback = myg!10
	,frame hidden
	,boxrule = 0sp
	,borderline west = {2pt}{0pt}{myg}
	,sharp corners
	,detach title
	,before upper = \tcbtitle\par\smallskip
	,coltitle = myg!85!black
	,fonttitle = \bfseries\sffamily
	,description font = \mdseries
	,separator sign none
	,segmentation style={solid, myg!85!black}
}
{th}



%================================
% Exercise
%================================

\tcbuselibrary{theorems,skins,hooks}
\newtcbtheorem[number within=section]{Exercise}{Exercise}
{%
	enhanced,
	breakable,
	colback = myexercisebg,
	frame hidden,
	boxrule = 0sp,
	borderline west = {2pt}{0pt}{myexercisefg},
	sharp corners,
	detach title,
	before upper = \tcbtitle\par\smallskip,
	coltitle = myexercisefg,
	fonttitle = \bfseries\sffamily,
	description font = \mdseries,
	separator sign none,
	segmentation style={solid, myexercisefg},
}
{th}

\tcbuselibrary{theorems,skins,hooks}
\newtcbtheorem[number within=chapter]{exercise}{Exercise}
{%
	enhanced,
	breakable,
	colback = myexercisebg,
	frame hidden,
	boxrule = 0sp,
	borderline west = {2pt}{0pt}{myexercisefg},
	sharp corners,
	detach title,
	before upper = \tcbtitle\par\smallskip,
	coltitle = myexercisefg,
	fonttitle = \bfseries\sffamily,
	description font = \mdseries,
	separator sign none,
	segmentation style={solid, myexercisefg},
}
{th}

%================================
% EXAMPLE BOX
%================================

\newtcbtheorem[number within=section]{Example}{Example}
{%
	colback = myexamplebg
	,breakable
	,colframe = myexamplefr
	,coltitle = myexampleti
	,boxrule = 1pt
	,sharp corners
	,detach title
	,before upper=\tcbtitle\par\smallskip
	,fonttitle = \bfseries
	,description font = \mdseries
	,separator sign none
	,description delimiters parenthesis
}
{ex}

\newtcbtheorem[number within=chapter]{example}{Example}
{%
	colback = myexamplebg
	,breakable
	,colframe = myexamplefr
	,coltitle = myexampleti
	,boxrule = 1pt
	,sharp corners
	,detach title
	,before upper=\tcbtitle\par\smallskip
	,fonttitle = \bfseries
	,description font = \mdseries
	,separator sign none
	,description delimiters parenthesis
}
{ex}

%================================
% DEFINITION BOX
%================================

\newtcbtheorem[number within=section]{Definition}{Definition}{enhanced,
	before skip=2mm,after skip=2mm, colback=red!5,colframe=red!80!black,boxrule=0.5mm,
	attach boxed title to top left={xshift=1cm,yshift*=1mm-\tcboxedtitleheight}, varwidth boxed title*=-3cm,
	boxed title style={frame code={
					\path[fill=tcbcolback]
					([yshift=-1mm,xshift=-1mm]frame.north west)
					arc[start angle=0,end angle=180,radius=1mm]
					([yshift=-1mm,xshift=1mm]frame.north east)
					arc[start angle=180,end angle=0,radius=1mm];
					\path[left color=tcbcolback!60!black,right color=tcbcolback!60!black,
						middle color=tcbcolback!80!black]
					([xshift=-2mm]frame.north west) -- ([xshift=2mm]frame.north east)
					[rounded corners=1mm]-- ([xshift=1mm,yshift=-1mm]frame.north east)
					-- (frame.south east) -- (frame.south west)
					-- ([xshift=-1mm,yshift=-1mm]frame.north west)
					[sharp corners]-- cycle;
				},interior engine=empty,
		},
	fonttitle=\bfseries,
	title={#2},#1}{def}
\newtcbtheorem[number within=chapter]{definition}{Definition}{enhanced,
	before skip=2mm,after skip=2mm, colback=red!5,colframe=red!80!black,boxrule=0.5mm,
	attach boxed title to top left={xshift=1cm,yshift*=1mm-\tcboxedtitleheight}, varwidth boxed title*=-3cm,
	boxed title style={frame code={
					\path[fill=tcbcolback]
					([yshift=-1mm,xshift=-1mm]frame.north west)
					arc[start angle=0,end angle=180,radius=1mm]
					([yshift=-1mm,xshift=1mm]frame.north east)
					arc[start angle=180,end angle=0,radius=1mm];
					\path[left color=tcbcolback!60!black,right color=tcbcolback!60!black,
						middle color=tcbcolback!80!black]
					([xshift=-2mm]frame.north west) -- ([xshift=2mm]frame.north east)
					[rounded corners=1mm]-- ([xshift=1mm,yshift=-1mm]frame.north east)
					-- (frame.south east) -- (frame.south west)
					-- ([xshift=-1mm,yshift=-1mm]frame.north west)
					[sharp corners]-- cycle;
				},interior engine=empty,
		},
	fonttitle=\bfseries,
	title={#2},#1}{def}



%================================
% Solution BOX
%================================

\makeatletter
\newtcbtheorem{question}{Question}{enhanced,
	breakable,
	colback=white,
	colframe=myb!80!black,
	attach boxed title to top left={yshift*=-\tcboxedtitleheight},
	fonttitle=\bfseries,
	title={#2},
	boxed title size=title,
	boxed title style={%
			sharp corners,
			rounded corners=northwest,
			colback=tcbcolframe,
			boxrule=0pt,
		},
	underlay boxed title={%
			\path[fill=tcbcolframe] (title.south west)--(title.south east)
			to[out=0, in=180] ([xshift=5mm]title.east)--
			(title.center-|frame.east)
			[rounded corners=\kvtcb@arc] |-
			(frame.north) -| cycle;
		},
	#1
}{def}
\makeatother

%================================
% SOLUTION BOX
%================================

\makeatletter
\newtcolorbox{solution}{enhanced,
	breakable,
	colback=white,
	colframe=myg!80!black,
	attach boxed title to top left={yshift*=-\tcboxedtitleheight},
	title=Solution,
	boxed title size=title,
	boxed title style={%
			sharp corners,
			rounded corners=northwest,
			colback=tcbcolframe,
			boxrule=0pt,
		},
	underlay boxed title={%
			\path[fill=tcbcolframe] (title.south west)--(title.south east)
			to[out=0, in=180] ([xshift=5mm]title.east)--
			(title.center-|frame.east)
			[rounded corners=\kvtcb@arc] |-
			(frame.north) -| cycle;
		},
}
\makeatother

%================================
% Question BOX
%================================

\makeatletter
\newtcbtheorem{qstion}{Question}{enhanced,
	breakable,
	colback=white,
	colframe=mygr,
	attach boxed title to top left={yshift*=-\tcboxedtitleheight},
	fonttitle=\bfseries,
	title={#2},
	boxed title size=title,
	boxed title style={%
			sharp corners,
			rounded corners=northwest,
			colback=tcbcolframe,
			boxrule=0pt,
		},
	underlay boxed title={%
			\path[fill=tcbcolframe] (title.south west)--(title.south east)
			to[out=0, in=180] ([xshift=5mm]title.east)--
			(title.center-|frame.east)
			[rounded corners=\kvtcb@arc] |-
			(frame.north) -| cycle;
		},
	#1
}{def}
\makeatother

\newtcbtheorem[number within=chapter]{wconc}{Wrong Concept}{
	breakable,
	enhanced,
	colback=white,
	colframe=myr,
	arc=0pt,
	outer arc=0pt,
	fonttitle=\bfseries\sffamily\large,
	colbacktitle=myr,
	attach boxed title to top left={},
	boxed title style={
			enhanced,
			skin=enhancedfirst jigsaw,
			arc=3pt,
			bottom=0pt,
			interior style={fill=myr}
		},
	#1
}{def}



%================================
% NOTE BOX
%================================

\usetikzlibrary{arrows,calc,shadows.blur}
\tcbuselibrary{skins}
\newtcolorbox{note}[1][]{%
	enhanced jigsaw,
	colback=gray!20!white,%
	colframe=gray!80!black,
	size=small,
	boxrule=1pt,
	title=\textbf{Note:-},
	halign title=flush center,
	coltitle=black,
	breakable,
	drop shadow=black!50!white,
	attach boxed title to top left={xshift=1cm,yshift=-\tcboxedtitleheight/2,yshifttext=-\tcboxedtitleheight/2},
	minipage boxed title=1.5cm,
	boxed title style={%
			colback=white,
			size=fbox,
			boxrule=1pt,
			boxsep=2pt,
			underlay={%
					\coordinate (dotA) at ($(interior.west) + (-0.5pt,0)$);
					\coordinate (dotB) at ($(interior.east) + (0.5pt,0)$);
					\begin{scope}
						\clip (interior.north west) rectangle ([xshift=3ex]interior.east);
						\filldraw [white, blur shadow={shadow opacity=60, shadow yshift=-.75ex}, rounded corners=2pt] (interior.north west) rectangle (interior.south east);
					\end{scope}
					\begin{scope}[gray!80!black]
						\fill (dotA) circle (2pt);
						\fill (dotB) circle (2pt);
					\end{scope}
				},
		},
	#1,
}

%%%%%%%%%%%%%%%%%%%%%%%%%%%%%%
% SELF MADE COMMANDS
%%%%%%%%%%%%%%%%%%%%%%%%%%%%%%


\newcommand{\thm}[2]{\begin{Theorem}{#1}{}#2\end{Theorem}}
\newcommand{\cor}[2]{\begin{Corollary}{#1}{}#2\end{Corollary}}
\newcommand{\mlenma}[2]{\begin{Lenma}{#1}{}#2\end{Lenma}}
\newcommand{\mprop}[2]{\begin{Prop}{#1}{}#2\end{Prop}}
\newcommand{\clm}[3]{\begin{claim}{#1}{#2}#3\end{claim}}
\newcommand{\wc}[2]{\begin{wconc}{#1}{}\setlength{\parindent}{1cm}#2\end{wconc}}
\newcommand{\thmcon}[1]{\begin{Theoremcon}{#1}\end{Theoremcon}}
\newcommand{\ex}[2]{\begin{Example}{#1}{}#2\end{Example}}
\newcommand{\dfn}[2]{\begin{Definition}[colbacktitle=red!75!black]{#1}{}#2\end{Definition}}
\newcommand{\dfnc}[2]{\begin{definition}[colbacktitle=red!75!black]{#1}{}#2\end{definition}}
\newcommand{\qs}[2]{\begin{question}{#1}{}#2\end{question}}
\newcommand{\pf}[2]{\begin{myproof}[#1]#2\end{myproof}}
\newcommand{\nt}[1]{\begin{note}#1\end{note}}

\newcommand*\circled[1]{\tikz[baseline=(char.base)]{
		\node[shape=circle,draw,inner sep=1pt] (char) {#1};}}
\newcommand\getcurrentref[1]{%
	\ifnumequal{\value{#1}}{0}
	{??}
	{\the\value{#1}}%
}
\newcommand{\getCurrentSectionNumber}{\getcurrentref{section}}
\newenvironment{myproof}[1][\proofname]{%
	\proof[\bfseries #1: ]%
}{\endproof}

\newcommand{\mclm}[2]{\begin{myclaim}[#1]#2\end{myclaim}}
\newenvironment{myclaim}[1][\claimname]{\proof[\bfseries #1: ]}{}

\newcounter{mylabelcounter}

\makeatletter
\newcommand{\setword}[2]{%
	\phantomsection
	#1\def\@currentlabel{\unexpanded{#1}}\label{#2}%
}
\makeatother




\tikzset{
	symbol/.style={
			draw=none,
			every to/.append style={
					edge node={node [sloped, allow upside down, auto=false]{$#1$}}}
		}
}


% deliminators
\DeclarePairedDelimiter{\abs}{\lvert}{\rvert}
\DeclarePairedDelimiter{\norm}{\lVert}{\rVert}

\DeclarePairedDelimiter{\ceil}{\lceil}{\rceil}
\DeclarePairedDelimiter{\floor}{\lfloor}{\rfloor}
\DeclarePairedDelimiter{\round}{\lfloor}{\rceil}

\newsavebox\diffdbox
\newcommand{\slantedromand}{{\mathpalette\makesl{d}}}
\newcommand{\makesl}[2]{%
\begingroup
\sbox{\diffdbox}{$\mathsurround=0pt#1\mathrm{#2}$}%
\pdfsave
\pdfsetmatrix{1 0 0.2 1}%
\rlap{\usebox{\diffdbox}}%
\pdfrestore
\hskip\wd\diffdbox
\endgroup
}
\newcommand{\dd}[1][]{\ensuremath{\mathop{}\!\ifstrempty{#1}{%
\slantedromand\@ifnextchar^{\hspace{0.2ex}}{\hspace{0.1ex}}}%
{\slantedromand\hspace{0.2ex}^{#1}}}}
\ProvideDocumentCommand\dv{o m g}{%
  \ensuremath{%
    \IfValueTF{#3}{%
      \IfNoValueTF{#1}{%
        \frac{\dd #2}{\dd #3}%
      }{%
        \frac{\dd^{#1} #2}{\dd #3^{#1}}%
      }%
    }{%
      \IfNoValueTF{#1}{%
        \frac{\dd}{\dd #2}%
      }{%
        \frac{\dd^{#1}}{\dd #2^{#1}}%
      }%
    }%
  }%
}
\providecommand*{\pdv}[3][]{\frac{\partial^{#1}#2}{\partial#3^{#1}}}
%  - others
\DeclareMathOperator{\Lap}{\mathcal{L}}
\DeclareMathOperator{\Var}{Var} % varience
\DeclareMathOperator{\Cov}{Cov} % covarience
\DeclareMathOperator{\E}{E} % expected

% Since the amsthm package isn't loaded

% I prefer the slanted \leq
\let\oldleq\leq % save them in case they're every wanted
\let\oldgeq\geq
\renewcommand{\leq}{\leqslant}
\renewcommand{\geq}{\geqslant}

% % redefine matrix env to allow for alignment, use r as default
% \renewcommand*\env@matrix[1][r]{\hskip -\arraycolsep
%     \let\@ifnextchar\new@ifnextchar
%     \array{*\c@MaxMatrixCols #1}}


%\usepackage{framed}
%\usepackage{titletoc}
%\usepackage{etoolbox}
%\usepackage{lmodern}


%\patchcmd{\tableofcontents}{\contentsname}{\sffamily\contentsname}{}{}

%\renewenvironment{leftbar}
%{\def\FrameCommand{\hspace{6em}%
%		{\color{myyellow}\vrule width 2pt depth 6pt}\hspace{1em}}%
%	\MakeFramed{\parshape 1 0cm \dimexpr\textwidth-6em\relax\FrameRestore}\vskip2pt%
%}
%{\endMakeFramed}

%\titlecontents{chapter}
%[0em]{\vspace*{2\baselineskip}}
%{\parbox{4.5em}{%
%		\hfill\Huge\sffamily\bfseries\color{myred}\thecontentspage}%
%	\vspace*{-2.3\baselineskip}\leftbar\textsc{\small\chaptername~\thecontentslabel}\\\sffamily}
%{}{\endleftbar}
%\titlecontents{section}
%[8.4em]
%{\sffamily\contentslabel{3em}}{}{}
%{\hspace{0.5em}\nobreak\itshape\color{myred}\contentspage}
%\titlecontents{subsection}
%[8.4em]
%{\sffamily\contentslabel{3em}}{}{}  
%{\hspace{0.5em}\nobreak\itshape\color{myred}\contentspage}



%%%%%%%%%%%%%%%%%%%%%%%%%%%%%%%%%%%%%%%%%%%
% TABLE OF CONTENTS
%%%%%%%%%%%%%%%%%%%%%%%%%%%%%%%%%%%%%%%%%%%

\usepackage{tikz}
\definecolor{doc}{RGB}{0,60,110}
\usepackage{titletoc}
\contentsmargin{0cm}
\titlecontents{chapter}[3.7pc]
{\addvspace{30pt}%
	\begin{tikzpicture}[remember picture, overlay]%
		\draw[fill=doc!60,draw=doc!60] (-7,-.1) rectangle (-0.9,.5);%
		\pgftext[left,x=-3.5cm,y=0.2cm]{\color{white}\Large\sc\bfseries Chapter\ \thecontentslabel};%
	\end{tikzpicture}\color{doc!60}\large\sc\bfseries}%
{}
{}
{\;\titlerule\;\large\sc\bfseries Page \thecontentspage
	\begin{tikzpicture}[remember picture, overlay]
		\draw[fill=doc!60,draw=doc!60] (2pt,0) rectangle (4,0.1pt);
	\end{tikzpicture}}%
\titlecontents{section}[3.7pc]
{\addvspace{2pt}}
{\contentslabel[\thecontentslabel]{2pc}}
{}
{\hfill\small \thecontentspage}
[]
\titlecontents*{subsection}[3.7pc]
{\addvspace{-1pt}\small}
{}
{}
{\ --- \small\thecontentspage}
[ \textbullet\ ][]

\makeatletter
\renewcommand{\tableofcontents}{%
	\chapter*{%
	  \vspace*{-20\p@}%
	  \begin{tikzpicture}[remember picture, overlay]%
		  \pgftext[right,x=15cm,y=0.2cm]{\color{doc!60}\Huge\sc\bfseries \contentsname};%
		  \draw[fill=doc!60,draw=doc!60] (13,-.75) rectangle (20,1);%
		  \clip (13,-.75) rectangle (20,1);
		  \pgftext[right,x=15cm,y=0.2cm]{\color{white}\Huge\sc\bfseries \contentsname};%
	  \end{tikzpicture}}%
	\@starttoc{toc}}
\makeatother


%From M275 "Topology" at SJSU
\newcommand{\id}{\mathrm{id}}
\newcommand{\taking}[1]{\xrightarrow{#1}}
\newcommand{\inv}{^{-1}}

%From M170 "Introduction to Graph Theory" at SJSU
\DeclareMathOperator{\diam}{diam}
\DeclareMathOperator{\ord}{ord}
\newcommand{\defeq}{\overset{\mathrm{def}}{=}}

%From the USAMO .tex files
\newcommand{\ts}{\textsuperscript}
\newcommand{\dg}{^\circ}
\newcommand{\ii}{\item}

% % From Math 55 and Math 145 at Harvard
% \newenvironment{subproof}[1][Proof]{%
% \begin{proof}[#1] \renewcommand{\qedsymbol}{$\blacksquare$}}%
% {\end{proof}}

\newcommand{\liff}{\leftrightarrow}
\newcommand{\lthen}{\rightarrow}
\newcommand{\opname}{\operatorname}
\newcommand{\surjto}{\twoheadrightarrow}
\newcommand{\injto}{\hookrightarrow}
\newcommand{\On}{\mathrm{On}} % ordinals
\DeclareMathOperator{\img}{im} % Image
\DeclareMathOperator{\Img}{Im} % Image
\DeclareMathOperator{\coker}{coker} % Cokernel
\DeclareMathOperator{\Coker}{Coker} % Cokernel
\DeclareMathOperator{\Ker}{Ker} % Kernel
\DeclareMathOperator{\rank}{rank}
\DeclareMathOperator{\Spec}{Spec} % spectrum
\DeclareMathOperator{\Tr}{Tr} % trace
\DeclareMathOperator{\pr}{pr} % projection
\DeclareMathOperator{\ext}{ext} % extension
\DeclareMathOperator{\pred}{pred} % predecessor
\DeclareMathOperator{\dom}{dom} % domain
\DeclareMathOperator{\ran}{ran} % range
\DeclareMathOperator{\Hom}{Hom} % homomorphism
\DeclareMathOperator{\Mor}{Mor} % morphisms
\DeclareMathOperator{\End}{End} % endomorphism

\newcommand{\eps}{\epsilon}
\newcommand{\veps}{\varepsilon}
\newcommand{\ol}{\overline}
\newcommand{\ul}{\underline}
\newcommand{\wt}{\widetilde}
\newcommand{\wh}{\widehat}
\newcommand{\vocab}[1]{\textbf{\color{blue} #1}}
\providecommand{\half}{\frac{1}{2}}
\newcommand{\dang}{\measuredangle} %% Directed angle
\newcommand{\ray}[1]{\overrightarrow{#1}}
\newcommand{\seg}[1]{\overline{#1}}
\newcommand{\arc}[1]{\wideparen{#1}}
\DeclareMathOperator{\cis}{cis}
\DeclareMathOperator*{\lcm}{lcm}
\DeclareMathOperator*{\argmin}{arg min}
\DeclareMathOperator*{\argmax}{arg max}
\newcommand{\cycsum}{\sum_{\mathrm{cyc}}}
\newcommand{\symsum}{\sum_{\mathrm{sym}}}
\newcommand{\cycprod}{\prod_{\mathrm{cyc}}}
\newcommand{\symprod}{\prod_{\mathrm{sym}}}
\newcommand{\Qed}{\begin{flushright}\qed\end{flushright}}
\newcommand{\parinn}{\setlength{\parindent}{1cm}}
\newcommand{\parinf}{\setlength{\parindent}{0cm}}
% \newcommand{\norm}{\|\cdot\|}
\newcommand{\inorm}{\norm_{\infty}}
\newcommand{\opensets}{\{V_{\alpha}\}_{\alpha\in I}}
\newcommand{\oset}{V_{\alpha}}
\newcommand{\opset}[1]{V_{\alpha_{#1}}}
\newcommand{\lub}{\text{lub}}
\newcommand{\del}[2]{\frac{\partial #1}{\partial #2}}
\newcommand{\Del}[3]{\frac{\partial^{#1} #2}{\partial^{#1} #3}}
\newcommand{\deld}[2]{\dfrac{\partial #1}{\partial #2}}
\newcommand{\Deld}[3]{\dfrac{\partial^{#1} #2}{\partial^{#1} #3}}
\newcommand{\lm}{\lambda}
\newcommand{\uin}{\mathbin{\rotatebox[origin=c]{90}{$\in$}}}
\newcommand{\usubset}{\mathbin{\rotatebox[origin=c]{90}{$\subset$}}}
\newcommand{\lt}{\left}
\newcommand{\rt}{\right}
\newcommand{\bs}[1]{\boldsymbol{#1}}
\newcommand{\exs}{\exists}
\newcommand{\st}{\strut}
\newcommand{\dps}[1]{\displaystyle{#1}}

\newcommand{\sol}{\setlength{\parindent}{0cm}\textbf{\textit{Solution:}}\setlength{\parindent}{1cm} }
\newcommand{\solve}[1]{\setlength{\parindent}{0cm}\textbf{\textit{Solution: }}\setlength{\parindent}{1cm}#1 \Qed}

%--------------------------------------------------
% LIE ALGEBRAS
%--------------------------------------------------
\newcommand*{\kb}{\mathfrak{b}}  % Borel subalgebra
\newcommand*{\kg}{\mathfrak{g}}  % Lie algebra
\newcommand*{\kh}{\mathfrak{h}}  % Cartan subalgebra
\newcommand*{\kn}{\mathfrak{n}}  % Nilradical
\newcommand*{\ku}{\mathfrak{u}}  % Unipotent algebra
\newcommand*{\kz}{\mathfrak{z}}  % Center of algebra

%--------------------------------------------------
% HOMOLOGICAL ALGEBRA
%--------------------------------------------------
\DeclareMathOperator{\Ext}{Ext} % Ext functor
\DeclareMathOperator{\Tor}{Tor} % Tor functor

%--------------------------------------------------
% MATRIX & GROUP NOTATION
%--------------------------------------------------
\DeclareMathOperator{\GL}{GL} % General Linear Group
\DeclareMathOperator{\SL}{SL} % Special Linear Group
\newcommand*{\gl}{\operatorname{\mathfrak{gl}}} % General linear Lie algebra
\newcommand*{\sl}{\operatorname{\mathfrak{sl}}} % Special linear Lie algebra

%--------------------------------------------------
% NUMBER SETS
%--------------------------------------------------
\newcommand*{\RR}{\mathbb{R}}
\newcommand*{\NN}{\mathbb{N}}
\newcommand*{\ZZ}{\mathbb{Z}}
\newcommand*{\QQ}{\mathbb{Q}}
\newcommand*{\CC}{\mathbb{C}}
\newcommand*{\PP}{\mathbb{P}}
\newcommand*{\HH}{\mathbb{H}}
\newcommand*{\FF}{\mathbb{F}}
\newcommand*{\EE}{\mathbb{E}} % Expected Value

%--------------------------------------------------
% MATH SCRIPT, FRAKTUR, AND BOLD SYMBOLS
%--------------------------------------------------
\newcommand*{\mcA}{\mathcal{A}}
\newcommand*{\mcB}{\mathcal{B}}
\newcommand*{\mcC}{\mathcal{C}}
\newcommand*{\mcD}{\mathcal{D}}
\newcommand*{\mcE}{\mathcal{E}}
\newcommand*{\mcF}{\mathcal{F}}
\newcommand*{\mcG}{\mathcal{G}}
\newcommand*{\mcH}{\mathcal{H}}

\newcommand*{\mfA}{\mathfrak{A}}  \newcommand*{\mfB}{\mathfrak{B}}
\newcommand*{\mfC}{\mathfrak{C}}  \newcommand*{\mfD}{\mathfrak{D}}
\newcommand*{\mfE}{\mathfrak{E}}  \newcommand*{\mfF}{\mathfrak{F}}
\newcommand*{\mfG}{\mathfrak{G}}  \newcommand*{\mfH}{\mathfrak{H}}

\usepackage{bm} % Ensure bold math works correctly
\newcommand*{\bmA}{\bm{A}}
\newcommand*{\bmB}{\bm{B}}
\newcommand*{\bmC}{\bm{C}}
\newcommand*{\bmD}{\bm{D}}
\newcommand*{\bmE}{\bm{E}}
\newcommand*{\bmF}{\bm{F}}
\newcommand*{\bmG}{\bm{G}}
\newcommand*{\bmH}{\bm{H}}

%--------------------------------------------------
% FUNCTIONAL ANALYSIS & ALGEBRA
%--------------------------------------------------
\DeclareMathOperator{\Aut}{Aut} % Automorphism group
\DeclareMathOperator{\Inn}{Inn} % Inner automorphisms
\DeclareMathOperator{\Syl}{Syl} % Sylow subgroups
\DeclareMathOperator{\Gal}{Gal} % Galois group
\DeclareMathOperator{\sign}{sign} % Sign function


%\usepackage[tagged, highstructure]{accessibility}
\usepackage{tocloft}
\usepackage{arydshln}
\usetikzlibrary{arrows.meta, decorations.pathreplacing}
\usepackage{tikz-cd}
\usepackage{polynom}
\usepackage{pifont}
\newcommand{\pistar}{{\zf\symbol{"4A}}}
% a tiny helper for a stretched phantom (for the underbrace)
\newcommand\mc[1]{\multicolumn{1}{c}{#1}}



\begin{document}
\title{Linear Algebra I}
\author{Lecture Notes Provided by Dr.~Miriam Logan.}
\date{}
\maketitle
\tableofcontents
\newpage  
 Suppose $ T: V \to V$ is a linear operator defined on the complex vector space $ V$, and let $ p \left( x \right) \in \mathcal{P} \left[ x \right] $, where $ \mathcal{P} \left[ x \right] = \left\{ \sum\limits_{i=1}^{k} a_i x^i \mid a_i \in \mathbb{C}, k \in \mathbb{N} \right\} $ .\\
 For all polynomials $ p \left( x \right) \in \mathcal{P} \left[ x \right] $, $ p \left( T \right)  $ is a linear operator defined on $ V $ (follows from the fact taht $ T $ is linear).\\
 \[
 p \left( T \right) :V \to V
 .\] 
 \[
 p \left( T \right) = \left( \sum\limits_{i=1}^{n} a_i T^i \right) \left( \vec{ v}  \right) = \sum\limits_{i=1}^{n} a_i T^i \left( \vec{ v}  \right) 
  \right) 
 .\] 
 \thm{}
 {
 $ ker \left( p \left( T \right)  \right)  $ and $ Im \left( p \left( T \right)  \right) $ are invariant under $ T $.\\
 }
 \pf{Proof:}{
  Suppose $ \vec{ v} \in ker \left( p \left( T \right)  \right) $    then $ p \left( T \right) \left( \vec{ v}  \right) = \vec{ 0}  $ .\\
  Thus 
  \[
  p \left( T \right) \left( T \left( \vec{ v}  \right)  \right) = T \left( p \left( T \right) \left( \vec{ v}  \right)  \right) = T \left( \vec{ 0}  \right) = \vec{ 0}
  .\] 
  \[
  \text{ i.e. }     T \left( \vec{ v}  \right) \in ker \left( p \left( T \right)  \right) \implies ker \left( p \left( T \right)  \right) \text{ is invariant under } T
  .\] 
  Suppose $ \vec{ w} \in Im \left( p \left( T \right)  \right) $. Then $ \exists  \vec{ u} \in V$ such that $ p \left( T \right) \left( \vec{ u}  \right) = \vec{ w} $.\\
  \[
  T \left( \vec{ w}  \right) = T \left( p \left( T \right) \left( \vec{ u}  \right)  \right) = p \left( T \right) \left( T \left( \vec{ u}  \right)  \right)
  .\] 
  \[
  \implies T \left( \vec{ w}  \right) \in Im \left( p \left( T \right)  \right) \qquad  \implies Im \left( p \left( T \right)  \right) \text{ is invariant under } T
  .\]
  
 }
  \thm{}
  {
  Let $ n = dim V$, $ T: V \to V$ be a linear operator, 
  \[
   V = ker \left( T ^{n} \right) \oplus Im \left( T ^{n} \right)
  .\] 
  }
  \dfn{ :}{
  Suppose $ T: V \to V$ is a linear operator on the vector space $ V$ . Suppose $ U$ is an invariant subspace under $ T$, i.e. $ T \left( \vec{ u}  \right) \in U \qquad  \forall \vec{ u} \in U$. We define $ T \bigg|_{u}^{} $ to be the restriction of $ T$ to $ U$. $ T \bigg|_{u}^{} $ is a linear operator defined on $ U$.
  \[
  T \bigg|_{U}^{} \left( \vec{ u}  \right) = T \left( \vec{ u}  \right) \qquad  \forall \vec{ u} \in U
  .\] 
  Since $ T$ preserves addition and scalar multiplication, $ T \left( \vec{ u}  \right) \in U \qquad  \forall  \vec{ u} \in U$
  }

  \thm{Jordan Decomposition Theorem}
  {
      Suppose $ V$ is a complex vector space and $ T: V \to V$ is a linear operator.  Let $ \lambda_1, \ldots , \lambda_m$ be the distinct eigenvalues of $ T$. Then $ V \bigoplus_{i=1} ^{m} V \left( \lambda_i \right) $, where $ V \left( \lambda_i \right) $ is the generalized eigenspace associated with $ \lambda_i$.
  }
  \pf{Proof:}{
  We'll prove the result using induction on $ n = dim V$. When $ n=1$, $ T$ has one eigenvalue $ \lambda_1$ (fundamental theorem of algebra) and one linearly independent eigenvector and $ V = V \left( \lambda_1 \right) $ is the eigenspace associated with $ \lambda_i$.\\  
  We assume $ n>1$ , and that the result holds for all vector spaces of smaller dimension.\\
  Because $ V$ is a complex vector space, $ T$ has at least one eigenvalue, thus $ m \ge 1$. Applying the previous theorem to $ \left( T - \lambda_1 I \right) $ we get that 
  \[
  V = ker \left( T - \lambda_1 I \right) ^{n} \bigoplus Im \left( T - \lambda_1 I \right) ^{n} \qquad  \star
  .\] 
  \[
  \text{ i.e. } V = V \left( \lambda_1 \right) \bigoplus Im \left( T - \lambda_1 I \right) ^{n}
  .\] 
  Let $  p \left( z \right) = \left( z - \lambda_1 \right) ^{n} $ is invariant under $ T$ $ \implies$ we can consider $ T \bigg|_{u}^{} $ and since $ ker \left( T - \lambda_1 I \right) ^{n} \neq  { \vec{ 0} } \implies dim U < n$.\\
  Hence we can apply our inductive hypothesis to $ T \bigg|_{u}^{} $.\\
  None of the generalized eigenvectors of $ T \bigg|_{u}^{} $ correspond to $ \lambda_1$ since all generalized eigenvectors corresponding to $ \lambda_1$ are in $ ker \left( T - \lambda_1 I \right) ^{n} = V \left( \lambda_1 \right) $.\\
  Thus each eigenvalue of $ T \bigg|_{u}^{} $ is in $ \left\{ \lambda_2 , \ldots , \lambda_m \right\} $.\\
  By induction $ U = \bigoplus_{i=2} ^{m} ker \left( T \bigg|_{u}^{} - \lambda_i I \right) ^{n}$.\\
  Need to show that $ ker \left( T - \lambda_i \right) = ker \left( T \bigg|_{u}^{} - \lambda_i I \right)  ^{n} \qquad  \forall  i$ .\\
  Clearly $ ker \left( T - \lambda_i I \right)  ^{n} \subseteq ker \left( T \bigg|_{u}^{} - \lambda_i I \right)  ^{n} \qquad \forall  i $.\\
  For the opposite direction, fix $ k$ and suppose.
  $ \vec{ v} \in ker \left( T - \lambda_k I  \right) ^{n}$, i.e.  $ \vec{ v} $ is a generalized eigenvector associated with $ \lambda_k$. $ \vec{ v} \in V \implies$ by  $  \star$ $ \vec{ v} = \vec{ v_1} + \vec{ u} $ where $ \vec{ v_1} \in V \left( \lambda_1 \right) $  and $ \vec{ u} \in Im \left( T - \lambda_1 I \right) ^{n} = \bigoplus_{i=2} ^{m} ker \left( T \bigg|_{u}^{} - \lambda_i I \right)^{n} $ since the sum is direct, and $ \vec{ v}  \notin V \left( \lambda_i \right) $.\\
  Hence $ V = \bigoplus_{i=1} ^{m} ker \left( T - \lambda_i I \right) ^{n} = \bigoplus_{i=1} ^{m} V \left( \lambda_i \right) $
  }
  \nt{
  In general to show that a transformation  $ T: V \to W$ between complex vector spaces $ V, W$ is linear, it suffices to show that 
  \[
  T \left( \alpha \vec{ v_1} + \vec{ v_2}  \right) = \alpha T \left( \vec{ v_1}  \right) + T \left( \vec{ v_2}  \right) \qquad  \forall  \alpha \in \mathbb{C}, \vec{ v_1} , \vec{ v_2} \in V
  .\] 
  }



  \section{Bilinear Forms}
  \dfn{Bilinear Form :}{
  A bilinear form on a real vector space $ V$ is a function $ f : V \times  V \to \mathbb{R} $ that in linear in each variable i.e. 
  \[
  f \left( \vec{ v_1} + \vec{ v_2} , \vec{ w}\right)  = f \left( \vec{ v_1} , \vec{ w}  \right) + f \left( \vec{ v_2} , \vec{ w}  \right) 
  .\] 
  \[
  \text{ and }  f \left( \lambda \vec{ v_1} , \vec{ w} \right)  = \lambda f \left( \vec{ v} ,\vec{ w}  \right) 
  .\] 
  (linear in first variable)\\
  also,
  \[
  f \left( \vec{ v} , \vec{ w_1} +\vec{ w_2}  \right) = f \left( \vec{ v_1} , \vec{ w_1}  \right) + f \left( \vec{ v} , \vec{ w_2}  \right) 
  .\] 
  \[
   \text{ and }  f \left( \vec{ v} , \lambda \vec{w}\right) = \lambda f \left( \vec{ v} , \vec{ w}  \right)
  .\] 
  (linear in second variable)\\
  }
 \ex{Dot Product on $  \mathbb{R} ^{n}$}{
 \[
 f : \mathbb{R} ^{n} \times  \mathbb{R} ^{n} \to \mathbb{R}
 .\] 
 \[
 f \left( \vec{ x} , \vec{ y}  \right) = \vec{ x} \cdot \vec{ y} = \sum\limits_{i=1}^{n} x_i y_i
 .\] 
 We could represent the dot product as follows 
 \[
 f \left( \vec{ x} , \vec{ y}  \right) = \left( \vec{ x}  \right) ^{ T} \left( \vec{ y}  \right) = \begin{bmatrix}
 x_1 & x_2 & \ldots  & xn \\
 \end{bmatrix}
 \begin{bmatrix}
 y_1\\
 y_2\\
 \vdots\\
 y_n\\
 \end{bmatrix}
    = \sum\limits_{i=1}^{n} x_i y_i
 .\] 
 Bilinear? \\
 Let $ \vec{ x} , \vec{ w} , \vec{ y} \in \mathbb{R} ^n$

 \begin{align*}
  f \left( \vec{ x} + \vec{ w} , \vec{ y}  \right) &= \left( \vec{ x} + \vec{ w}  \right) ^{T} \left( \vec{ y}  \right) = \left( \vec{ x} ^{T}+ \vec{ w} ^{ T} \right) \left(  \vec{ y}  \right) \\
 &= \vec{ x} ^{T} \vec{ y} + \vec{ w} ^{T} \vec{ y} \\
 &= f \left( \vec{ x} , \vec{ y}  \right) + f \left( \vec{ w} , \vec{ y}  \right)
 .\end{align*}

 Let $ \lambda \in \mathbb{R}$, $  \vec{ x} , \vec{ y} \in \mathbb{R} ^{n} $
 \begin{align*}
  f \left( \lambda \vec{ x} , \vec{ y}  \right) &= \left( \lambda \vec{ x}  \right) ^{T} \left( \vec{ y}  \right) = \lambda \left( \vec{ x} ^{T}  \right) \left( \vec{ y}  \right) \\
  &= \lambda f \left( \vec{ x} , \vec{ y}  \right) = \lambda f \left( \vec{ x} , \vec{ y}  \right)
 .\end{align*}
 $ \implies$ linear in the first component. \\
 Similarly it can be shown that 
 \[
 f \left( \vec{ x} , \lambda \vec{ w} + \vec{ y}  \right) = \lambda f \left( \vec{ x} ,\vec{ w}  \right) + f \left( \vec{ x} , \vec{ y}  \right)
 .\] 
 }
 \ex{}{
 Suppose $ A \in M _{ n\times  n} \left( \mathbb{R} \right) . f : \mathbb{R} ^{n} \times  \mathbb{R} ^{n} \to \mathbb{R}$ defined by $ f \left( \vec{ x} , \vec{ y}  \right) = \left( \vec{ x}  \right) ^{T} A \vec{ y} $ is a bilinear form.\\
 \\
  $ 1 ^{ \text{ st}} $ variable: Let $  \alpha , \vec{ c_1} , \vec{ x_2}, \vec{ y} \in \mathbb{R} ^{n}$
  \begin{align*}
   f \left( \alpha \vec{ x_1}+ \vec{ x_2}, \vec{ y}    \right) &= \left( \alpha \vec{ x_1} + \vec{ x_2}  \right) ^{T} A \vec{ y}\\
   &= \left( \alpha \vec{ x_1}  \right) ^{T} A \vec{ y} + \left( \vec{ x_2}  \right) ^{T} A \vec{ y}  \\
   &= \alpha f \left( \vec{ x_1} , \vec{ y}  \right) + f \left( \vec{ x_2} , \vec{ y}  \right) 
  .\end{align*}
  Similarly it can be shown that $ f$ is linear in the second variable.\\
  Suppose the $ (i,j) $ entry of $ A$ is $ a_{ij}$.\\
  Fix $ \mathcal{E} = \left\{ \vec{ e_1}, \ldots , \vec{ e_n}  \right\} $ as the standard basis of $ \mathbb{R} ^{n}$.\\
  \[
  \left( \vec{ x} = \sum\limits_{i=1}^{n} x_i \vec{ e_i} , \qquad  \vec{ y} = \sum\limits_{j=1}^{n} y_j \vec{ e_j}  \right)
  .\] 
  \[
  \text{ then } \left( \vec{ x}  \right) ^{T} A \vec{ y} = \left( \sum\limits_{i=1}^{n} x_i \vec{ e_i}  \right) ^{T} A \left( \sum\limits_{j=1}^{n} y_j \vec{ e_j}  \right) = \sum\limits_{1 \leq i,j \leq n} x_i y_j \left( \vec{ e_i}  \right) ^{T} A \vec{ e_j}
   \right) 
  .\] 
  Note:
  \[
  \vec{ e_i} ^{T}A =  \bigl[\,0\;\cdots\;0\;
  \underset{\text{$i$-th pos.}}{1}\;
  0\;\cdots\;0\,\bigr]
\begin{bmatrix}
  a_{11} & \cdots & a_{1n} \\[2pt]
  \vdots & \ddots & \vdots \\[2pt]
  a_{n1} & \cdots & a_{nn}
\end{bmatrix}         = \begin{bmatrix}
a_{ i 1} & a_{ i 2} & \ldots & a_{ i n} \\
\end{bmatrix} = \text{ $i$-th row of } A
  .\] 

   \[
\bigl(\mathbf e_i^{\mathsf T}A\bigr)\,\mathbf e_j
  \;=\;
\begin{bmatrix}
  a_{i1} & a_{i2} & \cdots & a_{in}
\end{bmatrix}
\!
\underbrace{%
\begin{bmatrix}
  0 \\[-2pt]
  \vdots \\[-2pt]
  1 \\[-2pt]
  \vdots \\[-2pt]
  0
\end{bmatrix}}_{\text{$j$-th position}}       = a_{ i j}
\]


  \[
  \implies \vec{ x}  ^{T} A \vec{ y} = \sum\limits_{1 \leq i,j \leq n} x_i y_j a_{ij} 
  .\] 
  where $ a_{ i j} = \left( \vec{ e_i}  \right) ^{T} A \left( \vec{ e_j}  \right) = f \left( \vec{ e_i} , \vec{ e_j}  \right) $
 }
 
 \thm{}
 {
 Let $ \vec{ x} , \vec{ y} \in \mathbb{R} ^{n}, \vec{ x}  = \sum\limits_{i=1}^{n} x_i \vec{ e_i} , \qquad \vec{ y }  = \sum\limits_{j=1}^{n} y_j \vec{ e_j} $. Every bilinear form of $ \mathbb{R} ^{n}$ has the form $ f \left( \vec{ x} ,\vec{ y}  \right) = \sum\limits_{1 \eq i,j \leq n}^{}  a_{ i j } x_i y_j $, where $ a_{ i j} = f \left( \vec{ e_i} , \vec{ e_j}  \right) $.

 }
  
\pf{Proof:}{
 \begin{align*}
  f \left( \vec{ x} ,\vec{ y}  \right) &= f \left( \sum\limits_{i=1}^{n} x_i \vec{ e_i} , \sum\limits_{j=1}^{n} y_j \vec{ e_j}  \right) \\
  &= \sum\limits_{i=1}^{n} x_i f \left( \vec{ e_i} , \sum\limits_{j=1}^{n} y_j \vec{ e_j} 
   \right) \\
   &= \sum\limits_{i=1}^{n} \sum\limits_{j=1}^{n} x_i y_j f \left( \vec{ e_i} , \vec{ e_j}  \right) \\
 .\end{align*}
 Note:    the $ (i,j) $ entry of the matrix of the dot product is $ \vec{ e_i} \cdot  \vec{ e_j} $ and
 \[
 \vec{ e_i} \cdot  \vec{ e_j} = \begin{cases}
   1& \text{if } i=j \\
  0 & i \neq j
 \end{cases}
 .\]

}
 Hence the matrix of the dot product is the identity matrix.\\
 \ex{}{
  Let $ V = \mathcal{P} _2 \left[ x \right] $. Define $ \langle   \rangle : V \times  V \to \mathbb{R}$ by 
  \[
  \langle p,q  \rangle = \int_{0}^{1} p \left( x  \right) q \left( x  \right) dx \qquad  \forall  p,q \in \mathcal{P}_2 \left[ x \right]
  .\] 
 \underline{Claim} \\
 $ \langle p,q  \rangle $ is a bilinear form.\\
 \pf{Proof:}{
  Let $ \alpha \in \mathbb{R}$, $ p_1,p_2,q \in \mathcal{P} _2 \left[ x \right] $
  \begin{align*}
   \langle \alpha p_1 + p_2 , q  \rangle &- \int_{0}^{1} \left( \alpha p_1 + p_2  \right) q \left( x  \right) dx \\
   &= \int_{0}^{1} \alpha p_1 \left( x  \right) q \left( x  \right) + p_2 \left( x  \right) q \left( x  \right) dx \\
   &= \alpha \int_{0}^{1} p_1 \left( x  \right) q \left( x  \right) dx + \int_{0}^{1} p_2 \left( x  \right) q \left( x  \right) dx \\
   &= \alpha \langle p_1,q  \rangle + \langle p_2,q  \rangle
  .\end{align*}
  Similarly it can be shown that 
  \[
  \langle q , \alpha p_1 + p_2  \rangle = \alpha \langle q,p_1  \rangle + \langle q,p_2  \rangle
  .\] 
  The matrix of the bilinear form with respect to the basis $ \mathcal{B} = \left\{ 1,x,x^2  \right\} $ of $ \mathcal{P} _2 \left[ x \right] $ is given by
  \[
   A = \left[ a_{ i j } \right] _{ 1 \leq i,j \leq 3}  \qquad  \text{ where } a_{ i j} = \langle b_i,b_j  \rangle
  .\] 
  Note that $ \vec{ b_1} = x^{0}$ , $ \vec{ b_2} = x^{1}$ , $ \vec{ b_3} = x^{2}$  and $ \vec{ b_i} = x ^{i-1}$ \\
  \begin{align*}
   \langle \vec{ b_i} , \vec{ b_j}   \rangle = \langle x ^{i-1}, x ^{ j-1}  \rangle &= \int_{0}^{1} x ^{ i-1} x ^{ j-1} dx \\
   &= \int_{0}^{1} x ^{ i + j -2} dx = \left[ \frac{x ^{ i + j -1}}{i + j -1} \right] _{0}^{1} = \frac{1}{i + j -1}
  .\end{align*}
  \[
  \implies A = \begin{bmatrix}
  1 & \frac{1}{2} & \frac{1}{3} \\
   \frac{1}{2}& \frac{1}{3} & \frac{1}{4} \\
   \frac{1}{3}& \frac{1}{4} & \frac{1}{5}\\
  \end{bmatrix}
  .\] 
 }
 
  
  
  
}
  \ex{}{
  Let $ V = \mathbb{R} ^2$ \\
  \begin{align*}
   \langle  \vec{ x} , \vec{ y}   \rangle = 2 x_1 y_1 + x_1 y_2 + 3 x_2 y_1 + 4 x_2 y_2  \\
   = \left[ \vec{ x}  \right] ^{ T}  \begin{bmatrix}
   2 & 1\\
   3 & 4\\
  \end{bmatrix} \left[ \vec{ y}  \right] 
  .\end{align*}
  \[
  \implies \langle ,  \rangle \text{ is a bilinear form on } \mathbb{R} ^2 \times  \mathbb{R} ^2
  .\] 
  }
  \section{Change of Basis and how its affects on the matric of a bilinear form}
  Suppose $ \langle ,  \rangle $ is a bilinear form on  $ \mathbb{R} ^{n}$ and $ A$ is the matrix of $ \langle ,  \rangle $ with respect to the standard basis $ \mathcal{E} = \left\{ \vec{ e_1}, \ldots , \vec{ e_n}  \right\} $\\
  i.e.  $ \left[ A \right] _{ i j } = \langle \vec{ e_i} , \vec{ e_j}   \rangle $ \\
  Suppose $ \mathcal{B} = \left\{ \vec{ b_1}, \ldots , \vec{ b_n}  \right\} $ is another basis of $ \mathbb{R} ^{n}$, and 
  
    \[
B \;=\;
\bigl[\,\vec b_{1}\;\vec b_{2}\;\dots\;\vec b_{n}\,\bigr]
\;=\;
P_{\mathcal B \to \varepsilon}.
\]

  Then 
  \[
   B \left[ \vec{ x}  \right]  _{ \mathcal{B}} = \left[ \vec{ x}  \right] _{ \mathcal{E}} \qquad  \star
  .\] 
 Hence 
 \begin{align*}
  \langle \vec{ x} , \vec{ y}\rangle  = \left[ \vec{ x}  \right] _{ \mathcal{E}} ^{t} A  \left[ \vec{ y}  \right] _{ \mathcal{E}} \\
  &= \left( \mathcal{B} \left[ \vec{ x}  \right] _{ \mathcal{B}} \right) ^{T} A \left( \mathcal{B} \left[ \vec{ y}  \right] _{ \mathcal{B}} \right) \qquad  \text{ from } \star\\
  &= \left[ \vec{ x}  \right] _{ \mathcal{B}} ^{T} \left( B ^{T} A B \right) \left[ \vec{ y}  \right] _{ \mathcal{B}}\\
 .\end{align*}
 Thus the matrix of $ \langle  , \rangle $ with respect to the basis $ \mathcal{B}$ is $ B ^{T}A B$.\\
 \section{Positive Definite}
  
 \dfn{Positive Definite :}{
   \begin{enumerate}[label=(\arabic*).]  
     \item A bilinear form $ \langle ,  \rangle : V \times  V \to \mathbb{R}$ is said to be \underline{positive definite} if
      \[
      \langle \vec{ v} , \vec{ v}   \rangle \qquad  \forall  \vec{ v} \in V \text{ and } \vec{ v} \neq \vec{ 0}
      .\] 
      Note: this is equivalent to saying that 
      \[
       \langle \vec{ v} , \vec{ v}   \rangle \ge 0 \forall  \vec{ v} \in V \text{ with equality } \iff vec{ v} = \vec{ 0}
      .\] 
     \item               $ A \in M _{ n \times  n} \left( \mathbb{R} \right) $ is said to be \underline{positive definite} if 
      \[
       \vec{ x} ^{T} A \vec{ x} > 0 \qquad  \forall  \vec{ x} \in \mathbb{R} ^{n} \text{ and } \vec{ x} \neq \vec{ 0}
      .\] 
   \end{enumerate}
   A bilinear form $ \langle ,  \rangle$ is positive definite if and only if the matrix form with respect to the standard basis of $ V$ is positive definite.\\
   
 }
 \ex{}{
 Dot product on $ \mathbb{R} ^{n}$ is positive definite since $ \langle \vec{ x} ,\vec{ x}   \rangle = \sum\limits_{i=1}^{n} x_i ^2
 =0\qquad  \forall  \vec{ x} \neq \vec{ 0} $
 }
 \ex{}{
  Let $ V = \mathcal{P} _{n} \left[ x \right]$. Suppose $ a < b$, 
  \[
  \langle f,g  \rangle = \int_{a}^{b} f \left( x  \right) g \left( x  \right) dx
  .\] 
  defined on $ V $ is a positive definite bilinear form
  \[
   \langle f,f  \rangle = \int_{a}^{b} f \left( x  \right) ^2 dx \ge 0 \text{ with equality } \iff f \left( x \right) =0 \text{ on } \left[ a,b \right]
  .\] 
  In general it can be hard to check if a bilinear form is positive definite.\\
 }
 \ex{}{
 $ \langle ,  \rangle : \mathbb{R} ^2 \to \mathbb{R} ^2 \to \mathbb{R}$.\\
 Is $ \langle \vec{ x} , \vec{ y}   \rangle = x_1 y_1 + x_1 y_2 + 3 x_2 y_1 + 6 x_2 y_2$ positive definite?\\
 \begin{align*}
  \langle \vec{ x} , \vec{ x}   \rangle &= x_1 ^2 + 4 x_1 x_2 + 6 x_2 ^2 \\
  &= \left( x_1 + 2 x_2 \right) ^{2} + 2 x_2 ^{2} =0 \\
  &\iff x_1 + 2 x_2 =0 \text{ and } x_2 =0 \\
  & \text{ i.e. } x_1 = 0 \text{ and } x_2 = 0 \\
  & \text{ i.e.  } \vec{ x} = \vec{ 0}
  \implies \langle ,  \rangle  \text{ is positive definite.}
 .\end{align*}
 }
 \ex{}{
    Is $ \langle \vec{ x} , \vec{ y}   \rangle = x_1 y_1 + x_1 y_2 + 3 x_2 y_1 + 3 x_2 y_2$   positive definite?\\
    \begin{align*}
     \langle \vec{ x} ,\vec{ x}   \rangle &= x_1 ^2 + 4 x_1 x_2 + 3 x_2 ^2 \\
     &= \left( x_1 + 2 x_2 \right) ^{2} + x_2 ^{2} \\
    .\end{align*}
    Not positive definite since 
    \[
    \langle  \begin{bmatrix}
    -2\\
    1\\
    \end{bmatrix}
    , \begin{bmatrix}
    -2\\
    1\\
    \end{bmatrix}
      \rangle  = -1 
    .\] 
 }
  \section{ Symmetric Bilinear Forms}
 \dfn{ Symmetric Bilinear Forms :}{
       A bilinear form $ \langle ,  \rangle $ on a real vector space $ V$ is said to be \underline{symmetric} if
       \[
       \langle \vec{ x} ,\vec{ y}   \rangle = \langle  \vec{ y} , \vec{ x}   \rangle \qquad  \forall  \vec{ x} , \vec{ y} \in V
       .\] 
 }
 \ex{}{
 Dot product on $ \mathbb{R} ^{n}$ is symmetric:
 \[
 \langle \vec{ x} ,\vec{ y}   \rangle = \sum\limits_{i=1}^{n} x_i y_i = \sum\limits_{i=1}^{n} y_i x_i = \langle \vec{ y} ,\vec{ x}   \rangle
 .\] 
 }
 
 \ex{}{
 \[
 \langle f,g  \rangle = \int_{a}^{b} f \left( x  \right) g \left( x  \right) dx \text{ defined on } \mathcal{P} _{n} \left[ x \right] \text{ is symmetric.}
 .\] 
 }
 
 \underline{Fact:} \\
 A bilinear form $ \langle ,  \rangle$ defined on a real vector space $ V$ is symmetric if and only if the matrix of the bilinear form with respect to some basis of $ V$ is symmetric.\\
 ( An $n \times n$  matrix $ A$ is symmetric if $ A = A ^{T}$, i.e. $ a_{ij} = a_{ji} \qquad  \forall  i,j$.)\\
   
 \pf{Proof:}{
  suppose $ A$ is the matrix of $ \langle ,  \rangle $ with respect to the basis $ \mathcal{B} $ of $ V$.\\
  \[
   \langle \vec{ x} ,\vec{ y}   \rangle = \left( \left[ \vec{ x}  \right] _{ \mathcal{B}}\right)  ^{T} A \left( \left[ \vec{ y}  \right] _{ \mathcal{B}} \right) = \sum\limits_{1 \leq i , j \leq n}^{}  x_i y_j a_{ij}
  .\] 
  \[
  \langle \vec{ y} ,\vec{ x}   \rangle = \left( \left[ \vec{ y}  \right] _{ \mathcal{B}}\right)  ^{T} A \left( \left[ \vec{ x}  \right] _{ \mathcal{B}} \right) = \sum\limits_{1 \leq i , j \leq n}^{}  y_i x_j a_{ij}
  .\] 
  \[
  \langle \vec{ x} , \vec{ y}   \rangle = \langle  \vec{ y} , \vec{ x}   \rangle \iff a_{ i j } = a_{ j i} \qquad  \forall  i,j
  .\] 
  \[
  \text{ i.e.  when } A = A ^{T} \text{ i.e.  $ A$ is symmetric.}
  .\] 
 }
 \section{Inner Product }
 \dfn{Inner Product :}{
 An inner product on a real vector space is a bilinear form which is positive definite and symmetric.\\
 }
 \textbf{Example:} \\
 \begin{enumerate}[label=(\arabic*).]  
   \item Dot product on $ \mathbb{R} ^{n}$ 
   \item \[
   \langle f,g  \rangle = \int_{a}^{b} f \left( x  \right) g \left( x  \right) dx \text{ defined on } \mathcal{P} _{n} \left[ x \right] 
   .\] 
 \end{enumerate}
  \\
  \textit{Which of the following define an inner product on $ \mathbb{R} ^{2}$?}\\
  \[
  \vec{ x} = \begin{bmatrix}
  x_1\\
  x_2\\
  \end{bmatrix}
   \qquad  \vec{ y} = \begin{bmatrix}
   y_1\\
   y_2\\
   \end{bmatrix}
   
  .\] 
  \begin{enumerate} [label=(\alph*)]
    \item 
     \[
     \langle \vec{ x} ,\vec{ y}   \rangle = x_1 y_2 + x_2 y_1
     .\] 
     Bilinear?\\
     Yes, matrix $ = \begin{bmatrix}
     0 & 1\\
     1 & 0\\
     \end{bmatrix} = A$ \\
     Symmetric?\\
     Yes, $ A = A ^{T}$\\
     also, $ \langle \vec{ y} ,\vec{ x}   \rangle = y_1 x_2 + y_2 x_1 = \langle \vec{ x} ,\vec{ y}   \rangle $ \\
     positive definite?\\
     $ \langle \vec{ x} , \vec{ x}   \rangle = 2 x_1 x_2$ NO\\
     \[
     \langle  \begin{bmatrix}
     1\\
     0\\
     \end{bmatrix}
     , \begin{bmatrix}
     1\\
     0\\
     \end{bmatrix}
       \rangle =0  \quad \text{ but } \begin{bmatrix}
       1\\
       0\\
       \end{bmatrix}
        \neq \begin{bmatrix}
        0\\
        0\\
        \end{bmatrix}
     .\] 
     \[
     \text{ or}               \langle \begin{bmatrix}
     1\\
     -1\\
     \end{bmatrix}
     , \begin{bmatrix}
     1\\
     -1\\
     \end{bmatrix}
       \rangle = -2 < 0
     .\] 
    \item 
     \[
     \langle \vec{ x} ,\vec{ y}   \rangle =x_1 y_1 \qquad  A = \begin{bmatrix}
     1 & 0\\
     0 & 0\\
     \end{bmatrix} 
     .\] 
     Bilinear? Yes\\
     Symmetric? Yes $ A^{T}=A$ \\
     positive definite?\\
     \[
     \text{ No } \qquad  \langle  \begin{bmatrix}
     0\\
     1\\
     \end{bmatrix}
     , \begin{bmatrix}
     0\\
     1\\
     \end{bmatrix}
       \rangle =0  \quad \text{ but } \begin{bmatrix}
       0\\
       1\\
       \end{bmatrix}
        \neq \begin{bmatrix}
        0\\
        0\\
        \end{bmatrix}
        
     .\] 
    \item 
     \[
     \langle \vec{ x} ,\vec{ y}   \rangle = x_1 y_1 + 7 x_2 y_2 \qquad  A = \begin{bmatrix}
     1 & 0\\
     0 & 7\\
     \end{bmatrix}
     .\] 
     bilinear? Yes\\
     symmetric? Yes $ A^{T}=A$ \\
     positive definite? $ \langle \vec{ x} ,\vec{ x}   \rangle = x_1^2 + 7 x_2 ^2 > 0$ with equality  $ \iff$ $ x_1 = x_2 =0$.\\
     $ \implies \langle \vec{ x} ,\vec{ y}   \rangle $ is an inner product 
    \item $ \langle \vec{ x} , \vec{ y}   \rangle = x_1 y_1 + x_1 y_2 + x_2 y_1 + x_2 y_2$
     \[
     A = \begin{bmatrix}
     1 & 1\\
     1 & 1\\
     \end{bmatrix}
     .\] 
     bilinear? Yes\\
     symmetric? Yes $ A^{T}=A$ \\
     positive definite?\\
     \begin{align*}
      \langle \vec{ x} ,\vec{ x}   \rangle &= x_1^2 + 2 x_1 x_2 + x_2 ^2 \\
      &= \left( x_1 + x_2 \right) ^2 
     .\end{align*}
     \[
     \text{No } \langle \begin{bmatrix}
     1\\
     -1\\
     \end{bmatrix}  , \begin{bmatrix}
     1\\
     -1\\
     \end{bmatrix}\rangle =0 \qquad  \text{ but } \begin{bmatrix}
     1\\
     -1\\
     \end{bmatrix}
     \neq \begin{bmatrix}
     0\\
     0\\
     \end{bmatrix}
        
     .\]
     $ \implies$ not an inner product.\\
    \end{enumerate}


    \dfn{Euclidean Length/ Vector Space :}{
    \begin{enumerate}[label=(\arabic*).]  
      \item A real vector space equipped with an inner product is called a \underline{Euclidean vector space}.\\
      \item   On a Euclidean vector space $ V$ we define the \underline{length} of a vector $ \vec{ v} \in V$ as
       \[
       \left\| \vec{ v}  \right\| = \sqrt{\langle \vec{ v} ,\vec{ v}   \rangle}
       .\] 
       A vector of length $ 1$ is called a \underline{unit vector}.\\
    \end{enumerate}

   }
      \thm{Cauchy -Schwarz Inequality :}
      {
         In a Euclidean vector space $ V$ we have that 
         \[
         |\langle \vec{ x} ,\vec{ y}   \rangle | \leq \left\| \vec{ x}  \right\| \left\| \vec{ y}  \right\| \qquad  \forall  \vec{ x} ,\vec{ y} \in V
         .\] 
      }
    \pf{Proof:}{
      Let $ \lambda \in \mathbb{R} $ 
      \[
      \left| \vec{ x} + \lambda \vec{ y} \right| ^2 \ge 0 \qquad  \forall  \lambda \in \mathbb{R}, \quad \vec{ x} , vec{ y} \in V
      .\] 
      \[
       \text{ i.e. } \langle \vec{ x} + \lambda \vec{ y} , \vec{ x} + \lambda \vec{ y}   \rangle \ge 0
      .\] 
      \[
      \langle \vec{ x} , \vec{ x}   \rangle + 2 \lambda \langle \vec{ x} , \vec{ y}   \rangle + \lambda ^2 \langle \vec{ y} ,\vec{ y}   \rangle \ge 0
      .\] 
      \[
      \left| \vec{ x} \right| ^2 + 2 \lambda \langle \vec{ x} , \vec{ y}   \rangle + \lambda ^2 \left| \vec{ y}  \right| ^2 \ge 0
      .\] 
      where the first term is a quadratic in $ \lambda$, which is non-negative.\\
      \[
      \iff \left( 2 \langle  \vec{ x} ,\vec{ y}   \rangle  \right) ^2 - 4 \left| \vec{ x}  \right| ^2 \left| \vec{ y}  \right| ^2 \le 0
      .\] 
      \begin{align*}
       \text{ i.e. } 4 \left( \langle \vec{ x} , \vec{ y}   \rangle  \right) ^2 &\leq 4 \left| \vec{ x}  \right| ^2 \left| \vec{ y}  \right| ^2 \\
       \left( \langle \vec{ x} ,\vec{ y}   \rangle  \right) ^2 &\leq \left| \vec{ x}  \right| ^2 \left| \vec{ y}  \right| ^2 \\
       \implies |\langle \vec{ x} ,\vec{ y}   \rangle | &\leq \left| \vec{ x}  \right| \left| \vec{ y}  \right|
      .\end{align*}
    }
   
    \dfn{ :}{
    We define the angle $ \theta $ between two vectors $ \vec{ v} ,\vec{ w} \in V$ in a Euclidean vector space $ V$ as
     \[
     \cos \theta = \frac{\langle \vec{ v} ,\vec{ w}   \rangle}{\left\| \vec{ v}  \right\| \left\| \vec{ w}  \right\|} 
     .\] 
    }
     \underline{Orthongonality:}\\
     Suppose $ V$ is a Euclidean vector space.\\
     \dfn{Orthonormal and Orthogonal Vectors :}{
          \begin{enumerate}[label=(\arabic*).]  
            \item $ \vec{ x} ,\vec{ y} \in V$ are said to be \underline{orthogonal} if $ \langle \vec{ x} ,\vec{ y}   \rangle =0$
            \item   A basis $ \left\{ \vec{ v_1} , \ldots , \vec{ v_n}  \right\} $  of $ V$ is said to be orthogonal if
              $ \langle \vec{ v_i} , \vec{ v_j}   \rangle =0\qquad  \forall  i \neq j$
             \item A basis $ \left\{ \vec{ v_1} , \ldots , \vec{ v_n}  \right\} $ of $ V$ is said to be orthonormal if
            the basis is orthogonal and $ \left\| \vec{ v_i}  \right\| =1$  i.e.  $ \left| \vec{ v_i} \right| \qquad  \forall  i$ 
          \end{enumerate}
     }

     \ex{}{
     \[
      \mathcal{B} = \left\{ \begin{bmatrix}
      2\\
      0\\
      \end{bmatrix}
      , \begin{bmatrix}
      0\\
      2\\
      \end{bmatrix}
       \right\}  \text{ is an orthogonal basis of } \mathbb{R} ^2
     .\] 
     \[
     \mathcal{C} = \left\{ \begin{bmatrix}
     1\\
     0\\
     \end{bmatrix}
     , \begin{bmatrix}
     0\\
     1\\
     \end{bmatrix}
      \right\}  \text{ is an orthonormal basis of } \mathbb{R} ^2
     .\] 
     }
     \thm{Orthogonal co-ordinates :}
     {
      If $ \left\{ \vec{ v_1} , \ldots \vec{ v_n}  \right\} $ forms an orthogonal basis of $ V$ and $ \vec{ v} \in V$, then \[
      \vec{ v} = \sum\limits_{i=1}^{n} \frac{\langle \vec{ v} , \vec{ v_i}   \rangle}{\left\| \vec{ v_i}  \right\| ^2} \vec{ v_i}
      .\] 
     }
     \pf{Proof:}{
      Since $ \left\{ \vec{ v_1} , \ldots \vec{ v_n}  \right\} $ forms a basis of $ V$,  there exists  scalars $ c_1, \ldots , c_n$ such that
      \[
      \vec{ v} = \sum\limits_{i=1}^{n} c_i \vec{ v_i}
      .\] 
      Taking the inner product with each $ \vec{ v_j} $ we get 
      \[
      \langle \vec{ v} , \vec{ v_j}   \rangle = \sum\limits_{i=1}^{n} c_i \langle \vec{ v_i} , \vec{ v_j}   \rangle= c_j \langle \vec{ v_j} , \vec{ v_i}   \rangle 
      .\] 
      since $ \langle \vec{ v_i} ,\vec{ v_j}\rangle =0 \qquad  \forall  i \neq j   $
      \[
      \implies \langle  \vec{ v} , \vec{ v_j}   \rangle = c_j \langle \vec{ v_j} ,\vec{ v_j}   \rangle 
      .\] 
      \[
      \implies \frac{  \langle \vec{ v} , \vec{ v_j}   \rangle   }{ \langle \vec{ v_j} ,\vec{ v_j}   \rangle } = c_j
      .\] 
      Hence, 
      \[
      \vec{ v} = \sum\limits_{i=1}^{n} \frac{\langle \vec{ v} , \vec{ v_i}   \rangle}{\langle \vec{ v_i} ,\vec{ v_i}   \rangle } \vec{ v_i}
      .\] 
     }
     \underline{Remark:}\\
     If the basis is orthonormal, then
     \[
     \langle \vec{ v_i} , \vec{ v_i}   \rangle =1 \qquad   \text{ and } \qquad  \vec{ v} = \sum\limits_{i=1}^{n} \langle \vec{ v} , \vec{ v_i}   \rangle \vec{ v_i} 
     .\] 
     \nt{
      $ \left\{ \vec{ e_1} , \vec{ e_2}  \right\}$  is an orthonormal basis of $ \mathbb{R} ^2 \qquad  \forall  \begin{bmatrix}
      x\\
      y\\
      \end{bmatrix}
      \in \mathbb{R} ^2$ 
      \[
      \begin{bmatrix}
      x\\
      y\\
      \end{bmatrix}
      = \langle \begin{bmatrix}
      x\\
      y\\
      \end{bmatrix}
      , \begin{bmatrix}
      1\\
      0\\
      \end{bmatrix}
        \rangle  \begin{bmatrix}
        1\\
        0\\
        \end{bmatrix}
        + \langle \begin{bmatrix}
        x\\
        y\\
        \end{bmatrix}
        , \begin{bmatrix}
        0\\
        1\\
        \end{bmatrix}
          \rangle \begin{bmatrix}
          0\\
          1\\
          \end{bmatrix}
          
      .\] 
     }
     \ex{}{
     Let 
     \[
     \vec{ v_1} = \frac{1} {   \sqrt{58} }     \begin{bmatrix}
     3\\
     7\\
     \end{bmatrix}
     \qquad \vec{ v_2} = \frac{1} {   \sqrt{58} }     \begin{bmatrix}
     -7\\
     3\\
     \end{bmatrix}
     
     .\] 
     $ \left\{ \vec{ v_1} , \vec{ v_2}  \right\}$ forms an orthonormal basis for $ \mathbb{R} ^2$.\\
     check! $ \langle \vec{ v_1} , \vec{ v_2}   \rangle = 0$ \\
     \[
     \left| \vec{ v_1} \right| = 1  \qquad \left| \vec{ v_2} \right| = 1 
     .\] 
     Given any other vector, $ \vec{ v} = \begin{bmatrix}
     2\\
     5\\
     \end{bmatrix}
     $
     \[
     \vec{ v} = \langle \begin{bmatrix}
     2\\
     5\\
     \end{bmatrix}
     , \frac{1}{  \sqrt{58} }     \begin{bmatrix}
     3\\
     7\\
     \end{bmatrix}
      \rangle  \frac{1}{  \sqrt{58} }     \begin{bmatrix}
      3\\
      7\\
      \end{bmatrix}
      + \langle \begin{bmatrix}
      2\\
      5\\
      \end{bmatrix}
      , \frac{1}{  \sqrt{58} }     \begin{bmatrix}
      -7\\
      3\\
      \end{bmatrix}
        \rangle  \frac{1}{  \sqrt{58} }     \begin{bmatrix}
        -7\\
        3\\
        \end{bmatrix}
        
     .\]
     \[
     \vec{ v} = \frac{ 6 + 35  }{ \sqrt{58}  } \vec{ v_1} + \frac{ -14 + 15  }{ \sqrt{58}  } \vec{ v_2}
     .\] 
     \[
     \begin{bmatrix}
     2\\
     5\\
     \end{bmatrix}
     = \vec{ v} = \frac{ 41 }{ \sqrt{58}  } \vec{ v_1} + \frac{ 1 }{ \sqrt{58}  } \vec{ v_2}
     .\] 
     }
     
     \section{Gram-Schmidt Orthogonalization}
     Suppose $ \left\{  \vec{ v_1} , \ldots , \vec{ v_n}  \right\}$ is an arbitrary basis of a vector space $ V$. We wish to use this basis to construct an orthogonal basis $ \left\{ \vec{ w_1} , \ldots , \vec{ w_n}  \right\} $ of $ V$.\\
     \\
     \\
     \underline{Case 1:} Two vectors \\
     \[
      \underbrace{ \left\{ \vec{ v_1} ,\vec{ v_2}  \right\} } _{ \text{ basis } }  \to  \underbrace{\left\{ \vec{ w_1} , \vec{ w_2}  \right\}  }_{ \text{ orthogonal basis } }
     .\] 
     Let $ \vec{ w_1} = \vec{ v_1} $ i.e.  keep the first basis vector.
              \begin{tikzpicture}[>=stealth,thick,line cap=round,line join=round]
  %--- coordinates -----------------------------------------------------------
  \coordinate (O)  at (0,0);          % origin
  \coordinate (V1) at (4,0);          % end-point of v1
  \coordinate (V2) at (3,2);          % end-point of v2
  \coordinate (P)  at (3,0);          % foot of the perpendicular

  %--- vectors ---------------------------------------------------------------
  \draw[->,blue]          (O) -- (V1) node[below right] {$\vec v_1 = \vec\omega_1$};
  \draw[->,blue]          (O) -- (V2) node[above]       {$\vec v_2$};
  \draw[dashed,magenta]   (V2) -- (P)  node[midway,right] {$\vec w_2$};
  \draw[->,ForestGreen]   (O) -- (P)  node[below]       {$\operatorname{proj}_{\vec v_1}\vec v_2$};

  %--- right-angle marker ----------------------------------------------------
  \draw ($(P)+(0,0.25)$) -- ($(P)+(0.25,0.25)$) -- ($(P)+(0.25,0)$);

  %--- angle θ at the origin -------------------------------------------------
  \pic [draw,->,blue,"\small$\theta$",angle radius=10,angle eccentricity=1.5]
       {angle = V1--O--V2};
\end{tikzpicture}



     Aim: to decompose $ \vec{ v_2} $ into a sum of some scalar multiple of $ \vec{ v_1} = \vec{ w_1} $ and some scalar multiple of a vector that is orthogonal to $ \vec{ w_1} $.\\
     \[
      \cos \theta = \frac{ \left| \text{ proj}_{ \vec{ w_1} \vec{ v_2} } \right| }{  \left| \vec{ v_2}  \right| }
     .\] 
     \[
      \implies \left| \text{ proj}_{ \vec{ w_1} \vec{ v_2} } \right| =  \left| \vec{ v_2}  \right| \cos \theta 
     .\] 
     \[
      \text{ proj}_{ \vec{ w_1} \vec{ v_2} } = \left| \vec{ v_2}  \right| \cos \theta \frac{ \vec{ w_1} }{ \left| \vec{ w_1}  \right| } = \left| \vec{ v_2}  \right| \frac{ \langle \vec{ v_2} , \vec{ w_1}   \rangle   }{ \left| \vec{ w_1}  \right| \left| \vec{ v_2} \right| }  \left(  \frac{ \vec{ w_1}   }{ \left| \vec{ w_1} \right| } \right) 
     .\] 
     \[
     \text{ Note: } \vec{ v_2} = \frac{ \langle  \vec{ v_2} , \vec{ w_1}   \rangle   }{ \langle  \vec{ w_1} , \vec{ w_1}   \rangle  } \vec{ w_1} + \vec{ w_2} 
     .\] 
     $ \implies$ Let $ \vec{ w_2} = \vec{ v_2} - \frac{  \langle \vec{ v_2} , \vec{ w_1}   \rangle   }{ \langle  \vec{ w_1} , \vec{ w_1}   \rangle  } \vec{ w_1} $ \\
     Claim: $ \vec{ w_2} $ is orthogonal to $ \vec{ w_1} $.\\
     \pf{Proof:}{
      \begin{align*}
       \langle \vec{ w_2} , \vec{ w_1}   \rangle &= \langle \vec{ v_2} - \frac{ \langle \vec{ v_2} , \vec{ w_1}    \rangle   }{ \langle \vec{ w_1} , \vec{ w_1}   \rangle  } \vec{ w_1} , \vec{ w_1}   \rangle \\
       &= \langle \vec{ v_2} , \vec{ w_1}   \rangle - \frac{ \langle \vec{ v_2} , \vec{ w_1}   \rangle   }{ \langle  \vec{ w_1} ,\vec{ w_1}   \rangle  } \langle  \vec{ w_1} ,\vec{ w_1}   \rangle  \\
       &= \langle \vec{ v_2} , \vec{ w_1}   \rangle - \langle \vec{ v_2} , \vec{ w_1}   \rangle =0
      .\end{align*}
      $ \vec{ w_2} $ lies in the same plane as $ \vec{ v_1} = \vec{ w_1} $ and $ \vec{ v_2} $ \\
     \[
     \implies span \left\{ \vec{ w_1} , \vec{ w_2}  \right\} = span \left\{ \vec{ v_1} , \vec{ v_2}  \right\}
     .\] 
     Since $  dim V = 2 \implies \left\{ \vec{ w_1} ,\vec{ w_2}  \right\} $ form an orthogonal basis for $ V$ \\
     \\
     Assume that given a basis of $ n$ vectors $ \left\{ \vec{ v_1} , \ldots , \vec{ v_n}  \right\} $ of $ V$, the first $ k$ vectors can be replaced by an orthogonal basis $ \left\{ \vec{ w_1} , \ldots , \vec{ w_k}  \right\} $ such that 
     \[
     span \left\{ \vec{ v_1} , \ldots , \vec{ v_k}  \right\} = span \left\{ \vec{ w_1} , \ldots , \vec{ w_k}  \right\}
     .\] 
     We decompose $ \vec{ v_{k+1}} $ as follows 
     \[
      \vec{ v_{k+1}} = \sum\limits_{i=1}^{k} \frac{ \langle  \vec{ V _{ k+1}} , \vec{ w_i}   \rangle   }{ \langle \vec{ w_i} , \vec{ w_i}   \rangle  }\vec{ w_i} + \left(  \vec{ v _{ k+1}} - \sum\limits_{i=1}^{k} \frac{ \langle \vec{ v_{k+1}}, \vec{ w_i}    \rangle   }{ \langle \vec{ w_i} , \vec{ w_i}   \rangle  } \vec{ w_i} \right)
       \right) 
     .\]
     where the firs term is the part of $ \vec{ v_{k+1}} $ that lies in the k-dimensional subspace spanned by $ \left\{ \vec{ w_1} , \ldots , \vec{ w_k}  \right\} $ and the second term is the perpendicular part.\\
     \\
     Define 
     \[
     \vec{ w_{k+1}} = \vec{ v_{k+1}} - \sum\limits_{i=1}^{k} \frac{ \langle \vec{ v_{k+1}}, \vec{ w_i}    \rangle   }{ \langle \vec{ w_i} , \vec{ w_i}   \rangle  } \vec{ w_i}
     .\] 
     Claim: $ \vec{ w_{k+1}} $ is orthogonal to $ \vec{ w_i} \qquad  \forall  i = 1, \ldots , k$\\
     \[
     \langle \vec{ w_{k+1}} , \vec{ w_i}   \rangle = \langle \vec{ v_{k+1}}, \vec{ w_j} \rangle - \sum\limits_{j=1}^{k} \frac{ \langle \vec{ v_{k+1}}, \vec{ w_j}    \rangle   }{ \langle \vec{ w_j} , \vec{ w_j}   \rangle  } \langle \vec{ w_j} ,\vec{ w_i}   \rangle
     .\] 
     \[
     \langle \vec{ w_{k+1}} , \vec{ w_j}   \rangle = \langle  \vec{ v_{k+1}} , \vec{ w_j}    \rangle - \frac{ \langle \vec{ v_{k+1}}, \vec{ w_j}    \rangle   }{ \langle \vec{ w_j} , \vec{ w_j}   \rangle  } \langle \vec{ w_j} ,\vec{ w_j}   \rangle =0
     .\] 
     Hence $ \left\{ \vec{ w_1} , \ldots , \vec{ w _{ k+1}}  \right\} $ is an orthogonal set and 
     \begin{align*}
      & span \left\{ \vec{ v_1} , \ldots , \vec{ v_{k+1}}  \right\} = span \left\{ \vec{ w_1} , \ldots , \vec{ w_{k+1}}  \right\} \\
      &= span \left\{ \vec{ w_1} , \ldots , \vec{ w_k}  , \vec{ w_{k+1}}  + \sum\limits_{j=1}^{k} c_j \vec{ w_j}  \right\} \\
      &= span \left\{ \vec{ w_1} , \ldots , \vec{ w_k}  , \vec{ w_{k+1}}  \right\}
     .\end{align*}
     }
     \ex{}{
     Let 
     \[
     \vec{ u_1} = \begin{bmatrix}
     1\\
     1\\
     1\\
     \end{bmatrix}
      , \qquad  \vec{ u_2} = \begin{bmatrix}
      0\\
      1\\
      1\\
      \end{bmatrix}
       , \qquad  \vec{ u_3} =  \begin{bmatrix}
       0\\
       0\\
       1\\
       \end{bmatrix}
     .\]
     $ \left\{ \vec{ u_1} , \vec{ u_2} , \vec{ u_3}  \right\} $ forms a basis of $ \mathbb{R} ^3$. Use the Gram Schmidt orthogonalization process to find an orthogonal basis of $ \mathbb{R} ^3$.\\
     Let $ \vec{ w_1} = \vec{ u_1} = \begin{bmatrix}
     1\\
     1\\
     1\\
     \end{bmatrix}
       $
       \[
       \vec{ w_2} = \vec{ u_2} - \frac{ \langle \vec{ u_2} , \vec{ w_1}   \rangle   }{ \langle \vec{ w_1} ,\vec{ w_1}   \rangle  } \vec{ w_1} = \begin{bmatrix}
       0\\
       1\\
       1\\
       \end{bmatrix}
         - \frac{2}{3} \begin{bmatrix}
         1\\
         1\\
         1\\
         \end{bmatrix}
          
       .\] 
       \[
       \vec{ w_2} = \begin{bmatrix}
       - \frac{2}{3}\\
       \frac{1}{3}\\
       \frac{1}{3}\\
       \end{bmatrix}
        
       .\] 
       \begin{align*}
        \vec{ w_3} &= \vec{ u_3} - \frac{ \langle \vec{ u_3} , \vec{ w_1}   \rangle   }{ \langle \vec{ w_1} ,\vec{ w_1}   \rangle  } \vec{ w_1} - \frac{ \langle \vec{ u_3} , \vec{ w_2}   \rangle   }{ \langle \vec{ w_2} ,\vec{ w_2}   \rangle  } \vec{ w_2} \\ 
        &=  \begin{bmatrix}
        0\\
        0\\
        1\\
        \end{bmatrix}
         - \frac{1}{3} \begin{bmatrix}
         1\\
         1\\
         1\\
         \end{bmatrix}
           - \frac{  \frac{ 1  }{ 3 }  }{  \frac{2}{3} } \begin{bmatrix}
           - \frac{2}{3}\\
           \frac{1}{3}\\
           \frac{1}{3}\\
           \end{bmatrix}
            \\    
            &=  \begin{bmatrix}
        0\\
        0\\
        1\\
        \end{bmatrix}
         - \frac{1}{3} \begin{bmatrix}
         1\\
         1\\
         1\\
         \end{bmatrix}
           - \frac{  1  }{  2 } \begin{bmatrix}
           - \frac{2}{3}\\
           \frac{1}{3}\\
           \frac{1}{3}\\
           \end{bmatrix}
            \\    
        \vec{ w_3} &= \begin{bmatrix}
        0 \\
        - \frac{1}{2}\\
        \frac{1}{2}\\
        \end{bmatrix}
       .\end{align*}
     }
     
     
      
     
     
     
     
     
      
    
    
    
 

  
 
 
 
 
 
 
























\end{document}   
